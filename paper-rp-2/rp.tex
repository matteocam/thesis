\section{Rational Proofs}

The following is the definition of Rational Proof from \cite{am}. As usual with $\negl(\cdot)$ we denote a {\em negligible} function, i.e. one that is asymptotically smaller than the inverse of any polynomial. Conversely a {\em noticeable} function is the inverse of a polynomial. 

% XXX: Do we need 
%In the following we will adopt a "concrete-security" version of the "asymptotic" 
%definitions and theorems in \cite{am1,ratargs}. We assume the reader is familiar with 
%the notion of interactive proofs \cite{gmr}. 

\noindent
\begin{definition}[Rational Proof]
\label{def:RP-delta}
\label{def:RP}
A function $f:$ $\bit^n$ $\to$ $\bit^*$ admits a rational proof if there exists an interactive proof $(P,V)$ and a randomized reward function
$\rew : \bits \to \posreals$ such that

\begin{enumerate}
\item \label{item:completeness} For any input $x \in 
\bit^n$, $\Pr[\out((P,V)(x)) = f(x)] \geq 1 - \negl(n).$

\item For every prover $\disP$, and for any input $x \in 
\bit^n$ there exists a $\delta_{\disP}(x) \geq 0$ such that 
$ \expRewProtDis + \delta_{\disP}(x) \leq \expRewProtHon. $
\end{enumerate}
The expectations and the probabilities are taken over the random coins of the prover and verifier.
\end{definition} 
We note that differently than \cite{am} we allow for non-perfect completeness: a negligible probability that even the correct prover will prove the wrong result. This will be necessary for our protocols for randomized computations. 


\medskip
\noindent
Let $\epsilon_{\disP} = \Pr[\out((P,V)(x)) \neq f(x)]$. 
Following \cite{ratargs} we define the {\sf reward gap} as 
\[ \Delta(x) = min_{P^* : \epsilon_{P^*}=1}[\delta_{P^*}(x)]  \]
i.e. the minimum reward gap over the provers that always report the incorrect value. 
It is easy to see that for arbitrary prover $\disP$ we have $\delta_{\disP}(x) \geq 
\epsilon_{\disP} \cdot \Delta(x)$. Therefore it suffices to prove that a protocol has 
a strictly positive reward gap $\Delta(x)$ for all $x$. 

% TODO: Ensure that $\Delta(x)$ is used consistently in all proofs.


\begin{definition}[\cite{am,am1,ratargs}]
The class $\DRMA[r, c, T]$ (Decisional Rational Merlin Arthur)
is the class of boolean functions $f : \bits \to \bit$ admitting a rational proof $\Pi = (P,V, \rew)$ s.t. on input $x$:
\begin{itemize}
    \item $\Pi$ terminates in $r(|x|)$ rounds;
    \item The communication complexity of $P$ is $c(|x|)$;
    \item The running time of $V$ is $T(|x|)$;
    \item The function $\rew$ is bounded by a polynomial;
    \item $\Pi$ has noticeable reward gap.
\end{itemize}
\end{definition}

\noindent
\begin{remark}
\label{rem:asy}
{\em The requirement that the reward gap must be noticeable was introduced in 
\cite{am1,ratargs} and is explained in Section~\ref{sec:proofs-seq-comp}.}
\end{remark}

%\subsection{A Warmup Example}
\label{sec:example}
\smallskip
\noindent
{\sc Examples of Rational Proofs.} For concreteness here we show the protocol for a single threshold gate (readers are referred to \cite{am,am1,rosen} for more examples). 

Let $G_{n,k}(x_1,\ldots,x_n)$ be a threshold gate with $n$ Boolean inputs, that evaluates to 1 if at least $k$ of the input bits are 1. The protocol in \cite{am1} to evaluate this gate goes as follows. The Prover announces the number $\tilde{m}$ of input bits equal to 1, which allows the Verifier to compute $G_{n,k}(x_1,\ldots,x_n)$. The Verifier select a random index $i \in [1..n]$ and looks at input bit $b=x_i$ and rewards the Prover using Brier's Rule $BSR(\tilde{p},b)$ where $\tilde{p}=\tilde{m}/n$ i.e. the probability claimed by the Prover that a randomly selected input bit be 1. Then
\[BSR(\tilde{p},1) = 2\tilde{p} - \tilde{p}^2 - (1-\tilde{p})^2 + 1 = 2\tilde{p}(2-\tilde{p}) \]
\[BSR(\tilde{p},0) = 2(1-\tilde{p}) - \tilde{p}^2 - (1-\tilde{p})^2 +1 = 2(1-\tilde{p}^2) \]
Let $m$ be the true number of input bits equal to 1, and $p=m/n$ the corresponding probability, then the expected reward of the Prover is
\begin{equation}
\label{eq:bsr}
p BSR(\tilde{p},1) + (1-p) BSR(\tilde{p},0) 
\end{equation}
which is easily seen to be maximized for $p=\tilde{p}$ i.e. when the Prover announces the correct result. Moreover one can see that when the Prover announces a wrong $\tilde{m}$ his reward goes down by $2(p-\tilde{p})^2 \geq 2/n^2$. In other words 
for all $n$-bit input $x$, we have $\Delta(x)=2/n^2$ and if a dishonest Prover $\disP$ cheats with probability $\epsilon_{\disP}$ then $\delta_{\disP} > 2\epsilon_{\disP}/n^2$. 




%
%Consider the function $f:$ $\bit^n$ $\to$ $[0 \ldots n]$ which on input $x$ outputs the 
%Hamming weight of $x$ (i.e. $\sum_{i} x_i$ where $x_i$ are the bits of $x$). 
%
%In \cite{am1} the prover announces a number $\tilde{m}$ which he claims to be equal to $m=f(x)$. This can be interpreted as the prover announcing that if one chooses an input bit $x_i$ at random it will be equal to 1 with probability $\tilde{p}=\tilde{m}/n$. The verifier then chooses a random input bit $x_i$ and uses $\tilde{m},x_i$ to compute the reward via a scoring rule. Since the scoring rule is maximized by the announcement of the correct $m$, a rational prover will announce the correct value. The scoring rule used in \cite{am1} (and in all other rational proofs based on scoring rules) is 
%Brier's rule where the reward is computed as $BSR(\tilde{p},x_i)$ where: 
%\[BSR(\tilde{p},1) = 2\tilde{p}(2-\tilde{p}) \; \; \mbox{ and } \;\; 
%BSR(\tilde{p},0) = 2(1-\tilde{p}^2) \]
%Notice that $p=m/n$ is the actual probability to get 1 when selecting an input bit at random so the expected reqward of the prover is 
%\begin{equation}
%\label{eq:bsr}
%p BSR(\tilde{p},1) + (1-p) BSR(\tilde{p},0) 
%\end{equation}
%which is easily seen to be maximized for $\tilde{p}=p$, i.e. $\tilde{m}=m$. 
%
%in \cite{cg15} we propose an alternative protocol for $f$ (motivated by the issues we
%discuss in Section~\ref{sec:proofs-seq-comp}). In our protocol we compute $f$ via an "addition circuit", organized as a complete binary tree with $n$ leaves which are the input, and where each internal node is a (fan-in 2) addition gate -- note that this circuit has depth $d=\log n$. The protocol has $d$ rounds: at the first round the prover announces $\tilde{m}$ (the claimed value of $f(x)$) and its two "children" $y_L,y_R$ in the output gate, i.e. the two input values of the last output gate $G$. The Verifier checks that 
%$y_L +y_R=\tilde{m}$, and then asks the Prover to verify that $y_L$ or $y_R$ (chosen a random) is correct, by recursing on the above test. At the end the verifier has to check the last addition gate on two input bits: she performs this test on her own by reading just those two bits. If any of the tests fails, the verifier pays a reward of 0, otherwise she will pay $R$. The intuition is that a cheating prover will be caught with probability $2^{-d}$ which is exactly the reward gap (and for log-depth circuits like this one is noticeable). 
%
%Note that the first protocol is a scoring-rule based one, while the second one is a weak-interactive proof. 
%

