% -*-latex-*-

\pagestyle{plain}

\newgeometry{left=3.5cm,bottom=0.1cm}

%\hyphenation{Efficient}
%\hyphenation{Verifiable}
%\hyphenation{Computation}


\title{Rationality and Efficient Verifiable Computation}

\titletufte{Rationality and\newline Efficient Verifiable\newline Computation}

\publisher{A 
	dissertation submitted to the Graduate Faculty 
	in Computer Science
	in partial fulfillment of the 
	requirements for 
	the	degree of Doctor of Philosophy,The City	University of New York}
\author{Matteo Campanelli}
% If you wish to list your previous degrees on the cover page, use the 
% previous degrees command:
% \prevdegrees{A.A., Harvard University (1985)}
% You can use the \\ command to list multiple previous degrees
\department{Department of Computer Science}
\date{}

% If the thesis is for two degrees simultaneously, list them both
% separated by \and like this:
% \degree{Doctor of Philosophy \and Master of Science}
\degree{Doctor of Philosophy in Computer Science}

% As of the 2007-08 academic year, valid degree months are September, 
% February, or June.  The default is June.
\degreemonth{June}
\degreeyear{2018}
\thesisdate{May 2, 2018}

%% By default, the thesis will be copyrighted to MIT.  If you need to copyright
%% the thesis to yourself, just specify the `vi' documentclass option.  If for
%% some reason you want to exactly specify the copyright notice text, you can
%% use the \copyrightnoticetext command.  
%\copyrightnoticetext{\copyright IBM, 1990.  Do not open till Xmas.}

% If there is more than one supervisor, use the \supervisor command
% once for each.
\supervisor{Rosario Gennaro}{Professor}

% This is the department committee chairman, not the thesis committee
% chairman.  You should replace this with your Department's Committee
% Chairman.
\chairman{Robert Haralick}{Chairperson}

% Make the titlepage based on the above information.  If you need
% something special and can't use the standard form, you can specify
% the exact text of the titlepage yourself.  Put it in a titlepage
% environment and leave blank lines where you want vertical space.
% The spaces will be adjusted to fill the entire page.  The dotted
% lines for the signatures are made with the \signature command.
\maketitlepage

% The abstractpage environment sets up everything on the page except
% the text itself.  The title and other header material are put at the
% top of the page, and the supervisors are listed at the bottom.  A
% new page is begun both before and after.  Of course, an abstract may
% be more than one page itself.  If you need more control over the
% format of the page, you can use the abstract environment, which puts
% the word "Abstract" at the beginning and single spaces its text.

%% You can either \input (*not* \include) your abstract file, or you can put
%% the text of the abstract directly between the \begin{abstractpage} and
%% \end{abstractpage} commands.

% First copy: start a new page, and save the page number.
\cleardoublepage
% Uncomment the next line if you do NOT want a page number on your
% abstract and acknowledgments pages.
% \pagestyle{empty}
%\setcounter{savepage}{\thepage}

\null\vfill
\noindent
\begin{center}
\copyright \ 2018 \\ Matteo Campanelli\\
All right reserved
\end{center}
\newpage

\begin{abstractpage}
% $Log: abstract.tex,v $
% Revision 1.1  93/05/14  14:56:25  starflt
% Initial revision
% 
% Revision 1.1  90/05/04  10:41:01  lwvanels
% Initial revision
% 
%
%% The text of your abstract and nothing else (other than comments) goes here.
%% It will be single-spaced and the rest of the text that is supposed to go on
%% the abstract page will be generated by the abstractpage environment.  This
%% file should be \input (not \include 'd) from cover.tex.
	\noindent
	%Delegating computation is a prominent computing paradigm 	
	In this thesis, we study protocols for delegating computation in a model where one of the parties is rational. In our model, a \textit{delegator} outsources the computation of a function $f$ on input $x$ to a \textit{worker}, who
	receives a (possibly monetary) reward. Our goal is to design \textit{very efficient} delegation schemes 
	where a worker is economically incentivized to provide the correct result
	$f(x)$. In this work we strive for not relying on cryptographic assumptions, in particular our results do not require the existence of one-way functions. 
	
	We provide several results within the framework of rational proofs introduced by Azar and Micali (STOC 2012).
	We make several contributions to efficient rational proofs for general feasible computations.
	First, we design schemes with a sublinear verifier with low round and communication complexity for
	space-bounded computations.
	Second, we provide evidence, as lower bounds, against the existence of rational proofs:
	with logarithmic communication and polylogarithmic verification for $\P$ and 
	with polylogarithmic communication for $\NP$.
	% Third, we generalize a scaling approach in Azar and Micali (STOC) ... [composition theorem]
	
	We then move to study the case where a delegator outsources multiple inputs.
	First, we formalize an extended notion of rational proofs for this scenario (sequential composability) and we
	show that existing schemes do not satisfy it. We show how these protocols incentivize workers
	to provide many ``fast'' incorrect answers which allow them to solve more problems and collect more rewards.
	We then design a $d$-rounds rational proof for sufficiently ``regular'' arithmetic circuit of depth $d = O(\log{n})$
	with sublinear verification. We show, that under certain cost assumptions, our scheme is sequentially composable,
	i.e. it can be used to delegate multiple inputs. We finally show that our scheme for space-bounded computations is also 
	sequentially composable under certain cost assumptions.
	
	%The results above have been published as proceedings in GameSec as Campanelli and Gennaro (GameSec 2015)
	%and Campanelli and Gennaro (GameSec 2017).
	
  In the last part of this thesis we initiate the study of {\em Fine Grained Secure Computation}: i.e. 
  the construction of {\em secure computation primitives} against ``moderately complex" adversaries. Such fine-grained protocols can be used to obtain sequentially composable rational proofs. We present definitions and constructions for \textit{compact} Fully Homomorphic Encryption and Verifiable Computation secure against (\textit{non-uniform}) $\mathsf{NC}^1$ adversaries. Our results hold under a widely believed separation assumption, namely $\fgAssump$. We also present two application scenarios for our model: \textit{(i)} hardware chips that prove their own correctness, and \textit{(ii)} protocols against rational adversaries potentially relevant to the {\em Verifier's Dilemma} in smart-contracts transactions such as Ethereum. 

\end{abstractpage}

% Additional copy: start a new page, and reset the page number.  This way,
% the second copy of the abstract is not counted as separate pages.
% Uncomment the next 6 lines if you need two copies of the abstract
% page.
% \setcounter{page}{\thesavepage}
% \begin{abstractpage}
% % $Log: abstract.tex,v $
% Revision 1.1  93/05/14  14:56:25  starflt
% Initial revision
% 
% Revision 1.1  90/05/04  10:41:01  lwvanels
% Initial revision
% 
%
%% The text of your abstract and nothing else (other than comments) goes here.
%% It will be single-spaced and the rest of the text that is supposed to go on
%% the abstract page will be generated by the abstractpage environment.  This
%% file should be \input (not \include 'd) from cover.tex.
	\noindent
	%Delegating computation is a prominent computing paradigm 	
	In this thesis, we study protocols for delegating computation in a model where one of the parties is rational. In our model, a \textit{delegator} outsources the computation of a function $f$ on input $x$ to a \textit{worker}, who
	receives a (possibly monetary) reward. Our goal is to design \textit{very efficient} delegation schemes 
	where a worker is economically incentivized to provide the correct result
	$f(x)$. In this work we strive for not relying on cryptographic assumptions, in particular our results do not require the existence of one-way functions. 
	
	We provide several results within the framework of rational proofs introduced by Azar and Micali (STOC 2012).
	We make several contributions to efficient rational proofs for general feasible computations.
	First, we design schemes with a sublinear verifier with low round and communication complexity for
	space-bounded computations.
	Second, we provide evidence, as lower bounds, against the existence of rational proofs:
	with logarithmic communication and polylogarithmic verification for $\P$ and 
	with polylogarithmic communication for $\NP$.
	% Third, we generalize a scaling approach in Azar and Micali (STOC) ... [composition theorem]
	
	We then move to study the case where a delegator outsources multiple inputs.
	First, we formalize an extended notion of rational proofs for this scenario (sequential composability) and we
	show that existing schemes do not satisfy it. We show how these protocols incentivize workers
	to provide many ``fast'' incorrect answers which allow them to solve more problems and collect more rewards.
	We then design a $d$-rounds rational proof for sufficiently ``regular'' arithmetic circuit of depth $d = O(\log{n})$
	with sublinear verification. We show, that under certain cost assumptions, our scheme is sequentially composable,
	i.e. it can be used to delegate multiple inputs. We finally show that our scheme for space-bounded computations is also 
	sequentially composable under certain cost assumptions.
	
	%The results above have been published as proceedings in GameSec as Campanelli and Gennaro (GameSec 2015)
	%and Campanelli and Gennaro (GameSec 2017).
	
  In the last part of this thesis we initiate the study of {\em Fine Grained Secure Computation}: i.e. 
  the construction of {\em secure computation primitives} against ``moderately complex" adversaries. Such fine-grained protocols can be used to obtain sequentially composable rational proofs. We present definitions and constructions for \textit{compact} Fully Homomorphic Encryption and Verifiable Computation secure against (\textit{non-uniform}) $\mathsf{NC}^1$ adversaries. Our results hold under a widely believed separation assumption, namely $\fgAssump$. We also present two application scenarios for our model: \textit{(i)} hardware chips that prove their own correctness, and \textit{(ii)} protocols against rational adversaries potentially relevant to the {\em Verifier's Dilemma} in smart-contracts transactions such as Ethereum. 

% \end{abstractpage}

\cleardoublepage

% READER PAGE
\reader{Reader 1 Name}{Reader 1 Title}{Reader 1 Affiliation}
\reader{Reader 2 Name}{Reader 2 Title}{Reader 2 Affiliation}

\myreaderpage

\cleardoublepage

\section*{Acknowledgments}

%List:
%L,
%Alessio,
%
%Mariana, Itai,
%Kelly,
%Marios,
%
%Konstantinos, Anoop.
%Kahn Fellowship.
%Domenico, Rita.
%Thanks to the whole group.


This is one of the most important sections in a dissertation. In fact, although one's technical contributions may be modest or become obsolete soon after the time of writing, the acts by and the interactions with the people mentioned here are the real timeless stuff. G.H. Hardy would have disagreed, but that's another matter.

First, a special thanks to my advisor Rosario Gennaro. Thanks for many things, but especially for one. It is a hard search problem that of finding the intersection of what we can solve and what is interesting to us. Rosario constantly made an effort to give me the time,  the space and the travel stipends (an often overlooked complexity measure) to eventually encounter some progress on it. 

Thanks to the people that, directly or not, acted as mentors and teachers: Amotz Bar-Noy, Nelly Fazio, Dario Fiore, Leonid Gurvits, Tancr\`ede Lepoint, Jesper Buus Nielsen, William Skeith, Noson Yanofsky. Noson taught the computational complexity course in my first term as a Ph.D. student. I owe to this course a stretch in my interests of geological proportions. My friends and colleagues, Kelly Eckenrode and Alessio Sammartano, offered meta-technical support and precious conversations throughout my Ph.D.; thank you for being an inspiration.

The work in this thesis would not have been possible without the interaction with so many people. My thanks go: to Marios Georgiou, for stretching my mind at every opportunity and tolerating my research-related phone calls at uncomfortable times of the day;
to my colleagues at CCNY, Bertrand Ithurburn, Sima Jafarikhah, Nihal Vatandas, for the good chats and a great lab atmosphere; 
to my coauthors Steven Goldfeder and Luca Nizzardo, for their dedication and for making it fun. 

I am deeply indebted to my parents, Domenico and Rita, for their unconditional love and support. I want to thank Nigel Deans and Annika Karinen; that little passion for science I have, I owe it all to you. I am indebted to $\cal{L}$, for being an outstanding listener and for telling me about The Rest. 

Finally, thanks to Itai Feigenbaum and Mariana Raykova for being part of my committee, and to the indispensable support of the City College of New York through the Robert Kahn Fellowship and the NSF through grant 1545759.

\restoregeometry



%%%%%%%%%%%%%%%%%%%%%%%%%%%%%%%%%%%%%%%%%%%%%%%%%%%%%%%%%%%%%%%%%%%%%%
% -*-latex-*-
