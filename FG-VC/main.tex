%\documentclass[envcountsect]{llncs}
%%%%%%


\newcommand{\vect}[1]{\mathbf{#1}}
\newcommand{\mat}[1]{\MakeUppercase{\vect{#1}}}
\newcommand{\Group}{\ensuremath{\mathbb{G}}}
\newcommand{\powervect}[2][n]{\vect{#2}^{#1}}
\newcommand{\commitment}{\ensuremath{\mathbf{com}}}

\newcommand{\Ints}{\ensuremath{\mathbb{Z}_p}}
\newcommand{\IntsStar}{\ensuremath{\mathbb{Z}^\star_p}}
\newcommand{\dotprod}[2]{\langle#1,#2\rangle}
\newcommand{\matprod}[2]{#1\cdot #2}
\newcommand{\IsEqual}{\overset{?}{=}}
\newcommand{\gen}[1]{#1}
\newcommand{\com}[1]{\MakeUppercase{#1}}
\newcommand{\Left}[1][n^\prime]{[:#1]}
\newcommand{\Right}[1][n^\prime]{[#1:]}
\newcommand{\Relation}[3]{\{(#1;#2): #3\}}



%%%%

\newcommand{\headrow}[1]{\multicolumn{1}{c}{\adjustbox{angle=45,lap=\width-0.5em}{#1}}}

% = = = Table bullets: \full and \prt (full and part)

\newcommand{\full}{$\bullet$}
\newcommand{\prt}{$\circ$}

\newcommand{\fulltext}{\CIRCLE}
\newcommand{\prttext}{\LEFTcircle}

\newcommand{\fulltable}{$\tiny{\CIRCLE}$}
\newcommand{\prttable}{$\tiny{\LEFTcircle}$}

\newcommand{\matteo}[1]{{\color{red}{#1 -- MC}}}


\newcommand{\rosario}[1]{{\color{green}{#1 -- RG}}}



%
%\pagenumbering{arabic}
%\pagestyle{plain}

%\newcommand{\hi}[1]{\textcolor{red}{#1}}

%\newcommand{\qed}{\hfill \rule{7pt}{7pt}}

%\renewenvironment{proof}{\noindent{\bf Proof}\hspace*{1em}}{\qed\bigskip}


%\newcommand{\ceil}[1]{\left\lceil #1 \right\rceil}
% \begin{comment}

% \newcommand{\bitset}{\{0,1\}}

% \newcommand{\ignore}[1]{}
% \def\eqdef{:=}
% \def\eps{\epsilon}
% \def\GG{\mathbb{G}}
% \def\pk{{\sf pk}}
% \def\F{{\cal F}}
% \def\P{{\cal P}}
% \def\I{{\cal I}}
% \def\A{{\cal A}}
% \def\D{{\cal D}}
% \def\hD{\hat{{\cal D}}}
% \def\sk{{\sf sk}}
% \def\gen{\mathsf{Gen}}
% \def\cfgen{\mathsf{CFGen}}
% \def\sig{\mathsf{Sig}}
% \def\ver{\mathsf{Ver}}
% \def\negl{{\sf negl}}
% \def\gets{\leftarrow}
% %\newcommand{\vect}[1]{\left\langle{#1}\right\rangle}
% \newcommand{\jac}[2]{\left( \frac{#1}{#2} \right)}

% \end{comment}
\newtheorem{assump}{Assumption}
%\newtheorem{theorem}{Theorem}
%\newtheorem{lemma}{Lemma}
%\newtheorem{corollary}{Corollary}
%\newcommand\textline[5][t]{%
%  \par\smallskip\noindent\parbox[#1]{.333\textwidth}{\raggedright\text{}#2}%
%  \parbox[#1]{.333\textwidth}{\centering#4}%
%  \parbox[#1]{.333\textwidth}{\raggedleft\texttt{#5}}\par\smallskip%
%}

\newcommand*{\trightarrow}[1]{\xrightarrow{\mathmakebox[5cm]{#1}}}
\newcommand*{\tleftarrow}[1]{\xleftarrow{\mathmakebox[5cm]{#1}}}






%\maketitle
\begin{abstract}
  This paper initiates a study of {\em Fine Grained Secure Computation}: i.e. 
  the construction of {\em secure computation primitives} against ``moderately complex" adversaries. We present definitions and constructions for compact Fully Homomorphic Encryption and Verifiable Computation secure against (\textit{non-uniform}) $\mathsf{NC}^1$ adversaries under the assumption $\fgAssump$. We also present two application scenarios for our model: \textit{(i)} hardware chips that prove their own correctness, and \textit{(ii)} protocols against rational adversaries potentially relevant to the {\em Verifier's Dilemma} in smart-contracts transactions such as Ethereum. 
\end{abstract}



\section{Preliminaries}
\label{sec:prelim}

%This section presents all the necessary definitions. 

%\subsection{Adversaries and Circuit Families}

%\subsection{Security Definitions}

For a distribution $D$, we denote by $x \gets D$ the fact that $x$ is being sample according to $D$.
We remind the reader that an ensemble $\mathcal{X} = \ens{X}$ is a family of  probability distributions over a family of domains $\mathcal{D}=\ens{D}$. We say two ensembles $\mathcal{D} = \ens{D}$ and $\mathcal{D}' = \ens{D'}$ are statistically indistinguishable if $\frac{1}{2}\Sum_x |D(x)-D'(x)| < \negl(\lambda)$. 
Finally, we note that all arithmetic computations (such as sums, inner product, matrix products, etc.) in this work will be over $\GF(2)$ unless specified otherwise.

\begin{definition}[Function Family]
A {\em function family} is a family of (possibly randomized) functions $F = \funfam{f}$, where for each $\lambda$, $f_{\lambda}$ has domain $D^f_{\lambda}$ and co-domain $R^f_{\lambda}$. A {\em class} $\mathcal{C}$ 
is a collection of function families. 
\end{definition}
In most of our constructions $D^f_{\lambda}=\{0,1\}^{d_\lambda^f}$ and $R^f_{\lambda}=\{0,1\}^{r_\lambda^f}$ for sequences $\{d_\lambda^f\}_\lambda$, 
$\{d_\lambda^f\}_\lambda$. 

In the rest of the paper we will focus on the class of $\mathcal{C}=\mathsf{NC}^1$ of functions for which there is a polynomial $p(\cdot)$ and a constant $c$ such that for each $\lambda$, the function $f_\lambda$ can be computed by a Boolean (randomized) fan-in 2, circuit of size $p(\lambda)$ and depth $c \log(\lambda)$. In the formal statements of our results we will also use the following classes: $\AC^0$, the class of functions of polynomial size and constant depth with $\function{AND}, \function{OR}$ and $\function{NOT}$ gates with unbounded fan-in; $\ACzt$, the class of functions of polynomial size and constant depth with $\function{AND}, \function{OR}, \function{NOT}$ and $\function{PARITY}$ gates with unbounded fan-in. For a gate $g$ we denote by $\func{type}_C(g)$ the type of the gate $g$ in the circuit $C$ and by $\func{parents}_C(g)$ the list of gates of $C$ whose output is an input to $C$ (such list may potentially contain duplicates). 

Given a function $f$, we can think of its \textit{multiplicative depth} as the degree of the lowest-degree polynomial in $\GF(2)$ that evaluates to $f$. Similarly, we define the multiplicative depth of a circuit as follows:

\begin{definition}[Multiplicative Depth]
Let $C$ be a circuit, we define the multiplicative depth of $C$ as $\func{md}(g_{out})$ where $g_{out}$ is its output gate and the function $\func{md}$, from the set of gates to the set of natural numbers is recursively defined as follows:

\[
\func{md}(g) \eqdef
    \begin{cases*}
      1 & \mbox{if } $\func{type}_C(g) = \func{input}$ \\
      \max\{\func{md}(g') : g' \in \func{parents}_C(g)\} & \mbox{if } $\func{type}_C(g) = \func{XOR}$ \\
      \Sum_{g' \in \func{parents}_C(g)}\func{md}(g') & \mbox{if } $\func{type}_C(g) \in \{ \ANDgt, \ORgt \}$
    \end{cases*}
\]
\end{definition}

\medskip
\noindent
The following two circuit classes will appear in several of our results.

\begin{definition}[Circuits with Constant Multiplicative Depth]
\label{def:ACztcm}
We denote by $\ACztcm$ the class of circuits in $\ACzt$ with \textit{constant multiplicative depth}.
\end{definition}

\begin{definition}[Circuits with Quasi-Constant Multiplicative Depth]
\label{def:ACztq}
\label{def:quasi-constant}
For a circuit $C$ we denote by $S_{\omega(1)}(C)$ the set of $\ANDgt$ and $\ORgt$ gates in $C$ with non-constant fan-in. We say that $C$ has \textit{quasi-constant multiplicative depth} if $|S_{\omega(1)}(C)| = O(1)$.
We shall denote by $\ACztq$ the class of circuits in $\ACzt$ with quasi-constant multiplicative depth.
\end{definition}

\medskip
\noindent
{\sc Limited Adversaries.}
We define adversaries also as families of randomized algorithms $\{A_\lambda\}_\lambda$, 
one for each security parameter (note that this is a non-uniform notion of security). We denote
the class of adversaries we consider as $\mathcal{A}$, and in the rest of the
paper we will also restrict $\mathcal{A}$ to $\mathsf{NC}^1$.


\medskip
\noindent
{\sc Infinitely-Often Security.}
We now move to define security against all adversaries $\{A_\lambda\}_\lambda$ that belong to a class $\mathcal{A}$. 

%We note that the statement $\{A_\lambda\}_\lambda \notin \mathcal{C}$ implies 
%that for all $\{g_\lambda\}_\lambda \in \mathcal{C}$ there exists an infinite 
%number of values $\lambda$ such that $A_\lambda \neq g_\lambda$. Therefore we 
%will use this "infintely often" notion of security, which states that for all 
%adversaries outside of our permitted class our security property holds 
%infinitely often (i.e. for an infinite sequence of security parameters rather 
%than for every sufficiently large security parameter\footnote{
%As in \cite{dvv17} if we make the stronger assumption that $A_\lambda \neq 
%g_\lambda$ for all sufficiently large $\lambda$ then we would be able to 
%obtain the standard notion of security for all sufficiently large security 
%parameters.}
%). 

Our results achieve an "infinitely often" notion of security, which states that for all adversaries outside of our permitted class $\mathcal{A}$ our security property holds infinitely often (i.e. for an infinite sequence of security parameters rather than for every sufficiently large security parameter. We 
inherit this limitation from the techniques of \cite{fgcrypto}. 

\begin{definition}[Infinitely-Often Computational Indistinguishability]
Let $\mathcal{X} = \ens{X}$ Let $\mathcal{Y} = \ens{Y}$ be ensembles over the same domain family, $\mathcal{A}$ a class of adversaries, and $\Lambda$ an infinite subset of $\naturals$. 
We say that $\mathcal{X}$ and $\mathcal{Y}$ are infinitely often computational indistinguishable with respect to set $\Lambda$ and the class $\mathcal{A}$,
denoted by
$\mathcal{X} \sim_{\Lambda,\mathcal{A}} \mathcal{Y}$
if there exists a negligible function $\nu$ such that for any $\lambda \in \Lambda$ and for any adversary $A=\{A_\lambda\}_\lambda \in \mathcal{A}$
\[
| \Pr[A_{\lambda}(X_{\lambda}) = 1] - \Pr[A_{\lambda}(Y_{\lambda}) = 1]| < \nu(\lambda)
\]
\end{definition}
When $\mathcal{A} = \NC^1$ we will keep it implicit and use the notation $\mathcal{X} \sim_{\Lambda} \mathcal{Y}$ and say that $\mathcal{X} \text{ and } \mathcal{Y}$ are $\Lambda$-computationally indistinguishable.


% \begin{remark}{On proofs of security for infinitely many values of $\lambda$}
% As standard practice in cryptography we will use reduction-based proofs. These will have a slightly different flavor than usual as our security notion is for infinitely many values of the security parameter (rather than the standard asymptotic notion). However, this does not change the essence of the usual logic.
% Suppose you want to prove that $Y$ is (infinitely many times) secure based on the assumption that $X$ is secure. 
% To proceed by contradiction you will assume there exist an adversary $\adv_Y$ and $\lambda^*$ such that for all values of $\lambda > \lambda^*$ the advantage of $\adv_Y$ is 
% \end{remark}

In our proofs we will use the following facts on infinitely-often computationally indistinguishable ensembles. We skip their proof as, except for a few technicalities, it is analogous to the corresponding properties for standard computational indistinguishability\footnote{We refer the reader to \cite{goldreich2009foundations1}.}.

\begin{lemma}[Facts on $\Lambda$-Computational Indistinguishability]
\label{lemma:hyb-arg}
\label{lemma:facts-lind}
\begin{itemize}
\item \textbf{Transitivity:}
Let $m = \poly(\lambda)$ and $\mathcal{X}^{(j)}$ with $j \in \{0,\dots,m \} $ be ensembles.
If for all $j \in [m]$ $\mathcal{X}^{(j-1)} \lind \mathcal{X}^{(j)}$, then $\mathcal{X}^{(0)} \lind \mathcal{X}^{(m)}$.
\item \textbf{Weaker than statistical indistinguishability:} Let $\mathcal{X}, \mathcal{Y}$ be statistically indistinguishable ensembles. Then $\mathcal{X} \lind \mathcal{Y}$ for any infinite $\Lambda \subseteq \naturals$
\item \textbf{Closure under $\NC^1$:}  Let $\mathcal{X}, \mathcal{Y}$ be ensembles and $\funfam{f} \in \NC^1$. If $\mathcal{X} \lind \mathcal{Y}$ for some $\Lambda$ then 
$f_{\lambda}(\mathcal{X}) \lind f_{\lambda}(\mathcal{Y})$.
\end{itemize}
\end{lemma}


%\subsection{Fine-Grained Encryption}

\subsection{Public-Key Encryption}
A public-key encryption scheme \\
$\PKE = (\PKEKeygen, \PKEEnc, \PKEDec)$ is a triple of algorithms which operate as follow:
\begin{itemize}
\item \textbf{Key Generation.} The algorithm $(\pk,\sk)\leftarrow$ \PKEKeygen$(1^{\lambda})$ takes a unary representation of the security parameter and outputs a public key encryption key $\pk$ and a secret decryption key $\sk$.
\item \textbf{Encryption.} The algorithm $c\leftarrow$ \PKEEnc$_{\pk}(\mu)$ takes the public key $\pk$ and a single bit message $\mu\in\bit$ and outputs a ciphertext $c$. The notation \PKEEnc$_{\pk}(\mu;r)$ will be used to represent the encryption of a bit $\mu$ using randomness $r$. 
\item \textbf{Decryption.} The algorithm $\mu^*\leftarrow$ \PKEDec$_{\sk}(c)$ takes the secret key $\sk$ and a ciphertext $c$ and outputs a message $\mu^*\in\bit$. 
\end{itemize}
Obviously we require that $\mu=$\PKEDec$_{\sk}($\PKEEnc$_{\pk}(\mu))$

\begin{definition}[CPA Security for PKE]
\label{def:cpa-pke}
\label{def:cpa-he}
A scheme $\PKE$ is IND-CPA secure if for an infinite $\Lambda \subseteq \naturals$ we have
$$ (\pk,\PKEEnc_{\pk}(0)) \lind (\pk,\PKEEnc_{\pk}(1)) $$
where $(\pk,\sk)\leftarrow$ \PKEKeygen($1^{\lambda}$).
\end{definition}

\begin{remark}[Security for Multiple Messages]
\label{rem:cpa-multiple}
Notice that by a standard hybrid argument and Lemma \ref{lemma:hyb-arg} we can prove that any scheme secure according to Definition \ref{def:cpa-pke} is also secure for multiple messages (i.e. the two sequences of encryptions bit by bit of two bit strings are computationally indistinguishable).
We will use this fact in the proofs in Section $\ref{sec:vc}$, but we do not provide the formal definition for this type of security. We refer the reader to 5.4.2 in~\cite{goldreich2009foundations2}.
\end{remark}

%\subsection{Public-Key Encryption}
A public-key encryption scheme \\
$\PKE = (\PKEKeygen, \PKEEnc, \PKEDec)$ is a triple of algorithms which operate as follow:
\begin{itemize}
\item \textbf{Key Generation.} The algorithm $(\pk,\sk)\leftarrow$ \PKEKeygen$(1^{\lambda})$ takes a unary representation of the security parameter and outputs a public key encryption key $\pk$ and a secret decryption key $\sk$.
\item \textbf{Encryption.} The algorithm $c\leftarrow$ \PKEEnc$_{\pk}(\mu)$ takes the public key $\pk$ and a single bit message $\mu\in\bit$ and outputs a ciphertext $c$. The notation \PKEEnc$_{\pk}(\mu;r)$ will be used to represent the encryption of a bit $\mu$ using randomness $r$. 
\item \textbf{Decryption.} The algorithm $\mu^*\leftarrow$ \PKEDec$_{\sk}(c)$ takes the secret key $\sk$ and a ciphertext $c$ and outputs a message $\mu^*\in\bit$. 
\end{itemize}
Obviously we require that $\mu=$\PKEDec$_{\sk}($\PKEEnc$_{\pk}(\mu))$

\begin{definition}[CPA Security for PKE]
\label{def:cpa-pke}
\label{def:cpa-he}
A scheme $\PKE$ is IND-CPA secure if for an infinite $\Lambda \subseteq \naturals$ we have
$$ (\pk,\PKEEnc_{\pk}(0)) \lind (\pk,\PKEEnc_{\pk}(1)) $$
where $(\pk,\sk)\leftarrow$ \PKEKeygen($1^{\lambda}$).
\end{definition}

\begin{remark}[Security for Multiple Messages]
\label{rem:cpa-multiple}
Notice that by a standard hybrid argument and Lemma \ref{lemma:hyb-arg} we can prove that any scheme secure according to Definition \ref{def:cpa-pke} is also secure for multiple messages (i.e. the two sequences of encryptions bit by bit of two bit strings are computationally indistinguishable).
We will use this fact in the proofs in Section $\ref{sec:vc}$, but we do not provide the formal definition for this type of security. We refer the reader to 5.4.2 in~\cite{goldreich2009foundations2}.
\end{remark}

\subsubsection{Somewhat Homomorphic Encryption}
A public-key encryption scheme is said to be homomorphic if there is an 
additional algorithm $\mathsf{Eval}$ which takes a input the public key $\pk$, 
the representation of a function $f:\bit^l\rightarrow\bit$ and a set of $l$ ciphertexts $c_1,\ldots,c_l$, and outputs a ciphertext $c_f$\footnote{Notice that the syntax of $\mathsf{Eval}$ can also be extended to return a sequence of encryptions for the case of multi-output functions. We will use this fact in Section \ref{sec:beyond-cm}. See also Remark \ref{rem:cpa-multiple}.}. 
%$\HE = (\HEKeygen, \HEEnc, \HEDec, \HEEval)$ is a public-key encryption scheme 
%with a few additions. It is a quadruple of algorithms which operate as follow:
%\begin{itemize}
%\item \textbf{Key Generation.} The algorithm $(\pk,\evk,\sk)\leftarrow$ %\HEKeygen$(1^{\lambda})$ takes a unary representation of the security parameter 
%and outputs a public key encryption key $\pk$, a public evaluation key $\evk$ and 
%a secret decryption key $\sk$.
%\item \textbf{Encryption.} The algorithm $c\leftarrow$ \HEEnc$_{\pk}(\mu)$ 
%operates exactly as $\PKEEnc$.
%\item \textbf{Decryption.} The algorithm $\mu^*\leftarrow$ \HEDec$_{\sk}(c)$ 
%operates exactly as $\PKEDec$.
%\item \textbf{Homomorphic Evaluation} The algorithm $c_{f}\leftarrow$ 
%$\HEEval}_{\pk}(f,c_1,\ldots,c_l)$ takes the evaluation key $\evk$, a function 
%$f:\bit^l\rightarrow\bit$ and a set of $l$ ciphertexts $c_1,\ldots,c_l$, and 
%outputs a ciphertext $c_f$. 

% It must be the case that:
% \begin{equation}
% \mathrm{\HEDec}_{\sk}(c_f) = f(\mathrm{\HEDec}_{\sk}(c_1),\ldots,\mathrm{\HEDec}_{\sk}(c_l))
% \end{equation}
% with probability $\geq 1- \gamma(\lambda)$. 
%\end{itemize}


%A homomorphic encryption scheme is said to be secure if it meets the following 
%notion of semantic security:

%\begin{definition}[CPA Security for HE]\label{def:cpa-he}
%A scheme $\HE$ is IND-CPA secure if for an infinite $\Lambda \subseteq \naturals$ 
%we have
%$$ (\pk,\evk,\mathrm{\HEEnc}_{\pk}(0)) \lind (\pk,\evk,\mathrm{\HEEnc}_{\pk}(1)) $$
%where $(\pk,\evk,\sk)\leftarrow$ \HEKeygen($1^{\lambda}$).
%\end{definition}

\newcommand{\cclass}{\mathcal{C}}
We proceed to define the homomorphism property. The next notion of $\cclass$-homomorphism is sometimes also referred to as ``somewhat homomorphism''.

\begin{definition}[$\cclass$-homomorphism]
Let $\cclass$ be a class of functions (together with their respective representations).
An encryption scheme $\PKE$ is $\cclass$-homomorphic (or, homomorphic for the class $\cclass$) if for every function $f_\lambda$ where  $f_\lambda \in \mathcal{F} \funfam{f} \in \cclass$ and respective inputs $\mu_1,\dots,\mu_l \in \bit$ (where $l = l(\lambda)$), it holds that if 
$(\pk, \sk) \gets \PKEKeygen(1^{\lambda})$ and $c_i \gets \PKEEnc_{\pk}(\mu_i)$
then 
\[
    \Pr[\PKEDec_{\sk}(\Eval_{\pk}(F, c_1,\dots, c_l)) \not = F(\mu_1,\dots,\mu_l)] = \negl(\lambda),
    % Note: this definition is not infinitely-often, but for large enough values. It's correct this way.
\]
\end{definition}


%As pointed out in~\cite{fhe-lwe}, there are two important properties that the 
%above definition does not require. First, it does not require that the ciphertexts
%$c_i$ are decryptable themselves, only that they become
%decryptable after homomorphic evaluation. Finally, it does not require that the 
%output of $\HEEval$
%can undergo additional homomorphic evaluation.


% \begin{definition}[CPA Security for HE]\label{def:cpa-he}
% A scheme \textnormal{\HE} is IND-CPA secure if, for any polynomial time adversary $\mathcal{A}$, there exists a negligible function $\mu(\cdot)$ such that 
% \begin{equation}
% \mathrm{Adv}_{\mathrm{CPA}}[\mathcal{A}] \EqDef  
% |\Pr[\mathcal{A}(\pk,\evk,\mathrm{\HEEnc}_{\pk}(0)) = 1] - 
% \Pr[\mathcal{A}(\pk,\evk,\mathrm{\HEEnc}_{\pk}(1)) = 1]|=\mu(\lambda)
% \end{equation}
% where $(\pk,\evk,\sk)\leftarrow$ \textnormal{\HEKeygen}($1^{\lambda}$). We further say that a scheme \textnormal{\HE} has CPA security $\delta$ for some negligible function $\delta(\cdot)$ if the above indistinguishability gap $\mu(\lambda)$ is smaller than $\delta(\lambda)^{\Omega(1)}$. 
% \end{definition}

As usual we require the scheme to be non-trivial by requiring that the output 
of $\mathsf{Eval}$ is compact:
\begin{definition}[\bfseries Compactness]\label{def:compact}
A homomorphic encryption scheme $\PKE$ is compact if there exists a polynomial $s$ in $\lambda$ such that the output length of $\mathsf{Eval}$ is at most $s(\lambda)$ bits long (regardless of the function $f$ being computed or the number of inputs). 
\end{definition}


\begin{definition}
\label{def:leveled}
Let $\cclass = \funfam{\cclass}$ of arithmetic circuits in $\GF(2)$. A scheme 
$\PKE$ is leveled  $\cclass$-homomorphic if it takes $1^L$ as additional input in 
key generation, and can only evaluate depth-$L$ arithmetic circuits from 
$\cclass$. The bound $s(\lambda)$ on the ciphertext must remain independent of 
$L$.
\end{definition}


\subsection{Verifiable Computation}
% We now describe our model for delegation of computation against bounded adversaries. We follow the description in \cite{ckv10} of delegation schemes against polynomial time adversaries. % XXX: check sentence
In a {\em Verifiable Computation} scheme a Client uses an untrusted server to compute a function $f$ over an input $x$. The goal is to prevent the Client from accepting an incorrect value $y'\neq f(x)$. We require that the Client's cost of running this protocol be smaller than the cost of computing the function on his own. The following definition is from \cite{ggp10} which allows the client to run a possibly expensive pre-processing step. 

\begin{definition}[Verifiable Computation Scheme]

A \emph{verifiable computation scheme} $\VC =
(\VCKG,\VCPG,\VCCompute,\VCVerif)$ consists of the four
algorithms defined below.  

\begin{enumerate}
\item $\VCKG(f, 1^\lambda) \to (\pkW, \skD)$: 
      Based on the security parameter $\lambda$, the randomized \emph{key generation} 
      algorithm generates a public key that encodes the target function $f$, 
      which is used by the Server to compute $f$. It also 
      computes a matching secret key, which is kept private by the Client. 

\item $\VCPG_{\skD}(x) \to (\qx,\sx)$:
      The \emph{problem generation} algorithm uses the secret key $\skD$ to encode the 
      function input $x$ as a public query $\qx$ which is given to the Server
      to compute with, and a secret value $\sx$ 
      which is kept private by the Client. 

\item $\VCCompute_{\pkW}(\qx) \to \ax$: 
      Using the Client's public key and the encoded input, the Server  \emph{computes} 
      an encoded version of the function's output $y = F(x)$. 

\item $\VCVerif_{\skD}(\sx,\ax) \to y\mbox{ }\cup \{ \fail \}$:
      Using the secret key $\skD$ and the secret ``decoding'' $\sx$, the 
      \emph{verification} algorithm converts the worker's encoded output into 
      the output of the function, e.g., $y = f(x)$ or outputs $\fail$ indicating
      that $\ax$ does not represent the valid output of $f$ on $x$.
\end{enumerate}      
% A delegation scheme is an interactive protocol $\Del =  \DW$ between a delegator $\D$ and a worker $\W$ with the following structure:

%  \begin{enumerate}
%      \item The scheme \Del~consists of two stages: an offline/preprocessing stage and an online stage. The offline stage is executed once before the online stage, whereas the online stage can be executed many times. 
%      \item In the offline stage, both the delegator $\D$ and the worker $\W$ receive a security parameter $\lambda$ and a function $F : \bit^n \to \bit^m $, represented by a boolean circuit $C$. At the end of the interaction, the delegator $\D$ decides whether to accept or reject. If $\D$ accepts, then $\D$ outputs a secret key $\skD$ and a public key $\pkW$. We will denote this by $(\skD, \pkW) = \DW(F, 1^{\lambda})$. We will use the notation $C, n \text{ and } m$ as the circuit and parameters associated with $F$ throughout the paper, and we will often omit the security parameter from the notation.
%      \item In the online stage, both parties receive $F, 1^{\lambda}$, and an input $x \in \bit^n$, and execute a one round communication protocol. Namely, $\D$ sends $q = \D(F, x, \skD)$ to $\W$, and then $\W$ sends 
%      $a = \W(F, x, \pkW, q)$ to $\D$. Then the delegator $\D$ either accepts or rejects. If $\D$ accepts, then $\D$ also generates a private output
%      $y = \D(F, x,\skD, q, a) \in \bit^m$, which is supposed to be $F(x)$.
%      For simplicity, we will omit the function $F$ and the security parameter from the input of the online stage.
%       \end{enumerate}
\end{definition}     
   
     

\noindent
The scheme should be complete, i.e. an honest Server should (almost) always return the correct value. 
\begin{definition}[Completeness]
\label{def:vc-completeness}
A delegation scheme $\VC = (\VCKG,\VCPG,\VCCompute,\VCVerif)$ has \textit{overwhelming completeness} for a class of functions $\mathcal{C}$ if  there is a function $\nu(n) = \negl(\lambda)$ such that for infinitely many values of $\lambda$, if $f_\lambda \in \mathcal{F} \in \mathcal{C}$, then for all 
inputs $x$ 
%for every function F and every $x \in \bit^n$, 
the following holds with probability at least $1-\nu(n)$:  
  $(\pkW, \skD) \gets \VCKG(f_\lambda, \lambda)$
$(\qx,\sx) \gets \VCPG_{\skD}(x)$ and 
  $\ax \gets \VCCompute_{\pkW}(\qx)$ then $y=f_{\lambda}(x) \gets \VCVerif_{\skD}(\sx,\ax)$.
\end{definition}




% \begin{center}
% \fbox{\pseudocode[syntaxhighlight=auto]%[syntaxhighlight=auto,head=$\function{Exp}^{\text{Verif}}_{\adv}(F, \lambda)$]
% {%
%       (\pkW, \skD) \sample \Doffline (F, \lambda) \\
%      (\hatx, \hata) \sample \adv^{\Donline(\skD, \cdot, \cdot)}(\pkW) \\
%      return \Dverif(\skD, \hatq, \hata) }}
% \end{center}



% \pseudocode[syntaxhighlight=auto]{$Exp^{\text{Verif}}_{\adv}$}{
%     $(\pkW, \skD)$ \sample \kgen $(\lambda)$ \\
%     $(\hatq, \hata)$ \sample $\adv^{\Donline(\skD, \cdot, \cdot)}(\pkW)$ \\
%     \pcreturn $\DVerif(\skD, \hatq, \hata)$ }

% \begin{comment}
% \begin{definition}{Security Game for Delegation Schemes}
% Let $\Del = \DW$ be a delegation scheme and $\lambda \in \mathbb{N}$ be the security parameter. The security game $\Gvc(\lambda)$ for $\Del$ is the following game played by a worker strategy $\Wstar$.
% \begin{itemize}
% \item The game starts with the offline stage of $\Del$, and is followed by many rounds of the online stage.
% % NB: it is the worker choosing the function F.
% \item $\Wstar(1^{\lambda})$ first chooses the delegation function $F$ and then $\D$ and $\Wstar$ interact in the offline stage of $\Del$ with input $F$.
% \item At the beginning of each round of the online stage (indexed by $l$), $\Wstar$ can either terminate the game or choose an input $x_l \in \bit^n$. If the game is not terminated, $\D$ and $\Wstar$ interact in the online stage of $\Del$ on input $x_l$.
% \item Whenever the delegator $\D$ rejects, the game terminates.
% \end{itemize}
% $\Wstar$ succeeds in the game $\Gvc(\lambda)$ if there exists a round $l$ of the online stage such that $\D$ accepts and outputs a wrong value $y_l \not = F(x_l)$, where $x_l$ is the delegated input chosen by $W^*$.
% \end{definition}
% \end{comment}

\def\fnFam{{\cal F}}

% Experiment
\def\accSet{{\cal I}}

\newcommand{\BatchVCPG}{\function{BatchProbGen}}

\renewcommand{\vec}[1]{\mathbf{#1}}

To define soundness we consider an adversary who plays the role of a malicious Server who tries to convince the Client of an incorrect output $y \neq f(x)$. The adversary is allowed to run the protocol on inputs of her choice, i.e. see the {\em queries} $q_{x_i}$ for adversarially chosen $x_{i}$'s before picking an 
input $x$ and attempt to cheat on that input. 
Because we are interested in the parallel complexity of the adversary we distinguish between two parameters $l$ and $m$. The adversary is allowed to do $l$ rounds of adaptive queries, and in each round she queries $m$ inputs. Jumping ahead, because our adversaries are restricted to $\mathsf{NC}^1$ circuits, we will have to bound $l$ with a constant, but we will be able to keep $m$ polynomially large. 

%Experiment for soundness below. $\BatchVCPG$ is a batch version of $\VCPG$ that 
%runs in parallel and returns a possibly polynomial number of outputs of $\VCPG$. %We model an adversary \adv as a tuple of circuits %(\adv^{(1)}_{q},\dots,\adv^{(L)}_{q}, \adv_{resp})$, where the $\adv^{(i)}_{q}$-s %generates the sets of inputs on which to  query the problem generator and the %$\adv_{resp}$ generates a response based on those queries.

% \vspace{2mm}
% \begin{tabular}{l}
% \hspace{1mm} Experiment ${\bf Exp}^{\function{Verif}}_{\adv}[\mathcal{VC}, F, \lambda, L]$\\
% \hspace{7mm} $(\pkW, \skD) \gets \VCKG(F, \lambda)$;\\
% \hspace{7mm} $\accSet \gets \emptyset$;\\

% \hspace{7mm} For $i=1,\ldots,i=L$;\\
% \hspace{13mm} $\vec{x}^{(i)} \gets \adv^{(i)}_{q}(\pkW,\accSet)$;\\
% \hspace{13mm} $(\vec{q}^{(i)},\vec{s}^{(i)}) \gets \BatchVCPG_{\skD}(\vec{x}^{(i)})$;\\
% \hspace{13mm} $\accSet \gets \accSet \cup \{ \vec{x}^{(i)}, \vec{q}^{(i)} \}$;\\
% \hspace{7mm} $(i,j,\hat{a}) \gets \adv_{resp}(\pkW, \accSet)$;\\
% \hspace{7mm} $\hat{y} \gets \VCVerif_{\skD}(s^{(i)}_j,\hat{a})$\\
% \hspace{7mm} If $\hat{y} \neq \fail$ and $\hat{y} \neq F(x^{i}_j)$, output 1, else 0;\\
% \end{tabular}
% \vspace{2mm}


\vspace{2mm}
\begin{tabular}{l}
\hspace{1mm} Experiment $\expVC{\VC}$\\
\hspace{7mm} $(\pkW, \skD) \gets \VCKG(f, \lambda)$;\\
\hspace{7mm} $\accSet \gets \emptyset$;\\

\hspace{7mm} For $i=1,\ldots,i=l$;\\
\hspace{13mm} $\{ x_{(i-1)m},\dots x_{im-1} \} \gets A_{\lambda}(\pkW,\accSet)$;\\
\hspace{13mm} $\{(q_{j},s_{j}) : (q_{j},s_{j}) \gets \VCPG_{\skD}(x_j), j \in \{(i-1)m,\dots,im\} \}$\\
\hspace{13mm} $\accSet \gets \accSet \cup \{ x_{(i-1)m},\dots x_{im-1} \} \cup \{ q_{(i-1)m},\dots q_{im-1} \}$;\\
\hspace{7mm} $\hat{a} \gets A_{\lambda}(\pkW, \accSet)$;\\
\hspace{7mm} $\hat{y} \gets \VCVerif_{\skD}(s_{lm},\hat{a})$\\
\hspace{7mm} If $\hat{y} \neq \fail$ and $\hat{y} \neq f(x_{lm})$, output 1, else 0.\\
\end{tabular}
\vspace{2mm}

\begin{remark}
In the experiment above the adversary "tries to cheat'' on the last input presented in the last round of queries (i.e. $x_lm$). This is without loss of generality. In fact, assume the adversary aimed at cheating on an input
presented before round $l$, then with one additional round it could present that same input once more as the last of the batch in that round. 
\end{remark}
% TODOs here
%\begin{definition}{One-Time Soundness}
%\label{def:one-time-sec}
%\matteo{TODO: soundness one-time defined with m=l=1}\\

%\end{definition}


%\begin{definition}{Many-Time Soundness}
%\label{def:many-times-sec}
%\matteo{TODO: soundness many-times, defined with $m = \poly(n), l=O(1)$}\\
%\end{definition}
%\matteo{TODO: Both definitions of soundness should have the constraint $\sum_i %DEPTH(\adv^{(i)}_{q}) + DEPTH(\adv_{resp}) = O(\log (n))$}

%\begin{remark}
%\matteo{To be fixed}
%Our definitions are equivalent to those of \cite{ckv10} if we restrict ourselves 
%to stateless algorithms (notice that the experiment has state though) and bounded
%depth. For this reason we impose a constraint on the number of rounds of queries 
%to $\BatchVCPG$
%\end{remark}

\begin{definition}[Soundness]
\label{def:vc-soundness}
We say that a verifiable computation scheme is $(l,m)$-sound against a class 
$\mathcal{A}$ of adversaries if there exists a negligible function $\negl(\lambda)$, such that for all $A = \{A_\lambda\}_\lambda \in \mathcal{A}$, and for infinitely many $\lambda$ we have that 
\[
\Pr[\expVC{\VC}=1] \leq \negl(\lambda)
\]
\end{definition}

Assume the function $f$ 
we are trying to compute belongs to a class $\mathcal{C}$ which is smaller than 
$\mathcal{A}$. Then our definition guarantees that the "cost" of cheating is higher than the cost of honestly computing $f$ and engaging in the Verifiable Computation protocol $\mathcal{VC}$. Jumping ahead, our scheme will allow us to compute the class $\mathcal{C}=\mathsf{AC}^0[2]$ against the class of 
adversaries $\mathcal{A}=\mathsf{NC}^1$.

\medskip
\noindent
{\sc Efficiency}
The last thing to consider is the efficiency of a VC protocol. Here we focus on the time complexity of computing the function $f$. Let $n$ be the number of input bits, and $m$ be the number of output bits, and $S$ be the size of the circuit computing $f$. 
     \begin{itemize}
        \item A verifiable computation scheme $\VC$ is \textbf{client-efficient} if circuit sizes of $\VCPG$ and $\VCVerif$ are $o(S)$. We say that it is \textbf{linear-client} if those sizes are $O(\poly(\lambda)(n + m))$. 
        
        \item A verifiable computation scheme $\VC$ is \textbf{server-efficient} if the circuit size of $\VCCompute$ is $O(\poly(\lambda)S)$.
        % \item A verifiable computation scheme $\VC$ has a \textbf{non-interactive} offline stage if $\D$ and $\W$
        % do not interact at all during the offline stage, and only $\D$ does some computation. Note that if $\Del$ has a non-interactive offline stage, then we can assume w.l.o.g. that $\D$ always accepts in the offline stage.
     \end{itemize}
We note that the key generation protocol $\VCKG$ can be expensive, and indeed in our protocol (as in \cite{ggp10,ckv10,aik10}) its cost is the same as computing $f$ -- this is OK as $\VCKG$ is only invoked once per function, and the cost can be amortized over several computations of $f$. 

%\subsection{Fine-Grained Two-Party Computation}

\section{Fine-Grained SHE}
\label{sec:he}
\label{sec:HE}

%\subsection{A Public-Key Encryption Scheme Secure Against $\NC^1$ adversaries %\cite{fgcrypto}}
We start by recalling the public key encryption from \cite{fgcrypto} which is 
secure against adversaries in $\mathsf{NC}^1$. 

The scheme is described in Figure \ref{fig:pke-dvv16}. Its security relies on the
 following result, implicit in \cite{ik00}\footnote{Stated as Lemma 4.3 in \cite{fgcrypto}.}. We will also use this lemma when proving the security of our construction in Section \ref{sec:he}.
 

\begin{lemma}[\cite{ik00}]
\label{lemma:ik00}
If $\fgAssump$ then there exist distribution $\mathcal{D}^{\text{kg}}_{\lambda}$ over  $\bit^{\lambda \times \lambda}$, distribution $\mathcal{D}^{f}_{\lambda}$ over matrices in $\bit^{\lambda \times \lambda}$ \textit{of full rank}, and infinite set $\Lambda \subseteq \naturals$ such that
\[
\Mkg \lind \MD
\]
where $\MD \gets \mathcal{D}^{f}_{\lambda}$ and $\Mkg \gets  \mathcal{D}^{\text{kg}}_{\lambda}$.
\end{lemma}

The following result is central to the correctness of the scheme $\PKE$ 
in Figure \ref{fig:pke-dvv16} and is implicit in \cite{fgcrypto}.

\begin{lemma}[\cite{fgcrypto}]
\label{lemma:dvv16-sampling}
There exists sampling algorithm $\KSample$ such that $(\M, \k) \gets \KSample(\unlambda)$, $\M$ is a matrix distributed according to $\mathcal{D}^{\text{kg}}_{\lambda}$ (as in Lemma \ref{lemma:ik00}), $\k$ is a vector in the kernel of $\M$ and has the form \\$\k = (r_1, \, r_2, \, \dots, \, r_{\lambda-1}, \, 1) \in \bit^{\lambda}$ where $r_i$-s are uniformly distributed bits.
\end{lemma}
%  % Matrices for sampling distributions
%  \def\visMz{
%  \begin{pmatrix}
%   0 & \   & \cdots & 0 & 0 \\
%   1 & 0 & \cdots & \   & 0 \\
%   0 & 1 & \ddots & \   & \vdots  \\
%   \vdots  & \ddots  & \ddots & \  & \\
%   0 & \cdots & 0 & 1 & 0 
%  \end{pmatrix}}


% % Sampling distributions
% \begin{figure}
% \label{fig:sampling-dists}
% \begin{framed}
% \begin{itemize}

% \item  \matteo{TODO: Find ways to set up matrices correctly}

% \end{itemize}
% \end{framed}
% \caption{Sampling Distributions from \cite{ik00}}
% \end{figure}
 
 
 % Definitions 
% \def\Rone{\mathbf{R}_{1}}
% \def\Rtwo{\mathbf{R}_{2}}


%  \newcommand{\LSamp}{\function{LSamp}}
%  \newcommand{\RSamp}{\function{RSamp}}
 
 
 
 % Actual scheme
\begin{figure}
\begin{framed}
% Let $\Mz$ be the $\lambda \times \lambda$ matrix  described in Figure \ref{fig:sampling-dists}.
\begin{itemize}
\item $\PKEKeygen_{\sk}(\unlambda):$
\begin{enumerate}
% \item Sample $\Rone \gets \LSamp (\unlambda)$ and 
% $\Rtwo \gets \RSamp(\unlambda)$;
% \item Let $\k = \transp{(\r \  1)}$ be the last column of $\Rtwo$;
\item Sample $(\M, \k) \gets \KSample(\unlambda)$;
\item Output $(\pk = \M, \sk = \k)$.
\end{enumerate}
\item $\PKEEnc_{\pk = \M}(\mu)):$
\begin{enumerate}
\item Sample $\r \sample \bit^{\lambda}$;
\item Let $\transp{t} = (0 \ \dots 0 \ 1) \in \bit^{\lambda}$;
\item Output $\transp{\c} = \transp{\r}  \M + \mu\transp{\t}$.
\end{enumerate}
\item $\PKEDec_{\sk = \k }(\c):$
\begin{enumerate}
\item Output $\inprod{\k}{\c}$
\end{enumerate}

\end{itemize}
\end{framed}
\caption{PKE construction \cite{fgcrypto}}
\label{fig:pke-dvv16}
\end{figure}


% % Sampling algorithms
% \begin{figure}
% \label{fig:sampling-algs}
% \begin{framed}
% \begin{itemize}

% \item  \matteo{TODO}

% \end{itemize}
% \end{framed}
% \caption{Sampling Algorithms (\cite{fgcrypto})}
% \end{figure}

\begin{theorem}[\cite{fgcrypto}]
Assume $\fgAssump$. Then, the scheme $\PKE = (\PKEKeygen, \PKEEnc, \PKEDec)$ defined in Figure \ref{fig:pke-dvv16} is a Public Key Encryption scheme secure against $\NC^1$ adversaries. All algorithms in the scheme are computable in $\ACzt$.
\end{theorem}

% To guarantee figures are flushed before the next subsection
%\clearpage
 
\subsection{Leveled Homomorphic Encryption for $\ACztcm$ Functions Secure against $\NC^1$}
\label{sec:leveled-he-simple}
%XXX: Can I  use such title even if it's only a subset of $\AC^0[2]$
 % We show how to adapt the relinearization technique from BV13
 
 % -- description of the scheme --

%To highlight what elements are elements in $\GF(2)$, in this section we will 
We denote by $\vec{x}[i]$ the $i$-th bit of a vector of bits $\vec{x}$ . Below, the scheme $\PKE = (\PKEKeygen,\PKEEnc, \PKEDec)$ is the one defined in Figure \ref{fig:pke-dvv16}.
% Here is the way we redefine keygen and how we define eval.
%\begin{figure}
%\label{fig:he-scheme}
%\begin{framed}

Our SHE scheme is defined by the following four algorithms: 
\begin{itemize}
\item $\HEKeygen_{\sk}(\unlambda, L):$ For key generation, sample $L+1$ key pairs
 $(\M_0, \k_0),\dots,\M_0, \k_L) \gets \PKEKeygen(\unlambda)$, and compute, for all $\ell \in \{0, \dots, L-1\}$, $i,j \in [\lambda]$, the value
\[ % XXX: See if something needs to become a vector
    \vec{a}_{\ell,i,j} \gets \PKEEnc_{\M_{\ell+1}}(\k_{\ell}[i]\cdot \k_{\ell}[j]) \in \lambdabits
\]
We define $\A \eqdef \{ a_{\ell,i,j} \}_{\ell,i,j}$ to be the set of all these values.
t then outputs the secret key $\sk = \k_L$, and the public key 
$\pk= (\M_0, \A)$. In the following we call $\evk = \A$ the evaluation key.

We point out a property that will be useful later: by the definition above, for all $\ell \in \{0, \dots, L-1 \}$ we have
\begin{align}
\label{he-keys-invariant}
\inprod{\klp}{\a_{\ell+1,i,j}} = \kl[i] \cdot \kl[j]\,.
\end{align}

\item $\HEEnc_{\pk}(\mu)):$
Recall that $\pk = \M_0$. To encrypt a message $\mu$ we compute 
$\v \gets \PKEEnc_{\M_0}(\mu)$. The output ciphertext contains $\v$ in addition to 
a ``level tag'', an index in $\{0, \dots, L \}$ denoting the ``multiplicative depth'' of the generated ciphertext. The encryption algorithm outputs $c \eqdef (\v, 0)$.

\item $\HEDec_{\k_L}(c):$ To decrypt a ciphertext\footnote{We are only requiring to decrypt ciphertexts that are output by $\HEEval(\cdots)$} $c = (\v,L)$ compute $\PKEDec_{\k_L}(\v)$, i.e.
\[
\inprod{\k_L}{\v}
\]

\item $\HEEval_{\evk}(f, c_1,\dots,c_t):$ where $F : \bit^t \to \bit$: We require that $f$ is represented as an arithmetic circuit in $\GF(2)$ with addition gates  of unbounded fan-in and multiplication gates of fan-in 2. We also require the circuit to be \textit{layered}, i.e. the set of gates can be partitioned in subsets (layers) such that wires are always between adjacent layers. Each layer should be composed homogeneously either of addition or multiplication gates. Finally, we require that the number of multiplications layers (i.e. the multiplicative depth) of $f$ is $L$.

We homomorphically evaluate $f$ gate by gate. We will show how to perform multiplication (resp. addition) of two (resp. many) ciphertexts. Carrying out this procedure recursively, we can homomorphically compute any circuit $f$ of multiplicative depth $L$.
\subsubsection{Ciphertext structure during evaluation.} During the homomorphic evaluation a ciphertext will be of the form $c = (\v, \ell)$ where $\ell$ is the ``level tag'' mentioned above. At any point of the evaluation we will have that $\ell$ is between $0$ (for fresh ciphertexts at the input layer) and $L$ (at the output layer). We define homomorphic evaluation only among ciphertexts at the same level. Since our circuit is layered we will not have to worry about homomorphic evaluation occurring among ciphertexts at different levels. 
Consistently with the fact a level tag represents the multiplicative depth of a ciphertext, addition gates will keep the level of ciphertexts unchanged, whereas multiplication gates will increase it by one.
% say about the invariant
Finally, we will keep the invariant that the output of each gate evaluation $c = (\v, \ell)$ is such that
\begin{align}
\label{he-invariant}
\inprod{\k_{\ell}}{\v} = \mu
\end{align}
where $\mu$ is the correct plaintext output of the gate.


\subsubsection{Homomorphic Evaluation of gates:}
\begin{itemize}
\item \textit{Addition gates.}
Homomorphic evaluation of an addition gates on inputs $c_1,\dots,c_t$ where $c_i = (\v_i, \ell)$ is performed by outputting
\[
 c_{\text{add}} = (\v_{\text{add}}, \ell) \eqdef \Big( \Sum_{i} \v_i, \ell \Big)
\]

Informally, one can see that
\[
\inprod{\k_{\ell}}{\v_{\text{add}}} =  \inprod{\k_{\ell}}{\Sum_i\v_i} = \Sum_i\inprod{\k_{\ell}}{\v_i} = \Sum_i \mu_i
\]
where $\mu_i$ is the plaintext corresponding to $\v_i$. This satisfies the invariant in Eq. \ref{he-invariant}.

\item \textit{Multiplication gates.}
We show how to multiply ciphertexts $c, c'$ where $c = (\v, \ell)$ and $c' = (\v', \ell)$ to obtain an output ciphertext $c_{\text{mult}} = (\v_{\text{mult}}, \ell+1)$.

The homomorphic multiplication algorithm will set
\[
\v_{\text{mult}} \eqdef \Sum_{i,j \in [\lambda]} h_{i,j}\cdot \a_{\ell+1,i,j}
\]
where $h_{i,j} = \v[i] \cdot \v'[j]$ for $i,j \in [\lambda]$.


The final output ciphertext will be
\[
c_{\text{mult}} \eqdef (\v_{\text{mult}}, \ell+1).
\]

This satisfies the invariant in Eq. \ref{he-invariant} as
\begin{align*}
 \inprod{\klp}{\v_{\text{mult}}} &=  \inprod{\klp}{\Sum_{i,j \in [\lambda]} h_{i,j}\cdot \a_{\ell+1,i,j}} \\
                                &= \Sum_{i,j \in [\lambda]} ( h_{i,j} \cdot \inprod{\klp}{\a_{\ell+1,i,j}} )\\
                                &=\Sum_{i,j \in [\lambda]} ( h_{i,j} \cdot \kl[i]\cdot\kl[j] )\\
                                &= \Sum_{i,j \in [\lambda]} ( \v[i] \cdot \v'[j] \cdot \kl[i]\cdot\kl[j] )\\
                                &= \Big(\Sum_{i \in [\lambda]} \v[i] \cdot \kl[i]\Big)\cdot\Big(\Sum_{j \in [\lambda]} \v'[j] \cdot \kl[j]\Big)\\
                                &= \inprod{\kl}{\v}\cdot\inprod{\kl}{\v'}\\
                                &= \mu \cdot \mu'
\end{align*}

where in the third and fourth equality we used respectively Eq. \ref{he-keys-invariant} and the definition of $h_{i,j}$, and $\mu, \mu'$ are the plaintexts corresponding to $\v$ $\v'$ respectively.

\end{itemize}
\end{itemize}

%\end{framed}
%\caption{Leveled Homomorphic Construction}
%\end{figure}

\subsection{Security Analysis}

\begin{theorem}[Security]
The scheme $\HE$ is $\text{CPA}$ secure against $\NC^1$ adversaries (Definition \ref{def:cpa-he}) under the assumption $\fgAssump$.
\end{theorem} 
\begin{proof}
We are going to prove that there exists infinite $\Lambda \subseteq \naturals$ such that $(\pk,\evk, \HEEnc_{\pk}(0)) \lind (\pk,\evk, \HEEnc_{\pk}(1))$.

When using the notations $\Mkg$ and $\MD$ we will always denote matrices to respectively distributed according to $\Dl$ and  $\mathcal{D}^{\text{kg}}$, where $\Dl$ and $\mathcal{D}^{\text{kg}}$ are the distributions defined in Lemma \ref{lemma:ik00}.



We will define the (randomized) encoding procedure $\hEnc : \bit^{\lambda \times \lambda} \to \bit{\lambda}$ defined as 
\[
    \hEnc(\M, b) = \transp{\r} \M + \transp{(0 \ \dots 0 \ b)} \, ,
\]
where $r$ is uniformly distributed in $\bit^{\lambda}$. The functions we will pass to $\hEnc$ will be distributed either according to $\Mkg$ or $\MD$. Notice that: \textit{(i)}  $\hEnc(\Mkg, b)$ is distributed identically to $\HEEnc_{\pk}(b)$; \textit{(ii)} $\hEnc(\MD, b)$ corresponds to the uniform distribution over $\bit^{\lambda}$ because (by Lemma \ref{lemma:ik00}) $\MD$ has full rank and hence $\transp{\r}\MD$ must be uniformly random.

We will denote with $\Mkg_1, \dots, \Mkg_{L}$  the matrices $\M_1, \dots, \M_{\ell}$ used to construct the evaluation key in $\HEKeygen$ (see definition). Recall these matrices are distributed according to $\mathcal{D}^{\text{kg}}$ as in Lemma \ref{lemma:ik00}.

We will also define the following vectors:
\[
\valphakg_{\ell} \eqdef \{ \hEnc(\Mkg_{\ell+1}, \kl[i] \cdot \kl[j]) \ | \ i,j \in [\lambda] \} \qquad \valphaD_{\ell} \eqdef \{ \hEnc(\MD_{\ell+1}, \kl[i] \cdot \kl[j]) \ | \ i,j \in [\lambda] \} \, ,
\]
where $\kl$ is defined as in $\HEKeygen$ and the matrices in input to $\hEnc$ will be clear from the context. Notice that all the elements of $\valphakg_{\ell}$ are encryptions, whereas all the elements of $\valphaD_{\ell}$ are uniformly distributed.

We will use a standard hybrid argument. Each of our hybrids is parametrized by a bit $b$. This bit informally marks whether the hybrid contains an element indistinguishable from an encryption of $b$.
\begin{itemize}

\item $\hE^b \eqdef (\Mkg_0, \hEnc(\Mkg_0, b), \valphakg_1, \dots, \valphakg_L) $ where 
$\Mkg_0$ corresponds to the public key of our scheme.
%$\Mkg_{\ell}$-s with $\ell \in [L]$ correspond to the elements of the evaluation keys as in the definition of $\HEKeygen$.
Notice that $\valphakg_{\ell} \equiv \{ \vec{a}_{\ell, i, j} \ | \ i,j \in [\lambda] \}$ where $ \vec{a}_{\ell, i, j}$ is as defined in $\HEKeygen$. This hybrid corresponds to the distribution $(\pk,\evk, \HEEnc_{\pk}(b))$.
\item $\hH^b_0 \eqdef (\MD_0, \hEnc(\MD, b), \valphakg_1, \dots, \valphakg_L)$. The only difference from $\hE$ is in the first two components where we replaced the actual public key and ciphertext with a full rank matrix distributed according to $\Dl$ and a random vector of bits.
\item For $\ell \in [L]$ we define
\[
\hH^b_{\ell} \eqdef (\MD_0, \hEnc(\MD, b), \valphaD_1, \dots, \valphaD_{\ell}, \valphakg_{\ell+1}, \dots, \valphakg_L) \, .
\]
%where $\valphaD_{l} \eqdef \{ \vec{a}_{\ell, i, j} \ | \ i,j \in [\lambda], \r \sample \bit^{\lambda} \}$
\end{itemize}

We will proceed proving that
\[
\hEz \lind \hHz_0 \lind \hHz_1 \lind \dots \lind \hHz_L \lind \hH^1_L \lind \dots \lind \hH^1_1 \lind \hH^1_0 \lind \hE^1
\]
through a series of smaller claims. In the remainder of the proof $\Lambda$ refers to the set in Lemma \ref{lemma:ik00}.

\begin{itemize}
\item $\hEz \lind \hHz_0$: if this were not the case we would be able to distinguish $\Mkg_0$ from $\MD_0$ for some of the values in the set $\Lambda$ thus contradicting Lemma \ref{lemma:ik00}.
\item $\hHz_{\ell-1} \lind \hHz_{\ell}$ for $\ell \in [L]$: assume by contradiction this statement is false for some $\ell \in [L]$. That is 
\[
(\MD_0, \hEnc(\MD_0, b), \valphaD_1, \dots, \valphaD_{\ell-1}, \valphakg_{\ell}, \dots, \valphakg_L) \not \lind 
(\MD_0, \hEnc(\MD_0, b), \valphaD_1, \dots, \valphaD_{\ell}, \valphakg_{\ell+1}, \dots, \valphakg_L) \, .
\]
Recall that, by definition, the elements of $\valphakg_{\ell}$ are all encryptions whereas the elements of $\valphaD_{\ell}$ are all randomly distributed values. This contradicts the the semantic security of the scheme $\PKE$ (by a standard hybrid argument on the number of ciphertexts). 
%Informally, it is then possible to  define a sub-hybrid that breaks .


\item $\hHz_{L} \lind \hH^1_{L}$: the distributions associated to these two hybrids are identical. In fact, notice the only difference between these two hybrids is in the second component: $\hEnc(\MD, 0)$ in $\hHz_{L}$ and $\hEnc(\MD, 1)$ in $\hH^1_{L}$. As observed above $\hEnc(\MD, b)$ is uniformly distributed, which proves the claim.
\end{itemize}

All the claims above can be proven analogously for  $\hE^1, \hH^1_{0}$  and $\hH^1_{\ell}$-s.
\qed
\end{proof}

\subsection{Efficiency and Homomorphic Properties of Our Scheme}

Our scheme is secure against adversaries in the class $\NC^1$. This implies that we can run $\HEEval$ only on functions $f$ that are in $\NC^1$, otherwise the evaluator would be able to break the semantic security of the scheme.
However we have to ensure that the \textit{whole} homomorphic evaluation stays in $\NC^1$. The problem is that homomorphically evaluating $f$ has an overhead with respect to the "plain" evaluation of $f$. Therefore, we need to determine 
for which functions $f$, we can guarantee that 
%the fact that $F$ is in $\NC^1$ may not be a guarantee for 
$\HEEval(F, \dots)$ will stay in $\NC^1$. 
%We now proceed to analyze for which classes of functions this is the case.

In terms of circuit depth, the main overhead when evaluating $f$ homomorphically is given by the multiplication gates (addition, on the other hand, is ``for free'' --- see definition of $\HEEval$ above). A single homomorphic multiplication can be performed by a depth two $\ACzt$ circuit, but this requires depth $\Omega(\log(n))$ with a circuit of fan-in two. Therefore, a circuit for $f$ with $\omega(1)$ multiplicative depth would require an evaluation of $\omega(\log(n))$ depth, which would be out of $\NC^1$. On the other hand, observe that for any function $f$ in $\ACzt$ with constant multiplicative depth, the evaluation stays in $\ACzt$. This because there is a constant number (depth) of homomorphic multiplications each requiring an $\ACzt$ computation. 

We can now state the following result, derived from the observations above and the fact that the invariant in Eq. \ref{he-invariant} is preserved throughout homomorphic evaluation.

\begin{theorem}
\label{thm:he-homomorphic}
Let $\ACztcm$ the family of circuits in $\ACzt$ with constant multiplicative depth (see Definition \ref{def:ACztcm}).
The scheme $\HE$ is leveled $\ACztcm$-homomorphic. Key generation, encryption, decryption and evaluation are all computable in $\ACztcm$. 
\end{theorem}

\subsection{Beyond Constant Multiplicative Depth}
\label{sec:beyond-cm}

In the previous section we saw how our scheme is homomorphic for a class of constant-depth, unbounded fan-in arithmetic circuits in $\GF(2)$ with \textit{constant multiplicative depth}, i.e. polynomials in $\GF(2)$ of constant degree. We now show how to overcome this limitation by slightly changing our scheme and using techniques from  \cite{razborov1987lower} to approximate $\ACzt$ circuits with low-degree polynomials.



% \begin{definition}[Quasi-Constant Multiplicative Depth]
% Let $C \in \ACzt$ be a circuit. Let $S$ be the number of $\function{AND}$ gates of non constant fan-in. If $S = O(1)$ we say that $C$ has quasi-constant multiplicative depth. We denote with $\ACztq$ the class of circuits with such property.
% \end{definition}

\begin{lemma}[\cite{razborov1987lower}]
\label{lemma:razborov}
Let $C$ be an $\ACztq$ circuit of depth $d$. Then there exists a randomized circuit $C' \in \ACztcm$ such that, for all x,
\[
\Pr[C'(x) \not = C(x)] \leq \epsilon \, ,
\]
where $\epsilon = O(1)$. The circuit $C'$ uses $O(n)$ random bits and its representation can be computed in $\NC^0$ from a representation of $C$.
%\matteo{TODO: Check if representation can be computed in AC0.}
\end{lemma}
\begin{proof}
Consider a circuit $C \in \ACztq$ and let $K = O(1)$ be the total number of $\ANDgt$ and $\ORgt$ gates with non-constant fan-in. We can replace every $\ORgt$ gate of fan-in $m = \omega(1)$ with a randomized ``gadget'' that takes in input $m$ additional random bits and computes the function
$$\hat{g}_{\ORgt}(x_1,\dots,x_m; r_1, \dots, r_m) \eqdef \sum_{i \in [m]} x_i r_i \, .$$
This function can  be implemented in constant multiplicative depth with one $\XORgt$ gate and $m$ $\ANDgt$ gates of fan-in two.
Let $\vec{x} = (x_1,\dots,x_m)$ and $\vec{r} = (r_1,\dots, r_m)$. The probabilistic gadget $\hat{g}_{\ORgt}$ has one-sided error. if  $x_i = 0$ (i.e. if $\ORgt(\vec{x}) = 0$) then $\Pr[\hat{g}_{\ORgt}(\vec{x}; \vec{r}) = 0] = 1$; otherwise $\Pr[\hat{g}_{\ORgt}(\vec{x}; \vec{r}) = 1] = \frac{1}{2}$.

In a similar fashion, we can replace every unbounded fan-in $\ANDgt$ gate with a randomized gadget in computing
$$\hat{g}_{\ANDgt}(x_1,\dots,x_m; r_1, \dots, r_m) \eqdef 1 - \sum_{i \in [m]} (1 - x_i) r_i \, .$$
This gadget can also be implemented in constant-multiplicative depth and has one-sided error $1/2$.
Finally, let us observe that $\Pr[C'(x) \not = C(x)] \leq \epsilon$ with $\epsilon$ being a constant, because
we have only a constant number of gates to be replaced with gadgets for $\hat{g}_{\ORgt}$ or $\hat{g}_{\ANDgt}$.

We only provide the intuition for why the transformations above can be carried out in $\NC^0$. Assume the encoding of a circuit as a list of gates in the form $(g, t_g, in_1, \dots, in_m)$ where $g$ and $t$ are respectively the index of the output wire of the gate and its type (possibly of the form ``input'' or ``random input'') and the $in_i$-s are the indices of the input wire of $g$. The transformation from $C$ to $C'$ needs to simply copy all the items in the list except for the gates of unbounded fan-in. We will assume the encoding conventions of $C$ always puts these gates at the end of the list\footnote{This allows our $\NC^0$ circuit to to ``know'' which gates to copy and which ones to transform based on their position only.}. For each of such gates the transformation circuit needs to: add appropriate $r_1,\dots, r_m$ to the list, add $m$ $\ANDgt$ gates and one $\XORgt$, possibly (if we are transforming an $\ANDgt$ gate) add negation gates. All this can be carried out based on wire connections and the type of the gate (a constant-size string) and thus in $\NC^0$.
% Say this later
% If we repeat the execution of the gadget $s$ times using every time fresh random bit vectors $\vec{r}^{(1)},\dots, \vec{r}^{(s)}$, then we can correctly compute $\ORgt(\vec{x})$ with overwhelming probability. Define $h_{\ORgt}(\vec{x}; \vec{r}^{(1)},\dots, \vec{r}^{(s)}) \eqdef \ORgt(\hat{g}_{\ORgt}(\vec{x}; \vec{r}^{(1)}), \dots, \hat{g}_{\ORgt}(\vec{x}; \vec{r}^{(s)}))$. Clearly $\Pr[h_{\ORgt}(\vec{x}; \vec{r}^{(1)},\dots, \vec{r}^{(s)}) = \ORgt(\vec{x})] \geq 1 - 2^{-s}$.
% Say that you replace every OR with $\hat{g}_{\func{or}}(x_1, \dots, x_m; r_1, \dots r_m) = \sum_i x_i\cdot r_i$ and that we can do this with the appropriate gates. The probability of success is 1/2. 
\end{proof}

In the construction above, we built $C'$ by replacing every gate $g \in \func{S}_{\omega(1)}(C)$ (as in Definition \ref{def:ACztq}) with a (randomized) gadget $G_g$. The output of each these gadgets will be useful in order to keep the low complexity of the decryption algorithm in our next homomorphic encryption scheme. We shall use an ``expanded'' version of $C'$, the multi-output circuit $C'_{exp}$. 

\def\GenApproxFun{\function{GenApproxFun}}
\def\SampleAuxRandomness{\function{SampleAuxRandomness}}
\def\EvalApprox{\function{EvalApprox}}

\begin{definition}[Expanded Approximating Function]
\label{def:exp-approx-fun}
Let $C$ be a circuit in $\ACztq$ and let $C'$ be a circuit as in the proof of Lemma \ref{lemma:razborov}.
We denote by $G_g(\vec{x}; \vec{r})$  the output of the gadget $G_g$ when $C'$ is evaluated on inputs $(\vec{x}; \vec{r})$. On input $(\vec{x}; \vec{r})$, the multi-output circuit $C'_{exp}$  output $C'(\vec{x};\vec{r})$ together with the outputs of the $O(1)$ gadgets $G_g$ for each $g \in \func{S}_{\omega(1)}(C)$. Finally, we denote with $\GenApproxFun$ the algorithm computeing a representation of $C'_{exp}$ from a representation  of $C$.
\end{definition}

\def\auxf{\mathbf{aux}_f}


\begin{lemma}
\label{lemma:decode-approx}

%\label{lemma:}
There exists a deterministic algorithm $\func{DecodeApprox}$  computable in $\ACzt$ with the following properties. 
For every circuit $C$ in $\ACztq$ computing the function $f$, there exists $\auxf \in \bit^{O(1)}$ such that for all $\vec{x} \in \bit^n$
$$\Pr[\func{DecodeApprox}(C'_{exp}(\vec{x}; \vec{r}^{(1)}), \dots, C'_{exp}(\vec{x}; \vec{r}^{(s)}) ) = C(\vec{x})] \geq 1 - \negl(s) \, ,$$
where $C'$ is an approximating circuit as in Lemma \ref{lemma:razborov}, the probability is taken over  the uniformly distributed bit vectors $\vec{r}^{(i)}$-s for $i \in [s]$, $C'_{exp}$ is as in Definition \ref{def:exp-approx-fun}.
Finally, there exists a function $\func{GenDecodeAux}$ that computes $\auxf$ from a representation of $C$ in $\NC^0$.
% \vec{y}^{(1)},\dots, \vec{y}^{(s)}, z^{(1)}, \dots, z^{(s)}
% $\vec{y}^{(i)} \eqdef \{ G_g(\vec{x}, \vec{r}^{(i)}) : g \in \func{S}_{\omega(1)}(C) \}$ 
\end{lemma}
\begin{proof}
Before we provide a construction for $\func{DecodeApprox}$, let us observe how we can amplify the error of $C'$.  Consider for example a gadget $\hat{g}_{\ORgt}$ constructed as in the proof of Lemma \ref{lemma:razborov}, approximating  an $\ORgt$ gate in $C$.
If we repeat the execution of the gadget $s$ times, every time using fresh random bit vectors $\vec{r}'^{(1)},\dots, \vec{r}'^{(s)}$, then we can correctly compute $\ORgt(\vec{x}')$ with overwhelming probability. Define $h_{\ORgt}(\vec{x}'; \vec{r}'^{(1)},\dots, \vec{r}'^{(s)}) \eqdef \ORgt(\hat{g}_{\ORgt}(\vec{x}; \vec{r}'^{(1)}), \dots, \hat{g}_{\ORgt}(\vec{x}'; \vec{r}'^{(s)}))$. Clearly $\Pr[h_{\ORgt}(\vec{x}'; \vec{r}'^{(1)},\dots, \vec{r}'^{(s)}) = \ORgt(\vec{x}')] \geq 1 - 2^{-s}$.
In a similar fashion we can define  $h_{\ANDgt}(\vec{x}'; \vec{r}'^{(1)},\dots, \vec{r}'^{(s)}) \eqdef \ANDgt(\hat{g}_{\ANDgt}(\vec{x}; \vec{r}'^{(1)}), \dots, \hat{g}_{\ANDgt}(\vec{x}'; \vec{r}'^{(s)}))$. It holds that $\Pr[h_{\ANDgt}(\vec{x}'; \vec{r}'^{(1)},\dots, \vec{r}'^{(s)}) = \ANDgt(\vec{x}')] \geq 1 - 2^{-s}$.

If $C'$ were composed by a single gadget $\hat{g}_{\ORgt}$ (resp. $\hat{g}_{\ANDgt}$) we could just let $\func{DecodeApprox}$ be the same as $h_{\ORgt}$ (resp. $h_{\ANDgt}$) and we would be done. To deal with multiple gadgets, however, we need a more general approach. For sake of presentation, assume there are only gadgets approximating $\ORgt$ gates and let us temporarily ignore $\auxf$. We can write each of the $C'_{exp}(\vec{x}; \vec{r}^{(j)})$ input to $\func{DecodeApprox}$ as $(z^{(j)}, y_1^{(j)}, \dots, y_K^{(j)})$ where $K \eqdef |\func{S}_{\omega(1)}|$, $z^{(j)}$ is the output of $C'(\vec{x}, \vec{r}^{(j)})$ and $y_i^{(j)}$ is the output of the $i-th$ gadget when provided random bits from $\vec{r}^{(j)}$. Define $y_i^*$ as $y_i^* \eqdef \ORgt(y_i^{(1)}, \dots,y_i^{(s)})$. 
We then let the output of $\func{DecodeApprox}$ be $z^{j^*}$ where $j^*$ is such that for all $i \in [K]$ it is the case that $y_i^{j^*} = y_i^*$. By the union bound the probability of $z^{j^*} \not = C(\vec{x})$ is upper bounded by $K\cdot 2^{-s}$, which is negligible since $K = O(1)$. 
To generalize this same approach to the scenario including both $\ORgt$ and $\ANDgt$ gadgets we let the string $\auxf$ include information on the type of gates in $\func{S}_{\omega(1)}$. This way  $\func{DecodeApprox}$ can use $\hat{g}_{\ORgt}$ or $\hat{g}_{\ANDgt}$ accordingly.
Clearly the  representation of $\auxf$ can be computed by a representation of $C$ in $\NC^0$.
\end{proof}


%The result above allows 

\subsubsection{Homomorphic Evaluations of $\ACztq$ Circuits}

Below is a variation of our homomorphic scheme that can evaluate all circuits in $\ACztq$ in $\ACzt$. This time, in order to evaluate circuit $C$, we perform several homomorphic evaluations of the randomized circuit $C'$ (as in Lemma \ref{lemma:razborov}). To obtain the plaintext output of $C$ we can decrypt all the ciphertext outputs and take the majority result. Notice that this scheme is still compact.
As we use a randomized approach to evaluate $f$, the scheme $\HEp$ will be implicitly parametrized by a soundness parameter $s$. Intuitively, the probability of a function $f$ being evaluated incorrectly will be upper bounded by $2^{-s}$.

%To simplify notation, in the following paragraphs and in Section \ref{sec:vc} we 
%will slightly abuse the syntax for homomorphic encryption schemes and consider 
%both the public key and evaluation key as part of $\pk$.


For our new scheme we will use the following auxiliary functions:
\begin{definition}[Auxiliary Functions for $\HEp$]
\label{def:aux-he-fns}
\item Let $f: \bit^t \to \bit$ be represented as an arithmetic circuit as in $\HE$ and $\pk$ a public key for the scheme $\HE$ that includes the evaluation key. Let $s$ be a soundness parameter.
We denote by  $f'$ the expanded randomized function approximating $f$ as in Definition \ref{def:exp-approx-fun}; let $t' = O(t)$ be the number of additional random bits $f'$ will take in input.
\begin{itemize}
\item $\GenApproxFun(f):$
\begin{itemize}
\item Computes and returns the representation of the expanded approximating function $f'$ as in Definition \ref{def:exp-approx-fun}.
\end{itemize}
\item $\func{GenDecodeAux}(f):$
\begin{itemize}
\item Computes and returns the auxiliary string $\auxf$ from a representation of $f$ as in Lemma \ref{lemma:decode-approx}.
\end{itemize}
\item $\SampleAuxRandomness_s(\pk, f'):$
\begin{enumerate}
\item We assume  $f'$ is the expanded randomized function approximating $f$ as in Definition \ref{def:exp-approx-fun}; let $t' = O(t)$ be the number of additional random bits $f'$ will take in input.
\item Sample $s\cdot t'$ random bits $r^{(1)}_1,\dots,r^{(1)}_{t'}, \dots, r^{(s)}_1,\dots,r^{(s)}_{t'}$;
\item Compute $\hatraux \eqdef \{  \hat{r}^{(i)}_j \ | \ \hat{r}^{(i)}_j \gets \HEEnc_{\pk}(r^{(i)}_j), i \in [s], j \in [t'] \} $;
\item Output $\hatraux$.
\end{enumerate}
\medskip
\item $\EvalApprox_s(\pk, f', c_1,\dots,c_t, \hatraux):$
\begin{enumerate}
\item Let $\hatraux = \{  \hat{r}^{(i)}_j \ | \  i \in [s], j \in [t'] \} $.
\item For $i \in [s]$, compute $\vec{c}^{\text{out}}_i \gets \HEEval_{\evk}(f', c_1, \dots c_t, \hat{r}^{(i)}_1, \dots, \hat{r}^{(i)}_{t'})$;
\item Output  $\vec{c} = (\vec{c}^{\text{out}}_1, \dots, \vec{c}^{\text{out}}_{s})$\footnote{Recall that the output of the expanded approximating function $f'$ is a bit string and thus each $\vec{c}^{\text{out}}_i$ encrypts a bit string.}.
\end{enumerate}
\end{itemize}
\end{definition}

\bigskip

The new scheme $\HEp$ with soundness parameter $s$ follows. Notice that the evaluation function outputs an auxiliary string $\auxf$ together with the proper ciphertext $\vec{c}$. This is necessary to have a correct decoding in decryption phase.
\begin{framed}
\begin{itemize}
\item Key generation and encryption are the same as in  $\HE$.
\item $\HEpEval_{\pk}(f, c_1,\dots,c_t)$:
\begin{enumerate}
\item Compute $f' \gets \GenApproxFun(f)$;
\item Compute $\hatraux \gets \SampleAuxRandomness_s(\pk, f')$;
\item $\auxf \gets \func{GenDecodeAux}(f)$;
\item $\vec{c} \gets \EvalApprox_s(\pk, f', c_1,\dots, c_t, \hatraux)$;
\item Output $(\vec{c}, \auxf)$.
%\item Output $\vec{c}$. % = (\vec{c}^{\text{out}}_1, \dots, \vec{c}^{\text{out}}_{s})$.
\end{enumerate}
\item $\HEpDec_{\sk}(\vec{c} = (\vec{c}^{\text{out}}_1, \dots, \vec{c}^{\text{out}}_{s}), \auxf)$:
\begin{enumerate}
%\item $\func{DecodeApprox}_f \gets \func{GenDecodeApproxFun}(\auxf)$
\item Let $\vec{y}^{\text{out}}_i \gets \HEDec_{\sk}(\vec{c}^{\text{out}}_i)$ for $i \in [s]$;
\item Output $\func{DecodeApprox}_f(\vec{y}^{\text{out}}_1, \dots, \vec{y}^{\text{out}}_s)$.
\end{enumerate}
\end{itemize}
\end{framed}
% \newcommand{\MACz}{\class{MAC}_0}
% In the theorem below, the complexity class $\MACz$ refers to the 

\begin{remark}
Given in input a function $f$ not necessarily of constant multiplicative depth,  $\GenApproxFun$ returns a function $f'$ of constant multiplicative depth that approximates it.
As stated in Lemma \ref{lemma:razborov}, $\GenApproxFun$ is computable in $\NC^0$ and so is $\func{GenDecodeAux}$. The function $\SampleAuxRandomness$ in $\ACztcm$ and $\EvalApprox$ makes parallel invocations to $\HEEval$ which is computable in $\ACztcm$ when provided in input a function in $\ACztcm$ (Theorem \ref{thm:he-homomorphic}). This fact will be useful when showing the completeness of our verifiable computation schemes in Section \ref{sec:vc}.
\end{remark}

\begin{theorem}
\label{thm:hep-homomorphic}
Let $\ACztq$ the family of circuits in $\ACzt$ with quasi-constant multiplicative depth as in Definition \ref{def:quasi-constant}.
The scheme $\HEp$ above with soundness parameter $s = \Omega(\lambda)$ is leveled $\ACztq$-homomorphic. Key generation, encryption and evaluation can be computed in $\ACztcm$. Decryption is computable in $\ACzt$.
\end{theorem}

\section{Fine-Grained Verifiable Computation}
\label{sec:vc}
\label{sec:VC}


In this section we describe our private verifiable computation scheme.
Our constructions are heavily based on the techniques in \cite{ckv10} to obtain (reusable) verifiable computation from fully homomorphic encryption.
In order to guarantee that these techniques also work within $\NC^1$  we prove that: \textit{(i)} the constructions can be computed in low-depth; \textit{(ii)} the reductions in the security proofs can be carried out in low-depth.

\medskip
\noindent{\sc The Scheme from \cite{ckv10}.}
To derive Verifiable Computation from Homomorphic Encryption, \cite{ckv10} follows this approach. The Client, in the expensive preprocessing phase, selects a random input $r$, encrypts it $c_r=E(r)$ and homomorphically compute $c_{f(r)}$ an encryption of $f(r)$. During the online phase, the Client, on input $x$, computes $c_x=E(x)$ and submits
the ciphertexts $c_x,c_r$ in random order to the Server, who homomorphically compute 
$c_{f(r)}=E(f(r))$ and $c_{f(x)}=E(f(x))$ and returns them to the Client. The Client given the message $c_0,c_1$ from the Server, checks that $c_b=c_{f(r)}$ (for the appropriate bit $b$) and if so accepts $y=D(c_{f(x)})$ as $y=f(x)$. The semantic security of $E$ guarantees that this protocol has soundness error $1/2$ (which can be reduced by parallel repetition). This scheme is however one-time, as a malicious server can figure out 
which one is the test ciphertext $c_{f(r)}$ if it is used again. 

To make this scheme ``many time secure", \cite{ckv10} uses the paradigm introduced in 
\cite{ggp10} of running the one-time scheme ``under the covers" of a different
homomorphic encryption key each time. 

\subsection{A One-time Verification Scheme}

Before we present our variant of the one-time construction in \cite{ckv10}, we present two auxiliary lemmas that guarantee that our protocols are computable in $\ACzt$. We refer the reader to \cite{hag91,matias91} for the proof Lemma \ref{lemma:perm-sampling}. % and we skip the proof of the second lemma since it is trivial.

\begin{lemma}{\cite{hag91,matias91}}
\label{lemma:perm-sampling}
There are uniform $\AC^0$ circuits $C: \bit^{\poly(l)}\to [l]^l$ of size $\poly(l)$ and depth $O(1)$ whose output distribution have statistical distance $\leq 2^{-l}$ from the uniform distribution over permutations of $[l]$.
\end{lemma}

\begin{lemma}
\label{lemma:perm-evaluation}
There are uniform $\AC^0[2]$ circuits $C: [l]^l \times \bit^l \to \bit^l$ of size $O(l^2)$ where $C(\pi, (x_1,\dots, x_l)) = (\pi(1),\dots,\pi(l))$ and $\pi$ is a permutation.
\end{lemma}
\begin{proof}
Let $\vec{x} = (x_1,\dots,x_l)$ the bits to permute and let $\pi$ be a permutation
If $\pi$ is represented as a permutation matrix with rows $\r_1,\dots,\r_l$, we can permute $\vec{x}$ by simply performing $l$ parallel inner products $\inprod{\vec{x}}{\r_i}$-s, which is in $\ACzt$.
We now describe how to generate the permutation matrix from a binary representations $x_1,\dots,x_{\function{lg}(l)}$ of the integers in $[l]$.
Let $f_i : \bit^{\function{lg}(l)} \to \bit^l$ be the function that computes the $i$-th row of the permutation matrix. We can define $f_i$ as follows:
\[
    f_i(x_1,\dots,x_{\function{lg}(l)}) \eqdef \function{eq}([i-1]_2, (x_1,\dots,x_{\function{lg}(l)}) \, ,
\]
where $[i-1]_2$ is the binary representation of $i-1$ and $\function{eq}$ returns 1 if its two inputs (each of lenght $\function{lg}(l)$) are equal.
The function $f_i$ is clearly in $\ACzt$.
% First, for each $i$, we generate the vector of hamming weight $1$ describing $\pi(i)$ by mapping 
\end{proof}

In Figure \ref{fig:one-time} we describe an adaptation of the one-time secure delegation scheme from \cite{ckv10}. We make non-black box use of our homomorphic encryption scheme $\HEp$ (Section \ref{sec:beyond-cm}) with soundness parameter $s= \lambda$. Notice that. during the preprocessing phase, we fix the ``auxiliary randomness'' for $\EvalApprox$ (and thus for $\HEpEval$) once and for all. We will use that same randomness for all the input instances. This choice does not affect the security of the construction.
We remind the reader that we will simplify notation by considering the evaluation key of our somewhat homomorphic encryption scheme as part of its public key.

If $x$ is a vector of bits $x_1, \dots, x_n$, below we will denote with $\HEpEnc(x)$ the concatenation of the bit by bit ciphertexts $\HEpEnc(x_1), \dots, \HEpEnc(x_n)$. We denote by $\HEpEnc(\zero)$ the concatenation of $n$ encryptions of $0$, $\HEpEnc(0)$.

% \item Compute $\hatraux \gets \SampleAuxRandomness_s(\pk, F)$;
% \item Compute $\vec{c} \gets \EvalApprox_s(\pk, F, \hatraux)$;
% \item Output $\vec{c} = (c^{\text{out}}_1, \dots, c^{\text{out}}_{s})$.

\begin{figure}
\begin{framed}
Let $f: \bit^n \to \bit^m$ be a function and $\GenApproxFun, \SampleAuxRandomness$ and $\EvalApprox$ as in Definition \ref{def:aux-he-fns}.
\begin{itemize}
\item $\VCKG(1^\lambda, f) \rightarrow (\pkW, \skD)$: We assume function $f$ represented as 
\begin{enumerate}
\item Generate a pair of keys $(\pk,\sk) \gets \HEpKeygen(1^\lambda)$.
\item Generate the approximating function $f' \gets \GenApproxFun(f)$ and auxiliary string $\auxf \gets \func{GenDecodeAux}(f)$;
\item Generate the ciphertext of the auxiliary random input for homomorphic evaluation $\hatraux \gets \SampleAuxRandomness_{\lambda}(\pk, f')$
\item Compute $t$ independent encryptions $\hatr_i = \HEpEnc_{\pk}(\zero)$ and the homomorphic evaluations $\hatw_i = \hatF(\hatr_i) =  \EvalApprox_s(\pk, f', \hatr_i, \hatraux)$ for $i \in [t]$;
\item $\pkW \gets (\pk, f', \hatraux), \skD \gets (\{ (\hatr_i, \hatw_i)_{i \in [t]}\}, \auxf)$.
\end{enumerate}
\item $\VCPG_{\skD}(x) \rightarrow (\qx, \sx)$: 
\begin{enumerate}
\item Compute $t$ independent encryptions $\hatr_{i+t} = \HEpEnc_{\pk}(x)$ for $i \in [t]$.
\item Sample a random permutation $\pi \sample S_{2t}$.
\item $\qx \gets (\hatzpi{1},\dots,\hatzpi{2t}) = (\hatr_1,\dots,\hatr_{2t})$; $\sx \gets \pi$
\end{enumerate}
\item $\VCCompute_{\pkW}(\qx) \rightarrow \ax$:
\begin{enumerate}
\item Compute $\hat{y}_i = \hatF(\hat{z}_i) = \EvalApprox_s(\pk, f', \hat{z}_i, \hatraux)$ for $i \in [2t]$.
\item $\ax = (\hat{y}_1,\dots,\hat{y}_{2t})$.
\end{enumerate}
\item $\VCVerif_{\skD}(\sx, \ax)$:
\begin{enumerate}
\item Check if $\hatw_i = \hat{y}_i$ for all $i \in [t]$.
\item Check if $\HEpDec_{\sk}(\hat{y}_{\pi(t+1)}, \auxf) = \dots = \HEpDec_{\sk}(\hat{y}_{\pi(2t)}, \auxf)$.
\item If either of the two tests above fails, return $\fail$; otherwise return $\HEpDec_{\sk}(\hat{y}_{\pi(t+1)}, \auxf)$.
\end{enumerate}
\end{itemize}
\end{framed}
\caption{One-Time Delegation Scheme}
\label{fig:one-time}
\end{figure}

\begin{remark}[On deterministic homomorphic evaluation]
As pointed out in \cite{ckv10}, one requirement for the approach in Figure \ref{fig:one-time} to work is for the homomorphic evaluation to be deterministic. We point out that once $\hatraux$ are fixed once and for all the homomorphic evaluation in $\VCCompute$ is deterministic.
\end{remark}

% \matteo{See if this should be fixed}
% \begin{remark}[On including $f'$ in $\pkW$]
% In the construction above we included $f'$ in the public key lengthening the size of the key. We point out this is not necessary and that $f'$ can be computed by the worker on her own during the execution of $\VCCompute$. However this would not allow us to simply homomorphically evaluate $\VCCompute$ in the definition of $\tVC$ in Section \ref{sec:vc-many}. This because the complexity of $\VCCompute$ would go from $\ACztcm$ to $\NC^1$, which our homomorphic schemes cannot handle.
% We point out that it would still be possible to modify the construction of $\tVC$  not including $f'$ in $\pkW$ to obtain the same completeness and soundness properties. However this would come at a cost of a more complex transformation in Figure \ref{fig:reusable-transform}. Including $f'$ in $\pkW$ allowed us to kept the transformation as simple and close to the original description in \cite{ckv10} as possible. 
% \end{remark}


\begin{lemma}[Completeness of $\VC$]
The verifiable computation scheme $\VC$ in Figure \ref{fig:one-time} has overwhelming completeness (Definition \ref{def:vc-completeness}) for the class $\ACztq$.
\end{lemma}
\begin{proof}
The proof is straightforward and stems directly from the homomorphic properties of $\HEp$ (Theorem \ref{thm:hep-homomorphic}).
In fact, by construction and by definition of $\HEp$ (Section \ref{sec:beyond-cm}), the distribution of the $\hatw_i$-s is identical to $\HEpEval_{\pk}(f, \hatr_i)$. Analogously, the distribution of $\hat{y}_i$-s is identical to $\HEpEval_{\pk}(f, \hat{z}_i)$.
\end{proof}


\begin{remark}[Efficiency of $\VC$]
In the following we consider the verifiable computation of a function $f: \bit^n \to \bit^m$ computable by an $\ACztq$ circuit of size $S$.
\begin{itemize}
\item  $\VCKG$ can be computed by an $\ACzt$ circuit of size $O(\poly(\lambda)S)$; 
\item $\VCPG$ can be computed by an $\ACzt$ circuit of size $O(\poly(\lambda)(m+n))$;
\item $\VCCompute$ can be computed by an $\ACzt$ circuit of size $O(\poly(\lambda)S)$; 
\item $\VCVerif$ can be computed by a $\ACzt$ circuit of size $O(\poly(\lambda)(m+n))$ and whose (constant) depth is independent of the depth of $f$.
\end{itemize}
\end{remark}

\begin{lemma}[One-time Soundness] 
\label{lemma:one-time}
Under the assumption that $\fgAssump$ the  scheme in Figure \ref{fig:one-time} is $(1,1)$-sound (one time secure) against $\NC^1$ adversaries whenever t is chosen to be  $\omega(\log(\lambda))$.
\end{lemma}
\begin{proof}
We follow the same proof structure as in the proof of Lemma $12$ in \cite{ckv10}. We will keep part of the analysis informal, emphasizing why this proof still works for low-depth circuits. We refer the reader to \cite{ckv10} for further details.

The following observation will be crucial in the rest of the proof. Notice that, by construction and by definition of $\HEp$ (Section \ref{sec:beyond-cm}), the distribution of the $\hatw_i$-s is identical to $\HEpEval_{\pk}(f, \hatr_i)$. Analogously, the distribution of $\hat{y}_i$-s is identical to $\HEpEval_{\pk}(f, \hat{z}_i)$.

Consider an $\NC^1$ adversary $\advstar$ that cheats with non-negligible probability in the one-time security experiment $\expVCone{\VC}$ (Definition \ref{def:vc-soundness}).
Let $(\hatr_1,\dots,\hatr_{t})$ be the independent copies of $\HEpEnc_{\pkW}(\zero)$ and  $(\hatr_{t+1},\dots,\hatr_{2t})$ the $t$ independent copies of
$\HEpEnc_{\pkW}(x)$ as above. 
Whenever the verification algorithm accepts, the adversary must have responded correctly on $\hatr_1,...,\hatr_t$ and incorrectly (and consistently) on $\hatr_{t+1},\dots,\hatr_{2t}$. 
Our goal is to bound the probability that the adversary succeeds in doing that. 

First, notice that the view of the adversary is
$(\pkW, \hatr_1,\dots,\hatr_{2t})$, and identical to $(\pkW, \HEpEnc_{\pkW}(\zero)^t, \HEpEnc_{\pkW}(x)^t)$.
By semantic security of the homomorphic encryption scheme, there exists an infinitely large set of parameters $\Lambda$ such that $ (\pkW, \HEpEnc_{\pkW}(\zero)^t, \HEpEnc_{\pkW}(x)^t) \lind (\pkW, \HEpEnc_{\pkW}(\zero)^{2t})$. Consider a modified game where the adversary receives  $(\pkW, \HEpEnc_{\pkW}(\zero)^{2t})$. Denote by $p$ the probability that the adversary succeeds in this game. By computational indistinguishability we have
\[
\Pr[\advstar \text{ is correct on } (\hatr_1,\dots,\hatr_{t}) \text{ and incorrect on } (\hatr_{t+1},\dots,\hatr_{2t}) ] \leq p + \negl(\lambda)
\]
for all $\lambda \in \Lambda$.
This inequality holds because we can test in $\NC^1$ whether $\advstar$ cheats only on $(\hatr_{t+1},\dots,\hatr_{2t})$. Therefore, if the adversary's behavior differed significantly between the two games, one would be able to break the semantic security of the homomorphic scheme. Here we made use of the third fact in Lemma $\ref{lemma:facts-lind}$.

We now proceed to upper bound $p$. Observe that 
\[ 
p = \Pr[\advstar \text{ is correct on } (\hatzpi{1},\dots,\hatzpi{t}) \text{ and incorrect on } (\hatzpi{t+1},\dots,\hatzpi{2t})] 
\]
where the $\hatzpi{i}$-s are defined as in Figure \ref{fig:one-time}.
Because of Lemma \ref{lemma:perm-sampling} that the distribution of $\pi$ is statistically indistinguishable from that of a uniformly random permutation. Also, observe that the answers $\hat{y}_i$ of the adversary are independent of $\pi$.
We can then conclude that $p \leq \frac{1}{\binom{2t} {t}} + \negl(t)$, which concludes the security analysis. 
% TODO: define z as in the other case
% \begin{align*}
%     p &= \Pr[\advstar \text{ is correct on } (\hatzpi{1},\dots,\hatzpi{t}) \text{ and incorrect on } (\hatzpi{t+1},\dots,\hatzpi{2t})] \\
%     &\leq \Pr[\advstar \text{ is correct on } (\hatzpi{1},\dots,\hatzpi{t}) \text{ and incorrect on } (\hatzpi{t+1},\dots,\hatzpi{2t}) \given[\Big] \sum^{2t}_{i=1} ]
% \end{align*}
\end{proof}


\subsection{A Reusable Verification Scheme}
\label{sec:vc-many}
We now describe how to obtain a reusable verification scheme $\tVC$ applying the transformation in \cite{ckv10} from one-time sound verification schemes through fully homomorphic encryption. The core idea behind the transformation in \cite{ckv10} is to encapsulate all the operations of a one-time verifiable computation scheme through homomorphic encryption. We instantiate this transformation with the one-time verifiable construction $\VC$, described in Figure \ref{fig:one-time}, and the simplest of our two somewhat homomorphic encryption schemes, $\HE$ (defined in Section \ref{sec:leveled-he-simple}). 

\begin{figure}
\begin{framed}
Let $\VC$ be the verifiable computation scheme defined in Figure \ref{fig:one-time}. The reusable verifiable computation scheme $\tVC = (\tVCKG,\tVCPG,\tVCCompute,\tVCVerif)$ is defined as follows.
\begin{itemize}
\item $\tVCKG(1^\lambda, f) \rightarrow (\tpkW, \tskD)$: The key generation stage is the same as in $\VC$.
\item $\tVCPG_{\tskD}(x) \rightarrow (\tqx, \tsx)$: 
\begin{enumerate}
\item $(\qx, \sx) \gets \VCPG_{\skD}(x)$;
\item Compute a fresh pair of keys $(\pk_x, \sk_x) \gets \HEKeygen(1^\lambda)$;
\item Compute $\hat{q}_x \gets \HEEnc_{\pk_x}(\qx)$;
\item $\tqx \gets (\pk_x, \hat{q}_x)$; $\tsx \gets (\sx, \sk_x)$
% \item Compute $t$ independent encryptions $\hatr_{i+t} = \HEpEnc_{\pk}(x)$ for $i \in [t]$.
% \item Sample a random permutation $\pi \sample S_{2t}$.
% \item $\qx \gets (\hatzpi{1},\dots,\hatzpi{2t}) = (\hatr_1,\dots,\hatr_{2t})$; $\sx \gets \pi$
\end{enumerate}
\item $\tVCCompute_{\tpkW}(\tqx) \rightarrow \tax$:
\begin{enumerate}
\item $\hat{a}_x \gets \HEEval_{\pk_x}(\VCCompute(\cdot, f), \hat{q}_x)$.
\item $\tax \gets \hat{a}_x$.
\end{enumerate}
\item $\tVCVerif_{\tskD}(\tsx, \tax)$:
\begin{enumerate}
\item $\ax \gets \HEDec_{\sk_x}(\hat{a}_x)$.
\item $\function{return}~\VCVerif_{\tskD}(\sx, \ax) $.
\end{enumerate}
\end{itemize}
\end{framed}
\caption{Transformation from one-time $\cal{VC}$ scheme to a \textit{reusable} $\cal{VC}$ scheme}
\label{fig:reusable-transform}
\end{figure}



\begin{corollary}[Completeness of $\tVC$]
The verifiable computation scheme $\tVC$ in Figure \ref{fig:reusable-transform} has overwhelming completeness (Definition \ref{def:vc-completeness}) for the class $\ACztq$.
\end{corollary}
\begin{proof}
The completeness of the scheme above follows directly from the completeness of $\VC$ and the homomorphic properties of $\HE$.
Notice that we can use $\HEEval$ to homomorphically compute $\VCCompute$ as the latter carries out a computation in $\ACztcm$ (although it is \textit{approximating} a computation in $\ACztq$). 
\end{proof}

\begin{remark}[Efficiency of $\tVC$]
The efficiency of $\tVC$ is analogous to that of $\VC$ with the exception of a circuit size overhead of a factor $O(\lambda)$ on the problem generation and verification algorithms and of $O(\lambda^2)$ for the computation algorithm.
All algorithms in $\tVC$ are computable by constant depth circuit (of unbounded fan-in) and the depth of the verification algorithm is independent of the function $F$.
\end{remark}


\begin{theorem}
Under the assumption that $\fgAssump$ the scheme $\tVC$ in Figure \ref{fig:reusable-transform} is $(O(1),\poly(\lambda))$-sound (many-times secure) against $\NC^1$ adversaries whenever t is chosen to be  $\omega(\log(\lambda))$ in the underlying scheme $\VC$. 
\end{theorem}
\begin{proof}
By Lemma \ref{lemma:one-time} there exists an infinite set $\Lambda \subseteq \naturals$ of security parameters for which $\VC$ ``is secure''.
By the proof of Lemma \ref{lemma:one-time}, this set is also the set of parameters where the somewhat homomorphic encryption scheme $\HE$ ``is secure''.
We will show that for all values in this same set $\Lambda$, the probability of success of any  $\NC^1$ adversary in $\expVCmany{\tVC}$ is negligible. 

Assume by contradiction there exists an $\NC^1$ adversary $\advstar$ that achieves non-negligible advantage in $\expVCmany{\tVC}$ for some $\lambda \in \Lambda$.

\textbf{Claim: If $\tVC$ is not secure for some $\lambda^* \in \Lambda$ then we can break the one-time security of $\VC$.}
Let $l = O(1)$ be the number of rounds in the many-time soundness experiment for $\tVC$. Consider the following $\NC^1$ adversary  $\adv_1$ for the experiment $\expVCone{\VC}$:
\begin{itemize}
\item $\adv_1$ obtains a pair a public key $\pkW$ and sends it to $\advstar$;
\item For all rounds $i \in \{1,\dots,l-1 \}$, $\adv_1$ replies to
$\advstar$ queries by generating a fresh pair of keys $(\pk, \sk)$ and sending back encryptions of $\HEEnc_{\pk}(\zero)$;
\item At round $l$, $\adv_1$ responds to all input queries but the last one as above. This, by experiment definition,  is the input where $\advstar$ will try to cheat; we denote this input by $x^*$. Now $\adv_1$ sends $x^*$ as the only input query in the one-time security experiment and will receive back $q^*$. It will then obtain a fresh pair of keys $(\pk^*, \sk^*)$ and send $\HEEnc_{\pk^*}(q^*)$ to $\advstar$.
\item $\advstar$ will respond with $\hata^*$ and $\adv_1$ will send $\HEDec_{\sk^*}(\hata)$ to the challenger for one-time security experiment.
\end{itemize}

The advantage of $\adv_1$ depends on how likely is  $\advstar$ can successfully cheat in that interaction. Let $p$ be the advantage of $\adv_1$ in the one-time security experiment. Clearly, if $p$ is close to the advantage of $\advstar$ in the many-times security experiment $\adv_1$ breaks the security of the one-time scheme.

\textbf{Claim: the advantage of $\adv_1$ is negligibly close to that of $\advstar$ in the many-time security game for security parameter $\lambda^*$.}
We can  prove this by relying on the semantic security of the homomorphic encryption and on a hybrid argument.
% Assume by contradiction that $p$ is noticeably far from the advantage of $\advstar$ in the many-time security game for all but a finite number of values of the security parameter.

Let $L = lm$, the total number of input queries in the many-times security experiment. We now define the hybrids $H^{(j)}$ with $j \in \{0, \dots, L\}$. We define $H^{(0)}$ to be the exactly the many-time security experiment. 
For $j \in [L]$ we define $H^{(j)}$ to be an experiment where we respond to input queries with $\HEEnc_{\pk_f}(\zero)$  where $\pk_f$ is a fresh public key up to input query $j$ and behaves the many-time security experiment from input query $j+1$ on. Notice that $H^{(L)}$ corresponds to the interaction with $\adv_1$ above.

Denote by $A^{(j)}$ the output distribution of $\advstar$ when interacting with $H^{(j)}$.
Intuitively, if the advantage of the $\adv_1$ in the one-time experiment is significantly different from the advantage of $\advstar$ in the many-times security games, then $A^{(0)}$ and $A^{(L)}$ are not $\Lambda$-computationally indistinguishable.

Therefore (by Lemma \ref{lemma:hyb-arg}), there exists $j \in [L]$ such that $A^{(j-1)} \not \sim_{\Lambda} A^{(j)}$.

\def\advcpa{\adv_{\text{CPA}}}
\textbf{Claim: If there exists $j \in [L]$ such that $A^{(j-1)} \not \sim_{\Lambda} A^{(j)}$ then we can break the semantic security of $\HE$.
}
Consider the following $\NC^1$ adversary $\advcpa$ which receives in input a ``challenge'' public key $\pk^*$. $\advcpa$ will interact with $\adv^*$ simulating $H^{(j)}$ until receiving input query $x_{j}$.
At this point it will compute $q_j$ from $\VCPG(x_j)$ and send to the CPA challenger (see Remark \ref{rem:cpa-multiple}) $q_j$ and $\zero$, receiving back an encryption $c^*$ of either message under the public key $\pk^*$. $\advcpa$ will now send $(\pk^*, c^*)$ to $\advstar$ and continue simulating $H^{(j)}$ till the end of the experiment.
The adversary $\advcpa$ will check whether $\advstar$ cheated successfully at the end of the experiment and output (in the multiple-message CPA experiment) $1$ if that is the case and $0$ otherwise. This would allow $\advcpa$ to have a noticeable advantage in the experiment thus breaking the semantic security of $\HE$. 
\end{proof}

% \begin{remark}
% On  the complexity of preparing a circuit version of $\HEpEval(\cdot, F)$ in the offline stage.
% \color{red}{TODO}
% \end{remark}





%\bibliographystyle{alpha}
%\bibliography{crypto,references}

%\end{document}
