The problem of securely outsourcing data and computation has received widespread attention due to the rise of {\em cloud computing:} a paradigm where businesses lease computing resources from a service (the {\em cloud provider}) rather than maintain their own computing infrastructure.  Small mobile devices, such as smart phones and netbooks, also rely on remote servers to store and perform computation on data that is too large to fit in the device. 

It is by now well recognized that these new scenarios have introduced new security problems that need to be addressed. When data is stored remotely, outside our control, how can we be sure of its integrity? Even more interestingly, how do we check that the results of outsourced computation on this remotely stored data are correct. And how do perform these tests while preserving the efficiency of the client (i.e. avoid retrieving the whole data, and having the client perform the computation) which was the initial reason data and computations were outsourced. 

{\sf Verifiable Outsourced Computation} is a very active research area in Cryptography and Network Security (see \cite{wb15} for a survey), with the goal of designing protocols where it is impossible (under suitable cryptographic assumptions) for a provider to "cheat" in the above scenarios. While much progress has been done in this area, we are still far from solutions that can be deployed in practice. 



