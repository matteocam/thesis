In the following lemmas and theorems, $\mathcal{L}$ is a language and $\{C_n\}$ 
is a family of circuits that decides $\mathcal{L}$ of depth $d(n)$.

\begin{mydef}[Dishonest prover]
We say that a prover $\disP$ is dishonest for $\mathcal{L}$ on input  $x \in 
\binstrings$ if $\Pr[\out(P,V)(x) \neq \mathcal{L}(x)] > 0$.

We say that a dishonest prover \textbf{successfully cheats} on input $x \in \binstrings$ if the verifier accepts after an interaction $(\disP, V)(x)$ such that $\out(P,V)(x) \neq \mathcal{L}(x)$.
\end{mydef}




 
\begin{mylemma}
Let $\disP$ be a dishonest prover for $\mathcal{L}$ on input $x$. If at the 
first round of the protocol $\disP$ sends $V$ two input wires inconsistent with 
the output of the final gate (i.e. $\mathcal{L}(x)$), then, if the circuit 
family for $\langL$ have depth greater than 1, there exists another 
dishonest prover $\disP'$ that cheats successfully on $x$ with probability 
higher or equal than $\disP$.
\end{mylemma}
\begin{proof}
This is trivial since, at any round, every prover that provides 
inputs wires which are inconsistent with the output of the corresponding gate 
has probability null of succeeding.

Observe that the probability that a dishonest prover cheats successfully is 
given by:
$$ \Pr[\disP \mbox{ cheats successfully}] = (1-p_{cor})\cdot\Pr[(\disP, V)(x) = 
accept | \out(\disP,V)(x)  \neq C_n(x)])  $$ 
where $p_{cor}$ is the probability that $\disP$ provides $V$ the correct answer.

Let us proceed by induction on the depth of the circuit.

If $d =1$ then if $\disP$ gives inconsistent answer, he has no possibility to 
convince the verifier. And the conditions of the theorem are trivially 
satisfied.

\end{proof}

\begin{mydef}
The prover $\disPSm$ is the dishonest prover that gives the following reply 
patterns when answering on gate $g$:
\begin{itemize}
\item If $g$ is an input gate then give the corresponding correct values;
\item Otherwise, let $g_L$ and $g_R$ be respectively the left and right "child" 
gates of $g$:
\begin{itemize}
\item $\disPSm$ provides an incorrect answer for $g$ that is however consistent 
with the correct output value 
\item $\disPSm$ provides the correct values for the whole subcircuit rooted in 
$g_L$.
\item $\disPSm$ recursively provides answers (as for $g$) for the subcircuit 
rooted at $g_R$
\end{itemize}
\end{itemize}
\end{mydef}

% CLAIM (Lemma 2): The prover \disPSm is the one dishonest one that maximizises 
%probability of cheating with probability below

\begin{mylemma}
Let $\mathcal{L}$ be a language decidable by a 
circuit of depth $d \geq 1$. Let $\disP$ be a dishonest prover for 
$\mathcal{L}$ and $x \in \binstrings$. If at the first round of the protocol 
$\disP$ 
sends $V$ two input wires such that neither of them is equal to the 
corresponding input wire in the actual computation, then there exists another 
dishonest prover $\disP'$ that cheats successfully with probability higher than 
$\disP$.
\end{mylemma}



\begin{mylemma}
\label{lemma:bound-adv-dis}
Let $\mathcal{L}$ be a language and let $\{C\}_n$ be a family of binary boolean 
circuits of  depth $d(n)$ that decides $\mathcal{L}$;
Let $\disP$ be a dishonest prover for input $x \in \binstring^n$.
Then the probability of $\disP$ successfully cheating is lower or equal than $1-2^{-d(n)}$, i.e:

$$\Pr[(\disP, V)(x) = accept | \out(\disP,V)(x) \neq C_n(x)] \leq 1-2^{-d(n)}$$
\end{mylemma}
\begin{proof}
Let us proceed by induction on the depth $d$ of the circuit.
If $d = 1$ then there is no possibility for the prover to cheat successfully 
and the conditions of the theorem trivially hold.
Assume $d > 1$. 
Let $\claimedy$ be the output provided by $\disP$.
Clearly $\claimedy \neq C(x)$ as we are assuming that $\disP$ is acting 
dishonestly. We can think of the binary circuit $C$ as composed by two 
subcircuits $C_L$ and $C_R$ and by a final gate $g$ such that
$  C(x) = g(C_L(x), C_R(x))$. The respective depths $d_L, d_R$ of these 
subcircuits are such that $0 \leq d_L, d_R \leq d-1$. After sending 
$\claimedy$, 
the protocol requires that
$\disP$ sends output values for $C_L(x)$ and
$C_R(x)$; let us denote these claimed values respectively with $\claimedy_L$ 
and  $\claimedy_R$. 
Notice that at least one of these alleged values will be different from the 
respective correct subcircuit output: if it were otherwise, $\disP$ would 
reject immediately as $g(\claimedy_L, \claimedy_R) = C(x) \neq \claimedy$.
Thus at most one of the two values $\claimedy_L$, $\claimedy_R$ is equal 
to the output of the corresponding subcircuit.
The probability that the $\disP$ cheats successfully is:
\begin{align}
\Pr[\mbox{V accepts}] & \leq 
\frac{1}{2}\cdot(\Pr[\mbox{V accepts on } C_L] + \Pr[\mbox{V accepts on } C_R]) 
\label{ineq:RP-formula-vs-circuit}
\\
& \leq \frac{1}{2}\cdot(1-2^{-\max(d_L, d_R)}) + \frac{1}{2} 
\label{ineq:RP-ind-hyp-bound} \\
& \leq \frac{1}{2}\cdot(1-2^{-d+1}) + \frac{1}{2} \\
& = 1 - 2^{-d}
\end{align}
Note that at line \ref{ineq:RP-formula-vs-circuit} the inequality is strict 
whenever 
the two probabilities are disjoint, i.e. when $C$ is not a formula.
At line \ref{ineq:RP-ind-hyp-bound} we used the inductive hypothesis and the 
fact 
that all probabilities are at most $1$.
   
\end{proof}


% Reward gap
In the following proofs it will be useful to talk about the advantage 
that 
the honest strategy has on all the other provers. In 
\cite{guo2014rational} it 
is 
introduced the concept of (one-time) reward gap, defined as the minimum 
loss 
incurred 
by an \emph{always dishonest prover}.

\begin{mydef}[One-time Reward Gap]
\label{def:one-time-rew-gap}
The one-time reward gap for a reward function $\rew$ and a protocol 
$(P,V)$ is defined as
$$ \rewGapOneTime(n) = \min_{x \in \binstring^n} \min_{\disP \in S_x} 
(\expRewProtHon - \expRewProtDis )  $$
where $ S_x = \{ \disP \ :\  \prOutProtDis = 1 \} $
\end{mydef}

As observed in \cite{guo2014rational}, for the one-time reward gap  the 
following important property holds:
\begin{mylemma}[A lower bound on reward losses for a dishonest prover]
\label{lemma:one-time-rew-gap-bound}
Let $(P,V)$ and $\rew$ be respectively a protocol and a reward function 
yielding a rational proof for some function $f \funonstrings$.
Let $\disP$ be an arbitrary prover, then the expected loss for $\disP$ 
on input $x \in \binstring^n$ is at 
least:
$$ \rewGapOneTime(n)\cdot\prOutProtDis  $$
\end{mylemma}
\begin{proof}
Let $p_{dis} = \prOutProtDis$ and $R = \expRewProtHon$; thus
\begin{align*}
& \expRewProtHon - \expRewProtDis   \\
& \geq   R - (1-p_{dis})\cdot R -
p_{dis}\cdot(R-\rewGapOneTime(n)) \\
%& =  p_{dis}\cdot(\rewGapOneTime(n) + 1) \\
& \geq  p_{dis}\cdot\rewGapOneTime(n)
\end{align*}

\end{proof}

From Lemmas \ref{lemma:bound-adv-dis} and \ref{lemma:one-time-rew-gap-bound}
we obtain the following corollaries.
\begin{mycorollary}
The one-time reward gap for the protocol in Section \ref{sec:our-protocol} on 
an input of size $n \in \naturals$ is such that:
$$ \rewGapOneTime \geq 2^{-d(n)}  $$
where $d(\cdot)$ is the depth of the circuit on which the protocol is executed.
\end{mycorollary}



\begin{mycorollary}[Noticeable losses for dishonest provers]
\label{cor:rew-gap-noticeable}
If $2^{-d(n)} = O(\poly(n))$ there exists $R(n) = \poly(n)$
%and $R(n) \leq 2^{-d(n)}$ 
such that any prover giving  an incorrect  answer with 
noticeable probability will receive a polynomially smaller reward w.r.t. a 
honest prover.
\end{mycorollary}

% OBS: Notice that the current reward function we are using now is 
%deterministic

\begin{mythm}
If a language $\langL$ is decidable by a family of 
%$O(\log(S(n)))$-space uniform
binary circuits of fan-in two, size $S(n)$ and 
depth $d(n) = O(\log(n))$, then $\langL \in DRMA[d(n), d(n), O(d(n)\cdot 
T_{C}(n)), 
\poly(n)]$ 
where $T(n)$ is the running time of the DTM returning the 
description of the circuits for $\langL$. 
% XXX: The definition above should be checked since it should account for 
%noticeable probabilitt
\end{mythm}
\begin{proof}
As observed earlier, the protocol described in Section \ref{sec:our-protocol} 
satisfies the complexity requirements for number of rounds, communication and 
$V$'s running time.
We are now to show that the same protocol yields a rational proof for $f$ and 
that the reward function is always polynomially bounded.

As observed in Section \ref{sec:our-protocol} the protocol is rationally 
complete. By Lemma \ref{lemma:dif-rew-RP} a honest prover earns a strictly 
higher reward than any dishonest prover (in expectation) and the second bullet 
in Definition \ref{def:RP-Rosen-style} holds.

If $d(n) = O(\log(n))$  we can apply Corollary 
\ref{cor:rew-gap-noticeable} and the last bullet in Definition 
\ref{def:RP-Rosen-style} holds.

\end{proof}

