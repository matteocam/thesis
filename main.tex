\documentclass[nobib]{tufte-handout}
%\documentclass[11pt]{tufte-book}
\usepackage{times}
\usepackage{url}
\usepackage{latexsym}
\usepackage[utf8]{inputenc}
\usepackage{natbib}
\usepackage{graphicx}
\usepackage{amsthm}
\usepackage{amssymb}
\usepackage{amsmath}
\usepackage{verbatim}
\usepackage{enumitem}
\usepackage{verbatim}
\usepackage{accents}
\usepackage{blindtext}
\usepackage{hyphenat}
\usepackage{comment}
\usepackage{complexity}
\usepackage{framed}
\usepackage{csquotes}
%s\usepackage[usenames, dvipsnames]{color}
\usepackage{fancyhdr}
\usepackage[normalem]{ulem}
%\setlength\titlebox{6.5cm}    % You can expand the title box if you
% really have to

\DeclareMathOperator*{\argmin}{arg\,min}
\DeclareMathOperator*{\argmax}{arg\,max}

\hyphenation{Efficient}
\hyphenation{Computation}

\title[Rationality and Efficient Verifiable Computation]{Rationality and Efficient Verifiable Computation\\ \noindent \small{Dissertation proposal}}


\author{Matteo Campanelli}
\date{}

\newtheorem{remark}{Remark}

\begin{document}
\maketitle

  
  \begin{abstract}
  	\noindent
  	We propose the study of protocols for delegating computation in a model in which one of the party is rational.
  	A \textit{delegator} outsources the computation of a function $f$ on input $x$ to a \textit{worker}, who
  	receives a (possibly monetary) reward. Our goal is to design \textit{very efficient} delegation schemes 
  	where a worker is economically incentivized to provide the correct result
  	$f(x)$. As a design choice, we do not want to rely on cryptographic assumptions
  	(i.e. we will not assume one-way functions exist).
  	
  	We provide several results within the framework of rational proofs introduced by Azar and Micali (STOC 2012).
  	We make several contributions to efficient rational proofs for general feasible computations.
  	First, we design schemes with a sublinear verifier with low round and communication complexity for
  	space-bounded computations.
  	Second, we provide evidence, as lower bounds, against the existence of rational proofs:
  	with logarithmic communication and polylogarithmic verification for $\P$ and 
  	with polylogarithmic communication for $\NP$.
  	% Third, we generalize a scaling approach in Azar and Micali (STOC) ... [composition theorem]
  	
  	We then move to study the case where a delegator outsources multiple inputs.
  	First, we formalize an extended notion of rational proofs for this scenario (sequential composability) and we
  	show that existing schemes do not satisfy it. We show how these protocols incentivize workers
  	to provide many ``fast'' incorrect answers which allow them to solve more problems and collect more rewards.
  	We then design a $d$-rounds rational proof for sufficiently ``regular'' arithmetic circuit of depth $d = O(\log{n})$
  	with sublinear verification. We show, that under certain cost assumptions, our scheme is sequentially composable,
  	i.e. it can be used to delegate multiple inputs. We finally show that our scheme for space-bounded computations is also 
  	sequentially composable under certain cost assumptions.
  	
  	%The results above have been published as proceedings in GameSec as Campanelli and Gennaro (GameSec 2015)
  	%and Campanelli and Gennaro (GameSec 2017).
  	
	Finally we discuss partial results for delegation of functions computable by low-depth circuits of unbounded fan-in.
	Our techniques rely on \textit{fine-grained} cryptographic schemes whose security guarantees do not require the existence of one-way functions
	and hold only against bounded adversaries. As a stepping stone for our delegation schemes, we also build 
	the first fine-grained somewhat homomorphic encryption schemes which we believe to be of independent interest.
	Our fine-grained protocols can be used to obtain sequentially composable rational proofs.
  \end{abstract}
  
  

%\abstract{
%This is a proposal.
%}

%\tableofcontents

% List of commands
% Tau
\newcommand{\Tau}{\mathcal{T}}

\newcommand{\function}[1]{\ensuremath{\mathsf{#1}}}

% underbar
%\newcommand{\ubar}[1]{\underaccent{\bar}{#1}}
\newcommand{\ubar}[1]{\uline{#1}}

% Theorems
\newtheorem{definition}{Definition}
\newtheorem{lemma}{Lemma}
\newtheorem{claim}{Claim}
\newtheorem{theorem}{Theorem}
\newtheorem{corollary}{Corollary}

%\newcommand{\expectation}{\mathbb{E}}
\newcommand{\expectation}{\mathop{\mathbb{E}}}

% Some shortcuts for math symbols
\newcommand{\binstrings}{\{0, 1\}^{*}}
%\newcommand{\iff}{\Leftrightarrow}
\newcommand{\bit}{\{0, 1\}}
\newcommand{\funonstrings}{: \binstrings \to \binstrings}

        % Announced distribution
        \newcommand{\annd}{\hat d}
        % Real distribution
        \newcommand{\reald}{d}

\newcommand{\naturals}{\mathbb{N}}
\newcommand{\reals}{\mathbb{R}}
\newcommand{\field}{\mathbb{F}}

% Circuit Family
\newcommand{\circfam}{\{C_n\}_{n=1}^{\infty}}

% Interactive Proofs
\newcommand{\transc}{\mathcal(T)}

\makeatletter
\newcommand{\verbatimfont}[1]{\def\verbatim@font{#1}}%
\makeatother

\newcommand{\DOM}{\function{DOM}}
\newcommand{\F}{\mathbb{F}}


\newcommand{\allhonest}{\function{all\_honest}}
\newcommand{\true}{\function{true}}
\newcommand{\false}{\function{false}}

\newcommand{\Enc}{\function{Enc}}
\newcommand{\pk}{\function{pk}}
\newcommand{\sk}{\function{sk}}


%% Shortcuts specific to Composition
\newcommand{\protOne}{\pi^{f_2}_{1}}
\newcommand{\protTwo}{\pi_{2}}
\newcommand{\rewGap}{\Delta}
\newcommand{\STEP}{STEP}

\newcommand{\disTransc}{\tilde{\Tau}}






% Introduction/Motivation:
%\chapter{Introduction}

\begin{comment}
In this section I describe the motivation of the problem and of the design choices.

Problem: Efficient Delegation of Computation 
[The motivation for this is straightforward]

Design Principles and Points of Focus:
- (almost) "lack" of cryptography (in the FG case, for reasons of both efficiency and assumptions)
- adversarial assumptions:
-- rationality, or
-- limited resources (circuit depth)


\end{comment}

%\section{Synopsis}
% -- Motivating Example -- i.e. why would we care?
% Nowadays a lot of computational resources are available. Yet, not evenly so.
% XXX: More about the transition back to "mainframe"-like era through cloud computing

Imagine this situation.
You are a researcher in a biological lab, you have plenty of data and a clear scientific hypothesis you want to test on them. All you need to test such hypothesis is running a program you wrote. Problem: the computers in your lab will take weeks to run it (the bliss and the curse of big data). If you had a large number of powerful CPUs, a parallel version of the program you wrote would run in only a few hours --- right on time for you to submit your findings for the next conference deadline.

Cloud computing solves a part of the problem above. If you cannot \textit{physically} run a program (or store data) on a computer you own, have someone ``rent'' you a computer (or \textit{many} computers) on which to run it. Let (say) Amazon EC2 \footnote{Amazon Elastic Cloud: https://aws.amazon.com/ec2/} run your beautiful code for you and write your papers with the results it will return. So far, so good. But there is a new problem now, a problem that intuitively has to do with \textit{trust}. 
What guarantees do you have on whether they actually ran your program? What guarantees do you have on how reliable the execution was? What guarantees do you have on whether anybody or anything compromised the execution? At times little, or none. 

This new problem we are facing can be summarized as follows: how can we \textit{verify} that an untrusted computer executed a program correctly just as if our local machine\footnote{Assuming our local machine is \textit{reliable} and \textit{uncompromised}.} would have executed it? 
% -- Connection of the example to wider research --
This question is at the core of the field of \textit{Verifiable Computation} (VC). The goal of VC is, informally, to provide \textit{efficient} methods to verify that a computation we \textit{delegate} is executed ``correctly''. It is important to stress that these methods should be efficient. Part of the reason being a very simple one: we do not have the resources to run our program ourselves. When I receive the alleged results from EC2, I will have to verify them through my \textit{limited} local machine\footnote{This trivially excludes re-executing the program ourselves. But couldn't we just ask a third party to run our program again? Here the question is again: with what guarantees? Plus, how do we know the two parties are not colluding with one another?} 
VC is a practical problem that has extremely benefited from a theoretical analysis through the lenses and the tools of computational complexity and cryptography. The range and the scope of these results can roughly be located on two dimensions: efficiency and ``extent'' of guarantees. Efficiency can be described in terms of: the running times of the Delegator and the Worker\footnote{In our example, respectively the lab's computer and the cloud.} and the amount of communication required. The other dimension deals with what type of guarantees the methods provide and under what assumptions. This is related to a classical dichotomy in cryptography, e.g. is an encryption scheme provably secure against \textit{any} adversary (no matter their computational resources) or only against \textit{efficient} adversaries\footnote{Usually modeled as probabilist Turing machines (also denoted as PPTs, for Probabilistic Polynomial Time).} running in polynomial time? These two notions are respectively known as \textit{information-theoretic} (or \textit{unconditional}) security and \textit{computational} security. Analogously, a delegator which verifies a computation can obtain either type of security guarantee. 

% An intuition on rationality
\newthought{Our Model: Rational Workers}
In this dissertation proposal we will slightly diverge from the types of security guarantees outlined above. Recall that in cloud computing, computation is rented from far away servers. This suggests an alternative approach: assume that the worker is economically motivated and seeks to maximize a monetary reward. In other words, we assume that the worker is \textit{rational}. Notice that we are still not trusting the worker. However, we diverge from unconditional and computational security guarantees which model a worker's behavior respectively as an arbitrary computable function and an arbitrary function computable by a PPT. In fact we will assume an utility function $u: \bits \to \reals$ which takes as input the ``interaction'' between the worker and the delegator. One way of thinking of this utility function is as the payment (or \textit{reward}) that the worker obtains by carrying out the computation for the delegator. Once fixed $u$, the worker's behavior as an arbitrary computable function that maximizes this utility function in expectation.


Once we assume the cloud is rational, we can reduce our security guarantee to the following condition:
\begin{displayquote}
	Any cheating worker will obtain a lower utility (in expectation) than an honest worker.
\end{displayquote}
 where by ``honest'' worker we mean one that carries out the computation correctly.
Our goal now becomes to design a protocol that no worker \textit{is incentivized} to deviate from, i.e. a protocol no \textit{rational worker} will deviate from. 

% == General question we are addressing ==

% -- What are we going to study here specifically --
\newthought{Our Research Problem: Efficient Verifiable Computation}

% -- Why non cryptography
\newthought{A Design Constraint: Non-Cryptographic Protocols}

% == A framework: rational proofs ==
\subsection{Rational Proofs}

% -- Stress important features of rational proofs: low cc, rounds and verifier's time

% == Q1: Expressivity of Rational Proofs ==
\newthought{Research Question 1: Expressivity}

% -- Our results --

% == Q2: Multiple Delegations == 

% -- Our results --

% == Transition to FG ==

% -- A limitation of our approach: the inner state assumption --

% -- One way to get beyond that: FG schemes --

% == Fine-Grained Homomorphic encryption as a basic tool to get to delegation schemes ==

% == Results on non-interactive Delegation schemes ==



% -- Begin Older draft --
\begin{comment}
\section{Verifiable Computation}

% NOTE: One possibility for the ouline is: first go all about rationality (both motivation and research questions) and then about fine-grained computation (both motivation and research questions)

% TODO: Say somewhere about the perspective "trading efficiency and assumptions for adversaries' bounds"? (thus stated is particularly FG specific)

In verifiable computation, a party with limited resources (called \textit{verifier}) delegates the computation of a function $f$ on input $x$ to a more powerful party (called \textit{prover}.)

There is a large body of research on this problem. Such research mixes theory of computing and cryptography.
[\textbf{TODO}]

The research on this problem branches according to types of assumptions done on the potentially malicious prover:
unconditional schemes or cryptographic schemes. Unconditional schemes are secure against any adversary, regardless of their computational power. These schemes make no cryptographic assumptions (e.g. the existence of one-way functions). Cryptographic schemes may require general cryptographic assumptions such as FHE (e.g. \cite{ggp09} ) or more specific ones, such as generalizations of Diffie-Hellman or ``knowledge assumptions" \CN (e.g. \cite{qsp}). Such schemes are secure against any polynomial-time adversary.

In this proposal, we focus on the problem of obtaining expressive delegation schemes (i.e. for general classes of computations) 
with very efficient verifiers.  Pursuing this goal, we will make assumptions on the adversary.
% TODO: Say somewhere once and for all what you mean with cryptographic assumptions: i.e. one-way functions or specific hardness assumptions
We will describe two different types of schemes from two different types of assumptions. Both assumptions allow us to obtain verifiers that, in many circumstances, are more efficient than the ones obtained so far in the  \XXX unconditional and cryptographic research mentioned above.
% XXX: Maybe you want to talk about changes for the prover's cost, if any? (At a shallow analysis, it seems that we gain in that dimension too)
% XXX: As a consequence, you may rephrase your goal as obtaining "doubly very efficient proofs"

\section{Assumptions on Adversaries}
\subsection{Adversaries with Incentives: Rationality}

% TODO: Give names to subsubsections maybe?
% XXX: Maybe just replace sections with chapters?

% What do we mean by this assumption?
In this work, we shall use rationality in the decision theoretic sense:
adversaries are agents that will act maximizing a certain utility function.
We will assume that the verifier will reward the prover for its services;
the prover will make sure to obtain the maximum reward possible.
We will not make assume the verifier to act honestly.

% How to exploit this assumption?
We know a potentially untruthful prover will not act in any other way. How to leverage this?
Our goal is to make the verifier confident that the result of a computation is correct.
We will do that in the following way.
The prover and the verifier will interact with each other, with
the verifier ``challenging'' the prover.
The verifier will then use the prover's responses to (probabilistically) decide its reward.
Thus, the reward is a function of the transcript of the interaction between prover and verifier.
As protocol designers, we will make sure that (in expectation) the reward of the prover is maximized
for the honest prover, i.e. the prover that answers truthfully and follows the protocol.

The security notion sketched above is formalized under the notion of \textit{rational proofs} \CN.

% What do we gain by such an assumption?
Delegation schemes based on rational proofs are very \textit{simple}, \textit{extremely efficient}
and for that do not require any hardness or cryptographic assumption\footnote{Some work \CN combined cryptography and rational proofs. In such \textit{rational arguments} the prover is assumed to be rational \textit{and} having limited computational power. Their round and communication complexity can be significantly lower than their information-theoretic counterparts. See \ref{sec:related-work} \XXX for more details. }. 
In particular, they achieve low communication and number of rounds and a \textit{strongly sublinear} (polylogarihtmic) verification time. In fact, the verifier --- assumed to have oracle access to the input --- could need sample as few as $O(1)$ input bits to compute the prover's reward.


% When can we afford such an assumption?
% pay-per-compute
Why can we afford to assume that our adversaries are rational? 
One typical context when this may be reasonable is in the context of cloud computing and pay-per-computation. 
% XXX: Say why it's intuitive that such parties may not be malicious, but simply negligent.
% XXX: Also mention in relation to this how "the cost of cheating" is then important and how it should be modeled.
In this case, rewards are directly related to the sums of money exchanged for the computation. 
Notice that right now, several commercial cloud computing services \XXX (Amazon, Microsoft Azure, Google, IBM?) follow
an approach based on "amount of time and resources". The concept of "being payed more or less according to the verification of the computation" is not a common payment model. \XXX
% Q: A natural question is, where is verifiable computation used currently?

% Volunteer computing
Another context is that of volunteer computing. \XXX (Examples?)
Here the reward for workers (provers) is some form of score. In turn, this score can at times (\XXX When?) exchanged
monetarily or provide someone with status/ranking. In this context we can assume that the users of the system are either honest or interested in exploiting the system for their own return.
% TODO: Talk about gridcoin and Proof-of-BOINC

% NOTE: Here is a question for Itai: how does one find out if these application domains satisfy the requirements?

% Cryptocurrencies (TODO)
A final potential application domain is that of cryptocurrencies \CN. A description of cryptocurrencies is out of the scope of this document. We will assume the reader is familiar with the basics of blockchains\footnote{See \CN for further details.}. In the following observations we will assume the existence of \textit{publicly verifiable rational proofs/arguments}, i.e. rational protocol that can be verified by any party, not necessarily by the verifier for which the proof is intended. 
% Explain connections between how transactions are verified and public verifiability

% Applications to mining
A first application is mining. A miner is usually rewarded a fee after completing two tasks: verifying that the transactions in a new block are legitimate and completing a proof of work, usually in the form of finding a hash preimage of a string with a specific prefix. Other miners in turn invest computational resources verifying that these two tasks were accomplished correctly. This can be computationally intensive. If the checks miners usually carry out admitted non-interactive publicly verifiable rational proofs, they could simply verify those. This could lead  enormous savings in computational resources. Utility-maximizing blockchain users who are trying to arrive first mining a new block 
Some further observations are in place. The approach alone would not protect against malicious miners whose goal is to disrupt the network. It is unclear how the incentives of miners would change as a consequence. Incentives to accurate verification in the blockchain may benefit from this approach, as it may solve the \textit{verifier's dilemma}(\cite{luu2015demystifying}). The verifier's dilemma is the observation that miners have incentive to skip the verification of expensive transactions to gain a competitive advantage in the race for the next block. We leave more sophisticated security and game-theoretic analysis that would shed light on this as an open problem.

% Applications to verifiable computation
% TODO: We could do unexpensive (computationally)


\subsection{Adversaries with Limited Parallel Time}


% What do we mean by this assumption?

% How to exploit this assumption?

% What do we gain by such an assumption?

% When can we afford such an assumption?


\section{Problem Statement}

\subsection{Rational Proofs: problem statement}

\subsection{Fine-Grained DoC: problem statement}

% TODO: Comparison between the "more efficient" schemes you get with your FG stuff and with what already exists
\end{comment}
% -- End Older draft --

\begin{comment}


\section{Related Work}
TODO...

\subsection{Work on Verifiable Computation}

% Different techniques here...

\subsection{Work on Fine-Grained Cryptography}

\subsection{Rationality and Cryptography}



\section{Organization}
\textbf{TODO...}
\end{comment}


% Rational Proofs: Preliminaries
%\chapter{Preliminaries and  Definitions}

%\input{rp-background.tex}


% Rational Proofs: Rational Proofs for Space-Bounded Computations
%\chapter{Rational Proofs for Space Bounded Computation}

%In this chapter \XXX

\section{Bounded-Space Computation}
\label{sec:protocol}
We are now ready to present our protocol. It uses the notion of a Turing Machine {\em configuration,} i.e. 
%The main idea it exploits is the concept of \emph{configuration graph} which we 
%will present here only informally\footnote{The interested reader may consult 
%an introductory textbook on computational complexity such as 
%\cite{arora2009computational}.}. A configuration of a Turing Machine is a 
the complete description of the current state of the computation: for a machine $M$, its state, the position of its heads, the non-blank values on its tapes.  

Let $L \in \DTISP(T(n),S(n))$ and $M$ be the deterministic TM that recognizes $L$. 
On input $x$, let $\gamma_1,\ldots,\gamma_N$ (where $N=T(|x|)$) be the 
configurations that $M$ goes through during the computation on input $x$, where 
$\gamma_{i+1}$ is reached from $\gamma_i$ according to the transition function of $M$. Note, first of all, that each configuration has size $O(S(n))$. Also if $x \in L$ (resp. $x \notin L$) then $\gamma_N$ is an accepting (resp. rejecting) configuration. 


The protocol presented below is a more general version of the one used in \cite{cg15} and described above. 
% and very simple in structure. 
The prover shows the claimed final configuration $\hat{\gamma}_N$ 
and then prover and 
verifier engage in a "chasing game", where the prover "commits" at each step to an intermediate configuration. If the prover is cheating (i.e. $\hat{\gamma}_N$ is wrong) then the intermediate configuration either does not follow from the initial configuration or does not lead to the final claimed configuration. At each step and after $P$ communicates the intermediate configuration $\gamma'$, the verifier then randomly chooses whether to continue invoking the protocol on the left or the right of $\gamma'$. The protocol terminates when $V$ ends up on two previously declared adjacent configurations that he can check.  Intuitively, the protocol works since, if $\hat{\gamma}_N$ is wrong, for any possible sequence of the prover's messages, there is at least one choice of random coins that allows $V$ to detect it; the space of such choices is polynomial in size.

We assume that $V$ has oracle access to the input $x$.
%\clearpage
\noindent What follows is a formal description of the protocol.
\begin{framed}
\begin{enumerate}
    \item $P$ sends to $V$:
    \begin{itemize}
    \item $\gamma_{N}$, the final accepting configuration (the starting configuration, $\gamma_1$, is known to the verifier);
    \item $N$, the number of steps between the two configurations. % Does it need to? sort of yes. What if the guy lies. It should be discussed possibly.
    \end{itemize}
    \item Then $V$ invokes the procedure $\PathCheck(N, \gamma_{1}, \gamma_{N})$.
\end{enumerate}
\end{framed}

\medskip
\noindent The procedure $\PathCheck(m,\gamma_l, \gamma_r)$ is defined for $1 \leq m \leq N$ as 
follows:
\begin{framed}
\begin{itemize}
    \item If $m > 1$, then:
    \begin{enumerate}
        \item $P$ sends intermediate configurations $\gamma_{p}$ and $\gamma_q$ (which may coincide) where $p = \lfloor \frac{l+m-1}{2} \rfloor$  and 
        $q = \lceil \frac{l+m-1}{2} \rceil$. % (m == r-l+1)
        \item If $p \neq q$, $V$ checks whether there is a transition leading from configuration $\gamma_p$ to configuration $\gamma_q$. If yes, $V$ accepts; otherwise $V$ halts and rejects.
	\item $V$ generates a random bit $b \in_R \bit$
        \item If  $b = 0$ then the protocol continues invoking $\PathCheck(\lfloor \frac{m}{2} \rfloor, \gamma_l, \gamma_p)$; If $b = 1$ the protocol continues invoking $\PathCheck(\lfloor \frac{m}{2} \rfloor, \gamma_q, \gamma_r)$
    \end{enumerate}
    \item If $m = 1$, then $V$ checks whether there is a transition leading from configuration $\gamma_l$ to configuration $\gamma_r$. If $l=1$, $V$ checks that $\gamma_l$ is indeed the initial configuration $\gamma_1$. If $r=N$, $V$ checks that $\gamma_r$ is indeed the final configuration sent by $P$ at the beginning. If yes, $V$ accepts; otherwise $V$ rejects.
\end{itemize}
\end{framed}

\medskip

\begin{theorem}
\label{thm:main}
$\DTISP[\poly(n), S(n)] \subseteq \DRMA[O(\log n), O(S(n)\log n), O(S(n)\log n)]$
\end{theorem}
\begin{proof}
% Efficiency
Let us consider the efficiency of the protocol above.
It requires $O(\log n)$ rounds.
Since the computation is in $\DTISP[\poly(n), S(n)]$, the configurations $P$ sends to $V$ at each round have size $O(S(n))$.
The verifier only needs to read the configurations and, at the last round, check the existence of a transition leading from $\gamma_l$ to $\gamma_r$. Therefore the total running time for $V$ is $O(S(n) \log n)$.

% Soundness
Let us now prove that this is a rational proof with noticeable reward gap.
%by showing the protocol satisfies the hypothesis of Lemma \ref{lemma:ip2rp}. 
Observe that the protocol has perfect completeness. 
Let us now prove that the soundness is at most $1 - 2^{-\log N} = 1 - \frac{1}{O(\poly(n))}$.
We aim at proving that, if there is no path between the configurations $\gamma_1$ and $\gamma_N$ then $V$ rejects with probability at least $2^{-\log N}$.
Assume, for sake of simplicity, that $N = 2^k$ for some $k$. We will proceed by induction on $k$. If $k=1$, $P$ provides the only intermediate configuration $\gamma'$ between $\gamma_1$ and $\gamma_N$. At this point $V$ flips a coin and the protocol will terminate after testing whether there exists a transition between $\gamma_1$ and $\gamma'$ or between $\gamma'$ and $\gamma_N$. Since we assume the input is not in the language, there exists at most one of such transitions and $V$ will detect this with probability $1/2$.

Now assume $k > 1$. At the first step of the protocol $P$ provides an intermediate configuration $\gamma'$. Either there is no path between $\gamma_1$ and $\gamma'$ or there is no path between $\gamma'$ and $\gamma_N$. Say it is the former: the protocol will proceed on the left with probability $1/2$ and then $V$ will detect $P$ cheating with probability $2^{-k+1}$ by induction hypothesis, which concludes the proof.

\end{proof}

\medskip
\noindent
The theorem above implies the results below. 
%In Corollary \ref{cor:L-NL} we use the fact $\NL = \coNL$.
%To the best of our knowledge the class $\NSC$ is still not known to be closed under 
%complement \cite{barrington1991oracle}.Therefore we are able to obtain only one-
%sided rational proofs for it.
\begin{corollary}
	\label{cor:L-NL}
$ \L \subseteq \DRMA[O(\log n), O(\log^2 n ), O(\log^2 n )]$
\end{corollary}

This improves over the construction of rational proofs for $\L$ in \cite{ratsumchecks} due to the better round complexity. 

\begin{corollary}
	\label{cor:SC}
$ \SC \subseteq \DRMA[O(\log n), O(\polylog(n)), O(\polylog(n))]$
\end{corollary}

No known result was known for $\SC$ before. 


\subsection{Rational Proofs for Randomized Bounded Space  Computation}
\label{sec:rand-space}

We now describe a variation of the above protocol, for the case of randomized bounded space computations. 

Let $\BPTISP[t,s]$ denote the class of languages recognized by randomized machines using time $t$ and space $s$ with error bounded by $1/3$ on both sides. 
In other words, $L \in \BPTISP[\poly(n), S(n)]$ if there exists a (deterministic) Turing Machine $M$ such that
for any $x \in \bits$ $\Pr_{r \in_R \bit^{\rnd(|x|)}}[M(x, r) = L(x)] \geq \frac{2}{3}$ and that runs in $S(|x|)$ space and polynomial time.
Let $\rho(n)$ be the maximum number of random bits used by $M$ for input $x \in \bit^n$; $\rho(\cdot)$ is clearly polynomial.

We can bring down the $2/3$ probability error to $\negl(n)$
by constructing a machine $M'$. $M'$ would simulate the $M$ on $x$ iterating the simulation $m = \poly(|x|)$ times
using fresh random bits at each execution and taking the majority output of $M(x;\cdot)$.
The machine $M'$ uses $m\rho(|x|)$ random bits and runs in polynomial time and $S(|x|) + O(\log(n))$ space.

The work in \cite{nisan1992pseudorandom} introduces pseudo-random generators (PRG) resistant against space bounded adversaries.
An implication of this result is that any randomized Turing Machine $M_1$ running in time $T$ and space $S$ can be simulated by a
 randomized Turing Machine $M_2$ running in time $O(T)$, space $O(S \log(T))$ and using only $O(S \log(T))$ random bits\footnote{We point out that the new machine $M_2$ introduces a small error. For our specific case this error keeps the overall error probability negligible and we can ignore it.}
(see in particular Theorem 3 in \cite{nisan1992pseudorandom}).
Let $L \in  \BPTISP[(\poly(n), S(n)]$ and $M'$ defined as above. We denote by $\hat{M}$ the simulation of $M'$ that uses Nisan's result described above.

By using the properties of the new machine $\hat{M}$, we can directly construct rational proofs for $\BPTISP(\poly(n), S(n))$.
We let the verifier picks a random string $r$ (of length $O(S \log(T))$) and sends it to the prover. They then invoke a rational
proof for the computation $\hat{M}(x;r)$.

By the observations above and Theorem \ref{thm:main} we have the following result:
\begin{corollary}
	$\BPTISP[\poly(n), S(n)] \subseteq \DRMA[\log(n), S(n) \log^2(n), S(n) \log^2(n)]$
\end{corollary}

We note that for this protocol, we need to allow for non-perfect completeness in the definition of $\DRMA$ in order to allow for the probability that the verifier chooses a bad random string $r$. 

%\begin{corollary}
%	\label{cor:space-bounded}
%	$ \NSC \subseteq \osDRMA[O(\log n), O(\polylog(n)), O(\polylog(n))]$
%\end{corollary}



\section{A Composition Theorem for Rational Proofs}
In this Section we prove a relatively simple {\em composition theorem} that states that while proving the value of a function $f$, we can replace oracle access to a function $g$, with a rational proof for $g$. The technically interesting part of the proof is to make sure that the {\em total} reward of the prover is maximized when the result of the computation of $f$ is correct. In other words, while we know that lying in the computation of $g$ will not be a rational strategy for just that computation, it may turn out to be the best strategy as it might increase the reward of an incorrect computation of $f$. A similar issue (arising in a particular rational proof for depth $d$ circuits) was discussed in \cite{am1}: our proof generalizes their technique. 


\begin{definition}
	\label{def:oracle-RP}
	We say that a rational proof $(P,V, \rew)$ for $f$ is a $g$-oracle rational proof if $V$ has oracle access to the function $g$ and carries out at most one oracle query.
	We allow the function $g$ to depend on the specific input $x$.
\end{definition}

% TODO: Put example here for definition above. (Maybe the PCP-like RP by Azar/Micali)


\newcommand{\rewf}{\rew^o_f}
\newcommand{\rewg}{\rew_g}
% TODO: Remove all the o-s in the notation
\begin{theorem}
	\label{thm:composition}
	Assume there exists a $g$-oracle rational proof $(P^o_f, V^o_f, \rewf)$ for $f$ with noticeable reward gap  and with round, communication and verification complexity respectively $r_f, c_f$ and $T_f$. Let $t_I$ the time necessary to invoke the oracle for $g$ and to read its output.Assume there exists a rational proof $(P_g, V_g, \rewg)$ with noticeable reward gap for $g$ with round, communication and verification complexity respectively $r_g, c_g$ and $T_g$. Then there exists a (non $g$-oracle) rational proof with noticeable reward gap for $f$ with round, communication and verification complexity respectively $r_f + 1 + r_g , c_f + t_I + c_g$ and $T_f - t_i + T_g$.
\end{theorem}

Before we embark on the proof of Theorem~\ref{thm:composition}, we prove a 
technical Lemma. 
The definition of rational proof requires that the expected reward of the honest prover is not lower than the expected reward of any other prover.
The following intuitive lemma states we necessarily obtain this property if an honest prover has a polynomial expected gain in comparison to provers
that \textit{always} provide a wrong output.

\begin{lemma}
	\label{lemma:noticeable-gap-implies-rp}
	Let $(P,V)$ be a protocol and $\rew$ a reward function as in Definition \ref{def:RP}.
	Let $f$ be a function s.t. $\forall x \Pr[\out(P,V)(x)] = 1$.
	Let $\rewGap$ be the corresponding reward gap w.r.t. the honest prover $P$ and $f$.
	If $\rewGap > \invPoly$ then $(P,V,\rew)$ is a rational proof for $f$ and admits noticeable reward gap.
\end{lemma}
\begin{proof}
	Assume w.l.o.g that for all $P' \not = P$ and such that $\forall x \Pr[\out(P',V)(x)] = 1$ it holds that
	$\expectation[\rew(P,V)(x)] \geq \expectation[\rew(P',V)(x)]$.
	
	Fix $x$. Let $\disP$ be an arbitrary prover, $R = \expectation[\rew(P,V)(x)] $, $\claimedy = \out(\disP, V)(x)$, $\disR = \expectation[\rew(\disP,V)(x)]$, $\disR_{corr} = \expectation[\rew(\disP,V)(x) | \claimedy = f(x)]$, $\disR_{err} = \expectation[\rew(\disP,V)(x) | \claimedy \not = f(x)]$. Then:
	
	\begin{align}
	& R - \disR & =  \\
	& R - \Pr[\claimedy = f(x)]\disR_{corr} - \Pr[\claimedy \not = f(x)] & = \\
	& \Pr[\claimedy = f(x)](R-\disR_{corr}) + \Pr[\claimedy \not = f(x)](R-\disR_{err}) & \geq \\
	& \Pr[\claimedy \not = f(x)](R-\disR_{err}) & \geq \\
	& \Pr[\claimedy \not = f(x)]\rewGap & > \\
	& 0 
	\end{align}
	
	The inequality above shows that $(P,V, \rew)$ is a rational proof for $f$.
	By the hypothesis on $\rewGap$ this protocol already admits a noticeable reward gap.
\end{proof}

% TODO: Clean up the proof above. Specify it has polynomial budget. Make it homogeneous with the definitions given in preliminaries. Q: Why does the def. of DRMA not mention poly budget?
Now we can start the proof of Theorem~\ref{thm:composition}.
\begin{proof}
% Introduce reward functions

Let $\rewf$ and $\rewg$ be the reward functions of the $g$-oracle rational proof for $f$ and the rational proof for $g$ respectively.
% construct new reward functions and protocol
We now construct a new verifier $V$ for $f$. This verifier runs exactly like the $g$-oracle verifier for $f$ except that every oracle query to $g$ is now replaced with an invocation of the rational proof for $g$.
The new reward function $\rew$ is defined as follows:
$$ \rew(\Tau) = \delta \rewf(\Tau^o_f \circ y_g)  + \rewg(\Tau_g)$$
where $\Tau$ is the complete transcript of the new rational proof, $\Tau^o_f$ is the transcript of the oracle rational proof for $f$,  $\Tau_g$ and $y_g$ are respectively the transcript and  the output of the rational proof for $g$. Finally $\delta$ is multiplicative factor in $(0,1])$. The intuition behind this formula is to "discount" the part of the reward from $f$ so that the prover is incentivized to provide the true answer for $g$. In turn, since $\rewf$ rewards the honest prover more when  the verifier has the right answer for a query to $g$ (by hypothesis), this entails that the whole protocol is rational proof for $f$.

% Say that, by the previous lemma, you just need to show that the resulting protocol has 
To prove the theorem we will use Lemma \ref{lemma:noticeable-gap-implies-rp} and it will suffice to prove that the new protocol has a noticeable reward gap.
\newcommand{\pg}{p_{g}}
\newcommand{\Rf}{R_f^o}
\newcommand{\disRf}{\tilde{R}_f^o}
\newcommand{\disRfGoodg}{\tilde{R}_f^{o,\text{good}(g)}}
\newcommand{\disRfWrongg}{\tilde{R}_f^{o,\text{wrong}(g)}}
\newcommand{\disRgGoodg}{\tilde{R}_g^{\text{good}(g)}}
\newcommand{\disRgWrongg}{\tilde{R}_g^{\text{wrong}(g)}}
\newcommand{\Rfmax}{b^o_f(n)}


Consider a prover $\disP$ that always answer incorrectly on the output of $f$. 
Let $\pg$ be the probability that the prover outputs a correct $y_g$. % TODO: Introduce variables
Then the difference between the expected reward of the honest prover and $\disP$ is:
\setcounter{equation}{0}
\begin{align}
 \delta(\Rf - \disRf) + (R_g - \disR_g)  = \\
 \delta(\Rf - \pg \disRfGoodg - (1-\pg)\disRfWrongg) + \nonumber  \\ (R_g - \pg \disRgGoodg - (1-\pg)\disRgWrongg)  = \\
 \delta(\pg(\Rf - \disRfGoodg) + (1-\pg)(R_f - \disRfWrongg)) \nonumber \\ + \pg(R_g - \disRgGoodg) + (1-\pg)(R_g-\disRgWrongg)  > \\
 \delta(\pg \rewGap^o_f + (1-\pg)(-\Rfmax)) + 0 + (1-\pg)\rewGap_g  = \\
 \pg\delta\rewGap^o_f + (1-\pg)(\rewGap_g - \delta \Rfmax)  \geq \\
 \min\{\delta \rewGap^o_f, \rewGap_g - \delta\Rfmax \}  > \\
\invPoly
\end{align}
Where the last inequality holds for $\delta = \frac{\rewGap_g}{2\Rfmax}$.

The round, communication and verification complexity of the construction is given by the sum of the respective complexities from the two rational proofs modulo minor adjustments. These adjustments account for the additional round by which the verifier communicates to the prover the requested instance for $g$. 

\begin{comment}
One final note: in the proof above we replaced the oracle query to $g$ with an invocation of the rational proof for it. In certain circumstances (see for example the proof of Theorem \ref{thm:crhf-p}), it would simplify a proof to assume that some of the messages of the rational proof for $g$ are sent before the protocol for $f$ is invoked.
The analysis above still holds if extending the rational proof for $f$ with such "preprocessing" messages still yields a $g$-oracle rational proof for $f$.
\end{comment}
\end{proof}

The theorem above can be used as design tool of rational proofs for a function $f$: first build a rational proof assuming the verifier has oracle access to a function $g$, then build a rational proof for $g$. This automatically provides a complete rational proof for $f$.

\begin{remark}
	Theorem \ref{thm:composition} assumes that verifier in the oracle rational proof for $f$ carries out a single oracle query. Notice however that the proof of the theorem can be generalized to any verifier carrying out a constant number of adaptive oracle queries, possibly all for distinct functions.
	This can be done by iteratively applying the theorem to a sequence of $m = O(1)$ oracle rational proofs for functions $f_1,...,f_m$ where the $i$-th rational proof is $f_{i+1}$-oracle for $1 \leq i < m$.
\end{remark}
% TODO: Is it possible to rewrite this in a  better way?


% Rational Proofs:  Sequential Composability
%\chapter{Sequentially Composable Rational Proofs}

%
\section{Profit vs. Reward}

Let us now define the {\sf profit} of the Prover as the difference between the reward paid by the verifier and the cost incurred by the Prover to compute $f$ and engage in the protocol. 
As already pointed out in \cite{am1,ratargs} the definition of Rational Proof is sufficiently robust to also maximize the {\sf profit} of the honest prover and not the reward. Indeed consider the case of a "lazy" prover $\disP$ that does not evaluate the function: even if 
$\disP$ collects a "small" reward, his total profit might still be higher than the profit of the honest prover $P$. 

Set $R(x)=\expRewProtHon$, $\tilde{R}(x)=\expRewProtDis$ and $C(x)$ [resp. $\tilde{C}(x)$] the cost for $P$ [resp. $\disP$] to engage in the protocol. Then we want
\[ R(x)-C(x) \geq \tilde{R}(x)-\tilde{C}(x) \; \Longrightarrow \; 
\delta_{\disP}(x) \geq C(x) -\tilde{C}(x) \]
In general this is not true (see for example the previous protocol), but it is always possible to change the reward by a multiplier $M$. Note that if $M \geq C(x)/\delta_{\disP}(x)$ then we have that 
\[ M(R(x) - \tilde{R}(x)) \geq C(x) \geq C(x) - \tilde{C}(x) \]
as desired. Therefore by using the multiplier $M$ in the reward, the honest prover 
$P$ maximizes its profit against all provers $\disP$ except those for which $\delta_{\disP}(x) \leq C(x)/M$, i.e. those who report the incorrect result with a "small" probability $\epsilon_{\disP}(x) \leq \frac{C(x)}{M \Delta(x)}$. 

We note that $M$ might be bounded from above, by budget considerations (i.e. the need to keep the total reward $MR(x) \leq B$ for some budget $B$). This point out to the importance of a large reward gap $\Delta(x)$ since the larger $\Delta(x)$ is, the smaller the probability of a cheating prover $\disP$ to report an incorrect result must be, in order for $\disP$ to achieve an higher profit than $P$. 

\smallskip
\noindent
{\sc Example.} In the above protocol we can assume that the cost of the honest prover is $C(x)=n$, and we know that $\Delta(x)=n^2$. Therefore the profit of the honest prover is maximized against all the provers that report an incorrect result with probability larger than $n^3/M$, which can be made sufficiently small by choosing the appropriate multiplier. 


\begin{remark}
\label{rem:asy}
{\em If we are interested in an asymptotic treatment, it is important to notice that as long as $\Delta(x) \geq 1/{\sf poly}(|x|)$ then it is possible to keep a polynomial reward budget, and maximize the honest prover profit against all provers who cheat with a substantial probability $\epsilon_{\disP} \geq 1/{\sf poly'}(|x|)$.}
\end{remark}

\section{Sequential Composition}
We now present the main results of our work. First we informally describe our notion of sequential composition of 
rational proof, via a motivating example and show that the protocols in \cite{am,am1,rosen} do not satisfy it. Then we present our definition of sequential rational proofs, and a protocol that achieves it for circuits of bounded depth. 


\subsection{Motivating Example}
Consider the protocol in the previous section for the computation of the function 
$G_{n,k}(\cdot)$. Assume that the honest execution of the protocol (including the computation of $G_{n,k}(\cdot)$) has cost $C=n$. 

Assume now that we are given a sequence of $n$ inputs 
$x^{(1)},\ldots,x^{(i)},\ldots$ where each $x^{(i)}$ is an $n$-bit string. 
In the following let $m_i$ be the Hamming weight of $x^{(i)}$ and $p_i=m_i/n$.

Therefore the honest prover investing $C=n$ cost, will be able to execute the protocol only only once, say on input $x^{(i})$.  By setting $p=\tilde{p}=p_i$ in Eq.~\ref{eq:bsr}, we see that $P$ obtains reward 
\[ R(x^{(i)}) = 2(p_i^2-p_i+1) \leq 2 \]
Consider instead a prover $\disP$ which in the execution of the protocol outputs a random value $\tilde{m} \in [0..n]$. The expected reward of $\disP$ on {\sf any} input $x^{(i)}$ is (by setting $p=p_i$ and $\tilde{p}=m/n$ in Eq.~\ref{eq:bsr} and taking 
expectations):
\begin{align*}
\tilde{R}(x^{(i)}) &= \expectation_{m,b}[BSR(\frac{m}{n}, b)] \\
                &= \frac{1}{n+1}\sum_{m=0}^{n}\expectation_b[BSR(\frac{m}{n},b] \\
                &= 
                \frac{1}{n+1}\sum_{m=0}^{n}(2(2p_i \cdot \frac{m}{n}-\frac{m^2}{n^2}-p_i+1))
                 \\
                 &=2-\frac{2n+1}{3n} > 1 \; \mbox{ for $n>1$.}
\end{align*}
Therefore by "solving" just two computations $\disP$ earns more than $P$. Moreover t
the strategy of $\disP$ has cost $1$ and 
therefore it earns more than $P$ by investing a lot less cost\footnote{
If we think of cost as time, then in the same time interval in which $P$ solves one problem, $\disP$ can solve up to $n$ problems, earning a lot 
more money, by answering fast and incorrectly.}.

Note that "scaling" the reward by a multiplier $M$ does not help in this case, since both the honest and dishonest prover's rewards would be multiplied by 
the same multipliers, without any effect on the above scenario. 

We have therefore shown a rational strategy, where cheating many times and collecting many rewards is more profitable than collecting a single reward for an honest computation. 


\subsection{Sequentially Composable Rational Proofs}

The above counterexample motivates the following Definition which formalizes 
that the reward of the honest prover $P$ must always be larger than the total 
reward of any prover $\disP$ that invests less computation cost than $P$. 

Technically this is not trivial to do, since it is not possible to claim the above for {\em any} prover $\disP$ and {\em any} sequence of inputs, because it
is possible that for a given input $\tilde{x}$, the prover $\disP$ has "hardwired" the correct value $\tilde{y}=f(\tilde{x})$ and can compute it without investing 
any work. We therefore propose a definition that holds for inputs randomly chosen according to a given probability distribution $\cal D$, and we allow for
the possibility that the reward of a dishonest prover can be "negligibly" larger than the reward of the honest prover (for example if $\disP$ is lucky and such 
"hardwired" inputs are selected by $\cal D$).

\noindent
\begin{definition}[Sequential Rational Proof]
	\label{def:SRP}
	A rational proof $(P,V)$ for a function $f:$ $\bit^n$ $\to$ 
	$\bit^n$ is $(\epsilon, K)$-{\sf sequentially composable} for an input distribution $\cal D$, if for every prover $\disP$, 
	for a sequence of inputs 
	$x,x_1,\ldots,x_k$ drawn according to ${\cal D}$ such that $C(x) \geq \sum_{i=1}^k 
	\tilde{C}(x_i)$ and $k \leq K$ we have that $\sum_{i}\tilde{R}(x_i) - R \leq \epsilon$.
\end{definition}

% Some properties of Sequential Rational Proofs
\noindent
A few sufficient conditions for sequential composability follow.

\begin{lemma}
\label{lemma:cost-rew-ratios}
% Rew ratio ineq. implies SRP
Let $(P,V)$ be a rational proof.
If for every input $x$  it holds that $R(x)=R$ and  $C(x)=C$ for constants 
$R \mbox{ and } C$, and the 
following inequality holds for every 
$\disP\neq 
P$ and input $x\in {\cal D}$:
\[ \frac{\tilde{R}(x)}{R} \leq \frac{\tilde{C}(x)}{C} + \epsilon\]
then $(P,V)$ is $(KR\epsilon, K)$-sequentially composable for $\cal D$
\end{lemma}
\begin{proof}
It suffices to observe that, for any $k$ inputs $x_1,...,x_k$, the inequality 
above 
implies
\begin{align*}
\sum_{i=1}^{k}\tilde{R}(x_i)  \leq R [\sum_{i=1}^{k} (\frac{\tilde{C}(x_i)}{C}  + \epsilon) ] 
\leq R +kR\epsilon
\end{align*}
where the last inequality holds whenever $\sum_{i=1}^{k} \tilde{C}(x_i) \leq C$ 
as 
in Definition \ref{def:SRP}.
\end{proof}

\begin{corollary}
	\label{cor:prob}
	Let $(P,V)$ and $\rew$ be respectively an interactive proof and a reward 
	function as in 
	Definition \ref{def:RP}; if $\rew$ can only assume the values $0$ and $R$ for 
	some constant $R$, let $\pDisR = \Pr[\rew((\disP,V)(x)) = R]$. If for $x \in {\cal D}$
	$$  \pDisR \leq \frac{\tilde{C}(x)}{C} + \epsilon $$
	then $(P,V)$ is $(KR\epsilon, K)$-sequentially composable for $\cal D$. 
	% If rew can be only R or 0 then the sufficient condition is on the probability
\end{corollary}
\begin{proof}
Observe that $\tilde{R}(x) = \pDisR\cdot R$ and then apply Lemma 
\ref{lemma:cost-rew-ratios}.
\end{proof}




\subsection{Sequential Rational Proofs in the PCP model}
\begin{comment}
In this subsection we:
- describe PCP's protocol by AM
- describe certain assumptions on cost
- we prove that it is a Sequential Rational Proofs
\end{comment}

We now describe a rational proof appeared in \cite{am1} and prove that is sequentially composable.
The protocol assumes the existence of a trusted memory storage to which both 
Prover and Verifier have access, to realize the so-called "PCP" (Probabilistically Checkable Proof) model. In this model, the Prover 
writes a very long proof of correctness, that the verifier checks only in a few randomly selected position. The trusted memory is needed to 
make sure that the prover is "committed" to the proof before the verifier starts querying it. 
% say that it is for logical circuits
% say about the abuse of notation in the theorem

The following protocol for proofs on a binary logical circuit $\cal C$ 
appeared  in  \cite{am1}. The Prover writes all the (alleged) values $\alpha_w$ for every wire in $w \in {\cal C}$, 
on the trusted memory location. The Verifier 
samples a single randome gate value to check its correctness and determines the reward accordingly:
\begin{enumerate}
\item The Prover writes the vector $\{ \alpha_w\}_{w \in {\cal C}}$
\item The Verifier samples a random gate $g \in \cal C$.
\begin{itemize}
\item  The Verifier reads $\alpha_{g_{out}}, \alpha_{g_L}, \alpha_{g_R}$, with $g_{out}, g_L, g_R$ being respectively the output, left and right 
input wires of $g$; the verifier checks that $\alpha_{g_{out}} = g(\alpha_{g_L}, 
\alpha_{g_R})$;
\item If $g$ in an input gate the Verifier also checks that  $\alpha_{g_L}, \alpha_{g_R}$ correspond to the correct input values;
\end{itemize}
The Verifier pays $R$ if both checks are satisfied, otherwise it pays $0$.
\end{enumerate}

\begin{theorem}[\cite{am1}]
The protocol above is a rational proof for any boolean function in $
P^{||NP}$, the class of all languages decidable by a polynomial time machine 
that can make non-adaptive queries to $NP$.
\end{theorem}

We will now show  a cost model where the rational proof above is sequentially 
composable.
We will assume that the cost for any prover is given by the number of gates he 
writes. Thus, for any input 
$x$, the costs for honest and dishonest provers are 
respectively $C(x) = S$, where $S = |\cal C|$, and $\tilde{C}(x) = \tilde{s}$ 
where $\tilde s$ is the number of gates written by the dishonest prover.
Observe that in this model a dishonest prover may not write all the $S$ 
gates, and that not all of the $\tilde{s}$ gates have to be correct. Let $\sigma \leq \tilde{s}$ the number of correct gates written by $\disP$. Then 

\begin{theorem}
In the cost model above the PCP protocol protocol in \cite{am1} is sequentially 
composable.
\end{theorem}
\begin{proof}
Observe that the probability $\pDisR$ that $\disP\neq P$ earns $R$ is such that
$$ \pDisR = \frac{\sigma}{S} \leq \frac{\tilde s}{S} = \frac{\tilde C}{C} $$
Applying Corollary \ref{cor:prob} completes the proof.
\end{proof}

The above cost model, basically says that the cost of writing down a gate dominates everything else, in particular the cost of {\em computing} that gate. 
In other cost models a proof of sequential composition may not 
be as straightforward. Assume, for example, that the honest prover pays $\$1$ to
compute the value of a single gate while writing down that gate is "free". Now $\pDisR$ is still equal to $\frac{\sigma}{S}$ but to 
prove that this is smaller than $\frac{\tilde C}{C}$ we need some additional assumption that limits the ability for $\disP$ to "guess" the 
right value of a gate without computing it (which we will discuss in the next Section). 




\subsection{Sequential Composition and the Unique Inner State Assumption}
\label{sec:uisa}
Definition~\ref{def:SRP} for sequential rational proofs requires a relationship between the reward earned by the prover and the amount of "work" the prover invested to produce that result. The intuition is that to produce the correct result, the prover must run the computation and incur its full cost. Unfortunately 
this intuition is difficult, if not downright impossible, to formalize. Indeed for a specific input $x$ a "dishonest" prover $\disP$ could have the correct 
$y=f(x)$ value "hardwired" and could answer correctly without having to perform any computation at all. Similarly, for certain inputs $x,x'$ and a certain 
function $f$, a prover $\disP$ after computing $y=f(x)$ might be able to "recycle" some of the computation effort (by saving some state) and compute 
$y'=f(x')$ incurring a much smaller cost than computing it from scratch. 

A way to circumvent this problem was suggested in \cite{b08} under the name of {\em Unique Inner State Assumption}: the idea is to assume a distribution 
$\cal D$ over the input space. When inputs $x$ are chosen according to $\cal D$, then we assume that computing $f$ requires cost $C$ from any party: 
this can be formalized by saying that if a party invests $\tilde{C}=\gamma C$ effort (for $\gamma \leq 1$), then it computes the correct value only with
probability negligibly close to $\gamma$ (since 
a party can always have a "mixed" strategy in which with probability $\gamma$ it runs the correct computation and with probability $1-\gamma$ does 
something else, like guessing at random). 

\begin{assumption}
\label{assump:uisa}
We say that the {\em ($C$,$\epsilon$)-Unique Inner State Assumption} holds for a function $f$ and a distribution $\cal D$ if for any algorithm $\disP$ with
cost $\tilde{C}=\gamma C$, the probability that on input $x \in {\cal D}$, $\disP$ outputs $f(x)$ is at most $\gamma+(1-\gamma)\epsilon$. 
\end{assumption}

Note that the assumption implicitly assumes a "large" output space for $f$ (since a random guess of the output of $f$ will be correct with probability 
$2^{-n}$ where $n$ is the binary length of $f(x)$). 

More importantly, note that Assumption~\ref{assump:uisa} immediately yields our notion of sequential composability, if the Verifier can detect if the 
Prover is lying or not. Assume, as a mental experiment for now, that given input $x$, the Prover claims that $\tilde{y}=f(x)$ and the Verifier
checks by recomputing $y=f(x)$ and paying a reward of $R$ to the Prover if $y=\tilde{y}$ and $0$ otherwise. Clearly this is not a very useful protocol, 
since the Verifier is not saving any computation effort by talking to the Prover. But it is sequentially composable according to our definition, since 
$\pDisR$, the probability that $\disP$ collects $R$, is equal to the probability that $\disP$ computes $f(x)$ correctly, and by using 
Assumption~\ref{assump:uisa} we have that 
$$ \pDisR = \gamma+(1-\gamma)\epsilon \leq \frac{\tilde{C}}{C} + \epsilon $$
satisfying Corollary~\ref{cor:prob}.

To make this a useful protocol we adopt a strategy from  \cite{b08}, which also uses this idea of verification by recomputing. Instead of checking 
every execution, we check only a random subset of them, and therefore we can amortize the Verifier's effort over a large number of computations. 
Fix a parameter $m$. The prover sends to the verifier the values $\tilde{y}_j$ which are claimed to be the result of computing $f$ over $m$ inputs $x_1,\ldots,x_m$. 
The verifier chooses one index $i$ 
randomly between $1$ and $m$, and computes $y_i=f(x_i)$. If $y_i=\tilde{y}_i$ the verifier pays $R$, otherwise it pays $0$. 

Let $T$ be the total cost by the honest prover to compute $m$ instances: cleary $T=mC$. Let $\tilde{T}=\Sigma_i \tilde{C}_i$ be the total effort invested by $\tilde{P}$, by investing $\tilde{C}_i$ on the computation of $x_i$. 
In order to satisfy Corollary~\ref{cor:prob} we need that $\pDisR$, the probability that $\disP$ collects $R$, be less than $\tilde{T}/T + \epsilon$. 

Let $\gamma_i = \tilde{C}_i/C$, then under Assumption~\ref{assump:uisa} we have that $\tilde{y}_i$ is correct with probability at most $\gamma_i+(1-\gamma_i)\epsilon$. Therefore if we set $\gamma = \sum_i \gamma_i/m$ we have 
$$ \pDisR = \frac{1}{m} \sum_i [\gamma_i+(1-\gamma_i)\epsilon] = \gamma + (1-\gamma) \epsilon \leq \gamma + \epsilon $$
But note that $\gamma=\tilde{T}/T$ as desired since
$$ \tilde{T} = \sum_i \tilde{C}_i = \sum_i \gamma_i C = T \sum_i \gamma_i/m $$

\medskip
\noindent
{\sc Efficiency of the Verifier.}
If our notion of "efficient Verifier" is a verifier who runs in time $o(C)$ where $C$ is the time to compute $f$, then in the above protocol $m$ must be sufficiently large to amortize the cost of computing one execution over many (in particular a constant -- in the input size $n$ -- value of $m$ would not work). In our "concrete analysis" treatment, if we requests that the Verifier runs in time $\delta C$ for an "efficiency" parameter $\delta \leq 1$, then we need $m \geq \delta^{-1}$. 

Therefore we are still in need of a protocol which has an efficient Verifier, and would still works for the "stand-alone" case ($m=1$) but also for the case
of sequential composability over any number $m$ of executions. 


\section{Our Protocol}
\label{sec:our-protocol}
We now present a protocol that works for functions $f: \binstring^n \to \binstring^n$ expressed by an arithmetic circuit $\cal C$ of size $C$ and depth $d$ and fan-in 2, given as a common input to both Prover and Verifier together with the input $x$. 

Intuitively the idea is for the Prover to provide the Verifier with the output value $y$ and its two "children" $y_L,y_R$ in the gate, i.e. the two input values of the last output gate $G$. The Verifier checks that $G(y_L,y_R)=y$, and then asks the Prover to verify that $y_L$ or $y_R$ (chosen a random) is correct, by 
recursing on the above test. The protocol description follows. 

%\smallskip
%\noindent

\begin{framed}
\begin{enumerate}
\item The Prover evaluates the circuit on $x$ and send the output value $y_1$ to the
Verifier. 

\item {\bf Repeat $r$ times:} The Verifier identifies the root gate $g_1$ and 
then invokes $Round(1,g_1, y_1)$,
\end{enumerate}
where the procedure $Round(i, g_i, y_i)$ is defined for $1 \leq i \leq d$ as 
follows:
\begin{enumerate}
\item The Prover sends the value of the input wires $z^0_i$ and $z^1_i$ of $g_i$ to the Verifier.  

\item The Verifiers performs the following 
\begin{itemize}
\item Check that $y_i$ is the result of the operation of gate $g_i$ on inputs 
$z^0_i$ and $z^1_i$. If not \textbf{STOP} and pay a reward of $0$.
\item If $i = d$ (i.e. if the inputs to $g_i$ are input wires), check that 
the values of $z^0_i$ and $z^1_i$ are equal to the corresponding bits of $x$. Pay
reward $R$ to Merlin if this is the case, nothing otherwise.
\item If $i < d$, choose a random bit $b$, send it to Merlin and invoke 
$Round(i+1, g^b_{i+1}, z^b_i)$ where $g^b_{i+1}$ is the child gate of $g_i$ whose output is
$z^b_{i}$.
\end{itemize}
\end{enumerate}
\end{framed}

\subsection{Efficiency} 
% Rounds
The protocol runs at most in $d$ rounds.
% Communication Complexity
In each round, the Prover sends a constant number of bits representing the values 
of specific input and output wires; The Verifier sends at most one bit per round, 
the choice of the child gate. Thus the communication complexity is $O(d)$ bits.

The computation of the Verifier in each round is: (i) computing the result of a gate 
and checking for bit equality; (ii) sampling a child.
Gate operations and equality are $O(1)$ per round.
We assume our circuits are $T$-uniform, which allows the Verifier to select the correct gate in time $T(n)$ \footnote{
We point out that the Prover can provide the Verifier with the requested gate and then the Verifier can use the uniformity of the circuit to check that the Prover has given him the correct gate in time $O(T(n))$) at each level in $O(T(n))$. }
Thus the Verifier runs in $O(r d\cdot T(n))$ with $r=O(\log C)$.



\subsection{Proofs of (stand-alone) Rationality}
\label{sec:proof-stand-alone}

\begin{theorem}
\label{thm:rp}
The protocol in Section~\ref{sec:our-protocol} for $r=1$ is a Rational Proof according to Def.~\ref{def:RP-delta}.
\end{theorem}
We prove the above theorem by showing that for every input $x$ the reward gap $\Delta(x)$ is positive. 

\begin{proof}
Let $\disP$ a prover that always reports $\claimedy \neq y_1 = f(x)$ at Round 1. 

Let us proceed by induction on the depth $d$ of the circuit.
If $d = 1$ then there is no possibility for $\disP$ to cheat successfully, and its 
reward is $0$. 

Assume $d > 1$. We can think of the binary circuit $\cal C$ as composed by two 
subcircuits ${\cal C}_L$ and ${\cal C}_R$ and the output gate $g_1$ such that
$f(x) = g_1({\cal C}_L(x), {\cal C}_R(x))$. The respective depths $d_L, d_R$ of these 
subcircuits are such that $0 \leq d_L, d_R \leq d-1$ and $max(d_L,d_R)=d-1$. After sending 
$\claimedy$, 
the protocol requires that
$\disP$ sends output values for $C_L(x)$ and
$C_R(x)$; let us denote these claimed values respectively with $\claimedy_L$ 
and  $\claimedy_R$. 
Notice that at least one of these alleged values will be different from the 
respective correct subcircuit output: if it were otherwise, $V$ would 
reject immediately as $g(\claimedy_L, \claimedy_R) = f(x) \neq \claimedy$.
Thus at most one of the two values $\claimedy_L$, $\claimedy_R$ is equal 
to the output of the corresponding subcircuit.
The probability that the $\disP$ cheats successfully is:
\begin{align}
\Pr[\mbox{V accepts}] & \leq 
\frac{1}{2}\cdot(\Pr[\mbox{V accepts on } C_L] + \Pr[\mbox{V accepts on } C_R]) 
\label{ineq:RP-formula-vs-circuit}
\\
& \leq \frac{1}{2}\cdot(1-2^{-\max(d_L, d_R)}) + \frac{1}{2} 
\label{ineq:RP-ind-hyp-bound} \\
& \leq \frac{1}{2}\cdot(1-2^{-d+1}) + \frac{1}{2} \\
& = 1 - 2^{-d}
\end{align}
At line \ref{ineq:RP-ind-hyp-bound} we used the inductive hypothesis and the 
fact 
that all probabilities are at most $1$.

Therefore the expected reward of $\disP$ is $\tilde{R} \leq R(1-2^{-d})$ and the reward 
gap is $\Delta(x) = 2^{-d}R$ (see Remark~\ref{rem:match} below to explain the equality). 
\end{proof}

The following useful corollary follows from the proof above.
\begin{corollary}
	\label{cor:repetitions}
	If the protocol described in Section~\ref{sec:our-protocol} is repeated $r \geq 1$ times a prover can cheat with probability at most $(1-2^{-d})^r$. 
\end{corollary}

\begin{remark}
\label{rem:match}
{\em 
We point out that one can always build a prover strategy 
$P^*$ which always answers incorrectly and achieves exactly the reward 
$R^*=R(1-2^{-d})$. This prover outputs an incorrect 
$\claimedy$ and then computes one of the subcircuits that results in one of the input values (so that at least one of the inputs is correct). This will allow him to recursively answer with values $z^0_i$ and $z^1_i$ where one of the two is correct, and therefore be caught only with probability $2^{-d}$.
}
\end{remark}

\begin{remark}
\label{rem:logd}
{\em 
In order to have a non-negligible reward gap (see Remark~\ref{rem:asy}) we need
to limit ourselves to circuits of depth $d=O(\log n)$. 
}
\end{remark}

\subsection{Proof of Sequential Composability}
\label{sec:proof-comp}

\newthought{General sufficient conditions for sequential composability}

\begin{lemma}
	Let $\cal{C}$ be a circuit of depth $d$. If the $(C,\epsilon)$ Unique Inner State Assumption (see Assumption~\ref{assump:uisa}) holds for the function $f$ computed by $\cal C$, and input distribution $\cal D$, then the protocol presented above with $r$ repetitions is a $k\epsilon R$-sequentially composable Rational Proof for $\cal{C}$ for $\cal D$ if the following inequality holds 
	$$ (1-2^{-d})^r \leq \frac{1}{C} $$
\end{lemma}
\begin{proof}
	Let $\gamma=\frac{\tilde{C}}{C}$. 
Consider $x \in \cal D$ and prover $\disP$ which invests effort $\tilde{C}\leq C$. Under Assumption~\ref{assump:uisa}, $\disP$ gives the correct outputs
with probability $\gamma+\epsilon$ -- assume that in this case $\disP$ collects the reward $R$. If $\disP$ gives an incorrect output we know (following Corollary \ref{cor:repetitions}) that he collects the reward 
$R$ with probability at most $(1-2^{-d})^r$ which by hypothesis is less than $\gamma$. So either way we have that $\tilde{R} \leq (\gamma + \epsilon)R$ 
and therefore applying Lemma~\ref{lemma:cost-rew-ratios} concludes the proof.
\end{proof}
The problem with the above Lemma is that it requires a large value of $r$ for the result to be true resulting in an inefficient Verifier. In the following
sections we discuss two approaches that will allows us to prove sequential composability even for an efficient Verifier:
\begin{itemize}
\item Limiting the class of provers we can handle in our security proof;
\item Limiting the class of functions/circuits.
\end{itemize}

% Repetitions

\newthought{Limiting the strategy of the prover: Non-adaptive Provers}

\smallskip
\noindent
In proving sequential composability it is useful to find a connection between the amount of work done by a dishonest prover and its probability of cheating. The more a dishonest prover works, the higher its probability of cheating. This is true for our protocol, since the more "subcircuits" the prover computes correctly, the higher is probability of convincing the verifier of an incorrect output becomes. The question then is how can a prover who has an "effort budget" to spend, can maximize its probability of success in our protocol. 

As we discussed in Remark~\ref{rem:match}, there is an {\em adaptive} strategy for the $\disP$ to maximize its probability of success: compute one subcircuit correctly at every round of the protocol. We call this strategy "adaptive", because the prover allocates its "effort budget" $\tilde{C}$ on the fly during the execution of the rational proof. Conversely a {\em non-adaptive} prover $\disP$ uses $\tilde{C}$ to compute some subcircuits in $\cal C$ before starting the protocol. Clearly an adaptive prover strategy is more powerful, than a non-adaptive one (since the adaptive prover can direct its computation effort where it matters most, i.e. where the Verifier "checks" the computation). 

Is it possible to limit the Prover to a non-adaptive strategy? This could be achieved by imposing some "timing" constraints to the execution of the protocol: to prevent the prover from performing large computations while interacting with the Verifier, the latter could request that prover's responses be delivered "immediately", and if a delay happens then the Verifier will not pay the reward. Similar timing constraints have been used before in the cryptographic literature, e.g. see the notion of {\em timing assumptions} in the concurrent zero-knowedge protocols in \cite{dns}.

Therefore in the rest of this subsection we assume that non-adaptive strategies are the only rational ones and proceed in analyzing our protocol 
under the assumption that the prover is adopting a non-adaptive strategy. 


Consider a prover $\disP$ with effort budget $\tilde{C} < C$. A \emph{DFS} (for "depth first search") prover  uses its effort budget $\tilde{C}$ to compute a whole subcircuit of size $\tilde{C}$ and maximal depth $d_{DFS}$. Call this subcircuit $\circDFS$. $\disP$ can answer correctly any verifier's query about a gate in $\circDFS$. During the interaction with $V$, the behavior of a DFS prover is as follows:
\begin{itemize}
\item At the beginning of the protocol send an arbitrary output value $y_1$.
\item During procedure $Round(i, g_i, y_i)$:
\begin{itemize}
\item If $g_i \in \circDFS$  then $\disP$ sends the two correct inputs $z^0_i$ and $z^1_i$.
\item If $g_i \not \in \circDFS$ and neither of $g_i$'s input gate belongs to $\circDFS$ then $\disP$ sends two arbitrary $z^0_i$ and $z^1_i$ that are consistent with $y_i$, i.e. $g_i(z^0_i,z^1_i) = y_i$.
\item $g_i \not \in \circDFS$ and one of $g_i$'s input gates belongs to $\circDFS$, then $\disP$ will send the correct wire known to him and another arbitrary value consistent with $y_i$ as above.
\end{itemize} 
\end{itemize}

\begin{lemma}[Advantage of a DFS prover]
\label{lem:DFS}
In one repetition of the protocol above, a DFS prover with effort budget $\tilde{C}$ investment has probability of cheating $\tilde{p}_{DFS}$ defined by 
$$ \tilde{p}_{FS} \leq 1 - 2^{-d_{DFS}}$$
\end{lemma}

The proof of Lemma~\ref{lem:DFS} follows easily from the proof of the stand-alone rationality of our protocol (see Theorem~\ref{thm:rp}). 

% BFS provers and their advantage
If a DFS prover focuses on maximizing the depth of a computed subcircuit given a certain investment, \emph{BFS} provers allot their resources to compute all subcircuits rooted at a certain height.
A BFS prover with effort budget $\tilde{C}$ computes the value of all gates up to the maximal height possible $d_{BFS}$. Note that $d_{BFS}$ is 
a function of the circuit $\cal C$ and of the effort $\tilde{C}$.  Let $ \circBFS$ be the 
collection of gates computed by the BFS prover. 
The interaction of a BFS prover with $V$ throughout the protocol resembles that of the DFS prover outlined above:
\begin{itemize}
	\item At the beginning of the protocol send an arbitrary output value $y_1$.
	\item During procedure $Round(i, g_i, y_i)$:
	\begin{itemize}
		\item If $g_i \in \circBFS$  then $\disP$ sends the two correct inputs $z^0_i$ and $z^1_i$.
		\item If $g_i \not \in \circBFS$ and neither of $g_i$'s input gate belongs to $\circBFS$ then $\disP$ sends two arbitrary $z^0_i$ and $z^1_i$ that are consistent with $y_i$, i.e. $g_i(z^0_i,z^1_i) = y_i$.
		\item $g_i \not \in \circBFS$ and both $g_i$'s input gates belong to $\circDFS$, then $\disP$ will send one of the correct wires known to him and another arbitrary value consistent with $y_i$ as above.
	\end{itemize} 
\end{itemize}
As before, it is not hard to see that the probability of successful cheating by a BFS prover can be bounded as follows: 

\begin{lemma}[Advantage of a BFS prover]
\label{lem:BFS}
	In one repetition of the protocol above, a BFS prover with effort budget $\tilde{C}$ has probability of cheating $\tilde{p}_{BFS}$ bounded by
	$$ \tilde{p} \leq 1 - 2^{-d_{BFS}}$$
\end{lemma}
%\begin{proof}
%	The proof proceeds analogously to that of Theorem \ref{thm:rp}, considering any subcircuit rooted at height $\deltaBfsInv+1$ instead of the whole circuit.
%\end{proof}

% TODO: write general prover and proof
BFS and DFS provers are both special cases of the general non-adaptive strategy which allots its investment $\tilde{C}$ among a general collection of subcircuits $\circNA$.
The interaction with $V$ of such a prover is analogous to that of a a BFS/DFS prover but with a collection of computed subcircuits not constrained by any specific height. We now try to formally define what the success probability of such a prover is. 

\begin{mydef}[Path experiment]
	\label{def:exp-path}
	Consider a circuit $\cal C$ and a collection $\circNA$ of subcircuits of $\cal C$. Perform the following experiment: starting from the output gate, flip a biased coin and choose the "left" subcircuit or the "right" subcircuit at random with probability 1/2. Continue until the random path followed by the experiment reaches a computed gate in $\circNA$. Let $i$ be the height of this gate, which is the output of the experiment. Define with $\Pi_i$ the probability that this experiment outputs $i$.
\end{mydef}

The proof of the following Lemma is a generalization of the proof of security of our scheme. Once the "verification path" chosen by the Verifier enters a fully computed subcircuit at height $i$ (which happens with probability $\Pi_i^{\circNA}$), the probability of success of the Prover is bounded by $(1 - 2^{-i})$

\begin{lemma}[Advantage of a non adaptive prover]
\label{lem:gen-adv}
	In one repetition of the protocol above, a generic prover with effort budget $\tilde{C}$ used to compute a collection $\circNA$ of subcircuits, has probability of cheating $\tilde{p}_{\circNA}$ bounded by
	$$ \tilde{p}_{\circNA} \leq \sum_{i=0}^{d} \Pi_i (1 - 2^{-i})$$
	where $\Pi_i$-s are defined as in Definition \ref{def:exp-path}.
\end{lemma}

\newthought{Limiting the class of functions: Regular Circuits}


\smallskip
\noindent
Lemma~\ref{lem:gen-adv} still does not produce a clear bound on the probability of success of a cheating prover. The reason is that it is not obvious
how to bound the probabilities $\Pi_i^{\circNA}$ that arise from the computed subcircuits $\circNA$ since those depends in non-trivial ways from the 
topology of the circuit $\cal C$. 

We now present a type of circuits for which it can be shown that the BFS strategy is optimal. The restriction on the circuit is surprisingly simple: we call them 
{\sf regular circuits}. In 
the next section we show examples of interesting functions that admit regular circuits. 

\begin{mydef}[Regular circuit]
\label{def:reg-circ}
	A circuit $\cal C$ is said to be regular if the following conditions hold:
	\begin{itemize}
		\item $\cal C$ is layered;
		\item every gate has fan-in 2;
		\item the inputs of every gate are the outputs of two distinct gates.
	\end{itemize}
\end{mydef}

The following lemma states that, in regular circuits, we can bound the advantage of any prover investing $\tilde{C}$ by looking at the advantage of a BFS prover with the same investment.

\begin{lemma}[A bound for provers' advantage in regular circuits]
	\label{lemma:bfs-bound-reg}
	Let $\disP$ be a prover investing $\tilde{C}$. Let $\cal C$ be the circuit being computed and $\delta=d_{BFS}({\cal C},\tilde{C})$. In one repetition of the  		protocol above, the advantage of $\disP$ is bounded by 
	$$ \tilde{p} \leq \tilde{p}_{BFS} = 1 - 2^{-\delta}$$ 
\end{lemma}

 \begin{proof}
Let $\circNA$ be the family of subcircuits compued by $\disP$ with effort $\tilde{C}$. As pointed out above the probability of success for $\disP$ is 
$$ \tilde{p} \leq \sum_{i=0}^{d} \Pi_i^{\circNA} (1 - 2^{-i})$$
Consider now a prover $\disP'$ which uses $\tilde{C}$ effort to compute a different collection of subcircuits $\circNA'$ defined as follows: 
\begin{itemize}
\item Remove a gate from a subcircuit of height $j$ in $\circNA$: this produces two subcircuits of height $j-1$. This is true because of the regularity of the circuit: since the inputs of every gate are the outputs of two distinct gates, when removing a gate of height $j$ this will produce two subcircuits of height $j-1$;
\item Use that computation to "join" two subcircuits of height $k$ into a single subcircuit of height $k+1$. Again we are using the regularity of the circuit here: since the circuit is layered, the only way to join two subcircuits into a single computed subcircuit is to take two subcircuits of the same height. 
\end{itemize}
What happens to the probability $\tilde{p}'$ of success of $\disP'$? Let $\ell$ be the number of possible paths generated by the experiment above
with $\circNA$. Then the probability of entering a computed subcircuit at height $j$ decreases by $1/\ell$ and that probability weight goes to entering at
height $j-1$. Similarly the probability of entering at height $k$ goes down by $2/\ell$ and that probability weight is shifted to entering at height $k+1$. 
Therefore 
$$ \tilde{p}' \leq \sum_{i \neq j,j-1,k,k+1} \Pi_i (1 - 2^{-i}) + $$
$$ + (\Pi_j - \frac{1}{\ell})(1-2^{-j}) + (\Pi_{j-1} + \frac{1}{\ell})(1-2^{-j+1}) + $$
$$ + (\Pi_k - \frac{2}{\ell})(1-2^{-k}) + (\Pi_{k+1} + \frac{2}{\ell})(1-2^{-k-1}) = $$
$$ = \tilde{p} + \frac{1}{\ell 2^j} - \frac{1}{\ell 2^{j-1}} + \frac{1}{\ell 2^{k-1}} - \frac{1}{\ell 2^k} $$
$$ = \tilde{p} +\frac{2^k-2^{k+1}+2^{j+1}-2^j}{\ell 2^{j+k}}
= \tilde{p} + \frac{2^j - 2^k}{\ell 2^{j+k}} $$
Note that $\tilde{p}'$ increases if $j > k$ which means that it's better to take "computation" away from tall computed subcircuits to make them shorter, and use the saved computation to increase the height of shorter computed subtrees, and therefore that the probability is maximized when all the subtrees are of the same height, i.e. by the BFS strategy which has probability of success $ \tilde{p}_{BFS} = 1 - 2^{-\delta}$.
\end{proof}

The above Lemma, therefore, yields the following 

\begin{theorem}
\label{thm:reg-circ}
Let $\cal{C}$ be a regular circuit of size $C$. If the $(C,\epsilon)$ Unique Inner State Assumption (see Assumption~\ref{assump:uisa}) holds for the function $f$ computed by $\cal C$, and input distribution $\cal D$, then the protocol presented above with $r$ repetitions is a $k\epsilon R$-sequentially composable Rational Proof for $\cal{C}$ for $\cal D$ if the prover follows a non-adaptive strategy and the following inequality holds for all $\tilde{C}$ 
	$$ (1-2^{-\delta})^r \leq \frac{\tilde{C}}{C} $$
where $\delta=d_{BFS}({\cal C},\tilde{C})$.
\end{theorem}

\begin{proof}
	Let $\gamma=\frac{\tilde{C}}{C}$. 
Consider $x \in \cal D$ and prover $\disP$ which invests effort $\tilde{C}\leq C$. Under Assumption~\ref{assump:uisa}, $\disP$ gives the correct outputs
with probability $\gamma+\epsilon$ -- assume that in this case $\disP$ collects the reward $R$. 

If $\disP$ gives an incorrect output we can invoke Lemma~\ref{lemma:bfs-bound-reg} and conclude that he collects reward  
$R$ with probability at most $(1-2^{-\delta})^r$ which by hypothesis is less than $\gamma$. So either way we have that $\tilde{R} \leq (\gamma + \epsilon)R$ 
and therefore applying Lemma~\ref{lemma:cost-rew-ratios} concludes the proof.
\end{proof}

\section{Results for FFT circuits}

In this section we apply the previous results to the problem of computing FFT circuits, and by extension to polynomial evaluations. 
% TODO: Intro to FFT, motivation and whatnot

\subsection{FFT circuit for computing a single coefficient}

The Fast Fourier Transform is an almost ubiquitous computational problem that appears in many applications, including many of the volunteer 
computations that motivated our work. As described in \cite{CLRS} a circuit to compute the FFT of a vector of $n$ input elements, consists of $\log n$ levels, 
where each level comprises $n/2$ {\em butterflies} gates. The output of the circuit is also a vector of $n$ input elements. 

Let us focus on the circuit that computes a single element of the output vector: it has $\log n$ levels, and at level $i$ it has $n/2^i$ butterflies gates. 
Moreover the circuit is regular, according to Definition~\ref{def:reg-circ}.

%As the theorem below shows, we can obtain sequential composability for this circuit with a 
%Verifier running in time $O(T\log n)$ where $T$ is the  uniformity parameter for the circuit.

% Seq. composability for this circuit
\begin{theorem}
Under the $(C,\epsilon)$-unique inner state assumption for input distribution $\cal D$,
the protocol in Section \ref{sec:our-protocol}, when repeated $r = O(1)$ times, yields sequentially composable rational proofs for the FFT, under input 
distribution $\cal D$ and assuming non-adaptive prover strategies. 
\end{theorem}
\begin{proof}
Since the circuit is regular we can prove sequential composability by invoking Theorem~\ref{thm:reg-circ} and proving that for $r = O(1)$, 
the following inequality holds 
$$ \tilde{p} = (1-2^{-\delta})^r \leq \frac{\tilde{C}}{C} $$
where $\delta=d_{BFS}({\cal C},\tilde{C})$.

But for any $\tildelta < d$, the structure of the FFT circuit inmplies that the number of gates below height $\tildelta$ is $\investTildelta = \Theta(C(1-2^{-\tildelta}))$.
Thus the inequality above can be satisfied with $r = \Theta(1)$.
\end{proof}	





\subsection{Mixed Strategies for Verification}


One of the typical uses of the FFT is to change representation for polynomials. Given a polynomial $P(x)$ of degree $n-1$ we can represent it 
as a vector of $n$ coefficients $[a_0,\ldots,a_{n-1}]$ or as a vector of $n$ points $[P(\omega_0),\ldots,P(\omega_{n-1})]$. If $\omega_i$ are the 
complext $n$-root of unity, the FFT is the algorithm that goes from one representation to the other in $O(n \log n)$ time, rather than the obvious $O(n^2)$. 

In this section we consider the following problem: given two  polynomial $P,Q$ of degree $n-1$ in point representation, compute the inner 
product of the coefficients of $P,Q$. A fan-in two circuit computing this function could be built as follows:
\begin{itemize}
\item two parallel FFT subcircuits computing the coefficient representation of $P,Q$ ($\log n$-depth and $n \log n)$ size total for the 2 circuits) ;
\item a subcircuit where at the first level the $i$-degree coefficients are multiplied with each other, and then all these products are added by a binary tree of
additions $O(\log n)$-depth and $O(n)$ size);
\end{itemize}
Note that this circuit is regular, and has depth $2 \log n +1$ and size $n \log n + n + 1$

Consider a prover $\disP$ who pays $\tilde{C}  < n \log n$ effort. Then, since the BFS strategy is optimal, the probability of convincing the Verifier of a wrong result of the FFT is $(1-2^{-\tilde{d}})^r$ where 
$\tilde{d}=c\log n$ with $c \leq 1$ . Note also that 
$\frac{\tilde{C}}{C} < 1$. Therefore with $r=O(n^c)$ repetitions, the probability of
success can be made smaller than $\frac{\tilde{C}}{C}$. The Verifier's complexity is $O(n^c \log n) = o(n \log n)$. 


\bigskip
\noindent
If $\tilde{C} \geq n \log n$ the analysis above fails since $\tilde{d} > \log n$. Here we observe that in order for $\disP$ to earn a larger reward than $P$, 
it must be that $P$ has run at least  $k=O(\log n)$ executions (since it is possible to find $k+1$ inputs such that $(k+1)\tilde{C} \leq k C$ only if $k> \log n$).
In this case we can use a "mixed" strategy for verification:
\begin{itemize}
\item The Verifier pays the Prover only after $k$ executions. Each execution is verified as above (with $n^c$ repetitions); 
\item Additionally the Verifier uses the "check by re-execution" (from Section~\ref{sec:uisa}) strategy every $k$ executions (verifiying one execution by
recomputing it);
\item The Verifier pays $R$ if all the checks are satisfied, $0$ otherwise;
\item The Verifier's complexity is $O(k n^c \log n + n \log n) = o(k n \log n)$ -- the latter being the complexity of computing $k$ instances. 
\end{itemize}
 




%  FG DoC
%\chapter{Fine-Grained Verifiable Computation}
%In this chapter we will show how to obtain Verifiable Computation from Fine-Grained SHE. 
The main idea is to use a technique from \cite{ckv10}.

% TODO: Possible applications: streaming



%  FG FHE
%\chapter{Fine-Grained Somewhat Homomorphic Encryption}
%In this chapter we show how to obtain somewhat homomorphic encryption secure against adversaries in

\section{Preliminaries}


\subsection{Public-Key Encryption}

\subsection{Public-key Encryption Secure Against $\NC^1$ Adversaries}

\subsection{Homomorphic Encryption}
% Here preliminaries on SHE

\section{Somewhat Homomorphic Encryption Secure Against $\NC^1$ Adversaries}

\subsection{A Relinerization Technique}

\subsection{Expressivity}






\bibliographystyle{plain}
\bibliography{references}


\end{document}
