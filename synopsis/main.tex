
  \begin{abstract}
  	\noindent
  	We propose the study of protocols for delegating computation in a model in which one of the party is rational.
  	A \textit{delegator} outsources the computation of a function $f$ on input $x$ to a \textit{worker}, who
  	receives a (possibly monetary) reward. Our goal is to design \textit{very efficient} delegation schemes 
  	where a worker is economically incentivized to provide the correct result
  	$f(x)$. As a design choice, we do not want to rely on cryptographic assumptions
  	(i.e. we will not assume one-way functions exist).
  	
  	We provide several results within the framework of rational proofs introduced by Azar and Micali (STOC 2012).
  	We make several contributions to efficient rational proofs for general feasible computations.
  	First, we design schemes with a sublinear verifier with low round and communication complexity for
  	space-bounded computations.
  	Second, we provide evidence, as lower bounds, against the existence of rational proofs:
  	with logarithmic communication and polylogarithmic verification for $\P$ and 
  	with polylogarithmic communication for $\NP$.
  	% Third, we generalize a scaling approach in Azar and Micali (STOC) ... [composition theorem]
  	
  	We then move to study the case where a delegator outsources multiple inputs.
  	First, we formalize an extended notion of rational proofs for this scenario (sequential composability) and we
  	show that existing schemes do not satisfy it. We show how these protocols incentivize workers
  	to provide many ``fast'' incorrect answers which allow them to solve more problems and collect more rewards.
  	We then design a $d$-rounds rational proof for sufficiently ``regular'' arithmetic circuit of depth $d = O(\log{n})$
  	with sublinear verification. We show, that under certain cost assumptions, our scheme is sequentially composable,
  	i.e. it can be used to delegate multiple inputs. We finally show that our scheme for space-bounded computations is also 
  	sequentially composable under certain cost assumptions.
  	
  	%The results above have been published as proceedings in GameSec as Campanelli and Gennaro (GameSec 2015)
  	%and Campanelli and Gennaro (GameSec 2017).
  	
	Finally we discuss partial results for delegation of functions computable by low-depth circuits of unbounded fan-in.
	Our techniques rely on \textit{fine-grained} cryptographic schemes whose security guarantees do not require the existence of one-way functions
	and hold only against bounded adversaries. As a stepping stone for our delegation schemes, we also build 
	the first fine-grained somewhat homomorphic encryption schemes which we believe to be of independent interest.
	Our fine-grained protocols can be used to obtain sequentially composable rational proofs.
  \end{abstract}
  
  

% -- Motivating Example -- i.e. why would we care?
% Nowadays a lot of computational resources are available. Yet, not evenly so.
% XXX: More about the transition back to "mainframe"-like era through cloud computing

Imagine this situation.
You are a researcher in a biological lab, you have plenty of data and a clear scientific hypothesis you want to test on them. All you need to test such hypothesis is running a program you wrote. Problem: the computers in your lab will take weeks to run it (the bliss and the curse of big data). If you had a large number of powerful CPUs, a parallel version of the program you wrote would run in only a few hours --- right on time for you to submit your findings for the next conference deadline.

Cloud computing solves a part of the problem above. If you cannot \textit{physically} run a program (or store data) on a computer you own, have someone ``rent'' you a computer (or \textit{many} computers) on which to run it. Let (say) Amazon EC2 \footnote{Amazon Elastic Cloud: https://aws.amazon.com/ec2/} run your beautiful code for you and write your papers with the results it will return. So far, so good. But there is a new problem now, a problem that intuitively has to do with \textit{trust}. 
What guarantees do you have on whether they actually ran your program? What guarantees do you have on how reliable the execution was? What guarantees do you have on whether anybody or anything compromised the execution? At times little, or none. 

This new problem we are facing can be summarized as follows: how can we \textit{verify} that an untrusted computer executed a program correctly just as if our local machine\footnote{Assuming our local machine is \textit{reliable} and \textit{uncompromised}.} would have executed it? 
% -- Connection of the example to wider research --
This question is at the core of the field of \textit{Verifiable Computation} (VC). The goal of VC is, informally, to provide \textit{efficient} methods to verify that a computation we \textit{delegate} is executed ``correctly''. It is important to stress that these methods should be efficient. Part of the reason being a very simple one: we do not have the resources to run our program ourselves. When I receive the alleged results from EC2, I will have to verify them through my \textit{limited} local machine\footnote{This trivially excludes re-executing the program ourselves. But couldn't we just ask a third party to run our program again? Here the question is again: with what guarantees? Plus, how do we know the two parties are not colluding with one another?} 
VC is a practical problem that has extremely benefited from a theoretical analysis through the lenses and the tools of computational complexity and cryptography. The range and the scope of these results can roughly be located on two dimensions: efficiency and ``extent'' of guarantees. Efficiency can be described in terms of: the running times of the Delegator and the Worker\footnote{In our example, respectively the lab's computer and the cloud.} and the amount of communication required. The other dimension deals with what type of guarantees the methods provide and under what assumptions. This is related to a classical dichotomy in cryptography, e.g. is an encryption scheme provably secure against \textit{any} adversary (no matter their computational resources) or only against \textit{efficient} adversaries\footnote{Usually modeled as probabilist Turing machines (also denoted as PPTs, for Probabilistic Polynomial Time).} running in polynomial time? These two notions are respectively known as \textit{information-theoretic} (or \textit{unconditional}) security and \textit{computational} security. Analogously, a delegator which verifies a computation can obtain either type of security guarantee. 

% An intuition on rationality
\newthought{Our Model: Rational Workers}
In this dissertation proposal we will slightly diverge from the types of security guarantees outlined above. Recall that in cloud computing, computation is rented from far away servers. This suggests an alternative approach: assume that the worker is economically motivated and seeks to maximize a monetary reward. In other words, we assume that the worker is \textit{rational}. Notice that we are still not trusting the worker. However, we diverge from unconditional and computational security guarantees which model a worker's behavior respectively as an arbitrary computable function and an arbitrary function computable by a PPT. In fact we will assume an utility function $u: \bits \to \reals$ which takes as input the ``interaction'' between the worker and the delegator. One way of thinking of this utility function is as the payment (or \textit{reward}) that the worker obtains by carrying out the computation for the delegator. Once fixed $u$, the worker's behavior as an arbitrary computable function that maximizes this utility function in expectation.


Once we assume the cloud is rational, we can reduce our security guarantee to the following condition:
\begin{displayquote}
	Any cheating worker will obtain a lower utility (in expectation) than an honest worker.
\end{displayquote}
 where by ``honest'' worker we mean one that carries out the computation correctly.
Our goal now becomes to design a protocol that no worker \textit{is incentivized} to deviate from, i.e. a protocol no \textit{rational worker} will deviate from. 

% == General question we are addressing ==

% -- What are we going to study here specifically --
\newthought{Our Research Problem: Efficient Verifiable Computation}

% -- Why non cryptography
\newthought{A Design Constraint: Non-Cryptographic Protocols}

% == A framework: rational proofs ==
\subsection{Rational Proofs}

% -- Stress important features of rational proofs: low cc, rounds and verifier's time

% == Q1: Expressivity of Rational Proofs ==
\newthought{Research Question 1: Expressivity}

% -- Our results --

% == Q2: Multiple Delegations == 

% -- Our results --

% == Transition to FG ==

% -- A limitation of our approach: the inner state assumption --

% -- One way to get beyond that: FG schemes --

% == Fine-Grained Homomorphic encryption as a basic tool to get to delegation schemes ==

% == Results on non-interactive Delegation schemes ==



