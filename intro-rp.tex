% NB: There is a little more stuff in the intro.tex-s. It is unclear it may help.
In the first two part of this thesis (Chapters \ref{chap:RP-expr} and \ref{chap:RP-seq}) we use the concept of {\sf Rational Proofs} introduced by Azar and Micali in \cite{am} and refined in a subsequent paper \cite{am1}. 

In a Rational Proof, given a function $f$ and an input $x$, the server returns the value $y=f(x)$, and (possibly) some auxiliary information, to the client. The client will in turn 
pay the server for its work with a reward which is a function of the messages 
sent by the server and some randomness chosen by the client.  The crucial 
property is that this reward is maximized in expectation when the server 
returns the correct value $y$. Clearly a rational prover who is only interested 
in maximizing his reward, will always answer correctly. 

The most striking feature of Rational Proofs is their simplicity. For example in \cite{am}, Azar and Micali show {\sf single-message} Rational Proofs for any problem in $\#P$, where an (exponential-time) prover convinces a (poly-time) verifier of the number of satisfying assignment of a Boolean formula. 

For the case of "real-life" computations, where the Prover is polynomial and the Verifier is as efficient as possible, Azar and Micali in \cite{am1} show $d$-round Rational Proofs for functions computed by (uniform) Boolean circuits of depth $d$, for $d=O(\log n)$ (which can be collapsed to a single round under some well-defined computational assumption as shown in \cite{ratargs}). The problem of rational proofs for any polynomial-time computable function remains tantalizingly open. 


Recent work \cite{ratsumchecks} shows how to obtain Rational Proofs with sublinear verifiers for languages in $\NC$. Recalling that $\L \subseteq \NL \subseteq \NC_2$, one can use the protocol  in \cite{ratsumchecks} to verify a logspace polytime computation (deterministic or nondeterministic) in $O(\log^2 n )$ rounds and $O(\log^2 n )$ verification.

The work by Chen et al. \cite{chen2016rational} focuses on rational proofs with multiple provers and the related class $\MRIP$ of languages decidable by a polynomial verifier interacting with an arbitrary number of provers. Under standard complexity assumptions, $\MRIP$ includes languages not decidable by a verifier interacting only with one prover. The class $\MRIP$ is equivalent to $\EXP^{||\NP}$.



\subsection{Contributions for Rational Proofs}

\subsubsection{Expressivity.}
We present new protocols for the verification of {\em space-bounded polytime computations} against a rational adversary. More specifically, consider a language $L \in \DTISP(T(n), S(n))$, i.e. recognized by a deterministic Turing Machine $M_L$ which runs in time $T(n)$ and space $S(n)$. 
In Section \ref{sec:rp-dtisp} we construct a protocol where a rational prover can
convince the verifier that $x \in L$ or $x \notin L$ with the following properties: 
\begin{itemize}
	\item The verifier runs in time $O(S(n) \log n)$
	\item The protocol has $O(\log n)$ rounds and communication complexity $O(S(n) \log n)$
	\item The prover simply runs $M_L(x)$ 
	%and stores all the intermediate configurations (i.e. requires space $O(S(n) T(n))$
\end{itemize}
%Our protocol can be proven to correctly incentivize a prover in {\bf both} the stand-alone model of \cite{am} and the sequentially composable definition of \cite{cg15}. This is the first protocol which is sequentially composable for a well-defined complexity class. 

For the case of ``real-life" computations (i.e. poly-time computations verified by a ``super-efficient" verifier) we 
note that for computations in sublinear space our general results yields a protocol in which the verifier is sublinear-time. Our protocols is the first rational proof for $\SC$ (also known as $\DTISP(\poly(n), \polylog(n))$) with polylogarithmic verification and logarithmic rounds. 
%Moreover, our results provide the first efficient rational proof for the 
%non-deterministic class $\NSC = \NTISP(\poly(n), \polylog(n) )$ . 

To compare this with the results in \cite{ratsumchecks}, we note that it is believed that $\NC \not = \SC$ and that the two classes are actually incomparable (see \cite{SCcompleteness} for a discussion). For these computations our results compare
favorably to the one in \cite{ratsumchecks} in at least one aspect: our protocol requires $O(\log n )$ rounds and has the same verification complexity.

We present several extensions of our main result:
\begin{itemize}
	
	\item Our main protocol can be extended to the case of space-bounded randomized computations using Nisan's 
	pseudo-random generator \cite{nisan1992pseudorandom} to derandomize the computation. 
	\item We also present a different protocol that works for BPNC (bounded error randomized NC) where the Verifier runs in polylog time (note that this class is not covered by our result since we do not know how to express NC with a polylog-space computation). This protocol uses in a crucial way a new {\em composition theorem} for rational proofs presented in this work and can be of independent interest. 
	\item Finally, we present lower bounds (i.e. conditional impossibility results) for Rational Proofs for various complexity classes.
\end{itemize}



\subsubsection{Repeated Executions and Costly Computation.}
Motivated by the problem of volunteer computation, our first
result is to show that the definition of Rational Proofs in \cite{am,am1} does not satisfy a basic compositional property which would make them applicable 
in that scenario. 
Consider the case where a large number of "computation problems" are outsourced. Assume that solving each problem takes time $T$. Then in a time interval of length $T$, the honest prover can only solve and receive the reward for a single problem. On the other hand a dishonest prover, can answer up to $T$ problems, for example by answering at random, a strategy that takes $O(1)$ time. To assure that answering correctly is a rational strategy, we 
need that at the end of the $T$-time interval the reward of the honest prover be larger than the reward of the dishonest one. But this is not necessarily the case: for some of the protocols in \cite{am,am1,ratargs} we can show that a ``fast" incorrect answer is more remunerable for the prover, by allowing him to solve more problems and collect more rewards.

The next questions, therefore, was to come up with a definition and a protocol that achieves rationality both in the stand-alone case, and in the composition
described above.  We first present an enhanced definition of Rational Proofs that removes the economic incentive  for the strategy of fast incorrect answers, and then we present a protocol that achieves it for the case of some (uniform) bounded-depth circuits.
Next, we design a $d$-rounds rational proof for sufficiently ``regular'' arithmetic circuit of depth $d = O(\log{n})$
with sublinear verification. We show, that under certain cost assumptions, our scheme is sequentially composable,
i.e. it can be used to delegate multiple inputs. We finally show that our scheme for space-bounded computations from Section \ref{sec:rp-dtisp} is also 
sequentially composable under certain cost assumptions.


\subsection{Comparison with Other Prior Work}
\label{sec:prior}

{\sc Other Decision-Theoretic Frameworks.}
An earlier work in the line of ``rational verifiable computation'' is \cite{b08} where the authors describe a system based on a scheme of rewards [resp. penalties] that the client assesses to the server for computing the function correctly [resp. incorrectly]. However in this system checking the computation may require re-executing it, something that the client does only on a randomized subset of cases, hoping that the penalty is sufficient to incentivize the server to perform honestly. Morever the scheme might require an "infinite" budget for the rewards, and has no way to "enforce" payment of penalties from cheating servers. For these reasons the best application scenario of this approach is the incentivization of volunteer computing schemes (such as SETI@Home or Folding@Home), where the rewards are non-fungible "points" used for "social-status". 

Because verification is performed by re-executing the computation, in this approach the client is "efficient" (i.e. does "less" work than the server) only in an 
amortized sense, where the cost of the subset of executions verified by the client is offset by the total number of computations performed by the server. This implies that the server must perform many executions for the client. 

{\sc Interactive Proofs.}
Obviously a ``traditional" interactive proof (where security holds against any adversary, even a computationally unbounded one) would work in our model. In this case the most relevant result is 
the recent independent work in \cite{rrr16} that presents breakthrough protocols for the deterministic (and randomized) restriction of the class of language we consider. If $L$ is a language which is recognized by a deterministic (or randomized) Turing Machine $M_L$ which runs in time $T(n)$ and space $S(n)$, then their protocol has the following properties: 
\begin{itemize}
	\item The verifier runs in 
	$O(\poly(S(n)) + n \cdot\polylog(n))$ time;
	\item The prover runs in polynomial time;
	\item The protocol runs in {\em constant} rounds, with communication complexity $O({\sf poly}(S(n)n^{\delta})$ for a constant $\delta$.
\end{itemize}
Apart from round complexity (which is the impressive breakthrough of the result in \cite{rrr16}) our protocols fares better in all other categories. Note in particular that a sublinear space computation does not necessarily yield a sublinear-time verifier in 
\cite{rrr16}. On the other hand, we stress that our protocol only considers weaker rational adversaries. 

\medskip
\noindent{\sc Computational Arguments.}
There is a large class of protocols for {\em arguments} of correctness (e.g. \cite{ggp10,ggpr13,krr14}) even in the rational model \cite{ratargs,ratsumchecks}. Recall that in an argument, security is achieved only against computationally bounded prover. In this case even single round solutions can be achieved. We will consider a variant of this model in Chapter \ref{chap:FG}.

\medskip
\noindent
{\sc Computational Decision Theory.}
Other works in theoretical computer science have studied the connections between cost of computation and utility in decision problems.
The work in \cite{halpern2011don} proposes a framework for \emph{computational decision problems}, where the Decision Maker's (DM) utility depends on the algorithm chosen for computing its strategy.
The Decision Maker runs the algorithm, assumed to be a Turing Machine, on the input to the computational decision problem.
The output of the algorithm determines the DM's strategy. 
Thus the choice of the DM reduces to the choice of a Turing Machine from a certain space. The DM will have beliefs on the running time (cost) of each Turing machine. The actual cost of running the chosen TM will affect the DM's reward.
Rational proofs with costly computation could be formalized in the language of \emph{computational decision problems} in \cite{halpern2011don}. There are similarities between the approach in this
work and that in \cite{halpern2011don}, as both take into account the cost of computation in a decision problem.


