\begin{comment}
In this section I describe the motivation of the problem and of the design choices.

Problem: Efficient Delegation of Computation 
[The motivation for this is straightforward]

Design Principles and Points of Focus:
- (almost) "lack" of cryptography (in the FG case, for reasons of both efficiency and assumptions)
- adversarial assumptions:
-- rationality, or
-- limited resources (circuit depth)


\end{comment}

%\section{Synopsis}
% -- Motivating Example -- i.e. why would we care?
% Nowadays a lot of computational resources are available. Yet, not evenly so.
% XXX: More about the transition back to "mainframe"-like era through cloud computing

{
\color{gray}


\section{Setting the Scene}

Imagine the following situation.
You are a researcher in a biological lab with plenty of data. You just formulated a clear scientific hypothesis and all you need to test it is to run a program you wrote. Problem is that any of the ordinary computers in your lab may take weeks to execute it (the bliss and the curse of big data). If you had a large number of powerful CPUs, a parallel version of the program you wrote would run in only a few hours --- right on time to submit your findings by the next conference deadline.

This scenario is the poster boy for cloud computing. If we cannot run a program (or store data) on a \textit{local} computer we can directly control, someone can ``rent'' us a machine (or \textit{many} machines) on which to run it on demand. Let us have (e.g.) Amazon EC2 \footnote{Amazon Elastic Cloud: https://aws.amazon.com/ec2/} run our beautiful code for us, then let us fill the tables in our paper with the returned output. This is extremely helpful, but now we are facing a new issue: \textit{trust}. 
What guarantees do we have on whether the cloud did actually run our program? What guarantees do we have on how reliable the execution was? And what guarantees do we have on whether no attacker compromised that execution? Possibly little, or none.

\newthought{Verifiable Computation}
This new problem can be summarized as follows: how can we \textit{verify} that an untrusted computer did \textit{correctly} execute a program, i.e. just as if it ran on our local machine\footnote{Assuming our local machine is \textit{reliable} and \textit{uncompromised}.}? 
% -- Connection of the example to wider research --
This question is at the core of the field of \textit{Verifiable Computation} (VC). The goal of VC is design protocols to verify a \textit{delegated} computation. These protocols should be \textit{efficient}. In our earlier example, once we receive the alleged results from the cloud, we will have to verify them through our \textit{limited} local machine\footnote{This trivially excludes re-executing the program ourselves. But couldn't we just delegate our program again to a third party and compare the results? Here the question is again: with what guarantees? Plus, how do we know the two parties are not colluding with one another?}. In the remainder of this document we will denote to these ``methods'' or protocols from VC interchangeably as \textit{delegation} or \textit{verification schemes}.

VC is a practical problem but has extremely benefited from a theoretical analysis through the cryptographic and complexity lenses. For a survey of the wide range of results in the field the reader is redirected to the excellent discussion in~\cite{wb15}. % Walfish
 The scope of these results can be arranged on at least three dimensions: how \textit{expressive} is a delegation scheme? How   efficient? What is the ``extent'' of its security guarantees?
 Expressivity refers to which classes of functions the results apply to (e.g. $\NC$ or $\P$).
 Efficiency can be described in terms of: the running times of the Delegator and the Worker\footnote{In our earlier example, the delegator and the worker are respectively the lab's computer and the cloud.} and the amount of communication required. Finally, results differ according to what type of guarantees their methods provide and under what assumptions. This is related to a classical dichotomy in cryptography, e.g. is an encryption scheme provably secure against \textit{any} adversary (regardless of their computational resources) or only against \textit{efficient} adversaries\footnote{Usually modeled as probabilist Turing machines (also denoted as PPTs, for Probabilistic Polynomial Time).}? These two notions are respectively known as \textit{information-theoretic} (or \textit{unconditional}) security and \textit{computational} security. Analogously to the encryption case, a verification scheme typically ensures either type of security guarantee. 

% An intuition on rationality
\newthought{Our Model: Rational Workers}
In this dissertation proposal we will slightly diverge from the types of security guarantees outlined above. Recall that in cloud computing, computation is \textit{rented} from far away servers. This suggests an alternative approach: assume that the worker is \textit{economically motivated} and seeks to maximize a monetary reward. In other words, assume it is \textit{rational}. Notice that we are still not trusting the worker and we are not assuming it will behave arbitrarily maliciously either. Thus we diverge from the unconditional and computational security guarantees sketched above. In fact these model a worker's behavior respectively as an arbitrary computable function and an arbitrary function computable by a PPT. In contrast we will assume a utility function $u: \bits \to \reals$ which takes as input the ``interaction'' between the worker and the delegator. One way of thinking of this utility function is as the payment (or \textit{reward}) that the worker obtains by carrying out the computation for the delegator. Once fixed $u$, the worker's behavior as an arbitrary computable function that maximizes this utility function in expectation.


Once we assume the cloud is rational, we can reduce our security guarantee to the following condition:
\begin{displayquote}
	Any cheating worker will obtain a lower utility (in expectation) than an honest worker.
\end{displayquote}
 where by ``honest'' worker we mean one that carries out the computation correctly.
Our goal now becomes to design a protocol that no worker \textit{is incentivized} to deviate from, i.e. a protocol no \textit{rational worker} will deviate from. 

% == General question we are addressing ==

% -- What are we going to study here specifically --
\newthought{Our Research Problem: Efficient Verifiable Computation}
In this proposal, our main question is how to design efficient delegation schemes in a rational setting, for a large class of computation? We will strive for efficiency in the typical key complexity dimensions in  verification schemes: the number of rounds of interaction between delegator and worker; the total number of bits exchanged (communication); the running time of the verification procedure; the overhead on the running time of the worker.

As unconditionally secure verification schemes (e.g. \cite{muggles,rrr16})  % Muggles, constant round proofs
imply security against rational workers, for our results to be of interest they should at least fare favorably against these.
Nonetheless it is important to stress that our focus in this work will not be on unconditionally secure results, but to leverage our assumptions on the worker's incentives as much as possible. 

% -- Why non cryptography
\newthought{A Design Constraint: Non-Cryptographic Protocols}
A long line of work in delegation of computation uses cryptography as a building tool. These schemes are secure against computationally bounded adversaries and under a variety of assumptions~\cite{ggp10,ckv10,qap,pinocchio,kalai2014delegate}.
In this work we will focus on solutions that do not assume full-blown cryptography, i.e. we shall not make any assumption that imply the existence of one-way functions, a fundamental primitive in complexity-based cryptography~\cite{impagliazzo1989one}\footnote{Equivalently, borrowing from  Impagliazzo's famous taxonomy of ``worlds''~\cite{impagliazzo1995personal}, we will not assume we are living in Cryptomania.}.

The motivation for this design choice is two-fold. From a theoretical perspective, we are interested in seeing what rationality alone as an additional assumption can help us get in terms of efficient protocols. From a practical point of view, cryptographic operations may be computationally expensive. As a consequence, a non-cryptographic verification scheme may gain in efficiency in practice.

In the second part of this work we will make use of tools from ``fine-grained'' cryptography~\cite{fgcrypto}. Loosely speaking, the term refers to cryptographic primitives (e.g. public-key encryption or pseudo-random functions) that are secure against a limited class of adversaries (e.g. circuits of constant depth) and whose security is unconditional (no assumptions are required) or based on very mild worst-case assumptions. This does not contradict our design criteria above as: \textit{(i)} fine-grained primitives may exist even in a world  where one-way functions are impossible to achieve; \textit{(ii)} the constructions for these primitives are extremely inexpensive, for example not requiring even integer addition.

% == A framework: rational proofs ==
\section{Complete Results}

The results in this section appeared in~\cite{cg15} and~\cite{cg17}.

\newthought{Rational Proofs}
Part of our verification schemes will be designed within the framework of \textit{rational proofs}~\cite{am}. Like the better known \textit{interactive proofs}~\cite{gmr}, rational proofs involve two parties, a \textit{prover} and a \textit{verifier}\footnote{The prover and verifier correspond respectively to the worker and the delegator of verifiable computation}, who talk back and forth. The prover is trying to convince the verifier that a certain theorem is true. At the end of the protocol the verifier will look at the transcript of her interaction with the prover and then decide how much to reward him.
By definition, rational proofs satisfy our goal above: a verifier will pay (in expectation) more for a ``correct proof'' than for an incorrect one. It follows immediately that a utility-maximizing prover would act honestly.
% -- Stress important features of rational proofs: low cc, rounds and verifier's time
So far, rational proofs offered simple and surprisingly efficient delegation schemes without cryptographic assumptions \cite{am1,ratargs,ratsumchecks}: a verifier provided random access to its input may have to only sample a constant number of input bits and run in polylogarithmic time. For  a formal definition of rational proofs the reader is referred to \cite{cg15}.

% == Q1: Expressivity of Rational Proofs ==
\newthought{The Expressivity of Efficient Rational Proofs}
Ideally we would like to be able to delegate any feasible computation, i.e. $\BPP$ and $\P$, or any computation that can be efficiently  \textit{verified} in the classical sense, i.e. $\NP$. Currently however we do not have \textit{efficient}\footnote{The work in \cite{am} describes rational proofs with relatively high budget and verification time for large complexity classes, e.g. $\#\P$.} rational proofs that include any of these important complexity classes in their entirety. The largest practical class admitting efficient rational proofs with low budget is $\NC$, the class of parallelizable languages. The construction is due to~\cite{ratsumchecks} and based on the celebrated interactive proofs for muggles\cite{muggles}.

\begin{question}
	Can we obtain efficient rational proofs for $\P, \BPP$ and $\NP$?
\end{question}

Although we do not settle the question, the work included in this thesis proposal provides new results in this direction.
In particular we provide efficient rational proofs for the class of space-bounded polynomial time computations.

% -- Our results --
\begin{result}[Rational Proofs for Space-Bounded Computations~\cite{cg17}]
	\label{ref:space-bounded}
	If $L$ is decidable in time $T = \poly(n)$ and space $S$, then there exists a rational proof with:
	\begin{itemize}
		\item logarithmic number of rounds and communication complexity equal to $S \log(n)$;
		\item a delegator running in time $S \log(n)$;
		\item a worker running in time $T$.
	\end{itemize}
\end{result}

As already mentioned, rational proofs can offer surprising efficiency compared to classical delegation schemes. How efficient can we expect rational proofs for $\P$ and $\NP$ to be? We offer some lower bounds in that direction. 

\begin{result}[Conditional Lower Bounds~\cite{cg17}]
	It is unlikely that:
	\begin{itemize}
		\item there exist rational proofs for $\P$ with polylogarithmic verification and logarithmic total communication;
		\item there exist rational proofs for $\NP$ with polylogarithmic total communication and polynomial verification.
	\end{itemize}
\end{result}

% More: result for composaition and bounded space computations

% == Q2: Multiple Delegations == 
\newthought{Rationality across Many Costly Delegations}

In the real world computation is costly, in terms of energy, time and money.
In their original definition from ~\cite{am}, rational proof not only can incentivize the honest worker who provides the correct $f(x)$, but they also guarantee that the honest behavior is \textit{individually rational}: the (expected) reward $R_h$ for the honest prover more than compensates the cost $\cost_h$ of computing $f(x)$. As an inequality, $R_h \geq C_h$. 
Thus, rational proofs can guarantee to maximize not only reward but also profit in \textit{a single delegation}.

Let us now assume that we will delegate not only a single $x$, but \textit{multiple inputs}, say two: $x_1$ and $x_2$. Consider a worker who \textit{does not} follow the honest strategy and invests fewer computational resources than the ones necessary to compute $f(x_i)$. Call $\disCost$ the ``lazy'' worker's cost \textit{per input} (we assume it's the same for each). Clearly, by the individual rationality above, this worker cannot obtain more than an honest worker answering correctly on both inputs. Instead our question is: is this ''lazy`` strategy for \textit{two inputs} still worse off when compared with the honest strategy for \textit{a single} input?\footnote{In other words, do I profit more for two sloppy jobs than for one done properly?}
Let us observe that after running the protocol for both $x_1$ and $x_2$ the delegator will reward him with an expected amount $2\disR$. What we want to check is that
\[
	2(\disR - \disCost) <  R_h - \cost_h
\]
Notice that we assumed that costs and reward are \textit{additive} throughout multiple delegation. We call \textit{sequential composability} the property of a protocol ensuring that honesty is rational throughout multiple delegation \footnote{The name refers to composition in cryptographic protocol literature to model a type of multiple invocations. At the same time the adjective ``sequential'' may be misleading in this case: in our model this ``sequential'' composition of rational proofs could as well be parallel and still convey the same notion.}.

\begin{question}\label{q:seq-comp}
	Are rational proofs guaranteed to be sequentially composable, i.e. to incentivize honesty throughout multiple delegations?
\end{question}

We formalize the notion of multiple delegation for rational proof and provide a negative answer to this question.

\begin{result}[~\cite{cg15}]
	Rational proofs such as the ones in \cite{am1} incentivize a worker
	to act ``negligently'' in the case of multiple delegations.
\end{result}

\begin{question}
	How to design rational proofs that are sequentially composable?
\end{question}

In~\cite{cg15} we design sequentially composable rational proofs for arithmetic circuits of logarithmic depth that are sufficiently ``regular'' (see~\cite{cg15} for a formal definition of regularity).

% -- Our results --
\begin{result}[Sequentially Composable Rational Proofs for Logarithmic depth Circuits~\cite{cg15}]\label{res:sq1}
	Under certain cost and structure assumptions, arithmetic circuits with depth $d = O(\log(n))$ admit rational proofs:
	\begin{itemize}
		\item that are sequentially composable;
		\item with $O(d)$ rounds of communication and $O(d \log(n))$ total bits exchanged;
		\item where the delegator runs in time $O(d)$\footnote{Plus a multiplicative factor depending on the uniformity of the circuit.} and the honest worker only has a constant overhead.
	\end{itemize}
\end{result}

\begin{result}[Sequentially Composable Rational Proofs for Space-Bounded Computations~\cite{cg17}]\label{res:sq2}
	Under certain cost assumptions the protocol of Result~\ref{ref:space-bounded} is sequentially composable.
\end{result}

\newthought{Discussion: Assumptions on Cost}
% -- A limitation of our approach: the inner state assumption --
\begin{comment}
Our results above assume a "No Free Computation" assumption stating roughly: ``if a machine uses only a fraction $\gamma$ of the resources expected to compute $f(x)$ then it should expect an error probability of at least roughly $1-\gamma$''\footnote{As a rough analogy, one can think about it as if the function $f$ acted as a random oracle and thus can only be brute-forced.}.
Such an assumption is required for multiple delegations as we need some guarantee on the error probability of a ``lazy'' worker.
This assumption has a few limitations: it is specific to a distribution thus not necessarily supporting arbitrary inputs; it may be strongly tied to a specific model of computation (e.g. circuit vs Turing machines); it is an open problem whether it naturally holds for certain classes of problems.
% XXX: Also add that this may make everything hold only for non-adaptive provers
\end{comment}

Why do we have assumptions on costs in the results above?
Intuitively because costs appear in the inequality right above Question \ref{q:seq-comp}, so we need to quantify them.
First, we need to quantify what the cost of the honest prover is. One way to define that cost is as ``the cost of the fastest algorithm that computes that function''.
And how much would that be? To quantify that notion of ``the cost of the fastest algorithm'', we need to somewhat couple our cost function with some model of computation.
For example, imagine the function we are delegating is sorting $n$ integers. What would you say the cost of the honest prover should be? Approximately $n$? Approximately $n \log(n)$?
It depends on the model. The first answer could be reasonable in a Word-RAM model of computation, the latter in the comparison model, where we are only allowed to compare two integers.
Unfortunately even if we had finally agreed on a specific model of computation, we are not done yet. In fact, by that same inequality above, we need to be able to quantify not only the cost of a (possibly lazy) prover
but also its expected reward. This is related intuitively to its chances of \emph{(i)} guessing $f(x)$ correctly, and \emph{(ii)} responding to the verifier's challenges
in a way similar the honest prover. Therefore, we need assumptions on how likely it is for any prover with a cost $\cost$ to ``approximate'' the honest prover's behavior.
Finally, the notion and the assumptions of cost should be with respect to a distribution of inputs. We need to be able to talk about the \emph{typical} cost of an input since some inputs may have different hardness and thus costs. 

\newcommand{\NoFLAVA}{\textsf{ NoFLAVA }}
What type of assumptions did we make in the results above (Results~\ref{res:sq1} and~\ref{res:sq2})? The two assumptions are relatively abstract and similar in fashion, but they refer to two different models of computations, respectively (arithmetic) circuits and Turing Machines. In the remainder of this section we will try and provide some model-agnostic intuitions on those assumptions, which, for lack of a better name, will collectively refer  to as ``No Free Lunch Assumption For Verifiable Answers'' (from now on, \NoFLAVA for short). For details refer to the original papers,~\cite{cg15} and~\cite{cg17} where this assumption is referred respectively as ``Unique Inner State Assumption'' and ''Hardness of State Guessing Assumption''.

One semi-formal way to summarize \NoFLAVA is the following: 
\begin{displayquote}
	If a machine uses only a fraction $\gamma$ of the cost to correctly compute $f(x)$, then its probability of successfully returning $f(x)$ is (approximately) at most $\gamma$''
\end{displayquote}

As a rough analogy, one can think about it as if the function $f$ were as hard as inverting a random function: the probability of succeeding grows linearly with the inverse of the effort.
This \emph{linearity} is useful because of our additive assumption on costs.
Here is a function where \NoFLAVA holds \textit{when the inputs are uniformly distributed}.
Let $\F$ be a finite field of any size. Let $\alpha_i \in \F$ for $i \in \nrange$ and define $f_{\alpha_1,\dots,\alpha_n}~:~\F^n \to \F^n$ as
$f_{\alpha_1,\dots,\alpha_n}(x_1,\dots,x_n) := (x_1+\alpha_1,\dots,x_n+\alpha_n)$.
Now, assume it costs a unitary operation to jointly read a single input component and compute the corresponding output component. Also assume that outputting anything at component $i$ without reading $x_i$ is completely free. The probability of a worker investing cost $\cost$ to output $f_{\alpha_1,\dots,\alpha_n}(x_1,\dots,x_n)$ is then lower than $\frac{\cost}{n}$. It is easy to see $f_{\alpha_1,\dots,\alpha_n}$ satisfies \NoFLAVA under the uniform distribution and the cost model above.
Finally $f_{\alpha_1,\dots,\alpha_n}$ enjoys one more property implicitly required $\NoFLAVA$ in~\cite{cg15} and ~\cite{cg17}: \textit{non amortizability}. Intuitively, a worker cannot reuse resources across computations. If the (possibly expected) cost of computing $f$ on a single instance is $\cost$, then the cost for a batch evaluation of $m$ instances has to be roughly $m \cost$.

% How to use \NoFLAVA?
% Possibility 1: directly
One way the results of sequential composability in~\cite{cg15} and~\cite{cg17} can be really applicable is to find functions, distributions and cost models for which some variant of \NoFLAVA holds (e.g.$f_{\alpha_1,\dots,\alpha_n}$ above). 
Obviously, this assumption cannot hold for languages or for functions with small ranges: the probability of error would be too little.
% Hard to identify such functions and maybe it does not apply for many of them
It is not clear to this author how such functions can be identified. Some work in this direction can be found in~\cite{fghardness} 
and~\cite{fggr}, notice however that the approaches in those works are for specific functions.
% Brittle: may be too tied to the model of computation
There is another reasons it is unlikely our protocols may be useful if applied ''directly`` to specific functions to which \NoFLAVA may hold. 
The approach may too brittle since the assumptions on cost would be \textit{very dependent on the cost model} (see example on the cost of sorting integers above). This limitation could be partially overcome if we had a theory of relations among different cost models. Such a theory should  provide insights similar how computational complexity connects hardness in different models of computation. However, given how fine-grained in  flavor the notion of cost is in our protocols, this author finds it unlikely a \textit{general} and \textit{generally useful} version of this theory would come about.

Given the limitations above, we argue instead that the results in ~\cite{cg15} and~\cite{cg17} should be used ``indirectly'' in the following sense. In fact, these protocols, would be useful immediately for a large class of functions if we had a result morally equivalent to the following:
\begin{displayquote}
For any function $f$ and input $x$ there exists a function $\hat{f}$ and input $\hat{x}$ such that  $\hat{f}(\hat{x})$ is correlated to $f(x)$ and a strong version of \NoFLAVA  holds for $\hat{f}$ and the distribution of $\hat{x}$.
\end{displayquote}
By ``a strong version of \NoFLAVA'' we informally mean ``not too dependent on the underlying computational model'' (e.g. $f_{\alpha_1,\dots,\alpha_n}$).
If such a result were available we could for example delegate $\hat{f}$ and $\hat{x}$ instead of $f$ and $x$ and apply (some variant of) our results.
Naturally for this to be useful $f(x)$ should extractable very efficiently from $\hat{f}(\hat{x})$.

We leave how to implement this approach as an interesting open problem. Nonetheless we identify \textit{randomized encodings}~\cite{re} and \textit{proofs of useful work}~\cite{ball2017proofs}
as promising approaches in this direction\footnote{For example, the randomized encoding of the $n$-ary NOR is obtained by checking that the function $f_{\alpha_1,\dots,\alpha_n}(x_1,\dots,x_n)$ (when $\alpha_1,\dots,\alpha_n$ are distributed uniformly) returns the zero vector.}.

In the remainder of this proposal we  will use another approach to achieve sequential composability that bypasses many of the limitations of \NoFLAVA applied to our protocols.

\section{Results in Progress}
% -- One way to get beyond that: FG schemes --

% == Transition to FG ==
\newthought{Fine-Grained Cryptography and Verifiable Computation}
What if, given an arbitrary distribution of inputs and an arbitrary computation, we could somehow enforce it to be hard for a worker to be lazy? Heading in this direction we shall use tools from the recently emerging field of fine-grained cryptography~\cite{fgcrypto}\footnote{Notice that none of the assumptions from these works imply the existence of one-way functions in the standard sense.}. A fine-grained cryptographic scheme, intuitively, is a scheme that is secure only against adversaries belonging to certain complexity classes, for example all adversaries running in time $\lambda^2$ for a security parameter $\lambda$. Unless stated otherwise, from now on we will focus on fine-grained adversaries with limited circuit depth under a widely-believed separation assumption\footnote{The assumption is implied by $L \not = \NC^1$.}.

% == Fine-Grained Homomorphic encryption as a basic tool to get to delegation schemes ==
There are  at least two ways to leverage fine-grained cryptography for rational verifiable computation:
\begin{enumerate}
	\item Consider a function $f$ and an input distribution where \NoFLAVA does not necessarily hold. We could hope to build a compiler that, given as input, e.g. a circuit $\circuit$, it converts it to a circuit $\circuit'$ where the assumption holds.
	This is similar in spirit to the ``indirect approach'' outlined above. %One way to think about this approach is by ensuring that every gate $g \in \circuit$ is replaced with a proof-of-useful work~\cite{upow} for $g$.
	% NB: Could we use randomized encodings for similar ideas?
	\item As an alternative we could actually build a delegation scheme that is secure against fine-grained adversaries. We can think of the resulting fine-grained scheme as a sequentially composable rational proof. In fact, suppose that any prover making the verifier accept\footnote{with high probability.} an incorrect answer had to run in time $n^{2+\epsilon}$ with $\epsilon > 0$. If we assume cost equals running time and that the honest prover's running time is $n^2$, then it rational to be honest. 
	%and if the fine-grained definition of security is robust enough this provides 
	%robust sequential composability. For example 	
%	can think of this would be a sequentially composable rational proof\footnote{Under certain cost assumptions.}.
\end{enumerate}

The results described in the remained of this document will follow the second approach.
We leave how to tackle the first approach as an interesting open problem.

\begin{question}
	What functions admit efficient fine-grained delegation schemes?
\end{question}


% == Results on non-interactive Delegation schemes ==

Recall that in $\AC^0[2]$ is the class of functions computable by constant-depth circuits with
parity gates. It is known that $\AC^0[2] \subsetneq \NC^1$.

Our results below will involve the following complexity class, which we will denote by $\ACzts$. This class includes languages such as $n$-ary AND, 3CNF formula evaluation and polynomials on $\F_2$ of constant depth. 

\begin{definition}
The class $\ACzts$ is the class of languages computable by $\ACzts$ circuit with only a constant number of AND/OR gates of $\omega(1)$ fan-in.
\end{definition}

\begin{result}
	\label{res:delegation}
	If $L \not = \NC^1$, there exists a non-interactive delegation scheme for computations in $\ACzts$ and secure against adversaries in $\NC^1$ where:
	\begin{itemize}
		\item The delegator is in $\TC^0$ with wire size $O(sn)$ (where $s$ is a parameter such that soundness  is $2^{-s}$);
		\item The worker is in $\ACzt$ and has gate size $O(s|\circuit|)$, where $|C|$ is the gate size of the circuit;
		\item Both the input and output remain private to the delegator.
	\end{itemize} 
\end{result}

Our techniques are heavily based on the ones in~\cite{ckv10} that obtain non-interactive delegation schemes from \textit{homomorphic encryption}. We observe the reduction in their proof of security holds even in a fine-grained model.

\newthought{Fine-Grained Somewhat Homomorphic Encryption}
A Somewhat Homomorphic Encryption Scheme (SHE) can be thought as a public-key encryption scheme endowed with a special algorithm $\Eval$. Let $\cal{F}$ be a family of functions. Informally, given in input a function $f \in \cal{F}$ and a encryption $c$ of a plaintext $x$, $\Eval(f, c)$ returns an encryption of $f(x)$. If $f$ is arbitrary we say the scheme is Fully-Homomorphic (FHE).

As a stepping stone  for Result~\ref{res:delegation} we  also show how to transform the fine-grained public-key encryption scheme in~\cite{fgcrypto} into a somewhat homomorphic one\footnote{Their scheme is already additively homomorphic, but not multiplicatively.}. Our technique follows the relinearization approach introduced in~\cite{fhe-lwe}. We believe this result to be of independent interest.

\begin{result}
	If $L \not = \NC^1$, there exist an homomorphic encryption scheme secure against $\NC^1$ adversaries where:
	\begin{itemize}
		\item Encryption is in $\NC^0$ and key generation is in $\ACzts$;
		\item If $f \in \ACzts$ then we can evaluate $f$ in $\ACzt$ and decrypt\footnote{With negligible error probability} the result in $\TC^0$.
	\end{itemize}
\end{result}


}


% -- From here on the new stuff --

% Overview and motivation

The problem of efficiently checking the correctness of a computation performed by an untrusted party has been central in Complexity Theory for the last 30 years since the introduction of Interactive Proofs by Babai and Goldwasser, Micali and Rackoff \cite{babai,gmr}. 

{\sf Verifiable Outsourced Computation} is now a very active research area in Cryptography and Network Security (see \cite{wb15} for a survey) with the aim to design protocols where it is impossible (under suitable cryptographic assumptions) for a provider to ``cheat" in the above scenarios. While much progress has been done in this area, we are still far from solutions that can be deployed in practice. 

Part of the reason is that Cryptographers consider a very strong adversarial model that prevents {\sf any} adversary from cheating. A different approach is to restrict ourselves to {\em rational adversaries}, whose motivation is not just to disrupt the protocol or computation, but simply to maximize a well defined utility function (e.g. profit).
%\subsection{Rational Proofs}

A different approach is to consider a model where ``cheating" might actually be possible, but the provider would have no motivation to do so. In other words while cryptographic protocols prevent {\sf any} adversary from cheating, one considers protocols that work against {\sf rational} adversaries whose motivation is to maximize a well defined utility function. 

We investigate this approach through two ``lenses'': \textit{(i)} \textit{rational proofs}, a variant of interactive proofs where provers lose (in economic terms) whenever offering a wrong proof; \textit{(ii)} \textit{fine-grained protocols}, a model where parties's resources are assumed to be limited \footnote{In a more specific sense than the usual ``probabilistic polynomial time''.} and cheating is possible only through a larger amount of resources (e.g. cheating requires time $\lambda^3$, for some parameter $\lambda$, but participating parties are assumed to be able to run in time at most $\lambda^2$). The connections between rationality and the latter model will be explored in the subsequent sections and in Chapter $\ref{chap:FG}$.

\subsection{Rational Proofs}

% NB: There is a little more stuff in the intro.tex-s. It is unclear it may help.
In the first two part of this thesis (Chapters \ref{chap:RP-expr} and \ref{chap:RP-seq}) we use the concept of {\sf Rational Proofs} introduced by Azar and Micali in \cite{am} and refined in a subsequent paper \cite{am1}. 

In a Rational Proof, given a function $f$ and an input $x$, the server returns the value $y=f(x)$, and (possibly) some auxiliary information, to the client. The client will in turn 
pay the server for its work with a reward which is a function of the messages 
sent by the server and some randomness chosen by the client.  The crucial 
property is that this reward is maximized in expectation when the server 
returns the correct value $y$. Clearly a rational prover who is only interested 
in maximizing his reward, will always answer correctly. 

The most striking feature of Rational Proofs is their simplicity. For example in \cite{am}, Azar and Micali show {\sf single-message} Rational Proofs for any problem in $\#P$, where an (exponential-time) prover convinces a (poly-time) verifier of the number of satisfying assignment of a Boolean formula. 

For the case of "real-life" computations, where the Prover is polynomial and the Verifier is as efficient as possible, Azar and Micali in \cite{am1} show $d$-round Rational Proofs for functions computed by (uniform) Boolean circuits of depth $d$, for $d=O(\log n)$ (which can be collapsed to a single round under some well-defined computational assumption as shown in \cite{ratargs}). The problem of rational proofs for any polynomial-time computable function remains tantalizingly open. 


Recent work \cite{ratsumchecks} shows how to obtain Rational Proofs with sublinear verifiers for languages in $\NC$. Recalling that $\L \subseteq \NL \subseteq \NC_2$, one can use the protocol  in \cite{ratsumchecks} to verify a logspace polytime computation (deterministic or nondeterministic) in $O(\log^2 n )$ rounds and $O(\log^2 n )$ verification.

The work by Chen et al. \cite{chen2016rational} focuses on rational proofs with multiple provers and the related class $\MRIP$ of languages decidable by a polynomial verifier interacting with an arbitrary number of provers. Under standard complexity assumptions, $\MRIP$ includes languages not decidable by a verifier interacting only with one prover. The class $\MRIP$ is equivalent to $\EXP^{||\NP}$.



\subsection{Contributions for Rational Proofs}

\subsubsection{Expressivity.}
We present new protocols for the verification of {\em space-bounded polytime computations} against a rational adversary. More specifically, consider a language $L \in \DTISP(T(n), S(n))$, i.e. recognized by a deterministic Turing Machine $M_L$ which runs in time $T(n)$ and space $S(n)$. 
In Section \ref{sec:rp-dtisp} we construct a protocol where a rational prover can
convince the verifier that $x \in L$ or $x \notin L$ with the following properties: 
\begin{itemize}
	\item The verifier runs in time $O(S(n) \log n)$
	\item The protocol has $O(\log n)$ rounds and communication complexity $O(S(n) \log n)$
	\item The prover simply runs $M_L(x)$ 
	%and stores all the intermediate configurations (i.e. requires space $O(S(n) T(n))$
\end{itemize}
%Our protocol can be proven to correctly incentivize a prover in {\bf both} the stand-alone model of \cite{am} and the sequentially composable definition of \cite{cg15}. This is the first protocol which is sequentially composable for a well-defined complexity class. 

For the case of ``real-life" computations (i.e. poly-time computations verified by a ``super-efficient" verifier) we 
note that for computations in sublinear space our general results yields a protocol in which the verifier is sublinear-time. Our protocols is the first rational proof for $\SC$ (also known as $\DTISP(\poly(n), \polylog(n))$) with polylogarithmic verification and logarithmic rounds. 
%Moreover, our results provide the first efficient rational proof for the 
%non-deterministic class $\NSC = \NTISP(\poly(n), \polylog(n) )$ . 

To compare this with the results in \cite{ratsumchecks}, we note that it is believed that $\NC \not = \SC$ and that the two classes are actually incomparable (see \cite{SCcompleteness} for a discussion). For these computations our results compare
favorably to the one in \cite{ratsumchecks} in at least one aspect: our protocol requires $O(\log n )$ rounds and has the same verification complexity.

We present several extensions of our main result:
\begin{itemize}
	
	\item Our main protocol can be extended to the case of space-bounded randomized computations using Nisan's 
	pseudo-random generator \cite{nisan1992pseudorandom} to derandomize the computation. 
	\item We also present a different protocol that works for BPNC (bounded error randomized NC) where the Verifier runs in polylog time (note that this class is not covered by our result since we do not know how to express NC with a polylog-space computation). This protocol uses in a crucial way a new {\em composition theorem} for rational proofs presented in this work and can be of independent interest. 
	\item Finally, we present lower bounds (i.e. conditional impossibility results) for Rational Proofs for various complexity classes.
\end{itemize}



\subsubsection{Repeated Executions and Costly Computation.}
Motivated by the problem of volunteer computation, our first
result is to show that the definition of Rational Proofs in \cite{am,am1} does not satisfy a basic compositional property which would make them applicable 
in that scenario. 
Consider the case where a large number of "computation problems" are outsourced. Assume that solving each problem takes time $T$. Then in a time interval of length $T$, the honest prover can only solve and receive the reward for a single problem. On the other hand a dishonest prover, can answer up to $T$ problems, for example by answering at random, a strategy that takes $O(1)$ time. To assure that answering correctly is a rational strategy, we 
need that at the end of the $T$-time interval the reward of the honest prover be larger than the reward of the dishonest one. But this is not necessarily the case: for some of the protocols in \cite{am,am1,ratargs} we can show that a ``fast" incorrect answer is more remunerable for the prover, by allowing him to solve more problems and collect more rewards.

The next questions, therefore, was to come up with a definition and a protocol that achieves rationality both in the stand-alone case, and in the composition
described above.  We first present an enhanced definition of Rational Proofs that removes the economic incentive  for the strategy of fast incorrect answers, and then we present a protocol that achieves it for the case of some (uniform) bounded-depth circuits.
Next, we design a $d$-rounds rational proof for sufficiently ``regular'' arithmetic circuit of depth $d = O(\log{n})$
with sublinear verification. We show, that under certain cost assumptions, our scheme is sequentially composable,
i.e. it can be used to delegate multiple inputs. We finally show that our scheme for space-bounded computations from Section \ref{sec:rp-dtisp} is also 
sequentially composable under certain cost assumptions.


\subsection{Comparison with Other Prior Work}
\label{sec:prior}

{\sc Other Decision-Theoretic Frameworks.}
An earlier work in the line of ``rational verifiable computation'' is \cite{b08} where the authors describe a system based on a scheme of rewards [resp. penalties] that the client assesses to the server for computing the function correctly [resp. incorrectly]. However in this system checking the computation may require re-executing it, something that the client does only on a randomized subset of cases, hoping that the penalty is sufficient to incentivize the server to perform honestly. Morever the scheme might require an "infinite" budget for the rewards, and has no way to "enforce" payment of penalties from cheating servers. For these reasons the best application scenario of this approach is the incentivization of volunteer computing schemes (such as SETI@Home or Folding@Home), where the rewards are non-fungible "points" used for "social-status". 

Because verification is performed by re-executing the computation, in this approach the client is "efficient" (i.e. does "less" work than the server) only in an 
amortized sense, where the cost of the subset of executions verified by the client is offset by the total number of computations performed by the server. This implies that the server must perform many executions for the client. 

{\sc Interactive Proofs.}
Obviously a ``traditional" interactive proof (where security holds against any adversary, even a computationally unbounded one) would work in our model. In this case the most relevant result is 
the recent independent work in \cite{rrr16} that presents breakthrough protocols for the deterministic (and randomized) restriction of the class of language we consider. If $L$ is a language which is recognized by a deterministic (or randomized) Turing Machine $M_L$ which runs in time $T(n)$ and space $S(n)$, then their protocol has the following properties: 
\begin{itemize}
	\item The verifier runs in 
	$O(\poly(S(n)) + n \cdot\polylog(n))$ time;
	\item The prover runs in polynomial time;
	\item The protocol runs in {\em constant} rounds, with communication complexity $O({\sf poly}(S(n)n^{\delta})$ for a constant $\delta$.
\end{itemize}
Apart from round complexity (which is the impressive breakthrough of the result in \cite{rrr16}) our protocols fares better in all other categories. Note in particular that a sublinear space computation does not necessarily yield a sublinear-time verifier in 
\cite{rrr16}. On the other hand, we stress that our protocol only considers weaker rational adversaries. 

\medskip
\noindent{\sc Computational Arguments.}
There is a large class of protocols for {\em arguments} of correctness (e.g. \cite{ggp10,ggpr13,krr14}) even in the rational model \cite{ratargs,ratsumchecks}. Recall that in an argument, security is achieved only against computationally bounded prover. In this case even single round solutions can be achieved. We will consider a variant of this model in Chapter \ref{chap:FG}.

\medskip
\noindent
{\sc Computational Decision Theory.}
Other works in theoretical computer science have studied the connections between cost of computation and utility in decision problems.
The work in \cite{halpern2011don} proposes a framework for \emph{computational decision problems}, where the Decision Maker's (DM) utility depends on the algorithm chosen for computing its strategy.
The Decision Maker runs the algorithm, assumed to be a Turing Machine, on the input to the computational decision problem.
The output of the algorithm determines the DM's strategy. 
Thus the choice of the DM reduces to the choice of a Turing Machine from a certain space. The DM will have beliefs on the running time (cost) of each Turing machine. The actual cost of running the chosen TM will affect the DM's reward.
Rational proofs with costly computation could be formalized in the language of \emph{computational decision problems} in \cite{halpern2011don}. There are similarities between the approach in this
work and that in \cite{halpern2011don}, as both take into account the cost of computation in a decision problem.

\subsection{Future Directions}

Our work leaves open a series of questions:
\begin{itemize}
	\item What is the relationship between scoring rule based protocols vs weak interactive proofs? Our work seems to indicate that the latter technique is more powerful (our work shows an example of a class of language which is not known to be recognizable using scoring rules and that scoring rules seem inherently insecure in a composable setting). Is it possible to show, however, that a scoring-rule based protocol can be transformed into a weak interactive proof (without a substantial loss of efficiency) therefore showing that it is enough to focus on the latter?
	
	\item Can we build efficient rational proofs for arbitrary poly-time computations, where the verifier runs in sub-linear, or even in linear, time? Even in the standalone model of \cite{am}?
	
	\item Our proof of sequential composability considers only non-adaptive adversaries, and enforces this condition by the use of timing assumptions or computationally bounded provers. Is it possible to construct protocols that are secure against adaptive adversaries? Or is it possible to relax the timing assumption to something less stringent than what is required in our protocol?
	
	\item It would be interesting to investigate the connection between the model of Rational Proofs and the work on Computational Decision Theory in  \cite{halpern2011don}. In particular it would be interesting to look at realistic cost models that could affect the choice of strategy by the prover particularly in the sequentially composable model. 
\end{itemize}



%\textbf{1}

%The problem of securely outsourcing data and computation has received widespread attention due to the rise of {\em cloud computing:} a paradigm where businesses lease computing resources from a service (the {\em cloud provider}) rather than maintain their own computing infrastructure.  Small mobile devices, such as smart phones and netbooks, also rely on remote servers to store and perform computation on data that is too large to fit in the device. 

It is by now well recognized that these new scenarios have introduced new security problems that need to be addressed. When data is stored remotely, outside our control, how can we be sure of its integrity? Even more interestingly, how do we check that the results of outsourced computation on this remotely stored data are correct. And how do perform these tests while preserving the efficiency of the client (i.e. avoid retrieving the whole data, and having the client perform the computation) which was the initial reason data and computations were outsourced. 

{\sf Verifiable Outsourced Computation} is a very active research area in Cryptography and Network Security (see \cite{wb15} for a survey), with the goal of designing protocols where it is impossible (under suitable cryptographic assumptions) for a provider to "cheat" in the above scenarios. While much progress has been done in this area, we are still far from solutions that can be deployed in practice. 

A different approach is to consider a model where "cheating" might actually be possible, but the provider would have no motivation to do so. In other words while cryptographic protocols prevent {\sf any} adversary from cheating, one considers protocols that work against {\sf rational} adversaries whose motivation is to maximize a well defined utility function. 


\smallskip
\noindent
{\sc Previous Work.}
An earlier work in this line is \cite{b08} where the authors describe a system based on a scheme of rewards [resp. penalties] that the client assesses to the server for computing the function correctly [resp. incorrectly]. However in this system checking the computation may require re-executing it, something that the client does only on a randomized subset of cases, hoping that the penalty is sufficient to incentivize the server to perform honestly. Morever the scheme might require an "infinite" budget for the rewards, and has no way to "enforce" payment of penalties from cheating servers. For these reasons the best application scenario of this approach is the incentivization of volunteer computing schemes (such as SETI@Home or Folding@Home), where the rewards are non-fungible "points" used for "social-status". 

Because verification is performed by re-executing the computation, in this approach the client is "efficient" (i.e. does "less" work than the server) only in an 
amortized sense, where the cost of the subset of executions verified by the client is offset by the total number of computations performed by the server. This implies that the server must perform many executions for the client. 

Another approach, instead, is the concept of {\sf Rational Proofs} introduced by Azar and Micali in \cite{am} and refined in subsequent papers \cite{am1,rosen}. This model captures, more accurately, real-world financial "pay-for-service" transactions, typical of cloud computing contractual arrangements, and security
holds for a single "stand-alone" execution.

In a Rational Proof, given a function $f$ and an input $x$, the server returns the value $y=f(x)$, and (possibly) some auxiliary information, to the client. The client will in turn 
pay the server for its work with a reward which is a function of the messages 
sent by the server and some randomness chosen by the client.  The crucial 
property is that this reward is maximized in expectation when the server 
returns the correct value $y$. Clearly a rational prover who is only interested 
in maximizing his reward, will always answer correctly. 

The most striking feature of Rational Proofs is their simplicity. For example in \cite{am}, Azar and Micali show {\sf single-message} Rational Proofs for any problem in $\#P$, where an (exponential-time) prover convinces a (poly-time) verifier of the number of satisfying assignment of a Boolean formula. 

For the case of "real-life" computations, where the Prover is polynomial and the Verifier is as efficient as possible, Azar and Micali in \cite{am1} show $d$-round Rational Proofs for functions computed by (uniform) Boolean circuits of depth $d$, for $d=O(\log n)$ (which can be collapsed to a single round under some well-defined computational assumption as shown in \cite{rosen}). The problem of rational proofs for any polynomial-time computable function remains tantalizingly open. 

\smallskip
\noindent
{\sc Our Results.} 
Motivated by the problem of volunteer computation, our first
result is to show that the definition of Rational Proofs in \cite{am,am1} does not satisfy a basic compositional property which would make them applicable 
in that scenario. 
Consider the case where a large number of "computation problems" are outsourced. Assume that solving each problem takes time $T$. Then in a time interval of length $T$, the honest prover can only solve and receive the reward for a single problem. On the other hand a dishonest prover, can answer up to $T$ problems, for example by answering at random, a strategy that takes $O(1)$ time. To assure that answering correctly is a rational strategy, we 
need that at the end of the $T$-time interval the reward of the honest prover be larger than the reward of the dishonest one. But this is not necessarily the case: for some of the protocols in \cite{am,am1,rosen} we can show that a "fast" incorrect answer is more remunerable for the prover, by allowing him to solve more problems and collect more rewards.

The next questions, therefore, was to come up with a definition and a protocol that achieves rationality both in the stand-alone case, and in the composition
described above.  We first present an enhanced definition of Rational Proofs that removes the economic incentive  for the strategy of fast incorrect answers, and then we present a protocol that achieves it for the case of some (uniform) bounded-depth circuits.
 


%\textbf{2}

%Consider the problem of {\sf Outsourced Computation} where a computationally  ``weak'' client hires a more  ``powerful'' server to store data and perform computations on its behalf. This paper is concerned with the problem of designing outsourced computation schemes that incentivize the server to perform correctly the tasks assigned by the client. 

The rise of the {\em cloud computing} paradigm where business do not maintain their own IT infrastructure, but rather hire  ``providers'' to run it, has brought this problem to the forefront of the research community. The goal is to find solutions that are efficient and feasible in practice for problems such as: How do we check the integrity of data that is stored remotely? How do we check computations performed on this remotely stored data? How can a client do this in the most efficient way possible? Or even more generally, how do we incentivize parties to perform correctly in such scenarios?


\subsection{Complexity Theory and Cryptography}

The problem of efficiently checking the correctness of a computation performed by an untrusted party has been central in Complexity Theory for the last 30 years since the introduction of Interactive Proofs by Babai and Goldwasser, Micali and Rackoff \cite{babai,gmr}. 

{\sf Verifiable Outsourced Computation} is now a very active research area in Cryptography and Network Security (see \cite{wb15} for a survey) with the aim to design protocols where it is impossible (under suitable cryptographic assumptions) for a provider to ``cheat" in the above scenarios. While much progress has been done in this area, we are still far from solutions that can be deployed in practice. 

Part of the reason is that Cryptographers consider a very strong adversarial model that prevents {\sf any} adversary from cheating. A different approach is to restrict ourselves to {\em rational adversaries}, whose motivation is not just to disrupt the protocol or computation, but simply to maximize a well defined utility function (e.g. profit).

\subsection{Rational Proofs}

In our work we use the concept of {\sf Rational Proofs} introduced by Azar and Micali in \cite{am} and refined in a subsequent paper \cite{am1}. 

In a Rational Proof, given a function $f$ and an input $x$, the server returns the value $y=f(x)$, and (possibly) some auxiliary information, to the client. The client will in turn 
pay the server for its work with a reward which is a function of the messages 
sent by the server and some randomness chosen by the client.  The crucial 
property is that this reward is maximized in expectation when the server 
returns the correct value $y$. Clearly a rational prover who is only interested 
in maximizing his reward, will always answer correctly. 

The most striking feature of Rational Proofs is their simplicity. For example in \cite{am}, Azar and Micali show {\sf single-message} Rational Proofs for any problem in $\#P$, where an (exponential-time) prover convinces a (poly-time) verifier of the number of satisfying assignment of a Boolean formula. 

For the case of  ``real-life" computations, Azar and Micali in \cite{am1} consider the case of efficient provers (i.e. poly-time) and ``super-efficient" (log-time) verifiers and present $d$-round Rational Proofs for functions computed by (uniform) Boolean circuits of depth $d$, for $d=O(\log n)$. 
% In this case the Verifier runs in logarithmic time.

Recent work \cite{ratsumchecks} shows how to obtain Rational Proofs with sublinear verifiers for languages in $\NC$. Recalling that $\L \subseteq \NL \subseteq \NC_2$, one can use the protocol  in \cite{ratsumchecks} to verify a logspace polytime computation (deterministic or nondeterministic) in $O(\log^2 n )$ rounds and $O(\log^2 n )$ verification.

The work by Chen et al. \cite{chen2016rational} focuses on rational proofs with multiple provers and the related class $\MRIP$ of languages decidable by a polynomial verifier interacting with an arbitrary number of provers. Under standard complexity assumptions, $\MRIP$ includes languages not decidable by a verifier interacting only with one prover. The class $\MRIP$ is equivalent to $\EXP^{||\NP}$.


\subsection{Repeated Executions with a Budget}
%\medskip
%\noindent
%{\sc Compositions of Rational Proofs.}
In \cite{cg15} 
%the authors 
we present a critique of the rational proof model in the case of ``repeated executions with a budget". This model arises in the context of ``volunteer computations" (\cite{seti,folding}) where many computational tasks are outsourced and provers compete in solving as many as possible to obtain rewards. In this scenario assume that a prover has a certain budget $B$ of ``computational effort": how can one  guarantee that the rational strategy is to provide the correct answer in {\em all} the proof he provides? The notion of rational proof guarantees that if the prover engages in a single rational proof then it is in his best interest to provide the correct output. But in \cite{cg15} 
%the authors
we show that in the presence of many computations, it might be more profitable for the prover to use his budget $B$ to provide many incorrect answers than to provide a single correct answer. That's because incorrect (e.g. random) answers are ``cheaper" to compute than the correct one and with the same budget $B$ the prover can provide many of them while the entire budget might be necessary to solve a single problem correctly. If the difference in reward between correct and incorrect answers is not high enough then many incorrect answers may be more profitable and a rational prover will choose that strategy, and indeed this is the case for many of the protocols in \cite{am,am1,ratargs,ratsumchecks}. 

In \cite{cg15} we put forward a stronger notion of {\em sequentially composable rational proofs} which avoids the above problem and guarantees that the rational strategy is always the one to provide correct answers. We also presented sequentially composable rational proofs, but only for some ad-hoc cases, and were not able to generalize them to well-defined complexity classes. 

%for a subset of bounded-depth circuit computations. 

\subsection{Our Contribution}

This paper presents new protocols for the verification of {\em space-bounded polytime computations} against a rational adversary. More specifically consider a language $L \in \DTISP(T(n), S(n))$, i.e. recognized by a deterministic Turing Machine $M_L$ which runs in time $T(n)$ and space $S(n)$. 
We construct a protocol where a rational prover can
convince the verifier that $x \in L$ or $x \notin L$ with the following properties: 
\begin{itemize}
	\item The verifier runs in time $O(S(n) \log n)$
	\item The protocol has $O(\log n)$ rounds and communication complexity $O(S(n) \log n)$
	\item The prover simply runs $M_L(x)$ 
	%and stores all the intermediate configurations (i.e. requires space $O(S(n) T(n))$
\end{itemize}
Our protocol can be proven to correctly incentivize a prover in {\bf both} the stand-alone model of \cite{am} and the sequentially composable definition of \cite{cg15}. This is the first protocol which is sequentially composable for a well-defined complexity class. 

For the case of ``real-life" computations (i.e. poly-time computations verified by a ``super-efficient" verifier) we 
note that for computations in sublinear space our general results yields a protocol in which the verifier is sublinear-time. More specifically, we introduce the first rational proof for $\SC$ (also known as $\DTISP(\poly(n), \polylog(n))$) with polylogarithmic verification and logarithmic rounds. 
%Moreover, our results provide the first efficient rational proof for the 
%non-deterministic class $\NSC = \NTISP(\poly(n), \polylog(n) )$ . 

To compare this with the results in \cite{ratsumchecks}, we note that it is believed that $\NC \not = \SC$ and that the two classes are actually incomparable (see \cite{SCcompleteness} for a discussion). For these computations our results compare
favorably to the one in \cite{ratsumchecks} in at least one aspect: our protocol requires $O(\log n )$ rounds and has the same verification complexity.

We present several extensions of our main result:
\begin{itemize}

	\item Our main protocol can be extended to the case of space-bounded randomized computations using Nisan's 
	pseudo-random generator \cite{nisan1992pseudorandom} to derandomize the computation. 
	\item We also present a different protocol that works for BPNC (bounded error randomized NC) where the Verifier runs in polylog time (note that this class is not covered by our result since we do not know how to express NC with a polylog-space computation). This protocol uses in a crucial way a new {\em composition theorem} for rational proofs which we present in this paper and can be of independent interest. 
	\item We discuss the notion of Rational Arguments (where the adversary is assumed to be both rational and computationally bounded, introduced in \cite{ratargs}) and show that it is achievable for all interesting complexity classes.
	\item Finally we present lower bounds (i.e. conditional impossibility results) for Rational Proofs for various complexity classes.
\end{itemize}

\subsection{The Landscape of Rational Proof Systems}

Rational Proof systems can be divided in roughly two categories, both of them presented in the original work \cite{am}. 

\medskip
\noindent
{\sc Scoring Rules.}
The more ``novel" approach in \cite{am} uses {\em scoring rules} to compute the reward paid by the verifier to the prover. A scoring rule is used to asses the ``quality" of a prediction of a randomized process. Assume that the prover declares that a certain random variable $X$ follows a particular probability distribution $D$. The verifier runs an ``experiment" (i.e. samples the random variable in question) and computes a ``reward" based on the distribution $D$ announced by the prover and the result of the experiment. A scoring rule is maximized if the prover announced the real distribution followed by $X$. The novel aspect of many of the protocols in \cite{am} was how to cast the computation of $y=f(x)$ as the announcement of a certain distribution $D$ that could be tested efficiently by the verifier and rewarded by a scoring rule. 

A simple example is the protocol for $\#P$ in \cite{am} (or its ``scaled-down" version for Hamming weight described more in detail in Section~\ref{sec:example}). Given a Boolean formula $\Phi(x_1,\ldots,x_n)$ the prover announces the number $m$ of satisfying assignments. This can be interpreted as the prover announcing that if one chooses an assignment at random it will be a satisfying one with probability $m \cdot 2^{-n}$. The verifier then chooses a random assignment and checks if it satisfies $\Phi$ or not and uses $m$ and the result of the test to compute the reward via a scoring rule. Since the scoring rule is maximized by the announcement of the correct $m$, a rational prover will announce the correct value. 

As pointed out in \cite{cg15} the problem with the scoring rule approach is that the reward declines slowly as the distribution announced by the Prover becomes more and more distant from the real one. The consequence is that incorrect results still get a substantial reward, even if not a maximal one. Since those incorrect results can be computed faster than the correct one, a Prover with ``budget" $B$ might be incentivized to produce many incorrect answers instead of a single correct one. All of the scoring rule based protocols in \cite{am,am1,ratargs,ratsumchecks} suffer from this problem. 

\medskip
\noindent
{\sc Weak Interactive Proofs.}
The definition of rational proofs requires that the expected reward is maximized for the honest prover. This definition can be made stronger (as done explicitly in \cite{ratargs}) and require a that every systematically dishonest prover would incur a polynomial loss (this property is usually described in terms of a \textit{noticeable reward gap}). As discussed above, the elegant device of scoring rules is the basis for most rational proof protocols in literature, some of which achieve noticeable reward gap. Another simple way in which we can obtain this stronger type of rational proof is the following. Imagine having a test where the prover can be caught cheating with ``low", but non-negligible probability, e.g. $n^{-k}$ for some $k \in \naturals$.
We will informally call this test a \textit{weak interactive proof}\footnote{This is basically the {\em covert adversary} model for multiparty computation introduced in \cite{AL10}.}. Indeed for such proofs we can always pay a fixed reward $R$ to the prover unless we catch him cheating in which case we pay $0$. These are rational proofs since obviously the expected reward of the prover is maximized by the honest behavior. Some of the proofs in \cite{am} and the proofs in \cite{cg15} are weak interactive proofs. Those proofs also turn out to be secure in the sequential model of \cite{cg15} (under appropriate assumptions). 

The protocols in this work are weak interactive proof, which is why we can prove them to be sequentially composable\footnote{
	One exception is the protocol for BPNC, which depends on the underlying protocol for (deterministic) NC. If we use the one in \cite{ratsumchecks}, then the resulting protocol uses scoring rules and is not sequentially composable. However an alterative protocol for NC can be used, based on our work in \cite{cg15}, which is a weak interactive proof and can be proven sequentially composable.}.


\medskip
\noindent
{\sc Scoring rules vs. weak interactive proofs.}
Comparing approaches based on scoring rules and weak interactive proofs the following two questions come up: 
\begin{itemize}
	\item Does one approach systematically lead to more efficient rational proofs (in terms of rounds, communication and verifying complexity) than the other? 
	\item Is one approach more suitable for sequential composability than the other?
\end{itemize}
We do not have a precise answer to the above questions, which we believe are interesting open problems to consider. However we can make the following statements. 

Regarding the first question: in the context of ``stand-alone" (non sequential) rational proofs it is not clear which approach is more powerful. We know that for every language class known to admit a scoring rule based protocol we also have a weak interactive proof with similar performance metrics (i.e. number of rounds, verifier efficiency, etc.). The result in this paper is the first example of a language class for which we have rational proofs based on weak interactive proofs but no example of a scoring rule based protocol exist\footnote{
	We stress that in this comparison we are interested in protocols with similar efficiency parameters. For example, the work in \cite{am} presents several large complexity classes for which we have rational proofs. However, these protocols require a polynomial verifier and do not obtain a noticeable reward gap.}.
This suggests that the weak interactive proof approach might be the more powerful technique. It would be interesting to prove that all rational proofs are indeed weak interactive proofs: i.e. that given a rational proof with certain efficiency parameters, one can construct a weak interactive proof with ``approximately" the same parameters. This question is left as future work.

On the issue of sequential composability, we have already proven in \cite{cg15} that some rational proofs based on scoring rules (such as Brier's scoring rule) are not  sequentially composable. 
This problem might be inherent at least for scoring rules that pay a substantial reward to incorrect computations. What we can say is that all known sequentially composable proofs are based on weak interactive proofs (\cite{cg15}, \cite{am1}\footnote{The construction in Theorem $5.1$ in \cite{am1} is shown to be sequentially composable in \cite{cg15}.} and this work). Again it would be interesting to prove that this is required, i.e. that all sequentially composable rational proofs are weak interactive proofs. 
%An important caveat to keep in mind, however, is that the notion of sequential 
%composability is to some extent flexible and depends
%on the cost model one assumes. % XXX: Maybe say something about the cost model in this work and in the previous one here?

% XXX: Rational arguments here.

\subsection{Other Related  Work}
\label{sec:prior}

{\sc Interactive Proofs.}
Obviously a ``traditional" interactive proof (where security holds against any adversary, even a computationally unbounded one) would work in our model. In this case the most relevant result is 
the recent independent work in \cite{rrr16} that presents breakthrough protocols for the deterministic (and randomized) restriction of the class of language we consider. If $L$ is a language which is recognized by a deterministic (or randomized) Turing Machine $M_L$ which runs in time $T(n)$ and space $S(n)$, then their protocol has the following properties: 
\begin{itemize}
	\item The verifier runs in 
	$O(\poly(S(n)) + n \cdot\polylog(n))$ time;
	\item The prover runs in polynomial time;
	\item The protocol runs in {\em constant} rounds, with communication complexity $O({\sf poly}(S(n)n^{\delta})$ for a constant $\delta$.
\end{itemize}
Apart from round complexity (which is the impressive breakthrough of the result in \cite{rrr16}) our protocols fares better in all other categories. Note in particular that a sublinear space computation does not necessarily yield a sublinear-time verifier in 
\cite{rrr16}. On the other hand, we stress that our protocol only considers weaker rational adversaries. 

\medskip
\noindent{\sc Computational Arguments.}
There is a large class of protocols for {\em arguments} of correctness (e.g. \cite{ggp10,ggpr13,krr14}) even in the rational model \cite{ratargs,ratsumchecks}. Recall that in an argument, security is achieved only against computationally bounded prover. In this case even single round solutions can be achieved. We do not consider this model in this paper, except in Section~\ref{sec:scproof} as one possible option to obtain sequential composability. 

\medskip
\noindent
{\sc Computational Decision Theory.}
Other works in theoretical computer science have studied the connections between cost of computation and utility in decision problems.
The work in \cite{halpern2011don} proposes a framework for \emph{computational decision problems}, where the Decision Maker's (DM) utility depends on the algorithm chosen for computing its strategy.
The Decision Maker runs the algorithm, assumed to be a Turing Machine, on the input to the computational decision problem.
The output of the algorithm determines the DM's strategy. 
Thus the choice of the DM reduces to the choice of a Turing Machine from a certain space. The DM will have beliefs on the running time (cost) of each Turing machine. The actual cost of running the chosen TM will affect the DM's reward.
Rational proofs with costly computation could be formalized in the language of \emph{computational decision problems} in \cite{halpern2011don}. There are similarities between the approach in this
work and that in \cite{halpern2011don}, as both take into account the cost of computation in a decision problem.







\subsection{Fine-Grained Verifiable Computation}

%\section{Introduction}

%Historically, Cryptography has been used to protect information (either in transit or stored) from unauthorized access. One of the most important developments in Cryptography in the last thirty years, has been the ability to protect not only information but also the {\em computations} that are performed on data that needs to be secure. Starting with the work on secure multiparty computation \cite{mpc}, and continuing with ZK proofs \cite{zk}, and more recently Fully Homomorphic Encryption \cite{gentry}, verifiable outsourcing computation \cite{muggles,ggp10}, SNARKs \cite{qap,snark-linear} and obfuscation \cite{garg2016candidate} we now have cryptographic tools that protect the secrecy and integrity not only of data, but also of the programs which run on that data. 

One of the crucial developments in Modern Cryptography has been the adoption of a more ``fine-grained" notion of computational hardness and security. The traditional cryptographic approach modeled computational tasks as ``easy" (for the honest parties to perform) and ``hard" (infeasible for the adversary). Yet we have also seen a notion of {\em moderately hard} problems being used to attain certain security properties. The best example of this approach might be the use of moderately hard inversion problems used in blockchain protocols such as Bitcoin. Although present in many works since the inception of Modern Cryptography, this approach was first 
formalized in a work of Dwork and Naor \cite{dn-spam}. 

In the second part of this thesis (Chapter \ref{chap:FG}) we consider the following model (which can be traced back to the seminal paper by Merkle \cite{merkle} on public key cryptography). Honest parties will run a protocol which will cost\footnote{
We intentionally refer to it as ``cost" to keep the notion generic. For concreteness one can think of $C$ as the running time required to run the protocol.} 
them $C$ while an adversary who wants to compromise the security of the protocol will incur a $C'=\omega(C)$ cost. Note that while $C'$ is asymptotically larger than $C$, it might still be a feasible cost to incur -- the only guarantee is that it is 
substantially larger than the work of the honest parties. For example in Merkle's original proposal for public-key cryptography the honest parties can exchange a key in time $T$ but the adversary can only learn the key in time $T^2$. Other examples include primitives introduced by Cachin and
Maurer \cite{maurer} and Hastad \cite{H87} where 
the cost is the space and parallel time complexity of the parties, respectively. 

Recently there has been renewed interest in this model. Degwekar et al. \cite{fgcrypto} show how to construct certain cryptographic 
primitives in $\NC^1$ [resp. $\AC^0$] which are secure against all adversaries in $\NC^1$ [resp. $\AC^0$]. In conceptually related work Ball et al. \cite{fghardness} present computational problems which are ``moderately hard'' on average, if they are moderately hard in the worst case, a useful property for such problems to be used as cryptographic primitives. 

The goal of this work is to initiate a study of {\em Fine Grained Secure Computation}. The question we ask is if it is possible to construct {\em secure computation primitives} that are secure against ``moderately complex" adversaries. We answer this question in the affirmative, by presenting definitions and constructions for the task of Fully Homomorphic Encryption and Verifiable Computation in the fine-grained model. We also present two application scenarios for our model: i) hardware chips that prove their own correctness and ii) protocols against rational adversaries including potential solutions to the {\em Verifier's Dilemma} in smart-contracts transactions such as Ethereum.

\subsubsection{Rationality and Fine-Grained Secure Computation.}
In Chapter \ref{chap:RP-seq} we studied ``sequentially composable'' rational protocols, i.e. protocols where the reward is strictly connected, not just to the correctness of the result, but to the amount of work done by the prover. Our work on Fine-Grained Secure Computation complements those results, which only apply to a limited class of computations. In fact, a protocol secure in a fine-grained sense is also a sequentially composable rational proof:
consider for example a protocol where the prover collects the reward only if he produces a proof of correctness of the result. Assume that the cost to  produce a valid proof for an incorrect result, is higher than just computing the correct result and the correct proof. Then obviously a rational prover will always answer correctly, because the above strategy of fast incorrect answers will not work anymore. 

\subsection{Contributions for Fine-Grained Secure Computation}

Our starting point is the work in \cite{fgcrypto} and specifically their public-key encryption scheme secure against $\NC^1$ circuits. Recall that $\ACzt$ is the class of Boolean circuits with constant depth, unbounded fan-in, 
augmented with parity gates. If the number of $\function{AND}$ gates of non constant fan-in is constant we say that the circuit belongs to the class $\ACztq \subset \ACzt$.

Our results can be summarized as follows
\begin{itemize}
\item We first show that the techniques in \cite{fgcrypto} can be used to build a somewhat homomorphic encryption (SHE) scheme. 
We note that because honest parties are limited to 
$\NC^1$ computations, the best we can hope is to have a scheme that is homomorphic for computations in $\NC^1$. However our scheme can only support computations that can be expressed in 
$\ACztq$. 

\item We then use our SHE scheme, in conjunction with protocols described in \cite{ggp10,ckv10,aik10}, to construct verifiable 
computation protocols for functions in $\ACztq$, secure and input/output private against any adversary in $\NC^1$.

\end{itemize}
Our somewhat homomorphic encryption also allows us to obtain the following protocols secure against $\NC^1$ adversaries: \textit{(i)} constant-round 2PC, secure in the presence of semi-honest static adversaries for functions in $\ACztq$; \textit{(ii)} \textit{Private Function Evaluation} in a two party setting for circuits of constant \textit{multiplicative} depth without relying on universal circuits. These results stem from well-known folklore transformations and we do not prove them formally.

The class $\ACztq$ includes many natural and interesting problems such as: fixed precision arithmetic, evaluation of formulas in 3CNF (or $k$CNF for any constant $k$), a representative subset of SQL queries, and S-Boxes~\cite{sboxes} for symmetric key encryption. 

Our results (like \cite{fgcrypto}) hold under the assumption that $\fgAssump$, a widely believed worst-case assumption on separation of complexity classes. Notice that this assumption does not imply the existence of one-way functions (or even $\P \not = \NP$). Thus, our work shows that it is possible to obtain ``advanced'' cryptographic schemes, such as somewhat homomorphic encryption and verifiable computation, even if we do not live in Minicrypt\footnote{This is a reference to Impagliazzo's ``five possible worlds''~\cite{impagliazzo1995personal}.}\footnote{Naturally the security guarantees of these schemes are more limited compared to their standard definitions.}.

\medskip
\noindent
{\sc Comparison with other approaches.}
One important question is: on what features are our schemes better than ``generic" cryptographic schemes that after all are secure against {\em any} polynomial time  adversary. 

One such feature is the type of assumption one must make to prove security. As we said above, our schemes rely on a very mild worst-case complexity assumption, while cryptographic SHE and VC schemes rely on very specific assumptions, which are much stronger than the above. 

For the case of Verifiable Computation, we also have information-theoretic
protocols which are secure against {\em any} (possibly computationally unbounded) adversary. For example the ``Muggles'' protocol in \cite{muggles} which 
can compute any (log-space uniform) $\NC$ function, 
and is also reasonably efficient
in practice \cite{CMT}.
Or, the more recent work \cite{grlocally}, which obtains efficient VC for functions in a subset of $\NC \cap \class{SC}$.
Compared to these results, one aspect in which our protocol fares better  is that
our Prover/Verifier can be implemented with a constant-depth circuit (in particular in 
$\ACzt$, see Section \ref{sec:VC}) which is not possible for the Prover/Verifier in \cite{muggles,grlocally}, which needs to be in $\TC^0$\footnote{The techniques in  \cite{muggles,grlocally} are based on properties of finite fields. Arithmetic in such fields can be carried out by threshold circuits of constant depth, but not in $\ACzt$.}. Moreover our protocol is non-interactive (while \cite{muggles,grlocally} requires
$\Omega(1)$ rounds of interaction) and because our protocols work in the ``pre-processing model" we do not require any uniformity or regularity condition on the circuit being outsourced (which are required by \cite{muggles} and \cite{CMT}). Finally, out verification scheme achieves input and output privacy.

Finally, we compare our results with the information-theoretic approaches (mostly based on randomized encodings) in  \cite{gghkr07,re,cryptoNC0,SYY}. From the techniques in these works one could obtain somewhat homomorphic encryption and verifiable computation in low-depth circuits (even in $\NC^0$). Here, however, we stress that we are interested in \textit{compact} homomorphic encryption schemes (where the ciphertexts do not grow in size with each homomorphic operation) and in verifiable computation schemes where the total work of the verifier approximately linear in the I/O size (i.e. the size of the verification circuit should be $O(\poly(\lambda)(n+m))$ where $n$ and $m$ are the size of the input and output respectively). The techniques in these works cannot directly achieve these goals. In fact, for homomorphic encryption, they lead to ciphertexts of size exponential in $d$, where $d$ is the depth of the (fan-in two) evaluation circuit. For verifiable computation, they  lead to verification with quadratic running time\footnote{On why this running time: in a straightforward application of these approaches we would have the verifier computing the (randomized) encoding of a function $f \in \NC^1$. The work necessary for this is quadratic in the size of the branching program computing $f$ \cite{re} (this is  cubic if we use the approach in \cite{gghkr07}, described in Guy Rothblum's thesis \cite{rothblum2009delegating}).}. 

% Interactive proofs\footnote{We stress again: with \textit{information-theoretic} soundness.} with verification in constant depth are discussed in \cite{gghkr07} (where the verifier is in 
% $\mathsf{NC}^0$). We point out that, besides achieving non-interactive, constant-depth verification, our schemes also have a verifier running in linear {\em sequential} time in the input/output size (i.e. in size $O(\lambda^c(n+m))$ where $\lambda$ is the security parameter, $n$ the input and $m$ the output sizes of the function being outsourced). 

\subsection{Technical Highlights}

In \cite{fgcrypto} the authors already point out that their scheme is 
linearly homomorphic. We make use of the {\em re-linearization} technique 
from \cite{fhe-lwe} to construct a leveled homomorphic encryption. 

Our scheme (as the one in \cite{fgcrypto}) is secure against adversaries in the class of (\textit{non-uniform}) $\NC^1$. This implies that we can only evaluate functions in $\NC^1$ otherwise the evaluator would be able to break the semantic security of the scheme.
However we have to ensure that the \textit{whole} homomorphic evaluation stays in $\NC^1$. The problem is that homomorphically evaluating a function $f$ might 
increase the depth of the computation. 

In terms of circuit depth, the main overhead will be (as usual)
%in the case of homomorphic encryption) 
the computation of multiplication gates. As we 
 show in Section \ref{sec:HE} a single homomorphic multiplication can be performed by a depth two $\ACzt$ circuit, but this requires depth $O(\log(n))$ with a circuit of fan-in two. Therefore, a circuit for $f$ with $\omega(1)$ multiplicative depth would require an evaluation of $\omega(\log(n))$ depth, which would be out of $\NC^1$. Therefore our first scheme can only evaluate
 functions with constant multiplicative depth, as in that case the evaluation stays in $\ACzt$. 
 
We then present a second scheme that extends the class of computable functions 
to $\ACztq$ by allowing for a negligible error in the correctness of the scheme. We use techniques from a work by Razborov \cite{razborov1987lower} on approximating $\ACzt$ circuits with low-degree polynomials -- the correctness of the approximation (appropriately amplified) will be the correctness of our scheme. 



\subsection{Application Scenarios}

The applications described in this section refer to the problem of Verifying Computation, where a Client outsources an algorithm $f$ and an input $x$ to a Server, who returns a value $y$ and a proof that $y=f(x)$. The security property is that it should be infeasible to convince the verifier to accept $y' \neq f(x)$, and the crucial efficiency property is that verifying the proof should cost less than computing $f$ (since avoiding that cost was the reason the Client hired the Server to compute $f$). 

\medskip
\noindent
{\sc Hardware Chips That Prove Their Own Correctness}
Verifiable Computation (VC) can be used to verify the execution of hardware chips designed by untrusted manufacturers. One could envision chips that provide (efficient) \emph{proofs of their correctness} for every input-output computation they perform. These proofs must be  \emph{efficiently verified} in less time and energy than it takes to re-execute the computation itself. 

When working in hardware, however, one may not need the full power of cryptographic protection against {\em any} malicious attacks since one could bound the computational power of the malicious chip. The bound could be obtained by making (reasonable and evidence-based) assumptions on how much computational power can fit in a given chip area. For example one could safely assume that a malicious chip can perform at most a constant factor more work than the original function because of the basic physics of the size and power constraints. In other words, if $C$ is the cost of the honest Server in a VC protocol, then in this model the adversary is limited to $O(C)$-cost computations, and therefore a protocol that guarantees that succesful cheating strategies require $\omega(C)$ cost, will suffice. This is exactly the model in our paper. Our results will apply to the case in which we define the cost as the depth (i.e. the parallel time complexity) of the computation implemented in the chip. 

%\medskip
%\noindent
%{\sc Rational Proofs.}
%The problem above is related to the notion of composable Rational Proofs defined in \cite{cg15}. In a Rational Proof (introduced by Azar and Micali~\cite{am,am1}), given a function $f$ and an input $x$, the Server returns the value $y=f(x)$, and (possibly) some auxiliary information, to the Client. The Client in turn 
%pays the Server for its work with a reward based on the transcript exchanged with the server and some randomness chosen by the client.  The crucial 
%property is that this reward is maximized in expectation when the server 
%returns the correct value $y$. Clearly a {\em rational} prover who is only interested 
%in maximizing his reward, will always answer correctly. 
%
%The authors of \cite{cg15} 
%show however that the definition of Rational Proofs in \cite{am,am1} does not satisfy a basic compositional property needed for the case in which many computations are outsourced to many servers who compete with each other for rewards (e.g. the case of volunteer computations \cite{seti}). 
%A ``rational proof" for the single-proof setting may no longer be rational when a large number of ``computation problems" are outsourced. %  Assume that solving each problem takes time $T$. Then in a time interval of length $T$, the honest prover can only solve and receive the reward for a single problem. On the other hand a dishonest prover, can answer up to $T$ problems, for example by answering at random, a strategy that takes $O(1)$ time. 
%If one can produce $T$ ``random guesses" to problems in the time it takes to solve 1 problem correctly,  it may be preferable to guess! That's because even if each individual reward for an incorrect answer is lower than the reward for a correct answer, the total reward of $T$ incorrect answers might be higher (and this is indeed the case for some of the protocols presented in \cite{am,am1}). 


\medskip
\noindent
{\sc The Verifier's Dilemma.}
In blockchain systems such as Ethereum, transactions can be expressed by arbitrary programs. To add a transaction to a block miners have to verify its validity, which could be too costly if the program is too complex. This creates the so-called {\em Verifier's Dilemma}~\cite{luu2015demystifying}: given a costly valid transaction $Tr$ a miner who spends time verifying it is at a disadvantage over a miner who does not verify it and accept it ``uncritically" since the latter will produce a valid block faster and claim the reward. On the other hand if the transaction is invalid, accepting it without verifying it first will lead to the rejection of the entire block by the blockchain and a waste of work by the uncritical miner.
The solution is to require efficiently verifiable proofs of validity for transactions, an approach already pursued by various startups in the Ethereum ecosystem (e.g. TrueBit\footnote{TrueBit: \textit{https://truebit.io/}}). We note that it suffices for these proofs to satisfy the condition above: i.e. we do not need the full power of information-theoretic or cryptographic security but it is enough to guarantee that to produce a proof of correctness for a false transaction is more costly than producing a valid transaction and its correct proof, which is exactly the model we are proposing. 


\subsection{Future Directions}

Our work opens up many interesting future directions. 

First of all, it would be nice to extend our results to the case where cost is the actual running time, rather than ``parallel running time"/``circuit depth" as in our model. The techniques in \cite{fghardness} (which presents problems conjectured to have $\Omega(n^2)$ complexity on the average), if not even the original work of Merkle \cite{merkle}, might be useful in building a verifiable computation scheme where if computing the function takes time $T$, then producing a false proof of correctness would have to take $\Omega(T^2)$. 

For the specifics of our constructions it would be nice to ``close the gap" between what we can achieve and the complexity assumption: our schemes can only compute $\ACztq$ against adversaries in $\NC^1$, and ideally we would like to be able to compute all of $\NC^1$ (or at the very least all of $\ACzt$). 

Finally, to apply these schemes in practice it is important to have tight concrete security reductions and a proof-of-concept implementations. 


%\subsection{Other Related Work}

%% Prior Work and Comparison to our results




\section{Notation and Common Preliminaries}




% -- Begin Older draft --
\begin{comment}
\section{Verifiable Computation}

% NOTE: One possibility for the ouline is: first go all about rationality (both motivation and research questions) and then about fine-grained computation (both motivation and research questions)

% TODO: Say somewhere about the perspective "trading efficiency and assumptions for adversaries' bounds"? (thus stated is particularly FG specific)

In verifiable computation, a party with limited resources (called \textit{verifier}) delegates the computation of a function $f$ on input $x$ to a more powerful party (called \textit{prover}.)

There is a large body of research on this problem. Such research mixes theory of computing and cryptography.
[\textbf{TODO}]

The research on this problem branches according to types of assumptions done on the potentially malicious prover:
unconditional schemes or cryptographic schemes. Unconditional schemes are secure against any adversary, regardless of their computational power. These schemes make no cryptographic assumptions (e.g. the existence of one-way functions). Cryptographic schemes may require general cryptographic assumptions such as FHE (e.g. \cite{ggp09} ) or more specific ones, such as generalizations of Diffie-Hellman or ``knowledge assumptions" \CN (e.g. \cite{qsp}). Such schemes are secure against any polynomial-time adversary.

In this proposal, we focus on the problem of obtaining expressive delegation schemes (i.e. for general classes of computations) 
with very efficient verifiers.  Pursuing this goal, we will make assumptions on the adversary.
% TODO: Say somewhere once and for all what you mean with cryptographic assumptions: i.e. one-way functions or specific hardness assumptions
We will describe two different types of schemes from two different types of assumptions. Both assumptions allow us to obtain verifiers that, in many circumstances, are more efficient than the ones obtained so far in the  \XXX unconditional and cryptographic research mentioned above.
% XXX: Maybe you want to talk about changes for the prover's cost, if any? (At a shallow analysis, it seems that we gain in that dimension too)
% XXX: As a consequence, you may rephrase your goal as obtaining "doubly very efficient proofs"

\section{Assumptions on Adversaries}
\subsection{Adversaries with Incentives: Rationality}

% TODO: Give names to subsubsections maybe?
% XXX: Maybe just replace sections with chapters?

% What do we mean by this assumption?
In this work, we shall use rationality in the decision theoretic sense:
adversaries are agents that will act maximizing a certain utility function.
We will assume that the verifier will reward the prover for its services;
the prover will make sure to obtain the maximum reward possible.
We will not make assume the verifier to act honestly.

% How to exploit this assumption?
We know a potentially untruthful prover will not act in any other way. How to leverage this?
Our goal is to make the verifier confident that the result of a computation is correct.
We will do that in the following way.
The prover and the verifier will interact with each other, with
the verifier ``challenging'' the prover.
The verifier will then use the prover's responses to (probabilistically) decide its reward.
Thus, the reward is a function of the transcript of the interaction between prover and verifier.
As protocol designers, we will make sure that (in expectation) the reward of the prover is maximized
for the honest prover, i.e. the prover that answers truthfully and follows the protocol.

The security notion sketched above is formalized under the notion of \textit{rational proofs} \CN.

% What do we gain by such an assumption?
Delegation schemes based on rational proofs are very \textit{simple}, \textit{extremely efficient}
and for that do not require any hardness or cryptographic assumption\footnote{Some work \CN combined cryptography and rational proofs. In such \textit{rational arguments} the prover is assumed to be rational \textit{and} having limited computational power. Their round and communication complexity can be significantly lower than their information-theoretic counterparts. See \ref{sec:related-work} \XXX for more details. }. 
In particular, they achieve low communication and number of rounds and a \textit{strongly sublinear} (polylogarihtmic) verification time. In fact, the verifier --- assumed to have oracle access to the input --- could need sample as few as $O(1)$ input bits to compute the prover's reward.


% When can we afford such an assumption?
% pay-per-compute
Why can we afford to assume that our adversaries are rational? 
One typical context when this may be reasonable is in the context of cloud computing and pay-per-computation. 
% XXX: Say why it's intuitive that such parties may not be malicious, but simply negligent.
% XXX: Also mention in relation to this how "the cost of cheating" is then important and how it should be modeled.
In this case, rewards are directly related to the sums of money exchanged for the computation. 
Notice that right now, several commercial cloud computing services \XXX (Amazon, Microsoft Azure, Google, IBM?) follow
an approach based on "amount of time and resources". The concept of "being payed more or less according to the verification of the computation" is not a common payment model. \XXX
% Q: A natural question is, where is verifiable computation used currently?

% Volunteer computing
Another context is that of volunteer computing. \XXX (Examples?)
Here the reward for workers (provers) is some form of score. In turn, this score can at times (\XXX When?) exchanged
monetarily or provide someone with status/ranking. In this context we can assume that the users of the system are either honest or interested in exploiting the system for their own return.
% TODO: Talk about gridcoin and Proof-of-BOINC

% NOTE: Here is a question for Itai: how does one find out if these application domains satisfy the requirements?

% Cryptocurrencies (TODO)
A final potential application domain is that of cryptocurrencies \CN. A description of cryptocurrencies is out of the scope of this document. We will assume the reader is familiar with the basics of blockchains\footnote{See \CN for further details.}. In the following observations we will assume the existence of \textit{publicly verifiable rational proofs/arguments}, i.e. rational protocol that can be verified by any party, not necessarily by the verifier for which the proof is intended. 
% Explain connections between how transactions are verified and public verifiability

% Applications to mining
A first application is mining. A miner is usually rewarded a fee after completing two tasks: verifying that the transactions in a new block are legitimate and completing a proof of work, usually in the form of finding a hash preimage of a string with a specific prefix. Other miners in turn invest computational resources verifying that these two tasks were accomplished correctly. This can be computationally intensive. If the checks miners usually carry out admitted non-interactive publicly verifiable rational proofs, they could simply verify those. This could lead  enormous savings in computational resources. Utility-maximizing blockchain users who are trying to arrive first mining a new block 
Some further observations are in place. The approach alone would not protect against malicious miners whose goal is to disrupt the network. It is unclear how the incentives of miners would change as a consequence. Incentives to accurate verification in the blockchain may benefit from this approach, as it may solve the \textit{verifier's dilemma}(\cite{luu2015demystifying}). The verifier's dilemma is the observation that miners have incentive to skip the verification of expensive transactions to gain a competitive advantage in the race for the next block. We leave more sophisticated security and game-theoretic analysis that would shed light on this as an open problem.

% Applications to verifiable computation
% TODO: We could do unexpensive (computationally)


\subsection{Adversaries with Limited Parallel Time}


% What do we mean by this assumption?

% How to exploit this assumption?

% What do we gain by such an assumption?

% When can we afford such an assumption?


\section{Problem Statement}

\subsection{Rational Proofs: problem statement}

\subsection{Fine-Grained DoC: problem statement}

% TODO: Comparison between the "more efficient" schemes you get with your FG stuff and with what already exists
\end{comment}
% -- End Older draft --

\begin{comment}


\section{Related Work}
TODO...

\subsection{Work on Verifiable Computation}

% Different techniques here...

\subsection{Work on Fine-Grained Cryptography}

\subsection{Rationality and Cryptography}



\section{Organization}
\textbf{TODO...}
\end{comment}