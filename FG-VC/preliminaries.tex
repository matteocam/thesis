\section{Preliminaries}
\label{sec:prelim}

%This section presents all the necessary definitions. 

%\subsection{Adversaries and Circuit Families}

%\subsection{Security Definitions}

In this chapter, all arithmetic computations (such as sums, inner product, matrix products, etc.) in this work will be over $\GF(2)$ unless specified otherwise.

\begin{definition}[Function Family]
A {\em function family} is a family of (possibly randomized) functions $F = \funfam{f}$, where for each $\lambda$, $f_{\lambda}$ has domain $D^f_{\lambda}$ and co-domain $R^f_{\lambda}$. A {\em class} $\mathcal{C}$ 
is a collection of function families. 
\end{definition}
In most of our constructions $D^f_{\lambda}=\{0,1\}^{d_\lambda^f}$ and $R^f_{\lambda}=\{0,1\}^{r_\lambda^f}$ for sequences $\{d_\lambda^f\}_\lambda$, 
$\{d_\lambda^f\}_\lambda$. 

In the rest of the paper we will focus on the class of $\mathcal{C}=\mathsf{NC}^1$ of functions for which there is a polynomial $p(\cdot)$ and a constant $c$ such that for each $\lambda$, the function $f_\lambda$ can be computed by a Boolean (randomized) fan-in 2, circuit of size $p(\lambda)$ and depth $c \log(\lambda)$. In the formal statements of our results we will also use the following classes: $\AC^0$, the class of functions of polynomial size and constant depth with $\function{AND}, \function{OR}$ and $\function{NOT}$ gates with unbounded fan-in; $\ACzt$, the class of functions of polynomial size and constant depth with $\function{AND}, \function{OR}, \function{NOT}$ and $\function{PARITY}$ gates with unbounded fan-in. For a gate $g$ we denote by $\func{type}_C(g)$ the type of the gate $g$ in the circuit $C$ and by $\func{parents}_C(g)$ the list of gates of $C$ whose output is an input to $C$ (such list may potentially contain duplicates). 

Given a function $f$, we can think of its \textit{multiplicative depth} as the degree of the lowest-degree polynomial in $\GF(2)$ that evaluates to $f$. Similarly, we define the multiplicative depth of a circuit as follows:

\begin{definition}[Multiplicative Depth]
Let $C$ be a circuit, we define the multiplicative depth of $C$ as $\func{md}(g_{out})$ where $g_{out}$ is its output gate and the function $\func{md}$, from the set of gates to the set of natural numbers is recursively defined as follows:

\[
\func{md}(g) \eqdef
    \begin{cases*}
      1 & \mbox{if } $\func{type}_C(g) = \func{input}$ \\
      \max\{\func{md}(g') : g' \in \func{parents}_C(g)\} & \mbox{if } $\func{type}_C(g) = \func{XOR}$ \\
      \Sum_{g' \in \func{parents}_C(g)}\func{md}(g') & \mbox{if } $\func{type}_C(g) \in \{ \ANDgt, \ORgt \}$
    \end{cases*}
\]
\end{definition}

\medskip
\noindent
The following two circuit classes will appear in several of our results.

\begin{definition}[Circuits with Constant Multiplicative Depth]
\label{def:ACztcm}
We denote by $\ACztcm$ the class of circuits in $\ACzt$ with \textit{constant multiplicative depth}.
\end{definition}

\begin{definition}[Circuits with Quasi-Constant Multiplicative Depth]
\label{def:ACztq}
\label{def:quasi-constant}
For a circuit $C$ we denote by $S_{\omega(1)}(C)$ the set of $\ANDgt$ and $\ORgt$ gates in $C$ with non-constant fan-in. We say that $C$ has \textit{quasi-constant multiplicative depth} if $|S_{\omega(1)}(C)| = O(1)$.
We shall denote by $\ACztq$ the class of circuits in $\ACzt$ with quasi-constant multiplicative depth.
\end{definition}

\medskip
\noindent
{\sc Limited Adversaries.}
We define adversaries also as families of randomized algorithms $\{A_\lambda\}_\lambda$, 
one for each security parameter (note that this is a non-uniform notion of security). We denote
the class of adversaries we consider as $\mathcal{A}$, and in the rest of the
paper we will also restrict $\mathcal{A}$ to $\mathsf{NC}^1$.


\medskip
\noindent
{\sc Infinitely-Often Security.}
We now move to define security against all adversaries $\{A_\lambda\}_\lambda$ that belong to a class $\mathcal{A}$. 

%We note that the statement $\{A_\lambda\}_\lambda \notin \mathcal{C}$ implies 
%that for all $\{g_\lambda\}_\lambda \in \mathcal{C}$ there exists an infinite 
%number of values $\lambda$ such that $A_\lambda \neq g_\lambda$. Therefore we 
%will use this "infintely often" notion of security, which states that for all 
%adversaries outside of our permitted class our security property holds 
%infinitely often (i.e. for an infinite sequence of security parameters rather 
%than for every sufficiently large security parameter\footnote{
%As in \cite{dvv17} if we make the stronger assumption that $A_\lambda \neq 
%g_\lambda$ for all sufficiently large $\lambda$ then we would be able to 
%obtain the standard notion of security for all sufficiently large security 
%parameters.}
%). 

Our results achieve an "infinitely often" notion of security, which states that for all adversaries outside of our permitted class $\mathcal{A}$ our security property holds infinitely often (i.e. for an infinite sequence of security parameters rather than for every sufficiently large security parameter. We 
inherit this limitation from the techniques of \cite{fgcrypto}. 

\begin{definition}[Infinitely-Often Computational Indistinguishability]
Let $\mathcal{X} = \ens{X}$ Let $\mathcal{Y} = \ens{Y}$ be ensembles over the same domain family, $\mathcal{A}$ a class of adversaries, and $\Lambda$ an infinite subset of $\naturals$. 
We say that $\mathcal{X}$ and $\mathcal{Y}$ are infinitely often computational indistinguishable with respect to set $\Lambda$ and the class $\mathcal{A}$,
denoted by
$\mathcal{X} \sim_{\Lambda,\mathcal{A}} \mathcal{Y}$
if there exists a negligible function $\nu$ such that for any $\lambda \in \Lambda$ and for any adversary $A=\{A_\lambda\}_\lambda \in \mathcal{A}$
\[
| \Pr[A_{\lambda}(X_{\lambda}) = 1] - \Pr[A_{\lambda}(Y_{\lambda}) = 1]| < \nu(\lambda)
\]
\end{definition}
When $\mathcal{A} = \NC^1$ we will keep it implicit and use the notation $\mathcal{X} \sim_{\Lambda} \mathcal{Y}$ and say that $\mathcal{X} \text{ and } \mathcal{Y}$ are $\Lambda$-computationally indistinguishable.


% \begin{remark}{On proofs of security for infinitely many values of $\lambda$}
% As standard practice in cryptography we will use reduction-based proofs. These will have a slightly different flavor than usual as our security notion is for infinitely many values of the security parameter (rather than the standard asymptotic notion). However, this does not change the essence of the usual logic.
% Suppose you want to prove that $Y$ is (infinitely many times) secure based on the assumption that $X$ is secure. 
% To proceed by contradiction you will assume there exist an adversary $\adv_Y$ and $\lambda^*$ such that for all values of $\lambda > \lambda^*$ the advantage of $\adv_Y$ is 
% \end{remark}

In our proofs we will use the following facts on infinitely-often computationally indistinguishable ensembles. We skip their proof as, except for a few technicalities, it is analogous to the corresponding properties for standard computational indistinguishability\footnote{We refer the reader to \cite{goldreich2009foundations1}.}.

\begin{lemma}[Facts on $\Lambda$-Computational Indistinguishability]
\label{lemma:hyb-arg}
\label{lemma:facts-lind}
\begin{itemize}
\item \textbf{Transitivity:}
Let $m = \poly(\lambda)$ and $\mathcal{X}^{(j)}$ with $j \in \{0,\dots,m \} $ be ensembles.
If for all $j \in [m]$ $\mathcal{X}^{(j-1)} \lind \mathcal{X}^{(j)}$, then $\mathcal{X}^{(0)} \lind \mathcal{X}^{(m)}$.
\item \textbf{Weaker than statistical indistinguishability:} Let $\mathcal{X}, \mathcal{Y}$ be statistically indistinguishable ensembles. Then $\mathcal{X} \lind \mathcal{Y}$ for any infinite $\Lambda \subseteq \naturals$
\item \textbf{Closure under $\NC^1$:}  Let $\mathcal{X}, \mathcal{Y}$ be ensembles and $\funfam{f} \in \NC^1$. If $\mathcal{X} \lind \mathcal{Y}$ for some $\Lambda$ then 
$f_{\lambda}(\mathcal{X}) \lind f_{\lambda}(\mathcal{Y})$.
\end{itemize}
\end{lemma}


%\subsection{Fine-Grained Encryption}

\subsection{Public-Key Encryption}
A public-key encryption scheme \\
$\PKE = (\PKEKeygen, \PKEEnc, \PKEDec)$ is a triple of algorithms which operate as follow:
\begin{itemize}
\item \textbf{Key Generation.} The algorithm $(\pk,\sk)\leftarrow$ \PKEKeygen$(1^{\lambda})$ takes a unary representation of the security parameter and outputs a public key encryption key $\pk$ and a secret decryption key $\sk$.
\item \textbf{Encryption.} The algorithm $c\leftarrow$ \PKEEnc$_{\pk}(\mu)$ takes the public key $\pk$ and a single bit message $\mu\in\bit$ and outputs a ciphertext $c$. The notation \PKEEnc$_{\pk}(\mu;r)$ will be used to represent the encryption of a bit $\mu$ using randomness $r$. 
\item \textbf{Decryption.} The algorithm $\mu^*\leftarrow$ \PKEDec$_{\sk}(c)$ takes the secret key $\sk$ and a ciphertext $c$ and outputs a message $\mu^*\in\bit$. 
\end{itemize}
Obviously we require that $\mu=$\PKEDec$_{\sk}($\PKEEnc$_{\pk}(\mu))$

\begin{definition}[CPA Security for PKE]
\label{def:cpa-pke}
\label{def:cpa-he}
A scheme $\PKE$ is IND-CPA secure if for an infinite $\Lambda \subseteq \naturals$ we have
$$ (\pk,\PKEEnc_{\pk}(0)) \lind (\pk,\PKEEnc_{\pk}(1)) $$
where $(\pk,\sk)\leftarrow$ \PKEKeygen($1^{\lambda}$).
\end{definition}

\begin{remark}[Security for Multiple Messages]
\label{rem:cpa-multiple}
Notice that by a standard hybrid argument and Lemma \ref{lemma:hyb-arg} we can prove that any scheme secure according to Definition \ref{def:cpa-pke} is also secure for multiple messages (i.e. the two sequences of encryptions bit by bit of two bit strings are computationally indistinguishable).
We will use this fact in the proofs in Section $\ref{sec:vc}$, but we do not provide the formal definition for this type of security. We refer the reader to 5.4.2 in~\cite{goldreich2009foundations2}.
\end{remark}

%\subsection{Public-Key Encryption}
A public-key encryption scheme \\
$\PKE = (\PKEKeygen, \PKEEnc, \PKEDec)$ is a triple of algorithms which operate as follow:
\begin{itemize}
\item \textbf{Key Generation.} The algorithm $(\pk,\sk)\leftarrow$ \PKEKeygen$(1^{\lambda})$ takes a unary representation of the security parameter and outputs a public key encryption key $\pk$ and a secret decryption key $\sk$.
\item \textbf{Encryption.} The algorithm $c\leftarrow$ \PKEEnc$_{\pk}(\mu)$ takes the public key $\pk$ and a single bit message $\mu\in\bit$ and outputs a ciphertext $c$. The notation \PKEEnc$_{\pk}(\mu;r)$ will be used to represent the encryption of a bit $\mu$ using randomness $r$. 
\item \textbf{Decryption.} The algorithm $\mu^*\leftarrow$ \PKEDec$_{\sk}(c)$ takes the secret key $\sk$ and a ciphertext $c$ and outputs a message $\mu^*\in\bit$. 
\end{itemize}
Obviously we require that $\mu=$\PKEDec$_{\sk}($\PKEEnc$_{\pk}(\mu))$

\begin{definition}[CPA Security for PKE]
\label{def:cpa-pke}
\label{def:cpa-he}
A scheme $\PKE$ is IND-CPA secure if for an infinite $\Lambda \subseteq \naturals$ we have
$$ (\pk,\PKEEnc_{\pk}(0)) \lind (\pk,\PKEEnc_{\pk}(1)) $$
where $(\pk,\sk)\leftarrow$ \PKEKeygen($1^{\lambda}$).
\end{definition}

\begin{remark}[Security for Multiple Messages]
\label{rem:cpa-multiple}
Notice that by a standard hybrid argument and Lemma \ref{lemma:hyb-arg} we can prove that any scheme secure according to Definition \ref{def:cpa-pke} is also secure for multiple messages (i.e. the two sequences of encryptions bit by bit of two bit strings are computationally indistinguishable).
We will use this fact in the proofs in Section $\ref{sec:vc}$, but we do not provide the formal definition for this type of security. We refer the reader to 5.4.2 in~\cite{goldreich2009foundations2}.
\end{remark}

\subsubsection{Somewhat Homomorphic Encryption}
A public-key encryption scheme is said to be homomorphic if there is an 
additional algorithm $\mathsf{Eval}$ which takes a input the public key $\pk$, 
the representation of a function $f:\bit^l\rightarrow\bit$ and a set of $l$ ciphertexts $c_1,\ldots,c_l$, and outputs a ciphertext $c_f$\footnote{Notice that the syntax of $\mathsf{Eval}$ can also be extended to return a sequence of encryptions for the case of multi-output functions. We will use this fact in Section \ref{sec:beyond-cm}. See also Remark \ref{rem:cpa-multiple}.}. 
%$\HE = (\HEKeygen, \HEEnc, \HEDec, \HEEval)$ is a public-key encryption scheme 
%with a few additions. It is a quadruple of algorithms which operate as follow:
%\begin{itemize}
%\item \textbf{Key Generation.} The algorithm $(\pk,\evk,\sk)\leftarrow$ %\HEKeygen$(1^{\lambda})$ takes a unary representation of the security parameter 
%and outputs a public key encryption key $\pk$, a public evaluation key $\evk$ and 
%a secret decryption key $\sk$.
%\item \textbf{Encryption.} The algorithm $c\leftarrow$ \HEEnc$_{\pk}(\mu)$ 
%operates exactly as $\PKEEnc$.
%\item \textbf{Decryption.} The algorithm $\mu^*\leftarrow$ \HEDec$_{\sk}(c)$ 
%operates exactly as $\PKEDec$.
%\item \textbf{Homomorphic Evaluation} The algorithm $c_{f}\leftarrow$ 
%$\HEEval}_{\pk}(f,c_1,\ldots,c_l)$ takes the evaluation key $\evk$, a function 
%$f:\bit^l\rightarrow\bit$ and a set of $l$ ciphertexts $c_1,\ldots,c_l$, and 
%outputs a ciphertext $c_f$. 

% It must be the case that:
% \begin{equation}
% \mathrm{\HEDec}_{\sk}(c_f) = f(\mathrm{\HEDec}_{\sk}(c_1),\ldots,\mathrm{\HEDec}_{\sk}(c_l))
% \end{equation}
% with probability $\geq 1- \gamma(\lambda)$. 
%\end{itemize}


%A homomorphic encryption scheme is said to be secure if it meets the following 
%notion of semantic security:

%\begin{definition}[CPA Security for HE]\label{def:cpa-he}
%A scheme $\HE$ is IND-CPA secure if for an infinite $\Lambda \subseteq \naturals$ 
%we have
%$$ (\pk,\evk,\mathrm{\HEEnc}_{\pk}(0)) \lind (\pk,\evk,\mathrm{\HEEnc}_{\pk}(1)) $$
%where $(\pk,\evk,\sk)\leftarrow$ \HEKeygen($1^{\lambda}$).
%\end{definition}

\newcommand{\cclass}{\mathcal{C}}
We proceed to define the homomorphism property. The next notion of $\cclass$-homomorphism is sometimes also referred to as ``somewhat homomorphism''.

\begin{definition}[$\cclass$-homomorphism]
Let $\cclass$ be a class of functions (together with their respective representations).
An encryption scheme $\PKE$ is $\cclass$-homomorphic (or, homomorphic for the class $\cclass$) if for every function $f_\lambda$ where  $f_\lambda \in \mathcal{F} \funfam{f} \in \cclass$ and respective inputs $\mu_1,\dots,\mu_l \in \bit$ (where $l = l(\lambda)$), it holds that if 
$(\pk, \sk) \gets \PKEKeygen(1^{\lambda})$ and $c_i \gets \PKEEnc_{\pk}(\mu_i)$
then 
\[
    \Pr[\PKEDec_{\sk}(\Eval_{\pk}(F, c_1,\dots, c_l)) \not = F(\mu_1,\dots,\mu_l)] = \negl(\lambda),
    % Note: this definition is not infinitely-often, but for large enough values. It's correct this way.
\]
\end{definition}


%As pointed out in~\cite{fhe-lwe}, there are two important properties that the 
%above definition does not require. First, it does not require that the ciphertexts
%$c_i$ are decryptable themselves, only that they become
%decryptable after homomorphic evaluation. Finally, it does not require that the 
%output of $\HEEval$
%can undergo additional homomorphic evaluation.


% \begin{definition}[CPA Security for HE]\label{def:cpa-he}
% A scheme \textnormal{\HE} is IND-CPA secure if, for any polynomial time adversary $\mathcal{A}$, there exists a negligible function $\mu(\cdot)$ such that 
% \begin{equation}
% \mathrm{Adv}_{\mathrm{CPA}}[\mathcal{A}] \EqDef  
% |\Pr[\mathcal{A}(\pk,\evk,\mathrm{\HEEnc}_{\pk}(0)) = 1] - 
% \Pr[\mathcal{A}(\pk,\evk,\mathrm{\HEEnc}_{\pk}(1)) = 1]|=\mu(\lambda)
% \end{equation}
% where $(\pk,\evk,\sk)\leftarrow$ \textnormal{\HEKeygen}($1^{\lambda}$). We further say that a scheme \textnormal{\HE} has CPA security $\delta$ for some negligible function $\delta(\cdot)$ if the above indistinguishability gap $\mu(\lambda)$ is smaller than $\delta(\lambda)^{\Omega(1)}$. 
% \end{definition}

As usual we require the scheme to be non-trivial by requiring that the output 
of $\mathsf{Eval}$ is compact:
\begin{definition}[\bfseries Compactness]\label{def:compact}
A homomorphic encryption scheme $\PKE$ is compact if there exists a polynomial $s$ in $\lambda$ such that the output length of $\mathsf{Eval}$ is at most $s(\lambda)$ bits long (regardless of the function $f$ being computed or the number of inputs). 
\end{definition}


\begin{definition}
\label{def:leveled}
Let $\cclass = \funfam{\cclass}$ of arithmetic circuits in $\GF(2)$. A scheme 
$\PKE$ is leveled  $\cclass$-homomorphic if it takes $1^L$ as additional input in 
key generation, and can only evaluate depth-$L$ arithmetic circuits from 
$\cclass$. The bound $s(\lambda)$ on the ciphertext must remain independent of 
$L$.
\end{definition}


\subsection{Verifiable Computation}
% We now describe our model for delegation of computation against bounded adversaries. We follow the description in \cite{ckv10} of delegation schemes against polynomial time adversaries. % XXX: check sentence
In a {\em Verifiable Computation} scheme a Client uses an untrusted server to compute a function $f$ over an input $x$. The goal is to prevent the Client from accepting an incorrect value $y'\neq f(x)$. We require that the Client's cost of running this protocol be smaller than the cost of computing the function on his own. The following definition is from \cite{ggp10} which allows the client to run a possibly expensive pre-processing step. 

\begin{definition}[Verifiable Computation Scheme]

A \emph{verifiable computation scheme} $\VC =
(\VCKG,\VCPG,\VCCompute,\VCVerif)$ consists of the four
algorithms defined below.  

\begin{enumerate}
\item $\VCKG(f, 1^\lambda) \to (\pkW, \skD)$: 
      Based on the security parameter $\lambda$, the randomized \emph{key generation} 
      algorithm generates a public key that encodes the target function $f$, 
      which is used by the Server to compute $f$. It also 
      computes a matching secret key, which is kept private by the Client. 

\item $\VCPG_{\skD}(x) \to (\qx,\sx)$:
      The \emph{problem generation} algorithm uses the secret key $\skD$ to encode the 
      function input $x$ as a public query $\qx$ which is given to the Server
      to compute with, and a secret value $\sx$ 
      which is kept private by the Client. 

\item $\VCCompute_{\pkW}(\qx) \to \ax$: 
      Using the Client's public key and the encoded input, the Server  \emph{computes} 
      an encoded version of the function's output $y = F(x)$. 

\item $\VCVerif_{\skD}(\sx,\ax) \to y\mbox{ }\cup \{ \fail \}$:
      Using the secret key $\skD$ and the secret ``decoding'' $\sx$, the 
      \emph{verification} algorithm converts the worker's encoded output into 
      the output of the function, e.g., $y = f(x)$ or outputs $\fail$ indicating
      that $\ax$ does not represent the valid output of $f$ on $x$.
\end{enumerate}      
% A delegation scheme is an interactive protocol $\Del =  \DW$ between a delegator $\D$ and a worker $\W$ with the following structure:

%  \begin{enumerate}
%      \item The scheme \Del~consists of two stages: an offline/preprocessing stage and an online stage. The offline stage is executed once before the online stage, whereas the online stage can be executed many times. 
%      \item In the offline stage, both the delegator $\D$ and the worker $\W$ receive a security parameter $\lambda$ and a function $F : \bit^n \to \bit^m $, represented by a boolean circuit $C$. At the end of the interaction, the delegator $\D$ decides whether to accept or reject. If $\D$ accepts, then $\D$ outputs a secret key $\skD$ and a public key $\pkW$. We will denote this by $(\skD, \pkW) = \DW(F, 1^{\lambda})$. We will use the notation $C, n \text{ and } m$ as the circuit and parameters associated with $F$ throughout the paper, and we will often omit the security parameter from the notation.
%      \item In the online stage, both parties receive $F, 1^{\lambda}$, and an input $x \in \bit^n$, and execute a one round communication protocol. Namely, $\D$ sends $q = \D(F, x, \skD)$ to $\W$, and then $\W$ sends 
%      $a = \W(F, x, \pkW, q)$ to $\D$. Then the delegator $\D$ either accepts or rejects. If $\D$ accepts, then $\D$ also generates a private output
%      $y = \D(F, x,\skD, q, a) \in \bit^m$, which is supposed to be $F(x)$.
%      For simplicity, we will omit the function $F$ and the security parameter from the input of the online stage.
%       \end{enumerate}
\end{definition}     
   
     

\noindent
The scheme should be complete, i.e. an honest Server should (almost) always return the correct value. 
\begin{definition}[Completeness]
\label{def:vc-completeness}
A delegation scheme $\VC = (\VCKG,\VCPG,\VCCompute,\VCVerif)$ has \textit{overwhelming completeness} for a class of functions $\mathcal{C}$ if  there is a function $\nu(n) = \negl(\lambda)$ such that for infinitely many values of $\lambda$, if $f_\lambda \in \mathcal{F} \in \mathcal{C}$, then for all 
inputs $x$ 
%for every function F and every $x \in \bit^n$, 
the following holds with probability at least $1-\nu(n)$:  
  $(\pkW, \skD) \gets \VCKG(f_\lambda, \lambda)$
$(\qx,\sx) \gets \VCPG_{\skD}(x)$ and 
  $\ax \gets \VCCompute_{\pkW}(\qx)$ then $y=f_{\lambda}(x) \gets \VCVerif_{\skD}(\sx,\ax)$.
\end{definition}




% \begin{center}
% \fbox{\pseudocode[syntaxhighlight=auto]%[syntaxhighlight=auto,head=$\function{Exp}^{\text{Verif}}_{\adv}(F, \lambda)$]
% {%
%       (\pkW, \skD) \sample \Doffline (F, \lambda) \\
%      (\hatx, \hata) \sample \adv^{\Donline(\skD, \cdot, \cdot)}(\pkW) \\
%      return \Dverif(\skD, \hatq, \hata) }}
% \end{center}



% \pseudocode[syntaxhighlight=auto]{$Exp^{\text{Verif}}_{\adv}$}{
%     $(\pkW, \skD)$ \sample \kgen $(\lambda)$ \\
%     $(\hatq, \hata)$ \sample $\adv^{\Donline(\skD, \cdot, \cdot)}(\pkW)$ \\
%     \pcreturn $\DVerif(\skD, \hatq, \hata)$ }

% \begin{comment}
% \begin{definition}{Security Game for Delegation Schemes}
% Let $\Del = \DW$ be a delegation scheme and $\lambda \in \mathbb{N}$ be the security parameter. The security game $\Gvc(\lambda)$ for $\Del$ is the following game played by a worker strategy $\Wstar$.
% \begin{itemize}
% \item The game starts with the offline stage of $\Del$, and is followed by many rounds of the online stage.
% % NB: it is the worker choosing the function F.
% \item $\Wstar(1^{\lambda})$ first chooses the delegation function $F$ and then $\D$ and $\Wstar$ interact in the offline stage of $\Del$ with input $F$.
% \item At the beginning of each round of the online stage (indexed by $l$), $\Wstar$ can either terminate the game or choose an input $x_l \in \bit^n$. If the game is not terminated, $\D$ and $\Wstar$ interact in the online stage of $\Del$ on input $x_l$.
% \item Whenever the delegator $\D$ rejects, the game terminates.
% \end{itemize}
% $\Wstar$ succeeds in the game $\Gvc(\lambda)$ if there exists a round $l$ of the online stage such that $\D$ accepts and outputs a wrong value $y_l \not = F(x_l)$, where $x_l$ is the delegated input chosen by $W^*$.
% \end{definition}
% \end{comment}

\def\fnFam{{\cal F}}

% Experiment
\def\accSet{{\cal I}}

\newcommand{\BatchVCPG}{\function{BatchProbGen}}

\renewcommand{\vec}[1]{\mathbf{#1}}

To define soundness we consider an adversary who plays the role of a malicious Server who tries to convince the Client of an incorrect output $y \neq f(x)$. The adversary is allowed to run the protocol on inputs of her choice, i.e. see the {\em queries} $q_{x_i}$ for adversarially chosen $x_{i}$'s before picking an 
input $x$ and attempt to cheat on that input. 
Because we are interested in the parallel complexity of the adversary we distinguish between two parameters $l$ and $m$. The adversary is allowed to do $l$ rounds of adaptive queries, and in each round she queries $m$ inputs. Jumping ahead, because our adversaries are restricted to $\mathsf{NC}^1$ circuits, we will have to bound $l$ with a constant, but we will be able to keep $m$ polynomially large. 

%Experiment for soundness below. $\BatchVCPG$ is a batch version of $\VCPG$ that 
%runs in parallel and returns a possibly polynomial number of outputs of $\VCPG$. %We model an adversary \adv as a tuple of circuits %(\adv^{(1)}_{q},\dots,\adv^{(L)}_{q}, \adv_{resp})$, where the $\adv^{(i)}_{q}$-s %generates the sets of inputs on which to  query the problem generator and the %$\adv_{resp}$ generates a response based on those queries.

% \vspace{2mm}
% \begin{tabular}{l}
% \hspace{1mm} Experiment ${\bf Exp}^{\function{Verif}}_{\adv}[\mathcal{VC}, F, \lambda, L]$\\
% \hspace{7mm} $(\pkW, \skD) \gets \VCKG(F, \lambda)$;\\
% \hspace{7mm} $\accSet \gets \emptyset$;\\

% \hspace{7mm} For $i=1,\ldots,i=L$;\\
% \hspace{13mm} $\vec{x}^{(i)} \gets \adv^{(i)}_{q}(\pkW,\accSet)$;\\
% \hspace{13mm} $(\vec{q}^{(i)},\vec{s}^{(i)}) \gets \BatchVCPG_{\skD}(\vec{x}^{(i)})$;\\
% \hspace{13mm} $\accSet \gets \accSet \cup \{ \vec{x}^{(i)}, \vec{q}^{(i)} \}$;\\
% \hspace{7mm} $(i,j,\hat{a}) \gets \adv_{resp}(\pkW, \accSet)$;\\
% \hspace{7mm} $\hat{y} \gets \VCVerif_{\skD}(s^{(i)}_j,\hat{a})$\\
% \hspace{7mm} If $\hat{y} \neq \fail$ and $\hat{y} \neq F(x^{i}_j)$, output 1, else 0;\\
% \end{tabular}
% \vspace{2mm}


\vspace{2mm}
\begin{tabular}{l}
\hspace{1mm} Experiment $\expVC{\VC}$\\
\hspace{7mm} $(\pkW, \skD) \gets \VCKG(f, \lambda)$;\\
\hspace{7mm} $\accSet \gets \emptyset$;\\

\hspace{7mm} For $i=1,\ldots,i=l$;\\
\hspace{13mm} $\{ x_{(i-1)m},\dots x_{im-1} \} \gets A_{\lambda}(\pkW,\accSet)$;\\
\hspace{13mm} $\{(q_{j},s_{j}) : (q_{j},s_{j}) \gets \VCPG_{\skD}(x_j), j \in \{(i-1)m,\dots,im\} \}$\\
\hspace{13mm} $\accSet \gets \accSet \cup \{ x_{(i-1)m},\dots x_{im-1} \} \cup \{ q_{(i-1)m},\dots q_{im-1} \}$;\\
\hspace{7mm} $\hat{a} \gets A_{\lambda}(\pkW, \accSet)$;\\
\hspace{7mm} $\hat{y} \gets \VCVerif_{\skD}(s_{lm},\hat{a})$\\
\hspace{7mm} If $\hat{y} \neq \fail$ and $\hat{y} \neq f(x_{lm})$, output 1, else 0.\\
\end{tabular}
\vspace{2mm}

\begin{remark}
In the experiment above the adversary "tries to cheat'' on the last input presented in the last round of queries (i.e. $x_lm$). This is without loss of generality. In fact, assume the adversary aimed at cheating on an input
presented before round $l$, then with one additional round it could present that same input once more as the last of the batch in that round. 
\end{remark}
% TODOs here
%\begin{definition}{One-Time Soundness}
%\label{def:one-time-sec}
%\matteo{TODO: soundness one-time defined with m=l=1}\\

%\end{definition}


%\begin{definition}{Many-Time Soundness}
%\label{def:many-times-sec}
%\matteo{TODO: soundness many-times, defined with $m = \poly(n), l=O(1)$}\\
%\end{definition}
%\matteo{TODO: Both definitions of soundness should have the constraint $\sum_i %DEPTH(\adv^{(i)}_{q}) + DEPTH(\adv_{resp}) = O(\log (n))$}

%\begin{remark}
%\matteo{To be fixed}
%Our definitions are equivalent to those of \cite{ckv10} if we restrict ourselves 
%to stateless algorithms (notice that the experiment has state though) and bounded
%depth. For this reason we impose a constraint on the number of rounds of queries 
%to $\BatchVCPG$
%\end{remark}

\begin{definition}[Soundness]
\label{def:vc-soundness}
We say that a verifiable computation scheme is $(l,m)$-sound against a class 
$\mathcal{A}$ of adversaries if there exists a negligible function $\negl(\lambda)$, such that for all $A = \{A_\lambda\}_\lambda \in \mathcal{A}$, and for infinitely many $\lambda$ we have that 
\[
\Pr[\expVC{\VC}=1] \leq \negl(\lambda)
\]
\end{definition}

Assume the function $f$ 
we are trying to compute belongs to a class $\mathcal{C}$ which is smaller than 
$\mathcal{A}$. Then our definition guarantees that the "cost" of cheating is higher than the cost of honestly computing $f$ and engaging in the Verifiable Computation protocol $\mathcal{VC}$. Jumping ahead, our scheme will allow us to compute the class $\mathcal{C}=\mathsf{AC}^0[2]$ against the class of 
adversaries $\mathcal{A}=\mathsf{NC}^1$.

\medskip
\noindent
{\sc Efficiency}
The last thing to consider is the efficiency of a VC protocol. Here we focus on the time complexity of computing the function $f$. Let $n$ be the number of input bits, and $m$ be the number of output bits, and $S$ be the size of the circuit computing $f$. 
     \begin{itemize}
        \item A verifiable computation scheme $\VC$ is \textbf{client-efficient} if circuit sizes of $\VCPG$ and $\VCVerif$ are $o(S)$. We say that it is \textbf{linear-client} if those sizes are $O(\poly(\lambda)(n + m))$. 
        
        \item A verifiable computation scheme $\VC$ is \textbf{server-efficient} if the circuit size of $\VCCompute$ is $O(\poly(\lambda)S)$.
        % \item A verifiable computation scheme $\VC$ has a \textbf{non-interactive} offline stage if $\D$ and $\W$
        % do not interact at all during the offline stage, and only $\D$ does some computation. Note that if $\Del$ has a non-interactive offline stage, then we can assume w.l.o.g. that $\D$ always accepts in the offline stage.
     \end{itemize}
We note that the key generation protocol $\VCKG$ can be expensive, and indeed in our protocol (as in \cite{ggp10,ckv10,aik10}) its cost is the same as computing $f$ -- this is OK as $\VCKG$ is only invoked once per function, and the cost can be amortized over several computations of $f$. 

%\subsection{Fine-Grained Two-Party Computation}