\subsection{Public-Key Encryption}
A public-key encryption scheme \\
$\PKE = (\PKEKeygen, \PKEEnc, \PKEDec)$ is a triple of algorithms which operate as follow:
\begin{itemize}
\item \textbf{Key Generation.} The algorithm $(\pk,\sk)\leftarrow$ \PKEKeygen$(1^{\lambda})$ takes a unary representation of the security parameter and outputs a public key encryption key $\pk$ and a secret decryption key $\sk$.
\item \textbf{Encryption.} The algorithm $c\leftarrow$ \PKEEnc$_{\pk}(\mu)$ takes the public key $\pk$ and a single bit message $\mu\in\bit$ and outputs a ciphertext $c$. The notation \PKEEnc$_{\pk}(\mu;r)$ will be used to represent the encryption of a bit $\mu$ using randomness $r$. 
\item \textbf{Decryption.} The algorithm $\mu^*\leftarrow$ \PKEDec$_{\sk}(c)$ takes the secret key $\sk$ and a ciphertext $c$ and outputs a message $\mu^*\in\bit$. 
\end{itemize}
Obviously we require that $\mu=$\PKEDec$_{\sk}($\PKEEnc$_{\pk}(\mu))$

\begin{definition}[CPA Security for PKE]
\label{def:cpa-pke}
\label{def:cpa-he}
A scheme $\PKE$ is IND-CPA secure if for an infinite $\Lambda \subseteq \naturals$ we have
$$ (\pk,\PKEEnc_{\pk}(0)) \lind (\pk,\PKEEnc_{\pk}(1)) $$
where $(\pk,\sk)\leftarrow$ \PKEKeygen($1^{\lambda}$).
\end{definition}

\begin{remark}[Security for Multiple Messages]
\label{rem:cpa-multiple}
Notice that by a standard hybrid argument and Lemma \ref{lemma:hyb-arg} we can prove that any scheme secure according to Definition \ref{def:cpa-pke} is also secure for multiple messages (i.e. the two sequences of encryptions bit by bit of two bit strings are computationally indistinguishable).
We will use this fact in the proofs in Section $\ref{sec:vc}$, but we do not provide the formal definition for this type of security. We refer the reader to 5.4.2 in~\cite{goldreich2009foundations2}.
\end{remark}