
\begin{remark}
\label{rem:asy}
{\em If we are interested in an asymptotic treatment, it is important to notice that as long as $\Delta(x) \geq 1/{\sf poly}(|x|)$ then it is possible to keep a polynomial reward budget, and maximize the honest prover profit against all provers who cheat with a substantial probability $\epsilon_{\disP} \geq 1/{\sf poly'}(|x|)$. This also allows us to consider the reward function to have range in $[0,1]$ without loss of generality. Such a reward function can always be scaled and keep being bounded by a polynomial. We will use this approach in some of the proofs. }
\end{remark}




\subsection{One-sided Rational Proofs}
Rational proofs as in Definition \ref{def:RP} impose a loss to provers that do not provide the right output for the computation of the function $f$. When dealing with non-deterministic computations, this might be too strong of a requirement, since it might be possible to prove that there is an accepting computation path but not the opposite. For those cases we introduce {\em one-sided} proofs: 


\begin{definition}[One-sided Rational Proof]
	\label{def:one-sided-RP}
	A language $L$ admits a one-sided rational proof if there exists an interactive proof $(P,V)$ and a randomized reward function
	$\rew : \bits \to \posreals$ such that
	
	\begin{enumerate}
		\item For any input $x \in 
		L \cap \bit^n$, $\Pr[\out((P,V)(x)) = 1] \geq 1 - \negl(n).$
		
		\item For every prover $\disP$, and for any input $x \in 
		L \cap \bit^n$ there exists a $\delta_{\disP}(x) \geq 0$ such that 
		$ \expRewProtDis + \delta_{\disP}(x) \leq \expRewProtHon. $
	\end{enumerate}
	The expectations and the probabilities are taken over the random coins of the prover and verifier.
\end{definition}
Note that if $x \notin L$ no guarantee can be made about the 

The class \textit{one-sided} $\DRMA$ ($\osDRMA$) is defined analogously to $\DRMA$ for one-sided rational proofs.

\begin{lemma}
	Let $L \subseteq \bits$. If $L, \bar{L} \in \osDRMA[r, C, T]$ then $L  \in \DRMA[r, C, T]$.
\end{lemma}
\begin{proof}
	Construct a rational proof for $L$ as in Definition \ref{def:RP} as follows. The prover first sends a bit $b$ indicating whether $x \in L$. Then invoke the one-sided rational proof for $L$ if $b = 1$ and the one for $\bar{L}$ otherwise. 
\end{proof}

\begin{corollary}
	Let $\Cclass$ be a complexity class closed under complement. If $\Cclass \subseteq \osDRMA[r, C, T]$ then $\Cclass \subseteq \DRMA[r, C, T]$.
\end{corollary}

\begin{comment}
A composition theorem (in its weak version: for yes/no Rational Proofs):
\begin{theorem}
Let $\protOne = (P_1, V_1^{L_2})$ be a (yes/no) rational proof for language $L_1$ with noticeable reward gap and let $V^{L_2}$ have oracle access to language $L_2$ with at most $O(1)$ queries.
Let $\protTwo (P_2, V_2)$ be a (yes/no) rational proof for $L_2$ with noticeable reward gap.
Then there exists a (yes/no) rational proof $\pi$ for $L_1$ with noticeable reward gap.
Moreover if $\protOne$ and $\protTwo$ have round, communication and verification complexity respectively $r_1(n), cc_1(n), T_1(n)$  and $r_2(n), cc_2(n), T_2(n)$ then language $L_1 \in DRMA[r_1(n) + O(r_2(n)), cc_1(n) + O(cc_2(n)), T_1(n) + O(T_2(n))]$
\end{theorem}
\end{comment}

% TODO: Observe that since rational proofs with poly budget without noticeable rew gap "hardly" make sense, we will implicitly consider rational proofs, always rational proofs with noticeable rew. gap.



\subsection{Sequential Composability}
\label{sec:sq-def}
Until now we have only considered agents who want to maximize their reward. But the reward alone, might not capture the complete utility function that the Prover is trying to maximize in his interaction with the Verifier. In particular we have not considered the 
{\em cost} incurred by the Prover to compute $f$ and engage in the protocol. It makes 
sense then to define the  {\em profit} of the Prover as the difference between the reward paid by the verifier and such cost. 


As already pointed out in \cite{am1,ratargs} the definition of Rational Proof is sufficiently robust to also maximize the profit of the honest prover and not just the reward. Indeed consider the case of a "lazy" prover $\tilde{P}$ that does not evaluate the function: let $\tilde{R}(x), \tilde{C}(x)$ be the reward and cost associated with $\tilde{P}$ on input $x$ (while $R(x),C(x)$ are the values associated with the honest prover). 

Obviously we want
\[ R(x)-C(x) \geq \tilde{R}(x)-\tilde{C}(x) \mbox{ or equivalently } 
R(x)-\tilde{R}(x) \geq C(x) - \tilde{C}(x) \]
Recall the notion of reward gap which is the minimum difference between the reward of the honest prover and any other prover
\[ \Delta(x) \leq R(x)-\tilde{R}(x) \]
To maximize the profit is therefore sufficient to change the reward by a 
a multiplier $M= C(x)/\Delta(x)$ since then we have that 
\[ M(R(x) - \tilde{R}(x)) \geq C(x) \geq C(x) - \tilde{C}(x) \]
as desired. This explains why we require the reward gap to be at least the inverse of a polynomial, since this will maintain the total reward paid by the Verifier bounded by a polynomial. 

\subsection{Profit in Repeated Executions}

In \cite{cg15} we showed how if Prover and Verifier engage in repeated execution of a Rational Proof, where the Prover has a "budget" of computation cost that he is willing to invest, then there is no guarantee anymore that the profit is maximized by the honest prover. The reason is that it might be more profitable for the prover to use his budget to provide many incorrect answers than to provide a single correct answer. That's because incorrect (e.g. random) answers are ``cheaper" to compute than the correct one and with the same budget $B$ the prover can provide many of them while the entire budget might be necessary to solve a single problem correctly. If incorrect answers still receive a substantial reward then many incorrect answers may be more profitable and a rational prover will choose that strategy. 

We refer the reader to \cite{cg15} for concrete examples of situations where this happens in many of the protocols in \cite{am,am1,ratargs,ratsumchecks}.

This motivated us to consider a stronger definition which requires the reward to be 
somehow connected to the "effort" paid by the prover. The definition (stated below) basically says that if a (possibly dishonest) prover invests less computation than the honest prover then he must collect a smaller reward. 

\noindent
\begin{definition}[Sequential Rational Proof]
\label{def:SRP}
A rational proof $(P,V)$ for a function $f:$ $\bit^n$ $\to$ 
$\bit^n$ is $(\epsilon, K)$-{\sf sequentially composable} for an input distribution $\cal D$, if for every prover $\disP$, 
and every sequence of inputs 
$x,x_1,\ldots,x_k$ drawn according to ${\cal D}$ such that $C(x) \geq \sum_{i=1}^k 
\tilde{C}(x_i)$ and $k \leq K$ we have that $\sum_{i}\tilde{R}(x_i) - R \leq \epsilon$.
\end{definition}

% Some properties of Sequential Rational Proofs

The following Lemma is from \cite{cg15}. 

\begin{lemma}
\label{cor:prob}
Let $(P,V)$ and $\rew$ be respectively an interactive proof and a reward 
function as in 
Definition \ref{def:RP}; if $\rew$ can only assume the values $0$ and $R$ for 
some constant $R$, let $\pDisR = \Pr[\rew((\disP,V)(x)) = R]$. If for $x \in {\cal D}$
$$  \pDisR \leq \frac{\tilde{C}(x)}{C} + \epsilon $$
then $(P,V)$ is $(KR\epsilon, K)$-sequentially composable for $\cal D$. 
% If rew can be only R or 0 then the sufficient condition is on the probability
\end{lemma}






It is not trivial to give a definition of sequential composability, since it is not possible to claim the above for {\em any} prover $\disP$ and {\em any} sequence of inputs. Indeed it
is possible that for a given input $\tilde{x}$, the prover $\disP$ has "hardwired" the correct value $\tilde{y}=f(\tilde{x})$ and can compute it without investing 
any work. We therefore propose a definition that holds for inputs randomly chosen according to a given probability distribution $\mathcal D$, and we allow for
the possibility that the reward of a dishonest prover can be "negligibly" larger than the reward of the honest prover (for example if $\disP$ is lucky and such 
"hardwired" inputs are selected by $\mathcal D$).

% Some additional points of discussion.
{\sc Cost Function}
In the most general version of our model (we will propose a slightly more concrete version in Section \ref{sec:proofs-seq-comp}), we use an abstract notion of cost, defined
as a function that, given an algorithm - a prover - and a binary string - an input - returns a positive integer.
We assume that cost is \textit{additive among inputs}. Also, we will assume that cost is the highest for the honest prover
(intuitively, a cheating prover is trying to save on computational resources). 

Examples of possible specific cost functions are:
\begin{itemize}
	\item in a simple threshold gate: a function that counts the number of input bits read;
	\item in a specific circuit, a function that counts the number of wires read from it (used in \cite{cg15});
	\item in a Turing Machine, the number of transitions computed (we will assume a variant of this cost function in Section \ref{sec:proofs-seq-comp});
\end{itemize} 

% TODO: Definition of sequential composability here?

To be  able to prove results on sequential composability we also have to augment our model in another dimension.
In certain domains, an algorithm that performs a "smaller computation" than the "best" correct one will have  a non-negligible error probability.
Such an intuition in the context of algorithms and complexity is captured by lower bounds, such as
computing the parity of a binary string requires reading all $n$ bits,
distributions of numbers hard for primality tests [\textbf{XXX: citation needed}],
depth-hardness in "regular circuits" such as the one to compute FFT (see Definition 4 and Lemma 6 in \cite{cg15}).
The challenge for sequential composability results is to leverage how worse a "lazy" prover will perform
given it has spent $\tilde{C}$. In order to do this we need to assume a connection between the
$\tilde{C}$, the amount/type of computation and the distribution of the inputs.

{\sc On cost functions and hardness}
In the real world  setting cost is plausibly an increasing function of "amount of computation" (for some definition of computation and complexity measures on it).
Thus the assumptions we will make on cost are strongly related to the notion of hardness.
As mentioned earlier, a realistic model could not be based on worst-case hardness, but would need to assume 
a distribution of inputs that are hard  for certain classes of algorithms.
In this paper we will use a simple approach based on the number of transitions computed by a prover (see Section \ref{sec:proofs-seq-comp}). 
The work in \cite{cg15} contains a more sophisticated approach (although limited to a small class of problems).
In particular the Unique Inner State Assumption in that work is strongly connected to the notion of
$\delta$-hardness in the research field of hardness amplification \cite{arora2009computational}.
Such connections are out of the scope of this paper, but are left as important directions for future work.

\textbf{TODO?: Introduce non-adaptive vs adaptive here or in the introduction?}

\textbf{TODO: Introduce notation R and C and similar...}




The definition of sequential rational proofs requires a relationship between the reward earned by the prover and the amount of "work" the prover invested to produce that result. In our protocol this means to demonstrate a link between the probability of success of a cheating prover and the amount of computation he invests.

