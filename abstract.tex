% $Log: abstract.tex,v $
% Revision 1.1  93/05/14  14:56:25  starflt
% Initial revision
% 
% Revision 1.1  90/05/04  10:41:01  lwvanels
% Initial revision
% 
%
%% The text of your abstract and nothing else (other than comments) goes here.
%% It will be single-spaced and the rest of the text that is supposed to go on
%% the abstract page will be generated by the abstractpage environment.  This
%% file should be \input (not \include 'd) from cover.tex.
	\noindent
	%Delegating computation is a prominent computing paradigm 	
	In this thesis, we study protocols for delegating computation in a model where one of the parties is rational. In our model, a \textit{delegator} outsources the computation of a function $f$ on input $x$ to a \textit{worker}, who
	receives a (possibly monetary) reward. Our goal is to design \textit{very efficient} delegation schemes 
	where a worker is economically incentivized to provide the correct result
	$f(x)$. In this work we strive for not relying on cryptographic assumptions, in particular our results do not require the existence of one-way functions. 
	
	We provide several results within the framework of rational proofs introduced by Azar and Micali (STOC 2012).
	We make several contributions to efficient rational proofs for general feasible computations.
	First, we design schemes with a sublinear verifier with low round and communication complexity for
	space-bounded computations.
	Second, we provide evidence, as lower bounds, against the existence of rational proofs:
	with logarithmic communication and polylogarithmic verification for $\P$ and 
	with polylogarithmic communication for $\NP$.
	% Third, we generalize a scaling approach in Azar and Micali (STOC) ... [composition theorem]
	
	We then move to study the case where a delegator outsources multiple inputs.
	First, we formalize an extended notion of rational proofs for this scenario (sequential composability) and we
	show that existing schemes do not satisfy it. We show how these protocols incentivize workers
	to provide many ``fast'' incorrect answers which allow them to solve more problems and collect more rewards.
	We then design a $d$-rounds rational proof for sufficiently ``regular'' arithmetic circuit of depth $d = O(\log{n})$
	with sublinear verification. We show, that under certain cost assumptions, our scheme is sequentially composable,
	i.e. it can be used to delegate multiple inputs. We finally show that our scheme for space-bounded computations is also 
	sequentially composable under certain cost assumptions.
	
	%The results above have been published as proceedings in GameSec as Campanelli and Gennaro (GameSec 2015)
	%and Campanelli and Gennaro (GameSec 2017).
	
  In the last part of this thesis we initiate the study of {\em Fine Grained Secure Computation}: i.e. 
  the construction of {\em secure computation primitives} against ``moderately complex" adversaries. Such fine-grained protocols can be used to obtain sequentially composable rational proofs. We present definitions and constructions for \textit{compact} Fully Homomorphic Encryption and Verifiable Computation secure against (\textit{non-uniform}) $\mathsf{NC}^1$ adversaries. Our results hold under a widely believed separation assumption, namely $\fgAssump$. We also present two application scenarios for our model: \textit{(i)} hardware chips that prove their own correctness, and \textit{(ii)} protocols against rational adversaries potentially relevant to the {\em Verifier's Dilemma} in smart-contracts transactions such as Ethereum. 
