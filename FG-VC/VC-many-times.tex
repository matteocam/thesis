
\subsection{A Reusable Verification Scheme}
\label{sec:vc-many}
We now describe how to obtain a reusable verification scheme $\tVC$ applying the transformation in \cite{ckv10} from one-time sound verification schemes through fully homomorphic encryption. The core idea behind the transformation in \cite{ckv10} is to encapsulate all the operations of a one-time verifiable computation scheme through homomorphic encryption. We instantiate this transformation with the one-time verifiable construction $\VC$, described in Figure \ref{fig:one-time}, and the simplest of our two somewhat homomorphic encryption schemes, $\HE$ (defined in Section \ref{sec:leveled-he-simple}). 

\begin{figure}
\begin{framed}
Let $\VC$ be the verifiable computation scheme defined in Figure \ref{fig:one-time}. The reusable verifiable computation scheme $\tVC = (\tVCKG,\tVCPG,\tVCCompute,\tVCVerif)$ is defined as follows.
\begin{itemize}
\item $\tVCKG(1^\lambda, f) \rightarrow (\tpkW, \tskD)$: The key generation stage is the same as in $\VC$.
\item $\tVCPG_{\tskD}(x) \rightarrow (\tqx, \tsx)$: 
\begin{enumerate}
\item $(\qx, \sx) \gets \VCPG_{\skD}(x)$;
\item Compute a fresh pair of keys $(\pk_x, \sk_x) \gets \HEKeygen(1^\lambda)$;
\item Compute $\hat{q}_x \gets \HEEnc_{\pk_x}(\qx)$;
\item $\tqx \gets (\pk_x, \hat{q}_x)$; $\tsx \gets (\sx, \sk_x)$
% \item Compute $t$ independent encryptions $\hatr_{i+t} = \HEpEnc_{\pk}(x)$ for $i \in [t]$.
% \item Sample a random permutation $\pi \sample S_{2t}$.
% \item $\qx \gets (\hatzpi{1},\dots,\hatzpi{2t}) = (\hatr_1,\dots,\hatr_{2t})$; $\sx \gets \pi$
\end{enumerate}
\item $\tVCCompute_{\tpkW}(\tqx) \rightarrow \tax$:
\begin{enumerate}
\item $\hat{a}_x \gets \HEEval_{\pk_x}(\VCCompute(\cdot, f), \hat{q}_x)$.
\item $\tax \gets \hat{a}_x$.
\end{enumerate}
\item $\tVCVerif_{\tskD}(\tsx, \tax)$:
\begin{enumerate}
\item $\ax \gets \HEDec_{\sk_x}(\hat{a}_x)$.
\item $\function{return}~\VCVerif_{\tskD}(\sx, \ax) $.
\end{enumerate}
\end{itemize}
\end{framed}
\caption{Transformation from one-time $\cal{VC}$ scheme to a \textit{reusable} $\cal{VC}$ scheme}
\label{fig:reusable-transform}
\end{figure}



\begin{corollary}[Completeness of $\tVC$]
The verifiable computation scheme $\tVC$ in Figure \ref{fig:reusable-transform} has overwhelming completeness (Definition \ref{def:vc-completeness}) for the class $\ACztq$.
\end{corollary}
\begin{proof}
The completeness of the scheme above follows directly from the completeness of $\VC$ and the homomorphic properties of $\HE$.
Notice that we can use $\HEEval$ to homomorphically compute $\VCCompute$ as the latter carries out a computation in $\ACztcm$ (although it is \textit{approximating} a computation in $\ACztq$). 
\end{proof}

\begin{remark}[Efficiency of $\tVC$]
The efficiency of $\tVC$ is analogous to that of $\VC$ with the exception of a circuit size overhead of a factor $O(\lambda)$ on the problem generation and verification algorithms and of $O(\lambda^2)$ for the computation algorithm.
All algorithms in $\tVC$ are computable by constant depth circuit (of unbounded fan-in) and the depth of the verification algorithm is independent of the function $F$.
\end{remark}


\begin{theorem}
Under the assumption that $\fgAssump$ the scheme $\tVC$ in Figure \ref{fig:reusable-transform} is $(O(1),\poly(\lambda))$-sound (many-times secure) against $\NC^1$ adversaries whenever t is chosen to be  $\omega(\log(\lambda))$ in the underlying scheme $\VC$. 
\end{theorem}
\begin{proof}
By Lemma \ref{lemma:one-time} there exists an infinite set $\Lambda \subseteq \naturals$ of security parameters for which $\VC$ ``is secure''.
By the proof of Lemma \ref{lemma:one-time}, this set is also the set of parameters where the somewhat homomorphic encryption scheme $\HE$ ``is secure''.
We will show that for all values in this same set $\Lambda$, the probability of success of any  $\NC^1$ adversary in $\expVCmany{\tVC}$ is negligible. 

Assume by contradiction there exists an $\NC^1$ adversary $\advstar$ that achieves non-negligible advantage in $\expVCmany{\tVC}$ for some $\lambda \in \Lambda$.

\textbf{Claim: If $\tVC$ is not secure for some $\lambda^* \in \Lambda$ then we can break the one-time security of $\VC$.}
Let $l = O(1)$ be the number of rounds in the many-time soundness experiment for $\tVC$. Consider the following $\NC^1$ adversary  $\adv_1$ for the experiment $\expVCone{\VC}$:
\begin{itemize}
\item $\adv_1$ obtains a pair a public key $\pkW$ and sends it to $\advstar$;
\item For all rounds $i \in \{1,\dots,l-1 \}$, $\adv_1$ replies to
$\advstar$ queries by generating a fresh pair of keys $(\pk, \sk)$ and sending back encryptions of $\HEEnc_{\pk}(\zero)$;
\item At round $l$, $\adv_1$ responds to all input queries but the last one as above. This, by experiment definition,  is the input where $\advstar$ will try to cheat; we denote this input by $x^*$. Now $\adv_1$ sends $x^*$ as the only input query in the one-time security experiment and will receive back $q^*$. It will then obtain a fresh pair of keys $(\pk^*, \sk^*)$ and send $\HEEnc_{\pk^*}(q^*)$ to $\advstar$.
\item $\advstar$ will respond with $\hata^*$ and $\adv_1$ will send $\HEDec_{\sk^*}(\hata)$ to the challenger for one-time security experiment.
\end{itemize}

The advantage of $\adv_1$ depends on how likely is  $\advstar$ can successfully cheat in that interaction. Let $p$ be the advantage of $\adv_1$ in the one-time security experiment. Clearly, if $p$ is close to the advantage of $\advstar$ in the many-times security experiment $\adv_1$ breaks the security of the one-time scheme.

\textbf{Claim: the advantage of $\adv_1$ is negligibly close to that of $\advstar$ in the many-time security game for security parameter $\lambda^*$.}
We can  prove this by relying on the semantic security of the homomorphic encryption and on a hybrid argument.
% Assume by contradiction that $p$ is noticeably far from the advantage of $\advstar$ in the many-time security game for all but a finite number of values of the security parameter.

Let $L = lm$, the total number of input queries in the many-times security experiment. We now define the hybrids $H^{(j)}$ with $j \in \{0, \dots, L\}$. We define $H^{(0)}$ to be the exactly the many-time security experiment. 
For $j \in [L]$ we define $H^{(j)}$ to be an experiment where we respond to input queries with $\HEEnc_{\pk_f}(\zero)$  where $\pk_f$ is a fresh public key up to input query $j$ and behaves the many-time security experiment from input query $j+1$ on. Notice that $H^{(L)}$ corresponds to the interaction with $\adv_1$ above.

Denote by $A^{(j)}$ the output distribution of $\advstar$ when interacting with $H^{(j)}$.
Intuitively, if the advantage of the $\adv_1$ in the one-time experiment is significantly different from the advantage of $\advstar$ in the many-times security games, then $A^{(0)}$ and $A^{(L)}$ are not $\Lambda$-computationally indistinguishable.

Therefore (by Lemma \ref{lemma:hyb-arg}), there exists $j \in [L]$ such that $A^{(j-1)} \not \sim_{\Lambda} A^{(j)}$.

\def\advcpa{\adv_{\text{CPA}}}
\textbf{Claim: If there exists $j \in [L]$ such that $A^{(j-1)} \not \sim_{\Lambda} A^{(j)}$ then we can break the semantic security of $\HE$.
}
Consider the following $\NC^1$ adversary $\advcpa$ which receives in input a ``challenge'' public key $\pk^*$. $\advcpa$ will interact with $\adv^*$ simulating $H^{(j)}$ until receiving input query $x_{j}$.
At this point it will compute $q_j$ from $\VCPG(x_j)$ and send to the CPA challenger (see Remark \ref{rem:cpa-multiple}) $q_j$ and $\zero$, receiving back an encryption $c^*$ of either message under the public key $\pk^*$. $\advcpa$ will now send $(\pk^*, c^*)$ to $\advstar$ and continue simulating $H^{(j)}$ till the end of the experiment.
The adversary $\advcpa$ will check whether $\advstar$ cheated successfully at the end of the experiment and output (in the multiple-message CPA experiment) $1$ if that is the case and $0$ otherwise. This would allow $\advcpa$ to have a noticeable advantage in the experiment thus breaking the semantic security of $\HE$. 
\end{proof}

% \begin{remark}
% On  the complexity of preparing a circuit version of $\HEpEval(\cdot, F)$ in the offline stage.
% \color{red}{TODO}
% \end{remark}
