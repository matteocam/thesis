\subsection{A One-time Verification Scheme}

Before we present our variant of the one-time construction in \cite{ckv10}, we present two auxiliary lemmas that guarantee that our protocols are computable in $\ACzt$. We refer the reader to \cite{hag91,matias91} for the proof Lemma \ref{lemma:perm-sampling}. % and we skip the proof of the second lemma since it is trivial.

\begin{lemma}{\cite{hag91,matias91}}
\label{lemma:perm-sampling}
There are uniform $\AC^0$ circuits $C: \bit^{\poly(l)}\to [l]^l$ of size $\poly(l)$ and depth $O(1)$ whose output distribution have statistical distance $\leq 2^{-l}$ from the uniform distribution over permutations of $[l]$.
\end{lemma}

\begin{lemma}
\label{lemma:perm-evaluation}
There are uniform $\AC^0[2]$ circuits $C: [l]^l \times \bit^l \to \bit^l$ of size $O(l^2)$ where $C(\pi, (x_1,\dots, x_l)) = (\pi(1),\dots,\pi(l))$ and $\pi$ is a permutation.
\end{lemma}
\begin{proof}
Let $\vec{x} = (x_1,\dots,x_l)$ the bits to permute and let $\pi$ be a permutation
If $\pi$ is represented as a permutation matrix with rows $\r_1,\dots,\r_l$, we can permute $\vec{x}$ by simply performing $l$ parallel inner products $\inprod{\vec{x}}{\r_i}$-s, which is in $\ACzt$.
We now describe how to generate the permutation matrix from a binary representations $x_1,\dots,x_{\function{lg}(l)}$ of the integers in $[l]$.
Let $f_i : \bit^{\function{lg}(l)} \to \bit^l$ be the function that computes the $i$-th row of the permutation matrix. We can define $f_i$ as follows:
\[
    f_i(x_1,\dots,x_{\function{lg}(l)}) \eqdef \function{eq}([i-1]_2, (x_1,\dots,x_{\function{lg}(l)}) \, ,
\]
where $[i-1]_2$ is the binary representation of $i-1$ and $\function{eq}$ returns 1 if its two inputs (each of lenght $\function{lg}(l)$) are equal.
The function $f_i$ is clearly in $\ACzt$.
% First, for each $i$, we generate the vector of hamming weight $1$ describing $\pi(i)$ by mapping 
\qed
\end{proof}

The following is an adaptation of the one-time secure delegation scheme from \cite{ckv10}. We make non-black box use of our homomorphic encryption scheme $\HEp$ (Section \ref{sec:beyond-cm}) with soundness parameter $s= \lambda$. Notice that. during the preprocessing phase, we fix the ``auxiliary randomness'' for $\EvalApprox$ (and thus for $\HEpEval$) once and for all. We will use that same randomness for all the input instances. This choice does not affect the security of the construction.
We remind the reader that we will simplify notation by considering the evaluation key of our somewhat homomorphic encryption scheme as part of its public key.

If $x$ is a vector of bits $x_1, \gets, x_n$, below we will denote with $\HEpEnc(x)$ the concatenation of the bit by bit ciphertexts $\HEpEnc(x_1), \dots, \HEpEnc(x_n)$. We denote by $\HEpEnc(\zero)$ the concatenation of $n$ encryptions of $0$, $\HEpEnc(0)$.

% \item Compute $\hatraux \gets \SampleAuxRandomness_s(\pk, F)$;
% \item Compute $\vec{c} \gets \EvalApprox_s(\pk, F, \hatraux)$;
% \item Output $\vec{c} = (c^{\text{out}}_1, \dots, c^{\text{out}}_{s})$.

\begin{figure}
\begin{framed}
Let $f: \bit^n \to \bit^m$ be a function and $\GenApproxFun, \SampleAuxRandomness$ and $\EvalApprox$ as in Definition \ref{def:aux-he-fns}.
\begin{itemize}
\item $\VCKG(1^\lambda, f) \rightarrow (\pkW, \skD)$: We assume function $F$ represented as 
\begin{enumerate}
\item Generate a pair of keys $(\pk,\sk) \gets \HEpKeygen(1^\lambda)$.
\item Generate the approximating function $f' \gets \GenApproxFun(f)$;
\item Generate the ciphertext of the auxiliary random input for homomorphic evaluation $\hatraux \gets \SampleAuxRandomness_{\lambda}(\pk, f')$
\item Compute $t$ independent encryptions $\hatr_i = \HEpEnc_{\pk}(\zero)$ and the homomorphic evaluations $\hatw_i = \hatF(\hatr_i) =  \EvalApprox_s(\pk, f', \hatr_i, \hatraux)$ for $i \in [t]$. 
\item $\pkW \gets (\pk, f', \hatraux), \skD \gets \{ (\hatr_i, \hatw_i)_{i \in [t]}\}$.
\end{enumerate}
\item $\VCPG_{\skD}(x) \rightarrow (\qx, \sx)$: 
\begin{enumerate}
\item Compute $t$ independent encryptions $\hatr_{i+t} = \HEpEnc_{\pk}(x)$ for $i \in [t]$.
\item Sample a random permutation $\pi \sample S_{2t}$.
\item $\qx \gets (\hatzpi{1},\dots,\hatzpi{2t}) = (\hatr_1,\dots,\hatr_{2t})$; $\sx \gets \pi$
\end{enumerate}
\item $\VCCompute_{\pkW}(\qx) \rightarrow \ax$:
\begin{enumerate}
\item Compute $\hat{y}_i = \hatF(\hat{z}_i) = \EvalApprox_s(\pk, f', \hat{z}_i, \hatraux)$ for $i \in [2t]$.
\item $\ax = (\hat{y}_1,\dots,\hat{y}_{2t})$.
\end{enumerate}
\item $\VCVerif_{\skD}(\sx, \ax)$:
\begin{enumerate}
\item Check if $\hatw_i = \hat{y}_i$ for all $i \in [t]$.
\item Check if $\HEpDec_{\sk}(\hat{y}_{\pi(t+1)}) = \dots = \HEpDec_{\sk}(\hat{y}_{\pi(2t)})$.
\item If either of the two tests above fails, return $\fail$; otherwise return $\HEpDec_{\sk}(\hat{y}_{\pi(t+1)})$.
\end{enumerate}
\end{itemize}
\end{framed}
\caption{One-Time Delegation Scheme}
\label{fig:one-time}
\end{figure}

\begin{remark}[On deterministic homomorphic evaluation]
As pointed out in \cite{ckv10}, one requirement for the approach above to work is for the homomorphic evaluation to be deterministic. We point out that once $\hatraux$ are fixed once and for all the homomorphic evaluation in $\VCCompute$ is deterministic.
\end{remark}

\begin{remark}[On including $f'$ in $\pkW$]
In the construction above we included $f'$ in the public key lengthening the size of the key. We point out this is not necessary and that $f'$ can be computed by the worker on her own during the execution of $\VCCompute$. However this would not allow us to simply homomorphically evaluate $\VCCompute$ in the definition of $\tVC$ in Section \ref{sec:vc-many}. This because the complexity of $\VCCompute$ would go from $\ACztcm$ to $\NC^1$, which our homomorphic schemes cannot handle.
We point out that it would still be possible to modify the construction of $\tVC$  not including $f'$ in $\pkW$ to obtain the same completeness and soundness properties. However this would come at a cost of a more complex transformation in Figure \ref{fig:reusable-transform}. Including $f'$ in $\pkW$ allowed us to kept the transformation as simple and close to the original description in \cite{ckv10} as possible. 
\end{remark}


\begin{lemma}[Completeness of $\VC$]
The verifiable computation scheme $\VC$ in Figure \ref{fig:one-time} has overwhelming completeness (Definition \ref{def:vc-completeness}) for the class $\ACztq$.
\end{lemma}
\begin{proof}
The proof is straightforward and stems directly from the homomorphic properties of $\HEp$ (Theorem \ref{thm:hep-homomorphic}).
In fact, by construction and by definition of $\HEp$ (Section \ref{sec:beyond-cm}), the distribution of the $\hatw_i$-s is identical to $\HEpEval_{\pk}(F, \hatr_i)$. Analogously, the distribution of $\hat{y}_i$-s is identical to $\HEpEval_{\pk}(F, \hat{z}_i)$. \qed
\end{proof}


\begin{remark}[Efficiency of $\VC$]
In the following we consider the verifiable computation of a function $F: \bit^n \to \bit^m$ computable by an $\ACztq$ circuit of size $S$.
\begin{itemize}
\item  $\VCKG$ can be computed by an $\NC^1$ circuit of size $O(\poly(\lambda)S)$; 
\item $\VCPG$ can be computed by an $\ACzt$ circuit of size $O(\poly(\lambda)(m+n))$;
\item $\VCCompute$ can be computed by an $\NC^1$ circuit of size $O(\poly(\lambda)S)$; 
\item $\VCVerif$ can be computed by a $\TC^0$ circuit of size $O(\poly(\lambda)(m+n))$ and whose (constant) depth is independent of the depth of $F$.
\end{itemize}
\end{remark}

\begin{lemma}[One-time Soundness] 
\label{lemma:one-time}
Under the assumption that $\fgAssump$ the  scheme in Figure \ref{fig:one-time} is $(1,1)$-sound (one time secure) against $\NC^1$ adversaries whenever t is chosen to be  $\omega(\log(\lambda))$.
\end{lemma}
\begin{proof}
We follow the same proof structure as in the proof of Lemma $12$ in \cite{ckv10}. We will keep part of the analysis informal, emphasizing why this proof still works for low-depth circuits. We refer the reader to \cite{ckv10} for further details.

The following observation will be crucial in the rest of the proof. Notice that, by construction and by definition of $\HEp$ (Section \ref{sec:beyond-cm}), the distribution of the $\hatw_i$-s is identical to $\HEpEval_{\pk}(F, \hatr_i)$. Analogously, the distribution of $\hat{y}_i$-s is identical to $\HEpEval_{\pk}(F, \hat{z}_i)$.

Consider an $\NC^1$ adversary $\advstar$ that cheats with non-negligible probability in the one-time security experiment $\expVCone{\VC}$ (Definition \ref{def:vc-soundness}).
Let $(\hatr_1,\dots,\hatr_{t})$ be the independent copies of $\HEpEnc_{\pkW}(\zero)$ and  $(\hatr_{t+1},\dots,\hatr_{2t})$ the $t$ independent copies of
$\HEpEnc_{\pkW}(x)$ as above. 
Whenever the verification algorithm accepts, the adversary must have responded correctly on $\hatr_1,...,\hatr_t$ and incorrectly (and consistently) on $\hatr_{t+1},\dots,\hatr_{2t}$. 
Our goal is to bound the probability that the adversary succeeds in doing that. 

First, notice that the view of the adversary is
$(\pkW, \hatr_1,\dots,\hatr_{2t})$, and identical to $(\pkW, \HEpEnc_{\pkW}(\zero)^t, \HEpEnc_{\pkW}(x)^t)$.
By semantic security of the homomorphic encryption scheme, there exists an infinitely large set of parameters $\Lambda$ such that $ (\pkW, \HEpEnc_{\pkW}(\zero)^t, \HEpEnc_{\pkW}(x)^t) \lind (\pkW, \HEpEnc_{\pkW}(\zero)^{2t})$. Consider a modified game where the adversary receives  $(\pkW, \HEpEnc_{\pkW}(\zero)^{2t})$. Denote by $p$ the probability that the adversary succeeds in this game. By computational indistinguishability we have
\[
\Pr[\advstar \text{ is correct on } (\hatr_1,\dots,\hatr_{t}) \text{ and incorrect on } (\hatr_{t+1},\dots,\hatr_{2t}) ] \leq p + \negl(\lambda)
\]
for all $\lambda \in \Lambda$.
This inequality holds because we can test in $\NC^1$ whether $\advstar$ cheats only on $(\hatr_{t+1},\dots,\hatr_{2t})$. Therefore, if the adversary's behavior differed significantly between the two games, one would be able to break the semantic security of the homomorphic scheme. Here we made use of the third fact in Lemma $\ref{lemma:facts-lind}$.

We now proceed to upper bound $p$. Observe that 
\[ 
p = \Pr[\advstar \text{ is correct on } (\hatzpi{1},\dots,\hatzpi{t}) \text{ and incorrect on } (\hatzpi{t+1},\dots,\hatzpi{2t})] 
\]
where the $\hatzpi{i}$-s are defined as in Figure \ref{fig:one-time}.
Because of Lemma \ref{lemma:perm-sampling} that the distribution of $\pi$ is statistically indistinguishable from that of a uniformly random permutation. Also, observe that the answers $\hat{y}_i$ of the adversary are independent of $\pi$.
We can then conclude that $p \leq \frac{1}{\binom{2t} {t}} + \negl(t)$, which concludes the security analysis. \qed
% TODO: define z as in the other case


% \begin{align*}
%     p &= \Pr[\advstar \text{ is correct on } (\hatzpi{1},\dots,\hatzpi{t}) \text{ and incorrect on } (\hatzpi{t+1},\dots,\hatzpi{2t})] \\
%     &\leq \Pr[\advstar \text{ is correct on } (\hatzpi{1},\dots,\hatzpi{t}) \text{ and incorrect on } (\hatzpi{t+1},\dots,\hatzpi{2t}) \given[\Big] \sum^{2t}_{i=1} ]
% \end{align*}

\end{proof}