%\documentclass{llncs}
%\usepackage{booktabs} % For formal tables
%
%\usepackage{times}
%\usepackage{complexity}
%\usepackage{url}
%\usepackage{latexsym}
%\usepackage[utf8]{inputenc}
%\usepackage{framed}
%\usepackage{natbib}
%\usepackage{graphicx}
%%\usepackage{amsthm}
%\usepackage{amssymb}
%\usepackage{amsmath}
%\usepackage{verbatim}
%%\usepackage{enumitem}
%\usepackage{verbatim}
%%\usepackage{accents}
%\usepackage{bm}
%%\usepackage[normalem]{ulem}
%%\setlength\titlebox{6.5cm}    % You can expand the title box if you
%% really have to
%
%\DeclareMathOperator*{\argmin}{arg\,min}
%\DeclareMathOperator*{\argmax}{arg\,max}
%
%
%
%\begin{document}
%\title{Efficient Rational Proofs for Space Bounded Computations}
%\author{Matteo Campanelli \and Rosario Gennaro}
%
%\institute{The City University of New York \\
%	{ mcampanelli@gradcenter.cuny.edu} \hspace{.5in} { rosario@ccny.cuny.edu}
%}
%\maketitle

%\begin{abstract}
%	We present new protocols for the verification of {\em space bounded polytime computations} against a rational adversary. For such computations requiring sublinear space our protocol requires only a verifier running in sublinear-time.
%	
%	We extend our main result in several directions: (i) we present protocols for randomized complexity classes, using a new {\em composition theorem} for rational proofs which is of independent interest; (ii) we present lower bounds (i.e. conditional impossibility results) for Rational Proofs for various complexity classes.
%	%we discuss the notion of Rational Arguments (where the adversary is assumed to 
%	%be both rational and computationally bounded) and show that it is achievable for 
%	%all interesting complexity classes; (iii)
%	
%	Our new protocol is the first rational proof not based on the circuit model of computation, and the first {\em sequentially composable} protocols for a well-defined language class.
%\end{abstract}



%% Theorems

\newenvironment{proofsketch}{\paragraph{\textbf{Proof Sketch:}}}{\hfill$\square$}

\newtheoremstyle{exampstyle}
{5pt} % Space above
{0pt} % Space below
{} % Body font
{} % Indent amount
{\bfseries} % Theorem head font
{.} % Punctuation after theorem head
{.5em} % Space after theorem head
{} % Theorem head spec (can be left empty, meaning `normal')

\theoremstyle{exampstyle}% default 
\newtheorem{theorem}{Theorem}[section] 
\newtheorem{lemma}[theorem]{Lemma} 
\newtheorem{proposition}[theorem]{Proposition} 
\newtheorem{property}[theorem]{Property}
\newtheorem{corollary}[theorem]{Corollary} 

\theoremstyle{definition} 
\newtheorem{definition}[theorem]{Definition}
\newtheorem{conjecture}[theorem]{Conjecture}
\newtheorem{assumption}[theorem]{Assumption}
\newtheorem{example}[theorem]{Example}
%\newtheorem{exer}[theorem]{Exercise}

\theoremstyle{plain}
\newtheorem{question}{Question}
\newtheorem{result}{Result}[question]
\newtheorem{resultprogress}{Result (in progress)}[question]

\theoremstyle{definition} 
\newtheorem{remark}{Remark} 



%\newcommand{\Sum}{\mathlarger{\sum}}

%\def\definition{definition}
%\def\theorem{theorem}
%\def\lemma{lemma}
%\def\corollary{corollary}
%\def\assumption{assumption}
%\def{\remark}{remark}
  
% Tau
\newcommand{\Tau}{\mathcal{T}}

% underbar
%\newcommand{\ubar}[1]{\underaccent{\bar}{#1}}
\newcommand{\ubar}[1]{\uline{#1}}

% Theorems
%\newtheorem{definition}{Definition}
%\newtheorem{lemma}{Lemma}
%\newtheorem{mytheorem}{Theorem}
%\newtheorem{corollary}{Corollary}



% New complexity classes
%\newclass{\DTISP}{DTISP}
\newclass{\NTISP}{NTISP}
%\newclass{\SC}{SC}
\newclass{\NSC}{NSC}
%\newclass{\NC}{NC}
\newclass{\DRMA}{DRMA}
\newclass{\MRIP}{MRIP}

% Rational Proofs
%\newcommand{\A}{Arthur }
%\newcommand{\M}{Merlin }

\newcommand{\Eval}{\mathsf{Eval}}

\newcommand{\nrange}{[n]}

\newcommand{\claimedy}{\tilde{y}}
\newcommand{\posreals}{\reals_{\geq 0}}

\newcommand{\cost}{c}
\newcommand{\disCost}{\tilde{c}}

\newcommand{\costP}{\cost(P,x)}
\newcommand{\xVec}{\widetilde{X}}
\newcommand{\costDisPMany}{m\cdot\costDisP}
\newcommand{\costDisP}{\cost(\disP, x )}

\newcommand{\disP}{\widetilde{P}}
\newcommand{\disPSm}{\widetilde{P}^*}
\newcommand{\bfsP}{\widetilde{P}_{BFS}}

\newcommand{\prCh}{p_{cheat}}
\newcommand{\ratioCosts}{\frac{\cost_H(x)}{\disCLB}}

\newcommand{\function}[1]{\ensuremath{\mathsf{#1}}}

\newcommand{\Size}{\function{Size}}
\newcommand{\out}{\function{out}}
\newcommand{\rew}{\function{rew}}
\newcommand{\profit}{\function{profit}}
%\newcommand{\poly}{\function{poly}}
\def\negl{\function{neg}}
%\newcommand{\reward}{\function{reward}}
%\newcommand{\log}{\function{log}}

\newcommand{\invPoly}{\frac{1}{\poly}}

\newcommand{\expRewProtDis}{\expectation[\rew((\disP,V)(x))]}
\newcommand{\expRewProtDisMany}{\expectation[\rew((\disP,V)(\xVec))]}
\newcommand{\expRewProtHon}{\expectation[\rew((P,V)(x))]}
\newcommand{\prOutProtDis}{\Pr[\out((\disP,V)(x)) \neq f(x)]}
\newcommand{\prOutProtDisMany}{\Pr[\out((\disP,V)(\xVec)) \neq f(\xVec)]}

% expected profit variants
\newcommand{\expProfitProtDis}{\expectation[\profit((\disP,V)(x))]}
\newcommand{\expProfitProtDisMany}{\expectation[\profit((\disP,V)(\xVec))]}
\newcommand{\expProfitProtHon}{\expectation[\profit((P,V)(x))]}

\newcommand{\pDisR}{\tilde{p}_x}

\newcommand{\rnd}{\rho}

\newcommand{\PathCheck}{\function{PathCheck}}

\newcommand{\circuit}{\cal{C}}

\newcommand{\circDFS}{\cal{C}_{DFS}}
\newcommand{\circBFS}{\overline{\cal{C}}_{BFS}}
\newcommand{\circNA}{\overline{\cal{C}}_{NA}}

\newcommand{\circEval}{\cal{C}_{\mbox{eval}}}
\newcommand{\circFFT}{\cal{C}_{\mbox{FFT}}}
\newcommand{\circPow}{\cal{C}_{\mbox{pow}}}

\newcommand{\tildeltaFFT}{\tildelta_{\mbox{FFT}}}

\newcommand{\probCheatFFT}{\tilde{p}_{\mbox{FFT}}}
\newcommand{\probCheatPow}{\tilde{p}_{\mbox{pow}}}

\newcommand{\tildelta}{\tilde{\delta}}
\newcommand{\investTildelta}{\tilde{C}_{\tildelta}}

\newcommand{\deltaBfsInv}{\delta_{BFS}(\tilde{C})}

\newcommand{\disR}{\tilde{R}}
\newcommand{\disRUB}{R'}
\newcommand{\disCLB}{\cost'}

\newcommand{\rewGapOneTime}{\Delta^{\rew}_{1}}
\newcommand{\profGapSeq}{\Delta^{\profit}_{\infty}}
\newcommand{\rewRatio}{\alpha^{\rew}}

% Announced distribution
\newcommand{\annd}{\hat d}
% Real distribution
\newcommand{\reald}{d}

%\newcommand{\expectation}{\mathbb{E}}
\newcommand{\expectation}{\mathop{\mathbb{E}}}

% Some shortcuts for math symbols
\newcommand{\binstrings}{\{0, 1\}^{*}}
\newcommand{\bits}{\{0, 1\}^{*}}
%\newcommand{\iff}{\Leftrightarrow}
\newcommand{\bit}{\{0, 1\}}
\newcommand{\funonstrings}{: \binstrings \to \binstrings}

\newcommand{\naturals}{\mathbb{N}}\label{key}
\newcommand{\reals}{\mathbb{R}}
\newcommand{\field}{\mathbb{F}}

\newcommand{\transp}[1]{{#1}^{\intercal}}


% Circuit Family
\newcommand{\circfam}{\{C_n\}_{n=1}^{\infty}}
\newcommand{\langL}{\mathcal{L}}
% Interactive Proofs
\newcommand{\transc}{\mathcal{T}}

\makeatletter
\newcommand{\verbatimfont}[1]{\def\verbatim@font{#1}}%
\makeatother

\newclass{\BPTISP}{BPTISP}
\newclass{\BPNC}{BPNC}
\newclass{\osDRMA}{osDRMA}
\newclass{\BPRMA}{BPRMA}
%\newclass{\BPQP}{BPQP}
\newclass{\Cclass}{C}

\newcommand{\cb}[1]{\colorbox{BurntOrange}{#1}}
\newcommand{\CN}{\cb{\textbf{[CN]}}}
\newcommand{\XXX}{\cb{\textbf{[XXX]}}}

\begin{comment}
\newtheorem{definition}{Definition}
\newtheorem{lemma}{Lemma}
\newtheorem{claim}{Claim}
\newtheorem{theorem}{Theorem}
\newtheorem{assumption}{Assumption}
\newtheorem{property}{Property}
\newtheorem{question}{Question}
\newtheorem{result}{Result}[question]
\newtheorem{resultprogress}{Result (in progress)}[question]
\newtheorem{corollary}{Corollary}
\end{comment}

\newcommand{\DOM}{\function{DOM}}
\newcommand{\F}{\mathbb{F}}

\newcommand{\allhonest}{\function{all\_honest}}
\def\true{\function{true}}
\def\false{\function{false}}

\newcommand{\Enc}{\function{Enc}}
\def\pk{\function{pk}}
\def\sk{\function{sk}}

 \newcommand{\inprod}[2]{\langle #1 \, , #2 \rangle}


%% Shortcuts specific to Composition
\newcommand{\protOne}{\pi^{f_2}_{1}}
\newcommand{\protTwo}{\pi_{2}}
\newcommand{\rewGap}{\Delta}
\newcommand{\STEP}{STEP}

\newcommand{\disTransc}{\tilde{\Tau}}


\newcommand{\eqdef}{\vcentcolon=}
\newcommand{\binstring}{\{0, 1\}}

\newcommand{\ACzt}{\AC^0[2]}
\newcommand{\ACztq}{\AC^0_{\text{Q}}[2]}
\newcommand{\ACztcm}{\AC^0_{\text{CM}}[2]}

\newcommand{\ACzts}{\AC^0[2]^*}

\def\ANDgt{\function{AND}}
\def\ORgt{\function{OR}}
\def\XORgt{\function{XOR}}


\newcommand{\EqDef}{\stackrel{\mathrm{def}}{=}}

\newcommand{\matr}[1]{\mathbf{#1}}

\newcommand{\unlambda}{1^{\lambda}}
\newcommand{\lambdabits}{\bit^{\lambda}}
\newcommand{\GF}{\text{GF}}
 

\newcommand{\ens}[1]{\{#1_{\lambda}\}_{\lambda\in\naturals}}
\newcommand{\funfam}[1]{\{#1_{\lambda}\}_{\lambda\in\naturals}}

\def\lind{\sim_{\Lambda}}


% == BEGIN Public-Key Encryption ==
\newcommand{\PKE}{\function{PKE}}
\newcommand{\PKEKeygen}{\function{PKE.Keygen}}
\newcommand{\PKEEnc}{\function{PKE.Enc}}
\newcommand{\PKEDec}{\function{PKE.Dec}}
% == END Public-key Encryption ==

% == BEGIN DVV16 ==
\def\M{\matr{M}}
\def\Mz{\matr{\hat{M}}_0^{\lambda}}
\def\Mo{\matr{\hat{M}}_1^{\lambda}}

\newcommand{\KSample}{\function{KSample}}

\def\r{\mathbf{r}}
\def\k{\mathbf{k}}
\def\c{\mathbf{c}}
\def\t{\mathbf{t}}

% == END DVV16 ==

% == BEGIN Homomorphic Encryption ==
\providecommand{\evk}{\pckeystyle{evk}}


\newcommand{\HE}{\function{HE}}
\newcommand{\HEKeygen}{\function{HE.Keygen}}
\newcommand{\HEEnc}{\function{HE.Enc}}
\newcommand{\HEDec}{\function{HE.Dec}}
\newcommand{\HEEval}{\function{HE.Eval}}

\newcommand{\HEp}{\function{HE'}}
\newcommand{\HEpKeygen}{\function{HE'.Keygen}}
\newcommand{\HEpEnc}{\function{HE'.Enc}}
\newcommand{\HEpDec}{\function{HE'.Dec}}
\newcommand{\HEpEval}{\function{HE'.Eval}}

\def\A{\mathbf{A}}
\def\v{\vect{v}}
\def\a{\vect{a}}
\def\Sum{\mathlarger{\sum}}
\def\kl{\k_{\ell}}
\def\klp{\k_{\ell+1}}


\def\Dl{\mathcal{D}^{\func{f}}_{\lambda}}
\def\MD{\M^{\mathcal{\func{f}}}}
\def\Mkg{\M^{\text{kg}}}

% == END Homomorphic Encryption ==

% == BEGIN proof security HE (crypto 2018) ==
\def\hE{\mathcal{E}}
\def\hEz{\hE^{0}}
\def\hH{\mathcal{H}}
\def\hHz{\hH^{0}}

\def\valpha{\pmb{\alpha}}
\def\valphakg{\valpha^{\func{kg}}}
\def\valphaD{\valpha^{\func{f}}}
\def\hEnc{\function{E}}
% == END proof security HE (crypto 2018) ==

% == BEGIN Delegation of Computation ==
\newcommand{\Del}{\function{Del}}
\def\D{\function{D}}
\def\W{\function{W}}
\newcommand{\DW}{\langle \D, \W \rangle}

\def\skD{\sk_{\D}}
\def\pkW{\pk_{\W}}

\newcommand{\advstar}{\adv^*}

 
\newcommand{\hatr}{\hat{r}}
\newcommand{\hatw}{\hat{w}}
\newcommand{\hatF}{\hat{f}}

\def\zero{\bar{0}}
\def\raux{\vect{r}_{\text{aux}}}
\newcommand{\hatraux}{\vect{\hat{r}}_{\text{aux}}}

\newcommand{\hatzpi}[1]{\hat{z}_{\pi(#1)}}

\def\Gvc{{\cal G}}
\def\Wstar{\W^*}

\def\hatx{\hat{x}}
\newcommand{\hatq}{\hat{q}}
\newcommand{\hata}{\hat{a}}
\def\hats{\hat{s}}
\def\Doffline{\D_{\text{off}}}
\def\Donline{\D_{\text{on}}}
\newcommand{\Dverif}{\D_{\text{ver}}}

\newcommand{\VCKG}{\function{VC.KeyGen}}
\newcommand{\VCPG}{\function{VC.ProbGen}}
\newcommand{\VCCompute}{\function{VC.Compute}}
\newcommand{\VCVerif}{\function{VC.Verify}}

\newcommand{\tVCKG}{\function{\overline{VC.KeyGen}}}
\newcommand{\tVCPG}{\function{\overline{VC.ProbGen}}}
\newcommand{\tVCCompute}{\function{\overline{VC.Compute}}}
\newcommand{\tVCVerif}{\function{\overline{VC.Verify}}}

\newcommand{\qx}{q_x}
\newcommand{\sx}{s_x}
\newcommand{\ax}{a_x}

\def\fail{\bot}

\def\VC{{\cal VC}}
\def\tVC{\overline{{\cal VC}}}


\newcommand{\expVC}[1]{{\bf Exp}^{\function{Verif}}_{A}[#1, f, \lambda, l,m]}
\newcommand{\expVCone}[1]{{\bf Exp}^{\function{Verif}}_{A}[#1, f, \lambda, 1,1]}
\newcommand{\expVCmany}[1]{{\bf Exp}^{\function{Verif}}_{A}[#1, f, \lambda, O(1), \poly(\lambda)]}


\newcommand{\hatzpip}[1]{\hat{z}_{\pi'(#1)}}
\newcommand\given[1][]{\:#1\vert\:}


\newcommand{\tpkW}{\pkW}
\newcommand{\tskD}{\skD}
\newcommand{\tqx}{\overline{\qx}}
\newcommand{\tsx}{\overline{\sx}}
\newcommand{\tax}{\overline{\ax}}


% == END Delegation of Computation==

\def\ANDgt{\function{AND}}
\def\ORgt{\function{OR}}
\def\XORgt{\function{XOR}}


\newcommand{\pLpoly}{\oplus \L / \poly}
\newcommand{\fgAssump}{\NC^1 \subsetneq \pLpoly}

%% Shortcuts specific to Composition
%\newcommand{\protOne}{\pi^{L_2}_{1}}
%\newcommand{\protTwo}{\pi_{2}}
%\newcommand{\rewGap}{\Delta}
%\newcommand{\STEP}{\function{STEP}}
%
%\newcommand{\disTransc}{\tilde{\Tau}}

	
%	\section{Introduction}
%	Consider the problem of {\sf Outsourced Computation} where a computationally  ``weak'' client hires a more  ``powerful'' server to store data and perform computations on its behalf. This paper is concerned with the problem of designing outsourced computation schemes that incentivize the server to perform correctly the tasks assigned by the client. 

The rise of the {\em cloud computing} paradigm where business do not maintain their own IT infrastructure, but rather hire  ``providers'' to run it, has brought this problem to the forefront of the research community. The goal is to find solutions that are efficient and feasible in practice for problems such as: How do we check the integrity of data that is stored remotely? How do we check computations performed on this remotely stored data? How can a client do this in the most efficient way possible? Or even more generally, how do we incentivize parties to perform correctly in such scenarios?


\subsection{Complexity Theory and Cryptography}

The problem of efficiently checking the correctness of a computation performed by an untrusted party has been central in Complexity Theory for the last 30 years since the introduction of Interactive Proofs by Babai and Goldwasser, Micali and Rackoff \cite{babai,gmr}. 

{\sf Verifiable Outsourced Computation} is now a very active research area in Cryptography and Network Security (see \cite{wb15} for a survey) with the aim to design protocols where it is impossible (under suitable cryptographic assumptions) for a provider to ``cheat" in the above scenarios. While much progress has been done in this area, we are still far from solutions that can be deployed in practice. 

Part of the reason is that Cryptographers consider a very strong adversarial model that prevents {\sf any} adversary from cheating. A different approach is to restrict ourselves to {\em rational adversaries}, whose motivation is not just to disrupt the protocol or computation, but simply to maximize a well defined utility function (e.g. profit).

\subsection{Rational Proofs}

In our work we use the concept of {\sf Rational Proofs} introduced by Azar and Micali in \cite{am} and refined in a subsequent paper \cite{am1}. 

In a Rational Proof, given a function $f$ and an input $x$, the server returns the value $y=f(x)$, and (possibly) some auxiliary information, to the client. The client will in turn 
pay the server for its work with a reward which is a function of the messages 
sent by the server and some randomness chosen by the client.  The crucial 
property is that this reward is maximized in expectation when the server 
returns the correct value $y$. Clearly a rational prover who is only interested 
in maximizing his reward, will always answer correctly. 

The most striking feature of Rational Proofs is their simplicity. For example in \cite{am}, Azar and Micali show {\sf single-message} Rational Proofs for any problem in $\#P$, where an (exponential-time) prover convinces a (poly-time) verifier of the number of satisfying assignment of a Boolean formula. 

For the case of  ``real-life" computations, Azar and Micali in \cite{am1} consider the case of efficient provers (i.e. poly-time) and ``super-efficient" (log-time) verifiers and present $d$-round Rational Proofs for functions computed by (uniform) Boolean circuits of depth $d$, for $d=O(\log n)$. 
% In this case the Verifier runs in logarithmic time.

Recent work \cite{ratsumchecks} shows how to obtain Rational Proofs with sublinear verifiers for languages in $\NC$. Recalling that $\L \subseteq \NL \subseteq \NC_2$, one can use the protocol  in \cite{ratsumchecks} to verify a logspace polytime computation (deterministic or nondeterministic) in $O(\log^2 n )$ rounds and $O(\log^2 n )$ verification.

The work by Chen et al. \cite{chen2016rational} focuses on rational proofs with multiple provers and the related class $\MRIP$ of languages decidable by a polynomial verifier interacting with an arbitrary number of provers. Under standard complexity assumptions, $\MRIP$ includes languages not decidable by a verifier interacting only with one prover. The class $\MRIP$ is equivalent to $\EXP^{||\NP}$.


\subsection{Repeated Executions with a Budget}
%\medskip
%\noindent
%{\sc Compositions of Rational Proofs.}
In \cite{cg15} 
%the authors 
we present a critique of the rational proof model in the case of ``repeated executions with a budget". This model arises in the context of ``volunteer computations" (\cite{seti,folding}) where many computational tasks are outsourced and provers compete in solving as many as possible to obtain rewards. In this scenario assume that a prover has a certain budget $B$ of ``computational effort": how can one  guarantee that the rational strategy is to provide the correct answer in {\em all} the proof he provides? The notion of rational proof guarantees that if the prover engages in a single rational proof then it is in his best interest to provide the correct output. But in \cite{cg15} 
%the authors
we show that in the presence of many computations, it might be more profitable for the prover to use his budget $B$ to provide many incorrect answers than to provide a single correct answer. That's because incorrect (e.g. random) answers are ``cheaper" to compute than the correct one and with the same budget $B$ the prover can provide many of them while the entire budget might be necessary to solve a single problem correctly. If the difference in reward between correct and incorrect answers is not high enough then many incorrect answers may be more profitable and a rational prover will choose that strategy, and indeed this is the case for many of the protocols in \cite{am,am1,ratargs,ratsumchecks}. 

In \cite{cg15} we put forward a stronger notion of {\em sequentially composable rational proofs} which avoids the above problem and guarantees that the rational strategy is always the one to provide correct answers. We also presented sequentially composable rational proofs, but only for some ad-hoc cases, and were not able to generalize them to well-defined complexity classes. 

%for a subset of bounded-depth circuit computations. 

\subsection{Our Contribution}

This paper presents new protocols for the verification of {\em space-bounded polytime computations} against a rational adversary. More specifically consider a language $L \in \DTISP(T(n), S(n))$, i.e. recognized by a deterministic Turing Machine $M_L$ which runs in time $T(n)$ and space $S(n)$. 
We construct a protocol where a rational prover can
convince the verifier that $x \in L$ or $x \notin L$ with the following properties: 
\begin{itemize}
	\item The verifier runs in time $O(S(n) \log n)$
	\item The protocol has $O(\log n)$ rounds and communication complexity $O(S(n) \log n)$
	\item The prover simply runs $M_L(x)$ 
	%and stores all the intermediate configurations (i.e. requires space $O(S(n) T(n))$
\end{itemize}
Our protocol can be proven to correctly incentivize a prover in {\bf both} the stand-alone model of \cite{am} and the sequentially composable definition of \cite{cg15}. This is the first protocol which is sequentially composable for a well-defined complexity class. 

For the case of ``real-life" computations (i.e. poly-time computations verified by a ``super-efficient" verifier) we 
note that for computations in sublinear space our general results yields a protocol in which the verifier is sublinear-time. More specifically, we introduce the first rational proof for $\SC$ (also known as $\DTISP(\poly(n), \polylog(n))$) with polylogarithmic verification and logarithmic rounds. 
%Moreover, our results provide the first efficient rational proof for the 
%non-deterministic class $\NSC = \NTISP(\poly(n), \polylog(n) )$ . 

To compare this with the results in \cite{ratsumchecks}, we note that it is believed that $\NC \not = \SC$ and that the two classes are actually incomparable (see \cite{SCcompleteness} for a discussion). For these computations our results compare
favorably to the one in \cite{ratsumchecks} in at least one aspect: our protocol requires $O(\log n )$ rounds and has the same verification complexity.

We present several extensions of our main result:
\begin{itemize}

	\item Our main protocol can be extended to the case of space-bounded randomized computations using Nisan's 
	pseudo-random generator \cite{nisan1992pseudorandom} to derandomize the computation. 
	\item We also present a different protocol that works for BPNC (bounded error randomized NC) where the Verifier runs in polylog time (note that this class is not covered by our result since we do not know how to express NC with a polylog-space computation). This protocol uses in a crucial way a new {\em composition theorem} for rational proofs which we present in this paper and can be of independent interest. 
	\item We discuss the notion of Rational Arguments (where the adversary is assumed to be both rational and computationally bounded, introduced in \cite{ratargs}) and show that it is achievable for all interesting complexity classes.
	\item Finally we present lower bounds (i.e. conditional impossibility results) for Rational Proofs for various complexity classes.
\end{itemize}

\subsection{The Landscape of Rational Proof Systems}

Rational Proof systems can be divided in roughly two categories, both of them presented in the original work \cite{am}. 

\medskip
\noindent
{\sc Scoring Rules.}
The more ``novel" approach in \cite{am} uses {\em scoring rules} to compute the reward paid by the verifier to the prover. A scoring rule is used to asses the ``quality" of a prediction of a randomized process. Assume that the prover declares that a certain random variable $X$ follows a particular probability distribution $D$. The verifier runs an ``experiment" (i.e. samples the random variable in question) and computes a ``reward" based on the distribution $D$ announced by the prover and the result of the experiment. A scoring rule is maximized if the prover announced the real distribution followed by $X$. The novel aspect of many of the protocols in \cite{am} was how to cast the computation of $y=f(x)$ as the announcement of a certain distribution $D$ that could be tested efficiently by the verifier and rewarded by a scoring rule. 

A simple example is the protocol for $\#P$ in \cite{am} (or its ``scaled-down" version for Hamming weight described more in detail in Section~\ref{sec:example}). Given a Boolean formula $\Phi(x_1,\ldots,x_n)$ the prover announces the number $m$ of satisfying assignments. This can be interpreted as the prover announcing that if one chooses an assignment at random it will be a satisfying one with probability $m \cdot 2^{-n}$. The verifier then chooses a random assignment and checks if it satisfies $\Phi$ or not and uses $m$ and the result of the test to compute the reward via a scoring rule. Since the scoring rule is maximized by the announcement of the correct $m$, a rational prover will announce the correct value. 

As pointed out in \cite{cg15} the problem with the scoring rule approach is that the reward declines slowly as the distribution announced by the Prover becomes more and more distant from the real one. The consequence is that incorrect results still get a substantial reward, even if not a maximal one. Since those incorrect results can be computed faster than the correct one, a Prover with ``budget" $B$ might be incentivized to produce many incorrect answers instead of a single correct one. All of the scoring rule based protocols in \cite{am,am1,ratargs,ratsumchecks} suffer from this problem. 

\medskip
\noindent
{\sc Weak Interactive Proofs.}
The definition of rational proofs requires that the expected reward is maximized for the honest prover. This definition can be made stronger (as done explicitly in \cite{ratargs}) and require a that every systematically dishonest prover would incur a polynomial loss (this property is usually described in terms of a \textit{noticeable reward gap}). As discussed above, the elegant device of scoring rules is the basis for most rational proof protocols in literature, some of which achieve noticeable reward gap. Another simple way in which we can obtain this stronger type of rational proof is the following. Imagine having a test where the prover can be caught cheating with ``low", but non-negligible probability, e.g. $n^{-k}$ for some $k \in \naturals$.
We will informally call this test a \textit{weak interactive proof}\footnote{This is basically the {\em covert adversary} model for multiparty computation introduced in \cite{AL10}.}. Indeed for such proofs we can always pay a fixed reward $R$ to the prover unless we catch him cheating in which case we pay $0$. These are rational proofs since obviously the expected reward of the prover is maximized by the honest behavior. Some of the proofs in \cite{am} and the proofs in \cite{cg15} are weak interactive proofs. Those proofs also turn out to be secure in the sequential model of \cite{cg15} (under appropriate assumptions). 

The protocols in this work are weak interactive proof, which is why we can prove them to be sequentially composable\footnote{
	One exception is the protocol for BPNC, which depends on the underlying protocol for (deterministic) NC. If we use the one in \cite{ratsumchecks}, then the resulting protocol uses scoring rules and is not sequentially composable. However an alterative protocol for NC can be used, based on our work in \cite{cg15}, which is a weak interactive proof and can be proven sequentially composable.}.


\medskip
\noindent
{\sc Scoring rules vs. weak interactive proofs.}
Comparing approaches based on scoring rules and weak interactive proofs the following two questions come up: 
\begin{itemize}
	\item Does one approach systematically lead to more efficient rational proofs (in terms of rounds, communication and verifying complexity) than the other? 
	\item Is one approach more suitable for sequential composability than the other?
\end{itemize}
We do not have a precise answer to the above questions, which we believe are interesting open problems to consider. However we can make the following statements. 

Regarding the first question: in the context of ``stand-alone" (non sequential) rational proofs it is not clear which approach is more powerful. We know that for every language class known to admit a scoring rule based protocol we also have a weak interactive proof with similar performance metrics (i.e. number of rounds, verifier efficiency, etc.). The result in this paper is the first example of a language class for which we have rational proofs based on weak interactive proofs but no example of a scoring rule based protocol exist\footnote{
	We stress that in this comparison we are interested in protocols with similar efficiency parameters. For example, the work in \cite{am} presents several large complexity classes for which we have rational proofs. However, these protocols require a polynomial verifier and do not obtain a noticeable reward gap.}.
This suggests that the weak interactive proof approach might be the more powerful technique. It would be interesting to prove that all rational proofs are indeed weak interactive proofs: i.e. that given a rational proof with certain efficiency parameters, one can construct a weak interactive proof with ``approximately" the same parameters. This question is left as future work.

On the issue of sequential composability, we have already proven in \cite{cg15} that some rational proofs based on scoring rules (such as Brier's scoring rule) are not  sequentially composable. 
This problem might be inherent at least for scoring rules that pay a substantial reward to incorrect computations. What we can say is that all known sequentially composable proofs are based on weak interactive proofs (\cite{cg15}, \cite{am1}\footnote{The construction in Theorem $5.1$ in \cite{am1} is shown to be sequentially composable in \cite{cg15}.} and this work). Again it would be interesting to prove that this is required, i.e. that all sequentially composable rational proofs are weak interactive proofs. 
%An important caveat to keep in mind, however, is that the notion of sequential 
%composability is to some extent flexible and depends
%on the cost model one assumes. % XXX: Maybe say something about the cost model in this work and in the previous one here?

% XXX: Rational arguments here.

\subsection{Other Related  Work}
\label{sec:prior}

{\sc Interactive Proofs.}
Obviously a ``traditional" interactive proof (where security holds against any adversary, even a computationally unbounded one) would work in our model. In this case the most relevant result is 
the recent independent work in \cite{rrr16} that presents breakthrough protocols for the deterministic (and randomized) restriction of the class of language we consider. If $L$ is a language which is recognized by a deterministic (or randomized) Turing Machine $M_L$ which runs in time $T(n)$ and space $S(n)$, then their protocol has the following properties: 
\begin{itemize}
	\item The verifier runs in 
	$O(\poly(S(n)) + n \cdot\polylog(n))$ time;
	\item The prover runs in polynomial time;
	\item The protocol runs in {\em constant} rounds, with communication complexity $O({\sf poly}(S(n)n^{\delta})$ for a constant $\delta$.
\end{itemize}
Apart from round complexity (which is the impressive breakthrough of the result in \cite{rrr16}) our protocols fares better in all other categories. Note in particular that a sublinear space computation does not necessarily yield a sublinear-time verifier in 
\cite{rrr16}. On the other hand, we stress that our protocol only considers weaker rational adversaries. 

\medskip
\noindent{\sc Computational Arguments.}
There is a large class of protocols for {\em arguments} of correctness (e.g. \cite{ggp10,ggpr13,krr14}) even in the rational model \cite{ratargs,ratsumchecks}. Recall that in an argument, security is achieved only against computationally bounded prover. In this case even single round solutions can be achieved. We do not consider this model in this paper, except in Section~\ref{sec:scproof} as one possible option to obtain sequential composability. 

\medskip
\noindent
{\sc Computational Decision Theory.}
Other works in theoretical computer science have studied the connections between cost of computation and utility in decision problems.
The work in \cite{halpern2011don} proposes a framework for \emph{computational decision problems}, where the Decision Maker's (DM) utility depends on the algorithm chosen for computing its strategy.
The Decision Maker runs the algorithm, assumed to be a Turing Machine, on the input to the computational decision problem.
The output of the algorithm determines the DM's strategy. 
Thus the choice of the DM reduces to the choice of a Turing Machine from a certain space. The DM will have beliefs on the running time (cost) of each Turing machine. The actual cost of running the chosen TM will affect the DM's reward.
Rational proofs with costly computation could be formalized in the language of \emph{computational decision problems} in \cite{halpern2011don}. There are similarities between the approach in this
work and that in \cite{halpern2011don}, as both take into account the cost of computation in a decision problem.






	
	%\section{Preliminaries}
	
\section{The Landscape of Rational Proof Systems}

Rational Proof systems can be divided in roughly two categories, both of them presented in the original work \cite{am}. 

\medskip
\noindent
{\sc Scoring Rules.}
The more ``novel" approach in \cite{am} uses {\em scoring rules} to compute the reward paid by the verifier to the prover. A scoring rule is used to asses the ``quality" of a prediction of a randomized process. Assume that the prover declares that a certain random variable $X$ follows a particular probability distribution $D$. The verifier runs an ``experiment" (i.e. samples the random variable in question) and computes a ``reward" based on the distribution $D$ announced by the prover and the result of the experiment. A scoring rule is maximized if the prover announced the real distribution followed by $X$. The novel aspect of many of the protocols in \cite{am} was how to cast the computation of $y=f(x)$ as the announcement of a certain distribution $D$ that could be tested efficiently by the verifier and rewarded by a scoring rule. 

A simple example is the protocol for $\#P$ in \cite{am} (or its ``scaled-down" version for Hamming weight described more in detail in Section~\ref{sec:example}). Given a Boolean formula $\Phi(x_1,\ldots,x_n)$ the prover announces the number $m$ of satisfying assignments. This can be interpreted as the prover announcing that if one chooses an assignment at random it will be a satisfying one with probability $m \cdot 2^{-n}$. The verifier then chooses a random assignment and checks if it satisfies $\Phi$ or not and uses $m$ and the result of the test to compute the reward via a scoring rule. Since the scoring rule is maximized by the announcement of the correct $m$, a rational prover will announce the correct value. 

As pointed out in \cite{cg15} the problem with the scoring rule approach is that the reward declines slowly as the distribution announced by the Prover becomes more and more distant from the real one. The consequence is that incorrect results still get a substantial reward, even if not a maximal one. Since those incorrect results can be computed faster than the correct one, a Prover with ``budget" $B$ might be incentivized to produce many incorrect answers instead of a single correct one. All of the scoring rule based protocols in \cite{am,am1,ratargs,ratsumchecks} suffer from this problem. 

\medskip
\noindent
{\sc Weak Interactive Proofs.}
The definition of rational proofs requires that the expected reward is maximized for the honest prover. This definition can be made stronger (as done explicitly in \cite{ratargs}) and require a that every systematically dishonest prover would incur a polynomial loss (this property is usually described in terms of a \textit{noticeable reward gap}). As discussed above, the elegant device of scoring rules is the basis for most rational proof protocols in literature, some of which achieve noticeable reward gap. Another simple way in which we can obtain this stronger type of rational proof is the following. Imagine having a test where the prover can be caught cheating with ``low", but non-negligible probability, e.g. $n^{-k}$ for some $k \in \naturals$.
We will informally call this test a \textit{weak interactive proof}\footnote{This is basically the {\em covert adversary} model for multiparty computation introduced in \cite{AL10}.}. Indeed for such proofs we can always pay a fixed reward $R$ to the prover unless we catch him cheating in which case we pay $0$. These are rational proofs since obviously the expected reward of the prover is maximized by the honest behavior. Some of the proofs in \cite{am} and the proofs in \cite{cg15} are weak interactive proofs. Those proofs also turn out to be secure in the sequential model of \cite{cg15} (under appropriate assumptions). 

The protocols in this work are weak interactive proof, which is why we can prove them to be sequentially composable\footnote{
	One exception is the protocol for BPNC, which depends on the underlying protocol for (deterministic) NC. If we use the one in \cite{ratsumchecks}, then the resulting protocol uses scoring rules and is not sequentially composable. However an alterative protocol for NC can be used, based on our work in \cite{cg15}, which is a weak interactive proof and can be proven sequentially composable.}.


\medskip
\noindent
{\sc Scoring rules vs. weak interactive proofs.}
Comparing approaches based on scoring rules and weak interactive proofs the following two questions come up: 
\begin{itemize}
	\item Does one approach systematically lead to more efficient rational proofs (in terms of rounds, communication and verifying complexity) than the other? 
	\item Is one approach more suitable for sequential composability than the other?
\end{itemize}
We do not have a precise answer to the above questions, which we believe are interesting open problems to consider. However we can make the following statements. 

Regarding the first question: in the context of ``stand-alone" (non sequential) rational proofs it is not clear which approach is more powerful. We know that for every language class known to admit a scoring rule based protocol we also have a weak interactive proof with similar performance metrics (i.e. number of rounds, verifier efficiency, etc.). The result in this paper is the first example of a language class for which we have rational proofs based on weak interactive proofs but no example of a scoring rule based protocol exist\footnote{
	We stress that in this comparison we are interested in protocols with similar efficiency parameters. For example, the work in \cite{am} presents several large complexity classes for which we have rational proofs. However, these protocols require a polynomial verifier and do not obtain a noticeable reward gap.}.
This suggests that the weak interactive proof approach might be the more powerful technique. It would be interesting to prove that all rational proofs are indeed weak interactive proofs: i.e. that given a rational proof with certain efficiency parameters, one can construct a weak interactive proof with ``approximately" the same parameters. This question is left as future work.

On the issue of sequential composability, we have already proven in \cite{cg15} that some rational proofs based on scoring rules (such as Brier's scoring rule) are not  sequentially composable. 
This problem might be inherent at least for scoring rules that pay a substantial reward to incorrect computations. What we can say is that all known sequentially composable proofs are based on weak interactive proofs (\cite{cg15}, \cite{am1}\footnote{The construction in Theorem $5.1$ in \cite{am1} is shown to be sequentially composable in \cite{cg15}.} and this work). Again it would be interesting to prove that this is required, i.e. that all sequentially composable rational proofs are weak interactive proofs. 
%An important caveat to keep in mind, however, is that the notion of sequential 
%composability is to some extent flexible and depends
%on the cost model one assumes. % XXX: Maybe say something about the cost model in this work and in the previous one here?

% XXX: Rational arguments here.


\section{Definitions and Preliminaries}

The following is the definition of Rational Proof from \cite{am}. As usual with $\negl(\cdot)$ we denote a {\em negligible} function, i.e. one that is asymptotically smaller than the inverse of any polynomial. Conversely a {\em noticeable} function is the inverse of a polynomial. 

% XXX: Do we need 
%In the following we will adopt a "concrete-security" version of the "asymptotic" 
%definitions and theorems in \cite{am1,ratargs}. We assume the reader is familiar with 
%the notion of interactive proofs \cite{gmr}. 

\noindent
\begin{definition}[Rational Proof]
\label{def:RP-delta}
\label{def:RP}
A function $f:$ $\bit^n$ $\to$ $\bit^*$ admits a rational proof if there exists an interactive proof $(P,V)$ and a randomized reward function
$\rew : \bits \to \posreals$ such that

\begin{enumerate}
\item \label{item:completeness} For any input $x \in 
\bit^n$, $\Pr[\out((P,V)(x)) = f(x)] \geq 1 - \negl(n).$

\item For every prover $\disP$, and for any input $x \in 
\bit^n$ there exists a $\delta_{\disP}(x) \geq 0$ such that 
$ \expRewProtDis + \delta_{\disP}(x) \leq \expRewProtHon. $
\end{enumerate}
The expectations and the probabilities are taken over the random coins of the prover and verifier.
\end{definition} 
We note that differently than \cite{am} we allow for non-perfect completeness: a negligible probability that even the correct prover will prove the wrong result. This will be necessary for our protocols for randomized computations. 


\medskip
\noindent
Let $\epsilon_{\disP} = \Pr[\out((P,V)(x)) \neq f(x)]$. 
Following \cite{ratargs} we define the {\sf reward gap} as 
\[ \Delta(x) = min_{P^* : \epsilon_{P^*}=1}[\delta_{P^*}(x)]  \]
i.e. the minimum reward gap over the provers that always report the incorrect value. 
It is easy to see that for arbitrary prover $\disP$ we have $\delta_{\disP}(x) \geq 
\epsilon_{\disP} \cdot \Delta(x)$. Therefore it suffices to prove that a protocol has 
a strictly positive reward gap $\Delta(x)$ for all $x$. 

% TODO: Ensure that $\Delta(x)$ is used consistently in all proofs.


\begin{definition}[\cite{am,am1,ratargs}]
The class $\DRMA[r, c, T]$ (Decisional Rational Merlin Arthur)
is the class of boolean functions $f : \bits \to \bit$ admitting a rational proof $\Pi = (P,V, \rew)$ s.t. on input $x$:
\begin{itemize}
    \item $\Pi$ terminates in $r(|x|)$ rounds;
    \item The communication complexity of $P$ is $c(|x|)$;
    \item The running time of $V$ is $T(|x|)$;
    \item The function $\rew$ is bounded by a polynomial;
    \item $\Pi$ has noticeable reward gap.
\end{itemize}
\end{definition}

\noindent
\begin{remark}
\label{rem:asy}
{\em The requirement that the reward gap must be noticeable was introduced in 
\cite{am1,ratargs} and is explained in Section~\ref{sec:proofs-seq-comp}.}
\end{remark}

%\subsection{A Warmup Example}
\label{sec:example}
\smallskip
\noindent
{\sc Examples of Rational Proofs.} For concreteness here we show the protocol for a single threshold gate (readers are referred to \cite{am,am1,rosen} for more examples). 

Let $G_{n,k}(x_1,\ldots,x_n)$ be a threshold gate with $n$ Boolean inputs, that evaluates to 1 if at least $k$ of the input bits are 1. The protocol in \cite{am1} to evaluate this gate goes as follows. The Prover announces the number $\tilde{m}$ of input bits equal to 1, which allows the Verifier to compute $G_{n,k}(x_1,\ldots,x_n)$. The Verifier select a random index $i \in [1..n]$ and looks at input bit $b=x_i$ and rewards the Prover using Brier's Rule $BSR(\tilde{p},b)$ where $\tilde{p}=\tilde{m}/n$ i.e. the probability claimed by the Prover that a randomly selected input bit be 1. Then
\[BSR(\tilde{p},1) = 2\tilde{p} - \tilde{p}^2 - (1-\tilde{p})^2 + 1 = 2\tilde{p}(2-\tilde{p}) \]
\[BSR(\tilde{p},0) = 2(1-\tilde{p}) - \tilde{p}^2 - (1-\tilde{p})^2 +1 = 2(1-\tilde{p}^2) \]
Let $m$ be the true number of input bits equal to 1, and $p=m/n$ the corresponding probability, then the expected reward of the Prover is
\begin{equation}
\label{eq:bsr}
p BSR(\tilde{p},1) + (1-p) BSR(\tilde{p},0) 
\end{equation}
which is easily seen to be maximized for $p=\tilde{p}$ i.e. when the Prover announces the correct result. Moreover one can see that when the Prover announces a wrong $\tilde{m}$ his reward goes down by $2(p-\tilde{p})^2 \geq 2/n^2$. In other words 
for all $n$-bit input $x$, we have $\Delta(x)=2/n^2$ and if a dishonest Prover $\disP$ cheats with probability $\epsilon_{\disP}$ then $\delta_{\disP} > 2\epsilon_{\disP}/n^2$. 




%\noindent
%Let $\epsilon_{\disP} = \Pr[\out((P,V)(x)) \neq f(x)]$. 
%Following \cite{rosen} we define the {\sf reward gap} as 
%\[ \Delta(x) = min_{P^* : \epsilon_{P^*}=1}[\delta_{P^*}(x)]  \]
%i.e. the minimum reward gap over the provers that always report the incorrect value. 
%It is easy to see that for arbitrary prover $\disP$ we have $\delta_{\disP}(x) \geq 
%\epsilon_{\disP} \cdot \Delta(x)$. Therefore it suffices to prove that a protocol has 
%a strictly positive reward gap $\Delta(x)$ for all $x$. 



%
%Consider the function $f:$ $\bit^n$ $\to$ $[0 \ldots n]$ which on input $x$ outputs the 
%Hamming weight of $x$ (i.e. $\sum_{i} x_i$ where $x_i$ are the bits of $x$). 
%
%In \cite{am1} the prover announces a number $\tilde{m}$ which he claims to be equal to $m=f(x)$. This can be interpreted as the prover announcing that if one chooses an input bit $x_i$ at random it will be equal to 1 with probability $\tilde{p}=\tilde{m}/n$. The verifier then chooses a random input bit $x_i$ and uses $\tilde{m},x_i$ to compute the reward via a scoring rule. Since the scoring rule is maximized by the announcement of the correct $m$, a rational prover will announce the correct value. The scoring rule used in \cite{am1} (and in all other rational proofs based on scoring rules) is 
%Brier's rule where the reward is computed as $BSR(\tilde{p},x_i)$ where: 
%\[BSR(\tilde{p},1) = 2\tilde{p}(2-\tilde{p}) \; \; \mbox{ and } \;\; 
%BSR(\tilde{p},0) = 2(1-\tilde{p}^2) \]
%Notice that $p=m/n$ is the actual probability to get 1 when selecting an input bit at random so the expected reqward of the prover is 
%\begin{equation}
%\label{eq:bsr}
%p BSR(\tilde{p},1) + (1-p) BSR(\tilde{p},0) 
%\end{equation}
%which is easily seen to be maximized for $\tilde{p}=p$, i.e. $\tilde{m}=m$. 
%
%in \cite{cg15} we propose an alternative protocol for $f$ (motivated by the issues we
%discuss in Section~\ref{sec:proofs-seq-comp}). In our protocol we compute $f$ via an "addition circuit", organized as a complete binary tree with $n$ leaves which are the input, and where each internal node is a (fan-in 2) addition gate -- note that this circuit has depth $d=\log n$. The protocol has $d$ rounds: at the first round the prover announces $\tilde{m}$ (the claimed value of $f(x)$) and its two "children" $y_L,y_R$ in the output gate, i.e. the two input values of the last output gate $G$. The Verifier checks that 
%$y_L +y_R=\tilde{m}$, and then asks the Prover to verify that $y_L$ or $y_R$ (chosen a random) is correct, by recursing on the above test. At the end the verifier has to check the last addition gate on two input bits: she performs this test on her own by reading just those two bits. If any of the tests fails, the verifier pays a reward of 0, otherwise she will pay $R$. The intuition is that a cheating prover will be caught with probability $2^{-d}$ which is exactly the reward gap (and for log-depth circuits like this one is noticeable). 
%
%Note that the first protocol is a scoring-rule based one, while the second one is a weak-interactive proof. 
%


	
	%\section{On designing rational proofs as interactive proofs}
	%One of the striking differences between the rational protocols in \cite{} % XXX: Works by Micali, Rosen et al.
and classical interactive proofs is that the verifier carries out no check.
Roughly, the verifier would first exchange messages with the prover, then it would pass the transcript and a sample of a few bits from the input to a "black box function" that decides how much to pay the prover.
This black box satisfies the rationality property because it would yield the largest output, and thus the largest reward, only in the case of the prover's correct answer. The tool used in these protocols are \emph{scoring rules} \cite{brier}.  
A scoring rules take as argument a distribution $\mathcal{D}$ and a sample $\sigma$, allegedly retrieved according to $\mathcal{D}$. The return value of a scoring function is maximized on expectation only if $\mathcal{D}$ corresponds to the distribution from which $\sigma$ was sampled. In a rational proof, $\sigma$ would correspond to a few bits of the input and the distribution $\mathcal{D}$ would depend on the transcript. Intuitively, if a prover is cheating then the distribution $\mathcal{D}$ gets farther from the "honest" distribution and the prover's reward would decrease.

Scoring rules are not the only way to design a rational proof protocol. In fact, the following (trivial) result shows that we can build efficient rational proofs from efficient interactive proofs with poor soundness.

%\medskip
%\noindent

% Implicit theorem in our previous work (and in general, as an obvious fact):
\begin{lemma}
\label{lemma:ip2rp}
Every public coin protocol for function $f: \bits \to \bits$ with completeness 1, soundness at most $1-\frac{1}{\poly(n)}$ and verification time $T_V$  is a rational proof with noticeable reward gap for $f$ where $V$ runs in time $T_V$.
\end{lemma}
\begin{proof}
The verifier can run the interactive protocol and pay the Prover $R = \poly(|x|)$ if it accepts and 0 otherwise. 
\end{proof}

% Something else to be said here?

% TODO: Point out that it is necessary to have rationality as an assumption in order to obtain recent delegation schemes from these "poor" interactive proofs 
This fact suggests an approach to designing rational proofs starting from weak interactive proofs. Variants of this approach have already been used in literature: in \cite{am1} to rationally delegate problems in P (under the assumption of a PCP-like tape); in \cite{chen2016rational} to design a simple multi-prover protocol; in \cite{cg15}, where it is used for delegating functions computable by low-depth circuits. The protocol presented in this paper also follows this approach. These  protocols intuitively differ from the ones based on scoring rules for the presence of explicit checks. These checks have comparatively low probability of detecting a cheating prover, yet high enough to provide noticeable reward gaps to a reward-incentivized prover.

There are two main advantages to designing Rational proofs as traditional interactive proofs following the approach above. First, it guarantees a certain reward $R$ to the honest prover. In fact, if the honest prover is paid by scoring rules its reward will depend on the verifier's randomness. Although maximized in expectation, the reward for the honest prover can be anything between 0 and some maximum value when using scoring rules. Second, it may provide better chances of achieving sequential composability. One of the requirements of sequential composability is that a cheating prover carrying out a "very low-cost computation" will get a very low reward. However, as explained above, scoring rules can provide surprisingly high rewards even to "lazy" provers (see Section 4.1 in \cite{cg15}). It is unclear how this can be avoided with protocols based on scoring rules. On the other hand, some of the protocols in literature based on traditional interactive proofs can be proven sequentially composable in reasonable cost models, specifically the PCP-like protocol in \cite{am1} and the protocols in \cite{cg15} and in the current paper. It is unknown to the authors whether the interactive proof used as rational proof in \cite{chen2016rational} is also sequentially composable. Finally, the analysis of sequential composability for this type of protocols are comparatively easier to carry out as one can simply look at the connections between soundness, i.e. the probability for a dishonest prover of getting a high reward, the and cost to achieve it (see Corollary \ref{cor:prob}).
A disadvantage of the approach based on interactive proofs is that it requires more checks from the Verifier, thus possibly increasing the verification time and number of rounds.

% Open problems: all rational protocols as interactive protocols? Sequential composability only from those?
\subsubsection{Open problems}
It is not clear whether protocols based on scoring rules cannot be  sequentially composable. However, as mentioned above, sequential composability might be easier to prove in protocols based on traditional interactive proofs. For this reason it would be interesting to solve another open problem: whether, for any rational proof with certain complexity for a language, there exists a rational proof based on interactive proofs for the same language and with almost the same complexity (for example, requiring only an additional constant factor in the number of rounds). 

	
	
	% Rational Proofs for NL 
	% Are these better than the ones we have already?? (not round complexity-wise)
	\section{Rational Proofs for Space-Bounded Computations }
	\label{sec:protocol}
\label{sec:rp-dtisp}
We are now ready to present our protocol. It uses the notion of a Turing Machine {\em configuration,} i.e. 
%The main idea it exploits is the concept of \emph{configuration graph} which we 
%will present here only informally\footnote{The interested reader may consult 
%an introductory textbook on computational complexity such as 
%\cite{arora2009computational}.}. A configuration of a Turing Machine is a 
the complete description of the current state of the computation: for a machine $M$, its state, the position of its heads, the non-blank values on its tapes.  

Let $L \in \DTISP(T(n),S(n))$ and $M$ be the deterministic TM that recognizes $L$. 
On input $x$, let $\gamma_1,\ldots,\gamma_N$ (where $N=T(|x|)$) be the 
configurations that $M$ goes through during the computation on input $x$, where 
$\gamma_{i+1}$ is reached from $\gamma_i$ according to the transition function of $M$. Note, first of all, that each configuration has size $O(S(n))$. Also if $x \in L$ (resp. $x \notin L$) then $\gamma_N$ is an accepting (resp. rejecting) configuration. 


The protocol presented below is a more general version of the one used in \cite{cg15} and described above. 
% and very simple in structure. 
The prover shows the claimed final configuration $\hat{\gamma}_N$ 
and then prover and 
verifier engage in a "chasing game", where the prover "commits" at each step to an intermediate configuration. If the prover is cheating (i.e. $\hat{\gamma}_N$ is wrong) then the intermediate configuration either does not follow from the initial configuration or does not lead to the final claimed configuration. At each step and after $P$ communicates the intermediate configuration $\gamma'$, the verifier then randomly chooses whether to continue invoking the protocol on the left or the right of $\gamma'$. The protocol terminates when $V$ ends up on two previously declared adjacent configurations that he can check.  Intuitively, the protocol works since, if $\hat{\gamma}_N$ is wrong, for any possible sequence of the prover's messages, there is at least one choice of random coins that allows $V$ to detect it; the space of such choices is polynomial in size.

We assume that $V$ has oracle access to the input $x$.
%\clearpage
\noindent What follows is a formal description of the protocol.
\begin{framed}
\begin{enumerate}
    \item $P$ sends to $V$:
    \begin{itemize}
    \item $\gamma_{N}$, the final accepting configuration (the starting configuration, $\gamma_1$, is known to the verifier);
    \item $N$, the number of steps between the two configurations. % Does it need to? sort of yes. What if the guy lies. It should be discussed possibly.
    \end{itemize}
    \item Then $V$ invokes the procedure $\PathCheck(N, \gamma_{1}, \gamma_{N})$.
\end{enumerate}
\end{framed}

\medskip
\noindent The procedure $\PathCheck(m,\gamma_l, \gamma_r)$ is defined for $1 \leq m \leq N$ as 
follows:
\begin{framed}
\begin{itemize}
    \item If $m > 1$, then:
    \begin{enumerate}
        \item $P$ sends intermediate configurations $\gamma_{p}$ and $\gamma_q$ (which may coincide) where $p = \lfloor \frac{l+m-1}{2} \rfloor$  and 
        $q = \lceil \frac{l+m-1}{2} \rceil$. % (m == r-l+1)
        \item If $p \neq q$, $V$ checks whether there is a transition leading from configuration $\gamma_p$ to configuration $\gamma_q$. If yes, $V$ accepts; otherwise $V$ halts and rejects.
	\item $V$ generates a random bit $b \in_R \bit$
        \item If  $b = 0$ then the protocol continues invoking $\PathCheck(\lfloor \frac{m}{2} \rfloor, \gamma_l, \gamma_p)$; If $b = 1$ the protocol continues invoking $\PathCheck(\lfloor \frac{m}{2} \rfloor, \gamma_q, \gamma_r)$
    \end{enumerate}
    \item If $m = 1$, then $V$ checks whether there is a transition leading from configuration $\gamma_l$ to configuration $\gamma_r$. If $l=1$, $V$ checks that $\gamma_l$ is indeed the initial configuration $\gamma_1$. If $r=N$, $V$ checks that $\gamma_r$ is indeed the final configuration sent by $P$ at the beginning. If yes, $V$ accepts; otherwise $V$ rejects.
\end{itemize}
\end{framed}

\medskip

\begin{theorem}
\label{thm:main}
$\DTISP[\poly(n), S(n)] \subseteq \DRMA[O(\log n), O(S(n)\log n), O(S(n)\log n)]$
\end{theorem}
\begin{proof}
% Efficiency
Let us consider the efficiency of the protocol above.
It requires $O(\log n)$ rounds.
Since the computation is in $\DTISP[\poly(n), S(n)]$, the configurations $P$ sends to $V$ at each round have size $O(S(n))$.
The verifier only needs to read the configurations and, at the last round, check the existence of a transition leading from $\gamma_l$ to $\gamma_r$. Therefore the total running time for $V$ is $O(S(n) \log n)$.

% Soundness
Let us now prove that this is a rational proof with noticeable reward gap.
%by showing the protocol satisfies the hypothesis of Lemma \ref{lemma:ip2rp}. 
Observe that the protocol has perfect completeness. 
Let us now prove that the soundness is at most $1 - 2^{-\log N} = 1 - \frac{1}{O(\poly(n))}$.
We aim at proving that, if there is no path between the configurations $\gamma_1$ and $\gamma_N$ then $V$ rejects with probability at least $2^{-\log N}$.
Assume, for sake of simplicity, that $N = 2^k$ for some $k$. We will proceed by induction on $k$. If $k=1$, $P$ provides the only intermediate configuration $\gamma'$ between $\gamma_1$ and $\gamma_N$. At this point $V$ flips a coin and the protocol will terminate after testing whether there exists a transition between $\gamma_1$ and $\gamma'$ or between $\gamma'$ and $\gamma_N$. Since we assume the input is not in the language, there exists at most one of such transitions and $V$ will detect this with probability $1/2$.

Now assume $k > 1$. At the first step of the protocol $P$ provides an intermediate configuration $\gamma'$. Either there is no path between $\gamma_1$ and $\gamma'$ or there is no path between $\gamma'$ and $\gamma_N$. Say it is the former: the protocol will proceed on the left with probability $1/2$ and then $V$ will detect $P$ cheating with probability $2^{-k+1}$ by induction hypothesis, which concludes the proof.

\end{proof}

\medskip
\noindent
The theorem above implies the results below. 
%In Corollary \ref{cor:L-NL} we use the fact $\NL = \coNL$.
%To the best of our knowledge the class $\NSC$ is still not known to be closed under 
%complement \cite{barrington1991oracle}.Therefore we are able to obtain only one-
%sided rational proofs for it.
\begin{corollary}
	\label{cor:L-NL}
$ \L \subseteq \DRMA[O(\log n), O(\log^2 n ), O(\log^2 n )]$
\end{corollary}

This improves over the construction of rational proofs for $\L$ in \cite{ratsumchecks} due to the better round complexity. 

\begin{corollary}
	\label{cor:SC}
$ \SC \subseteq \DRMA[O(\log n), O(\polylog(n)), O(\polylog(n))]$
\end{corollary}

No known result was known for $\SC$ before. 


\subsection{Rational Proofs for Randomized Bounded Space  Computation}
\label{sec:rand-space}

We now describe a variation of the above protocol, for the case of randomized bounded space computations. 

Let $\BPTISP[t,s]$ denote the class of languages recognized by randomized machines using time $t$ and space $s$ with error bounded by $1/3$ on both sides. 
In other words, $L \in \BPTISP[\poly(n), S(n)]$ if there exists a (deterministic) Turing Machine $M$ such that
for any $x \in \bits$ $\Pr_{r \in_R \bit^{\rnd(|x|)}}[M(x, r) = L(x)] \geq \frac{2}{3}$ and that runs in $S(|x|)$ space and polynomial time.
Let $\rho(n)$ be the maximum number of random bits used by $M$ for input $x \in \bit^n$; $\rho(\cdot)$ is clearly polynomial.

We can bring down the $2/3$ probability error to $\negl(n)$
by constructing a machine $M'$. $M'$ would simulate the $M$ on $x$ iterating the simulation $m = \poly(|x|)$ times
using fresh random bits at each execution and taking the majority output of $M(x;\cdot)$.
The machine $M'$ uses $m\rho(|x|)$ random bits and runs in polynomial time and $S(|x|) + O(\log(n))$ space.

The work in \cite{nisan1992pseudorandom} introduces pseudo-random generators (PRG) resistant against space bounded adversaries.
An implication of this result is that any randomized Turing Machine $M_1$ running in time $T$ and space $S$ can be simulated by a
 randomized Turing Machine $M_2$ running in time $O(T)$, space $O(S \log(T))$ and using only $O(S \log(T))$ random bits\footnote{We point out that the new machine $M_2$ introduces a small error. For our specific case this error keeps the overall error probability negligible and we can ignore it.}
(see in particular Theorem 3 in \cite{nisan1992pseudorandom}).
Let $L \in  \BPTISP[(\poly(n), S(n)]$ and $M'$ defined as above. We denote by $\hat{M}$ the simulation of $M'$ that uses Nisan's result described above.

By using the properties of the new machine $\hat{M}$, we can directly construct rational proofs for $\BPTISP(\poly(n), S(n))$.
We let the verifier picks a random string $r$ (of length $O(S \log(T))$) and sends it to the prover. They then invoke a rational
proof for the computation $\hat{M}(x;r)$.

By the observations above and Theorem \ref{thm:main} we have the following result:
\begin{corollary}
	$\BPTISP[\poly(n), S(n)] \subseteq \DRMA[\log(n), S(n) \log^2(n), S(n) \log^2(n)]$
\end{corollary}

We note that for this protocol, we need to allow for non-perfect completeness in the definition of $\DRMA$ in order to allow for the probability that the verifier chooses a bad random string $r$. 

%\begin{corollary}
%	\label{cor:space-bounded}
%	$ \NSC \subseteq \osDRMA[O(\log n), O(\polylog(n)), O(\polylog(n))]$
%\end{corollary}


	
	\section{A Composition Theorem for Rational Proofs}
	In this Section we prove a relatively simple {\em composition theorem} that states that while proving the value of a function $f$, we can replace oracle access to a function $g$, with a rational proof for $g$. The technically interesting part of the proof is to make sure that the {\em total} reward of the prover is maximized when the result of the computation of $f$ is correct. In other words, while we know that lying in the computation of $g$ will not be a rational strategy for just that computation, it may turn out to be the best strategy as it might increase the reward of an incorrect computation of $f$. A similar issue (arising in a particular rational proof for depth $d$ circuits) was discussed in \cite{am1}: our proof generalizes their technique. 


\begin{definition}
	\label{def:oracle-RP}
	We say that a rational proof $(P,V, \rew)$ for $f$ is a $g$-oracle rational proof if $V$ has oracle access to the function $g$ and carries out at most one oracle query.
	We allow the function $g$ to depend on the specific input $x$.
\end{definition}

% TODO: Put example here for definition above. (Maybe the PCP-like RP by Azar/Micali)


\newcommand{\rewf}{\rew^o_f}
\newcommand{\rewg}{\rew_g}
% TODO: Remove all the o-s in the notation
\begin{theorem}
	\label{thm:composition}
	Assume there exists a $g$-oracle rational proof $(P^o_f, V^o_f, \rewf)$ for $f$ with noticeable reward gap  and with round, communication and verification complexity respectively $r_f, c_f$ and $T_f$. Let $t_I$ the time necessary to invoke the oracle for $g$ and to read its output.Assume there exists a rational proof $(P_g, V_g, \rewg)$ with noticeable reward gap for $g$ with round, communication and verification complexity respectively $r_g, c_g$ and $T_g$. Then there exists a (non $g$-oracle) rational proof with noticeable reward gap for $f$ with round, communication and verification complexity respectively $r_f + 1 + r_g , c_f + t_I + c_g$ and $T_f - t_i + T_g$.
\end{theorem}

Before we embark on the proof of Theorem~\ref{thm:composition}, we prove a 
technical Lemma. 
The definition of rational proof requires that the expected reward of the honest prover is not lower than the expected reward of any other prover.
The following intuitive lemma states we necessarily obtain this property if an honest prover has a polynomial expected gain in comparison to provers
that \textit{always} provide a wrong output.

\begin{lemma}
	\label{lemma:noticeable-gap-implies-rp}
	Let $(P,V)$ be a protocol and $\rew$ a reward function as in Definition \ref{def:RP}.
	Let $f$ be a function s.t. $\forall x \Pr[\out(P,V)(x)] = 1$.
	Let $\rewGap$ be the corresponding reward gap w.r.t. the honest prover $P$ and $f$.
	If $\rewGap > \invPoly$ then $(P,V,\rew)$ is a rational proof for $f$ and admits noticeable reward gap.
\end{lemma}
\begin{proof}
	Assume w.l.o.g that for all $P' \not = P$ and such that $\forall x \Pr[\out(P',V)(x)] = 1$ it holds that
	$\expectation[\rew(P,V)(x)] \geq \expectation[\rew(P',V)(x)]$.
	
	Fix $x$. Let $\disP$ be an arbitrary prover, $R = \expectation[\rew(P,V)(x)] $, $\claimedy = \out(\disP, V)(x)$, $\disR = \expectation[\rew(\disP,V)(x)]$, $\disR_{corr} = \expectation[\rew(\disP,V)(x) | \claimedy = f(x)]$, $\disR_{err} = \expectation[\rew(\disP,V)(x) | \claimedy \not = f(x)]$. Then:
	
	\begin{align}
	& R - \disR & =  \\
	& R - \Pr[\claimedy = f(x)]\disR_{corr} - \Pr[\claimedy \not = f(x)] & = \\
	& \Pr[\claimedy = f(x)](R-\disR_{corr}) + \Pr[\claimedy \not = f(x)](R-\disR_{err}) & \geq \\
	& \Pr[\claimedy \not = f(x)](R-\disR_{err}) & \geq \\
	& \Pr[\claimedy \not = f(x)]\rewGap & > \\
	& 0 
	\end{align}
	
	The inequality above shows that $(P,V, \rew)$ is a rational proof for $f$.
	By the hypothesis on $\rewGap$ this protocol already admits a noticeable reward gap.
\end{proof}

% TODO: Clean up the proof above. Specify it has polynomial budget. Make it homogeneous with the definitions given in preliminaries. Q: Why does the def. of DRMA not mention poly budget?
Now we can start the proof of Theorem~\ref{thm:composition}.
\begin{proof}
% Introduce reward functions

Let $\rewf$ and $\rewg$ be the reward functions of the $g$-oracle rational proof for $f$ and the rational proof for $g$ respectively.
% construct new reward functions and protocol
We now construct a new verifier $V$ for $f$. This verifier runs exactly like the $g$-oracle verifier for $f$ except that every oracle query to $g$ is now replaced with an invocation of the rational proof for $g$.
The new reward function $\rew$ is defined as follows:
$$ \rew(\Tau) = \delta \rewf(\Tau^o_f \circ y_g)  + \rewg(\Tau_g)$$
where $\Tau$ is the complete transcript of the new rational proof, $\Tau^o_f$ is the transcript of the oracle rational proof for $f$,  $\Tau_g$ and $y_g$ are respectively the transcript and  the output of the rational proof for $g$. Finally $\delta$ is multiplicative factor in $(0,1])$. The intuition behind this formula is to "discount" the part of the reward from $f$ so that the prover is incentivized to provide the true answer for $g$. In turn, since $\rewf$ rewards the honest prover more when  the verifier has the right answer for a query to $g$ (by hypothesis), this entails that the whole protocol is rational proof for $f$.

% Say that, by the previous lemma, you just need to show that the resulting protocol has 
To prove the theorem we will use Lemma \ref{lemma:noticeable-gap-implies-rp} and it will suffice to prove that the new protocol has a noticeable reward gap.
\newcommand{\pg}{p_{g}}
\newcommand{\Rf}{R_f^o}
\newcommand{\disRf}{\tilde{R}_f^o}
\newcommand{\disRfGoodg}{\tilde{R}_f^{o,\text{good}(g)}}
\newcommand{\disRfWrongg}{\tilde{R}_f^{o,\text{wrong}(g)}}
\newcommand{\disRgGoodg}{\tilde{R}_g^{\text{good}(g)}}
\newcommand{\disRgWrongg}{\tilde{R}_g^{\text{wrong}(g)}}
\newcommand{\Rfmax}{b^o_f(n)}


Consider a prover $\disP$ that always answer incorrectly on the output of $f$. 
Let $\pg$ be the probability that the prover outputs a correct $y_g$. % TODO: Introduce variables
Then the difference between the expected reward of the honest prover and $\disP$ is:
\setcounter{equation}{0}
\begin{align}
 \delta(\Rf - \disRf) + (R_g - \disR_g)  = \\
 \delta(\Rf - \pg \disRfGoodg - (1-\pg)\disRfWrongg) + \nonumber  \\ (R_g - \pg \disRgGoodg - (1-\pg)\disRgWrongg)  = \\
 \delta(\pg(\Rf - \disRfGoodg) + (1-\pg)(R_f - \disRfWrongg)) \nonumber \\ + \pg(R_g - \disRgGoodg) + (1-\pg)(R_g-\disRgWrongg)  > \\
 \delta(\pg \rewGap^o_f + (1-\pg)(-\Rfmax)) + 0 + (1-\pg)\rewGap_g  = \\
 \pg\delta\rewGap^o_f + (1-\pg)(\rewGap_g - \delta \Rfmax)  \geq \\
 \min\{\delta \rewGap^o_f, \rewGap_g - \delta\Rfmax \}  > \\
\invPoly
\end{align}
Where the last inequality holds for $\delta = \frac{\rewGap_g}{2\Rfmax}$.

The round, communication and verification complexity of the construction is given by the sum of the respective complexities from the two rational proofs modulo minor adjustments. These adjustments account for the additional round by which the verifier communicates to the prover the requested instance for $g$. 

\begin{comment}
One final note: in the proof above we replaced the oracle query to $g$ with an invocation of the rational proof for it. In certain circumstances (see for example the proof of Theorem \ref{thm:crhf-p}), it would simplify a proof to assume that some of the messages of the rational proof for $g$ are sent before the protocol for $f$ is invoked.
The analysis above still holds if extending the rational proof for $f$ with such "preprocessing" messages still yields a $g$-oracle rational proof for $f$.
\end{comment}
\end{proof}

The theorem above can be used as design tool of rational proofs for a function $f$: first build a rational proof assuming the verifier has oracle access to a function $g$, then build a rational proof for $g$. This automatically provides a complete rational proof for $f$.

\begin{remark}
	Theorem \ref{thm:composition} assumes that verifier in the oracle rational proof for $f$ carries out a single oracle query. Notice however that the proof of the theorem can be generalized to any verifier carrying out a constant number of adaptive oracle queries, possibly all for distinct functions.
	This can be done by iteratively applying the theorem to a sequence of $m = O(1)$ oracle rational proofs for functions $f_1,...,f_m$ where the $i$-th rational proof is $f_{i+1}$-oracle for $1 \leq i < m$.
\end{remark}
% TODO: Is it possible to rewrite this in a  better way?
		
	%\section{Rational Proofs for Randomized Log-depth Circuits}
	


\subsection{Rational Proofs for Randomized Circuits}
\label{sec:bpnc}

As an application of the composition theorem described above we present an alternative approach to rational proofs for randomized computations. We show that by assuming the existence
of a {\em common reference string (CRS)} we obtain rational proofs for randomized circuits of polylogarithmic depth and polynomial size, i.e. $\BPNC$ the class of uniform $\polylog$-depth $\poly$-size randomized circuits with error bounded by $1/3$ on both sides.

If we insist on a "super-efficient" verifier (i.e. with sublinear running time) we cannot
use the same approach as in Section~\ref{sec:rand-space} since we do not know
how to bound the space $S(n)$ used by a computation in NC (and the verifier's 
complexity in our protocol for bounded space computations, depends on the 
space complexity of the underlying language).  We get around this problem by assuming a CRS, to which the verifier has oracle access. 


We start by describing a rational proof with oracle access for $\BPP$ and then we show how to remoe the oracle access (via our composition theorem) for the case of 
$\BPNC$. 

Let $L \in \BPP$ and let $M$ a PTM that decides $L$ in polynomial time and $\rnd(\cdot)$ the
randomness complexity of $M$.
For $x \in \bits$ we denote by $L_x$ the (deterministically
decidable) language $\{(x, r) : r \in \bit^{ \rnd(|x|)} \wedge M (x, r) = L(x)\}$.

\begin{lemma}
	\label{lemma:crs-rp-oracle}
	Let $L$ be a language in $\BPP$. Then there exists a $L_x$-oracle rational proof with CRS $\sigma$ for $L$ where $|\sigma|=\poly(n) \rho(n)$. 
%Such rational proof is one-round and has $O(\log n)$ communication and 
%verification complexity. The CRS has length $n\rho(n)$ and the verifier 
%has random access to it.
\end{lemma}
\begin{proof}
Our construction is as follows. W.l.o.g. we will assume $\sigma$ to be divided in 
$\ell=\poly(n)$  blocks  $r_1,...,r_\ell$, each of size $\rho(n)$. 
\begin{enumerate}
\item The honest prover $P$ runs $M(x,r_i)$ for $1 \leq i \leq \ell$ and announces $m$ 
the number of strings $r_i$ s.t. $M(x,r_i)$ accepts, i.e. $\sum_i M(x,r_i)$; 
\item $P$ sends $m$ to $x$.
\item The Verifier accepts if $m>\ell/2$
\end{enumerate}
We note that if we set $y_i=M(x,r_i)$ then the prover is announcing the Hamming weight of the string $y_1,\ldots,y_\ell$. At this point we can use the Hamming weight verification protocol in Section~\ref{sec:example} where the Verifier use the oracle for 
$L_x$ to verify on her own the value of $y_i$. 
\end{proof}
We note that no matter which protocol is used, round complexity, communication complexity and verifier running time (not counting the oracle calls) are all $\polylog(n)$. 

To obtain our result for $\BPNC$ we invoke the following result from \cite{ratsumchecks}:
\begin{theorem}
	$\NC \subseteq \DRMA[\polylog(n), \polylog(n), \polylog(n)]$
\end{theorem}
The theorem above, together with Theorem \ref{thm:composition} 
%(see also Remark \ref{rem:extensions-crs}) 
and Lemma \ref{lemma:crs-rp-oracle} yields:
\begin{corollary}
	Let $x \in \bit^n$ and $L \in \BPNC$. Assuming the existence of a (polynomially long) CRS then there exists a rational proof for $L$
	with polylogarithmically many rounds, polylogarithmic communication and verification complexity.
\end{corollary}
Notice that some problems (e.g. perfect matching) are not known to be in $\NC$ but are known to be in $\RNC \subseteq \BPNC$ \cite{karp1985constructing}.
			
	
	
\section{Lower Bounds for Rational Proofs}

In this section we discuss how likely it is will be able to find very efficient non-cryptographic rational protocols for the classes $\P$ and $\NP$.

We denote by $\BPQP$ the class of languages decidable by a randomized algorithm running in quasi-polynomial time, i.e. $\BPQP = \bigcup_{k>0}\BPTIME[2^{O(log^k(n))} ]$.
Our theorem follows the same approach of Theorem 16 in \cite{ratargs}\footnote{Since we only sketch our proof the reader is invited to see details of the proof \cite{ratargs}}. 

\begin{theorem}
	\label{thm:np-limits}
	$\NP \not \subseteq \DRMA[\polylog(n), \polylog(n), \poly(n)]$  unless $\NP \subseteq \BPQP$. 
\end{theorem}
\begin{proofsketch}
Assume there exists a rational proof $\pi_L$ for a language $L \in \NP$ with parameters as the ones above.
We can build a PTM $M$ to decide $L$ as follows:
\begin{itemize}
	\item $M$ generates all possible transcripts $\Tau$ for $\pi_L$;
	\item For each $\Tau$, $M$ estimates the expected reward $R_{\Tau}$ associated to that transcript by sampling $\rew(\Tau)$ $t$ times (recall the reward function is probabilistic);
	\item $M$ returns the output associated to transcript $\Tau^* = \argmax_\Tau R_{\Tau}$.
\end{itemize}
Consider that space of the transcripts of rational proof with a polylogarithmic number of rounds and bits exchanged.
The number of possible transcripts in such protocol is bounded by $(2^{\polylog(n)})^{\polylog(n)} = 2^{\polylog(n)}$. 
Let $\rewGap$ be the (noticeable) reward gap of the protocol. By using Hoeffding's inequality we can prove $M$ can approximate each $R_\Tau$ within
$\rewGap/3$ with probability $2/3$ after $t = \poly(n)$ samples. Recalling the definition of reward gap (see Remark \ref{rem:asy}), we conclude $M$ can decide $L$ in randomized time $2^{\polylog(n)}$. 

\end{proofsketch}

It is not known whether $\NP \not \subseteq \BPQP$ is true, although this assumption has  been used to show hardness of approximation results \cite{makarychev2010maximum,khot2006better}.
Notice that this assumption  implies $\NP \not \subseteq \BPP$ \cite{johnson2006np}. 

Let us now consider  rational proofs for $\P$. By the following theorem they might require $\omega(\log(n))$ total communication complexity (since we believe $\P \subseteq \BPNC$ to be unlikely \cite{papakonstantinou2010constructions} ).
\begin{theorem}
	\label{thm:p-limits}
	$\P \not \subseteq \DRMA[O(1), O(\log(n)), \polylog(n)]$  unless $\P \subseteq \BPNC$. 
\end{theorem}
\begin{proofsketch}
	Given a language $L \in \P$ we build a machine $M$ to decide $L$ as in the proof of Theorem \ref{thm:np-limits}.
	The only difference is that  $M$ can be simulated by a randomized circuit of $\polylog(n)$ depth and polynomial size.
	In fact, all the possible $2^{O(\log(n))} = \poly(n)$ transcripts can be simulated in parallel in $O(\log(n))$ sequential time. The same holds computing the $t = \poly(n)$ sample rewards for each of these transcripts. By assumption on the verifier's running time, each reward can be computed in polylogarithmic sequential time. Finally, the estimate of each transcript's expected reward and the maximum among them can be computed in $O(\log(n))$ depth.
\end{proofsketch}

\begin{remark}
	Theorem \ref{thm:p-limits} can be generalized to rational proofs with round and communication complexities $r$ and $c$ such that $r\cdot c = O(\log(n))$.
\end{remark}


	
	%\section{On Rational Arguments and cryptography}
\label{sec:ratarg}

In this section we general results using standard cryptographic assumptions such as OWF and CRHF for rational arguments for $\NP$.


\begin{definition}[Rational Argument (\cite{ratargs})]
	A function $f : \bits  \to \bits$ admits
a rational
	argument with security parameter $\kappa : \naturals \to \naturals$  if there exists a protocol $(P,V)$ and a
	randomized reward function $\rew : \bits \to \reals_{\geq 0}$
	such that for any input $x \in \bits$
	
	and any prover $P^*$ of size $\leq \poly(2^{\kappa(|x|)})$ the following hold:
	\begin{enumerate}
		\item $\Pr[\out((P,V)(x) = f(x)] = 1$.
		\item There exists a negligible function $\epsilon(\cdot)$  such that $\expectation[\rew((P,V)(x))] + \epsilon(|x|) \geq \expectation[\rew((P^*, V)(x))]$.
		\item If there exists a polynomial $p(\cdot)$ such that
		$\Pr[\out((P^*, V)(x))\not = f(x)] \geq p(|x|)^{-1}$ then there exists a polynomial $q(\cdot)$ such that
		$\expectation[\rew((P^*, V)(x)))] + q(|x|)^{-1} \leq \expectation[\rew((P, V)(x)))] $.
		The expectations and the probabilities are taken over the random coins of the respective prover and verifier.
	\end{enumerate}
\end{definition}

\begin{remark}[One-sided and Oracle Rational Arguments and Their Composition]
	\label{rem:oracle-rat-args}
	\label{rem:os-rat-args}
	We define the notion of oracle rational arguments extending Definition \ref{def:oracle-RP} to rational arguments in the natural way.
	It is immediate to see that the proof of Theorem \ref{thm:composition} extends to the composition of two rational protocols where one or both are arguments
	(the resulting composition is an argument in both cases).
	Finally, the notion of one-sided rational proofs (Definition \ref{def:os-rp}) can also be extended to rational arguments in the natural way.
\end{remark}

\textbf{TODO: Should there be an intro on CRHFs? Is the statement of the theorem below correct?} 
\newcommand{\hf}{\cal H}
\begin{lemma}
	\label{lemma:crhf-merkle-ratargs}
	Let $f: \bit^n \to \bit$ be a function that admits a $g$-oracle rational proof $\pi^g$, with $g: \bit^{O(\log(n))} \to \bit$. Let $r,C,T$ be respectively the round, communication and verification complexity of $\pi^g$.
	Assume the existence of a CRHF $\hf$ with output length $\kappa$, then $f$ admits a (non oracle) rational argument where:
	\begin{itemize}
		\item The verifier runs in time $T + O(\kappa \log(n))$;
		\item The communication complexity is $C + O(\kappa \log(n))$ and the number of rounds is $r+2$. % HF + Merkle root + queries + replies
	\end{itemize}
\end{lemma}
\begin{proofsketch}
	At the high-level we follow Kilian's construction for interactive arguments for $\NP$ \cite{kilian1992note}.
	We assume $P$ and $V$ agree on a representation of function $g$ as a (polynomially long) sequence of bits $s_g$.
	The interactive argument for $g$:
	\begin{enumerate}
	\item The verifier sends the prover a CRHF $\hf$;
	\item The prover sends the root of a Merkle Tree representing $s_g$;
	\item The verifier asks for value $i$;
	\item The prover sends the verifier $g(i)$ and the openings on the path from the root to the leaf $g(i)$.
	\item If the openings are correct, the verifier pays the prover a reward $R = \poly(n)$; otherwise it pays $0$.
	\end{enumerate}
	
	Notice that the one above is an argument in the classical sense for $g$ with soundness $\leq \negl(\kappa)$. If $\kappa = \omega(n)$ then the protocol
	above admits noticeable reward gap and is a rational argument (see \cite{kilian1992note} for details). To conclude the proof we use Theorem \ref{thm:composition} (see also Remark \ref{rem:oracle-rat-args}). Before applying the theorem we need to pay particular attention to a point. We need to make sure that the prover's "commitment" to the string $s_g$ does not depend adaptively on any of the queries by $V$ in the rest of the protocol. Thus we apply the theorem to 
\end{proofsketch}

\newcommand{\conn}{\function{conn}_n}
\begin{theorem}
	\label{thm:crhf-p}
	If CRHF exist then $\P$ admits a rational argument with security parameter $\kappa$, $O(1)$ rounds and polylogarithmic (in $n$ and $\kappa$) communication and verification time.
\end{theorem}
\begin{proofsketch}
	Our approach is inspired by the construction in \cite{am1} for rational proofs for $\P^{||\NP}$ (Theorem $5.1$). Let $L$ be a language in $\P$, then there exists a polynomially sized circuit $C_n$. Also, there exists a log-space computable language
	$\conn = \{ (\alpha, \beta, \gamma, \tau) : \alpha, \beta $ are inputs  $\tau$ gate whose output $\gamma$ in  $C_n \}$ where
	$\tau \in \{ \function{and}, \function{or}, \function{input} \}$. If $\gamma$ is an input wire then $\alpha, \beta$ are $\bot$.
	%Since, for a given wire $\gamma$, the verifier cannot know the values of the "parent" wires $\alpha$ and $\beta$, in order to be able to use Lemma %\ref{lemma:crhf-merkle-ratargs} we will also assume that the verifier has access to the function $\function{gate}(\cdot)$ that given in input $\gamma$ returns a pair
	%of wires that are input to $\gamma$ and the type of the gate of which $\gamma$ is an output in $C_n$.
	Fixed $x \in \bit^n$, we can represent the execution of $M$ on the string $x$ as a string $v$ where $v_{i}$ is the value of wire $i$ when the circuit is evaluated on input $x$. If $i$ is an input wire then $v_i = x_i$.
	
	We will now construct an oracle rational proof where we assume $V$ has oracle access to $v(\cdot)$ and $\conn$. Notice that both languages have polynomial size in $n$.
	By Lemma \ref{lemma:crhf-merkle-ratargs} and we will obtain a rational argument for $\NP$.
	
	The protocol proceeds as follows:
	\begin{itemize}
		\item The prover first sends the verifier a bit $b$ denoting whether $x \in L$;
		\item The verifier chooses a random wire $\gamma$ and sends it to the prover.
		\item The prover sends the verifier a quadruple $(y, \alpha, \beta, \tau)$, respectively the alleged value of wire $\gamma$, its parents and the gate type $\tau$.
		\item The verifier then proceeds in checking the results from $P$ by invokeing the oracles for $v$  and $\conn$. Then:
		\begin{itemize}
			\item If $\gamma$ is an input wire, $V$ checks whether $y = x_i$;
			\item Otherwise, $\gamma$ invokes the oracle for $v$ on inputs $\alpha$ and $\beta$ and checks that $\tau(v_\alpha, v_\beta) = v_\gamma$.
		\end{itemize}
		\item If all the tests pass the verifier pays a reward $R = \poly(n)$ otherwise it pays nothing.
	\end{itemize}
	
	
	In the protocol described above the verifier has a noticeable probability of finding out a cheating prover and thus it is a (oracle) rational proof with noticeable reward gap for $L$. Applying Lemma \ref{lemma:crhf-merkle-ratargs} to replace oracle queries to $v$ and $\conn$ concludes the proof.
	
\end{proofsketch}

\begin{theorem}
	\label{thm:crhf-np}
	If CRHF exist then $\NP$ admits a one-sided rational argument with security parameter $\kappa$, $O(\log(n))$ rounds and polylogarithmic (in $n$ and $\kappa$) communication and verification time.
\end{theorem}
\begin{proofsketch}
	The proof of this theorem is very close to that of Theorem \ref{thm:crhf-p}.
	
	We first build a one-sided oracle rational proof for 3SAT.
	Let $\phi$ be a 3SAT formula with $n$ variables. We assume the verifier has access to $v$, where $v$ is a satisfying assignment for $\phi$ or an arbitrary assignment if no satisfying assignment exists. The prover sends the verifier one bit $b$. The verifier randomly selects a clause and checks that the assignment satisfies the clause.
	If $b=1$ and the randomly selected clause is satisfied then it pays the prover $R = \poly(n)$, otherwise it pays 0.
	
	To see this is a one-sided rational proof, assume $\phi$ were not satisfiable. Then for any assignment $v$ there would be a non-satisfying clause.
	The verifier can guess such a clause with noticeable probability. This rational proof can be transformed in a rational argument for 3SAT using Lemma $\ref{lemma:crhf-merkle-ratargs}$.
	
	To obtain a rational argument for any $L \in \NP$, consider $f_L$, the Karp reduction from $L$ to 3SAT. 
	Fixed $x \in \bit^n$, let $\phi_i$ be the $i$-th bit of $f_L(x)$. Since $f_L$ can be computed in log-space there exists a rational
	proof for each $\phi_i$ that runs in $O(\log(n))$ rounds and with polylogarithmic communication and verification complexity
	(see Corollary \ref{cor:L-NL}). By using Theorem \ref{thm:composition} (see also Remark \ref{rem:oracle-rat-args}) to compose the rational argument above with
	a rational proof for $\phi_i$ we obtain a (one-sided) rational argument for any language in $\NP$. 
	% Refer to \ref{rem:os-rat-args}
\end{proofsketch}

\begin{comment}
\begin{remark}[Reducing round complexity of our arguments]
	Theorem 5 in $\cite{ratargs}$ allows to transform a rational proof with $r$ rounds in a 1-round rational arguments assuming Private Information Retrieval (PIR) schemes. Unfortunately such theorem is not applicable to the arguments above because they are not history-ignorant
\end{remark}
\end{comment}
% NOTE: Important: write comparison with Kilian, i.e. ours is a _sublinear_ verification time delegation scheme for NP.


	
	
	
	
	
	
	
%
%\subsubsection*{Acknowledgments}
%Authors would like to thank Jesper Buus Nielsen for suggesting the approach of the construction in Theorem \ref{thm:main}.
%

% Bibliography
%\bibliographystyle{plain}
%\bibliography{references}
%
%\end{document}
