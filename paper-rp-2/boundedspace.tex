\label{sec:protocol}
We are now ready to present our protocol. It uses the notion of a Turing Machine {\em configuration,} i.e. 
%The main idea it exploits is the concept of \emph{configuration graph} which we 
%will present here only informally\footnote{The interested reader may consult 
%an introductory textbook on computational complexity such as 
%\cite{arora2009computational}.}. A configuration of a Turing Machine is a 
the complete description of the current state of the computation: for a machine $M$, its state, the position of its heads, the non-blank values on its tapes.  

Let $L \in \DTISP(T(n),S(n))$ and $M$ be the deterministic TM that recognizes $L$. 
On input $x$, let $\gamma_1,\ldots,\gamma_N$ (where $N=T(|x|)$) be the 
configurations that $M$ goes through during the computation on input $x$, where 
$\gamma_{i+1}$ is reached from $\gamma_i$ according to the transition function of $M$. Note, first of all, that each configuration has size $O(S(n))$. Also if $x \in L$ (resp. $x \notin L$) then $\gamma_N$ is an accepting (resp. rejecting) configuration. 


The protocol presented below is a more general version of the one used in \cite{cg15} and described above. 
% and very simple in structure. 
The prover shows the claimed final configuration $\hat{\gamma}_N$ 
and then prover and 
verifier engage in a "chasing game", where the prover "commits" at each step to an intermediate configuration. If the prover is cheating (i.e. $\hat{\gamma}_N$ is wrong) then the intermediate configuration either does not follow from the initial configuration or does not lead to the final claimed configuration. At each step and after $P$ communicates the intermediate configuration $\gamma'$, the verifier then randomly chooses whether to continue invoking the protocol on the left or the right of $\gamma'$. The protocol terminates when $V$ ends up on two previously declared adjacent configurations that he can check.  Intuitively, the protocol works since, if $\hat{\gamma}_N$ is wrong, for any possible sequence of the prover's messages, there is at least one choice of random coins that allows $V$ to detect it; the space of such choices is polynomial in size.

We assume that $V$ has oracle access to the input $x$.
%\clearpage
\noindent What follows is a formal description of the protocol.
\begin{framed}
\begin{enumerate}
    \item $P$ sends to $V$:
    \begin{itemize}
    \item $\gamma_{N}$, the final accepting configuration (the starting configuration, $\gamma_1$, is known to the verifier);
    \item $N$, the number of steps between the two configurations. % Does it need to? sort of yes. What if the guy lies. It should be discussed possibly.
    \end{itemize}
    \item Then $V$ invokes the procedure $\PathCheck(N, \gamma_{1}, \gamma_{N})$.
\end{enumerate}
\end{framed}

\medskip
\noindent The procedure $\PathCheck(m,\gamma_l, \gamma_r)$ is defined for $1 \leq m \leq N$ as 
follows:
\begin{framed}
\begin{itemize}
    \item If $m > 1$, then:
    \begin{enumerate}
        \item $P$ sends intermediate configurations $\gamma_{p}$ and $\gamma_q$ (which may coincide) where $p = \lfloor \frac{l+m-1}{2} \rfloor$  and 
        $q = \lceil \frac{l+m-1}{2} \rceil$. % (m == r-l+1)
        \item If $p \neq q$, $V$ checks whether there is a transition leading from configuration $\gamma_p$ to configuration $\gamma_q$. If yes, $V$ accepts; otherwise $V$ halts and rejects.
	\item $V$ generates a random bit $b \in_R \bit$
        \item If  $b = 0$ then the protocol continues invoking $\PathCheck(\lfloor \frac{m}{2} \rfloor, \gamma_l, \gamma_p)$; If $b = 1$ the protocol continues invoking $\PathCheck(\lfloor \frac{m}{2} \rfloor, \gamma_q, \gamma_r)$
    \end{enumerate}
    \item If $m = 1$, then $V$ checks whether there is a transition leading from configuration $\gamma_l$ to configuration $\gamma_r$. If $l=1$, $V$ checks that $\gamma_l$ is indeed the initial configuration $\gamma_1$. If $r=N$, $V$ checks that $\gamma_r$ is indeed the final configuration sent by $P$ at the beginning. If yes, $V$ accepts; otherwise $V$ rejects.
\end{itemize}
\end{framed}

\medskip

\begin{theorem}
\label{thm:main}
$\DTISP[\poly(n), S(n)] \subseteq \DRMA[O(\log n), O(S(n)\log n), O(S(n)\log n)]$
\end{theorem}
\begin{proof}
% Efficiency
Let us consider the efficiency of the protocol above.
It requires $O(\log n)$ rounds.
Since the computation is in $\DTISP[\poly(n), S(n)]$, the configurations $P$ sends to $V$ at each round have size $O(S(n))$.
The verifier only needs to read the configurations and, at the last round, check the existence of a transition leading from $\gamma_l$ to $\gamma_r$. Therefore the total running time for $V$ is $O(S(n) \log n)$.

% Soundness
Let us now prove that this is a rational proof with noticeable reward gap.
%by showing the protocol satisfies the hypothesis of Lemma \ref{lemma:ip2rp}. 
Observe that the protocol has perfect completeness. 
Let us now prove that the soundness is at most $1 - 2^{-\log N} = 1 - \frac{1}{O(\poly(n))}$.
We aim at proving that, if there is no path between the configurations $\gamma_1$ and $\gamma_N$ then $V$ rejects with probability at least $2^{-\log N}$.
Assume, for sake of simplicity, that $N = 2^k$ for some $k$. We will proceed by induction on $k$. If $k=1$, $P$ provides the only intermediate configuration $\gamma'$ between $\gamma_1$ and $\gamma_N$. At this point $V$ flips a coin and the protocol will terminate after testing whether there exists a transition between $\gamma_1$ and $\gamma'$ or between $\gamma'$ and $\gamma_N$. Since we assume the input is not in the language, there exists at most one of such transitions and $V$ will detect this with probability $1/2$.

Now assume $k > 1$. At the first step of the protocol $P$ provides an intermediate configuration $\gamma'$. Either there is no path between $\gamma_1$ and $\gamma'$ or there is no path between $\gamma'$ and $\gamma_N$. Say it is the former: the protocol will proceed on the left with probability $1/2$ and then $V$ will detect $P$ cheating with probability $2^{-k+1}$ by induction hypothesis, which concludes the proof.

\end{proof}

\medskip
\noindent
The theorem above implies the results below. 
%In Corollary \ref{cor:L-NL} we use the fact $\NL = \coNL$.
%To the best of our knowledge the class $\NSC$ is still not known to be closed under 
%complement \cite{barrington1991oracle}.Therefore we are able to obtain only one-
%sided rational proofs for it.
\begin{corollary}
	\label{cor:L-NL}
$ \L \subseteq \DRMA[O(\log n), O(\log^2 n ), O(\log^2 n )]$
\end{corollary}

This improves over the construction of rational proofs for $\L$ in \cite{ratsumchecks} due to the better round complexity. 

\begin{corollary}
	\label{cor:SC}
$ \SC \subseteq \DRMA[O(\log n), O(\polylog(n)), O(\polylog(n))]$
\end{corollary}

No known result was known for $\SC$ before. 


\subsection{Rational Proofs for Randomized Bounded Space  Computation}
\label{sec:rand-space}

We now describe a variation of the above protocol, for the case of randomized bounded space computations. 

Let $\BPTISP[t,s]$ denote the class of languages recognized by randomized machines using time $t$ and space $s$ with error bounded by $1/3$ on both sides. 
In other words, $L \in \BPTISP[\poly(n), S(n)]$ if there exists a (deterministic) Turing Machine $M$ such that
for any $x \in \bits$ $\Pr_{r \in_R \bit^{\rnd(|x|)}}[M(x, r) = L(x)] \geq \frac{2}{3}$ and that runs in $S(|x|)$ space and polynomial time.
Let $\rho(n)$ be the maximum number of random bits used by $M$ for input $x \in \bit^n$; $\rho(\cdot)$ is clearly polynomial.

We can bring down the $2/3$ probability error to $\negl(n)$
by constructing a machine $M'$. $M'$ would simulate the $M$ on $x$ iterating the simulation $m = \poly(|x|)$ times
using fresh random bits at each execution and taking the majority output of $M(x;\cdot)$.
The machine $M'$ uses $m\rho(|x|)$ random bits and runs in polynomial time and $S(|x|) + O(\log(n))$ space.

The work in \cite{nisan1992pseudorandom} introduces pseudo-random generators (PRG) resistant against space bounded adversaries.
An implication of this result is that any randomized Turing Machine $M_1$ running in time $T$ and space $S$ can be simulated by a
 randomized Turing Machine $M_2$ running in time $O(T)$, space $O(S \log(T))$ and using only $O(S \log(T))$ random bits\footnote{We point out that the new machine $M_2$ introduces a small error. For our specific case this error keeps the overall error probability negligible and we can ignore it.}
(see in particular Theorem 3 in \cite{nisan1992pseudorandom}).
Let $L \in  \BPTISP[(\poly(n), S(n)]$ and $M'$ defined as above. We denote by $\hat{M}$ the simulation of $M'$ that uses Nisan's result described above.

By using the properties of the new machine $\hat{M}$, we can directly construct rational proofs for $\BPTISP(\poly(n), S(n))$.
We let the verifier picks a random string $r$ (of length $O(S \log(T))$) and sends it to the prover. They then invoke a rational
proof for the computation $\hat{M}(x;r)$.

By the observations above and Theorem \ref{thm:main} we have the following result:
\begin{corollary}
	$\BPTISP[\poly(n), S(n)] \subseteq \DRMA[\log(n), S(n) \log^2(n), S(n) \log^2(n)]$
\end{corollary}

We note that for this protocol, we need to allow for non-perfect completeness in the definition of $\DRMA$ in order to allow for the probability that the verifier chooses a bad random string $r$. 

%\begin{corollary}
%	\label{cor:space-bounded}
%	$ \NSC \subseteq \osDRMA[O(\log n), O(\polylog(n)), O(\polylog(n))]$
%\end{corollary}

