
% Overview and motivation

The problem of efficiently checking the correctness of a computation performed by an untrusted party has been central in Complexity Theory for the last 30 years since the introduction of Interactive Proofs by Babai and Goldwasser, Micali and Rackoff \cite{babai,gmr}. 

{\sf Verifiable Outsourced Computation} is now a very active research area in Cryptography and Network Security (see \cite{wb15} for a survey) with the aim to design protocols where it is impossible (under suitable cryptographic assumptions) for a provider to ``cheat" in the above scenarios. While much progress has been done in this area, we are still far from solutions that can be deployed in practice. 

Part of the reason is that Cryptographers consider a very strong adversarial model that prevents {\sf any} adversary from cheating. A different approach is to restrict ourselves to {\em rational adversaries}, whose motivation is not just to disrupt the protocol or computation, but simply to maximize a well defined utility function (e.g. profit).
%\subsection{Rational Proofs}

A different approach is to consider a model where ``cheating" might actually be possible, but the provider would have no motivation to do so. In other words while cryptographic protocols prevent {\sf any} adversary from cheating, one considers protocols that work against {\sf rational} adversaries whose motivation is to maximize a well defined utility function. 

We investigate this approach through two theoretical ``lenses'': \textit{(i)} \textit{rational proofs}, a variant of interactive proofs where provers lose (in economic terms) whenever they attempt to ``prove'' a false statement; \textit{(ii)} \textit{fine-grained protocols}, a model where parties's resources are assumed to be limited \footnote{In a more specific sense than the usual ``probabilistic polynomial time''.} and cheating is possible only through a larger amount of resources (e.g., cheating requires time $\lambda^3$, for some parameter $\lambda$, but all participating parties are assumed to be able to run in time at most $\lambda^2$). The connections between rationality and the latter model will be explored in the subsequent sections and in Chapter $\ref{chap:FG}$.

\subsection{Rational Proofs}

% NB: There is a little more stuff in the intro.tex-s. It is unclear it may help.
In the first two part of this thesis (Chapters \ref{chap:RP-expr} and \ref{chap:RP-seq}) we use the concept of {\sf Rational Proofs} introduced by Azar and Micali in \cite{am} and refined in a subsequent paper \cite{am1}. 

In a Rational Proof, given a function $f$ and an input $x$, the server returns the value $y=f(x)$, and (possibly) some auxiliary information, to the client. The client will in turn 
pay the server for its work with a reward which is a function of the messages 
sent by the server and some randomness chosen by the client.  The crucial 
property is that this reward is maximized in expectation when the server 
returns the correct value $y$. Clearly a rational prover who is only interested 
in maximizing his reward, will always answer correctly. 

The most striking feature of Rational Proofs is their simplicity. For example in \cite{am}, Azar and Micali show {\sf single-message} Rational Proofs for any problem in $\#P$, where an (exponential-time) prover convinces a (poly-time) verifier of the number of satisfying assignment of a Boolean formula. 

For the case of "real-life" computations, where the Prover is polynomial and the Verifier is as efficient as possible, Azar and Micali in \cite{am1} show $d$-round Rational Proofs for functions computed by (uniform) Boolean circuits of depth $d$, for $d=O(\log n)$ (which can be collapsed to a single round under some well-defined computational assumption as shown in \cite{ratargs}). The problem of rational proofs for any polynomial-time computable function remains tantalizingly open. 


Recent work \cite{ratsumchecks} shows how to obtain Rational Proofs with sublinear verifiers for languages in $\NC$. Recalling that $\L \subseteq \NL \subseteq \NC_2$, one can use the protocol  in \cite{ratsumchecks} to verify a logspace polytime computation (deterministic or nondeterministic) in $O(\log^2 n )$ rounds and $O(\log^2 n )$ verification.

The work by Chen et al. \cite{chen2016rational} focuses on rational proofs with multiple provers and the related class $\MRIP$ of languages decidable by a polynomial verifier interacting with an arbitrary number of provers. Under standard complexity assumptions, $\MRIP$ includes languages not decidable by a verifier interacting only with one prover. The class $\MRIP$ is equivalent to $\EXP^{||\NP}$.



\subsection{Contributions for Rational Proofs}

\subsubsection{Expressivity.}
We present new protocols for the verification of {\em space-bounded polytime computations} against a rational adversary. More specifically, consider a language $L \in \DTISP(T(n), S(n))$, i.e. recognized by a deterministic Turing Machine $M_L$ which runs in time $T(n)$ and space $S(n)$. 
In Section \ref{sec:rp-dtisp} we construct a protocol where a rational prover can
convince the verifier that $x \in L$ or $x \notin L$ with the following properties: 
\begin{itemize}
	\item The verifier runs in time $O(S(n) \log n)$
	\item The protocol has $O(\log n)$ rounds and communication complexity $O(S(n) \log n)$
	\item The prover simply runs $M_L(x)$ 
	%and stores all the intermediate configurations (i.e. requires space $O(S(n) T(n))$
\end{itemize}
%Our protocol can be proven to correctly incentivize a prover in {\bf both} the stand-alone model of \cite{am} and the sequentially composable definition of \cite{cg15}. This is the first protocol which is sequentially composable for a well-defined complexity class. 

For the case of ``real-life" computations (i.e. poly-time computations verified by a ``super-efficient" verifier) we 
note that for computations in sublinear space our general results yields a protocol in which the verifier is sublinear-time. Our protocols is the first rational proof for $\SC$ (also known as $\DTISP(\poly(n), \polylog(n))$) with polylogarithmic verification and logarithmic rounds. 
%Moreover, our results provide the first efficient rational proof for the 
%non-deterministic class $\NSC = \NTISP(\poly(n), \polylog(n) )$ . 

To compare this with the results in \cite{ratsumchecks}, we note that it is believed that $\NC \not = \SC$ and that the two classes are actually incomparable (see \cite{SCcompleteness} for a discussion). For these computations our results compare
favorably to the one in \cite{ratsumchecks} in at least one aspect: our protocol requires $O(\log n )$ rounds and has the same verification complexity.

We present several extensions of our main result:
\begin{itemize}
	
	\item Our main protocol can be extended to the case of space-bounded randomized computations using Nisan's 
	pseudo-random generator \cite{nisan1992pseudorandom} to derandomize the computation. 
	\item We also present a different protocol that works for BPNC (bounded error randomized NC) where the Verifier runs in polylog time (note that this class is not covered by our result since we do not know how to express NC with a polylog-space computation). This protocol uses in a crucial way a new {\em composition theorem} for rational proofs presented in this work and can be of independent interest. 
	\item Finally, we present lower bounds (i.e. conditional impossibility results) for Rational Proofs for various complexity classes.
\end{itemize}



\subsubsection{Repeated Executions and Costly Computation.}
Motivated by the problem of volunteer computation, our first
result is to show that the definition of Rational Proofs in \cite{am,am1} does not satisfy a basic compositional property which would make them applicable 
in that scenario. 
Consider the case where a large number of "computation problems" are outsourced. Assume that solving each problem takes time $T$. Then in a time interval of length $T$, the honest prover can only solve and receive the reward for a single problem. On the other hand a dishonest prover, can answer up to $T$ problems, for example by answering at random, a strategy that takes $O(1)$ time. To assure that answering correctly is a rational strategy, we 
need that at the end of the $T$-time interval the reward of the honest prover be larger than the reward of the dishonest one. But this is not necessarily the case: for some of the protocols in \cite{am,am1,ratargs} we can show that a ``fast" incorrect answer is more remunerable for the prover, by allowing him to solve more problems and collect more rewards.

The next questions, therefore, was to come up with a definition and a protocol that achieves rationality both in the stand-alone case, and in the composition
described above.  We first present an enhanced definition of Rational Proofs that removes the economic incentive  for the strategy of fast incorrect answers, and then we present a protocol that achieves it for the case of some (uniform) bounded-depth circuits.
Next, we design a $d$-rounds rational proof for sufficiently ``regular'' arithmetic circuit of depth $d = O(\log{n})$
with sublinear verification. We show, that under certain cost assumptions, our scheme is sequentially composable,
i.e. it can be used to delegate multiple inputs. We finally show that our scheme for space-bounded computations from Section \ref{sec:rp-dtisp} is also 
sequentially composable under certain cost assumptions.


\subsection{Comparison with Other Prior Work}
\label{sec:prior}

{\sc Other Decision-Theoretic Frameworks.}
An earlier work in the line of ``rational verifiable computation'' is \cite{b08} where the authors describe a system based on a scheme of rewards [resp. penalties] that the client assesses to the server for computing the function correctly [resp. incorrectly]. However in this system checking the computation may require re-executing it, something that the client does only on a randomized subset of cases, hoping that the penalty is sufficient to incentivize the server to perform honestly. Morever the scheme might require an "infinite" budget for the rewards, and has no way to "enforce" payment of penalties from cheating servers. For these reasons the best application scenario of this approach is the incentivization of volunteer computing schemes (such as SETI@Home or Folding@Home), where the rewards are non-fungible "points" used for "social-status". 

Because verification is performed by re-executing the computation, in this approach the client is "efficient" (i.e. does "less" work than the server) only in an 
amortized sense, where the cost of the subset of executions verified by the client is offset by the total number of computations performed by the server. This implies that the server must perform many executions for the client. 

{\sc Interactive Proofs.}
Obviously a ``traditional" interactive proof (where security holds against any adversary, even a computationally unbounded one) would work in our model. In this case the most relevant result is 
the recent independent work in \cite{rrr16} that presents breakthrough protocols for the deterministic (and randomized) restriction of the class of language we consider. If $L$ is a language which is recognized by a deterministic (or randomized) Turing Machine $M_L$ which runs in time $T(n)$ and space $S(n)$, then their protocol has the following properties: 
\begin{itemize}
	\item The verifier runs in 
	$O(\poly(S(n)) + n \cdot\polylog(n))$ time;
	\item The prover runs in polynomial time;
	\item The protocol runs in {\em constant} rounds, with communication complexity $O({\sf poly}(S(n)n^{\delta})$ for a constant $\delta$.
\end{itemize}
Apart from round complexity (which is the impressive breakthrough of the result in \cite{rrr16}) our protocols fares better in all other categories. Note in particular that a sublinear space computation does not necessarily yield a sublinear-time verifier in 
\cite{rrr16}. On the other hand, we stress that our protocol only considers weaker rational adversaries. 

\medskip
\noindent{\sc Computational Arguments.}
There is a large class of protocols for {\em arguments} of correctness (e.g. \cite{ggp10,ggpr13,krr14}) even in the rational model \cite{ratargs,ratsumchecks}. Recall that in an argument, security is achieved only against computationally bounded prover. In this case even single round solutions can be achieved. We will consider a variant of this model in Chapter \ref{chap:FG}.

\medskip
\noindent
{\sc Computational Decision Theory.}
Other works in theoretical computer science have studied the connections between cost of computation and utility in decision problems.
The work in \cite{halpern2011don} proposes a framework for \emph{computational decision problems}, where the Decision Maker's (DM) utility depends on the algorithm chosen for computing its strategy.
The Decision Maker runs the algorithm, assumed to be a Turing Machine, on the input to the computational decision problem.
The output of the algorithm determines the DM's strategy. 
Thus the choice of the DM reduces to the choice of a Turing Machine from a certain space. The DM will have beliefs on the running time (cost) of each Turing machine. The actual cost of running the chosen TM will affect the DM's reward.
Rational proofs with costly computation could be formalized in the language of \emph{computational decision problems} in \cite{halpern2011don}. There are similarities between the approach in this
work and that in \cite{halpern2011don}, as both take into account the cost of computation in a decision problem.





\subsection{Fine-Grained Verifiable Computation}

%\section{Introduction}

Historically, Cryptography has been used to protect information (either in transit or stored) from unauthorized access. One of the most important developments in Cryptography in the last thirty years, has been the ability to protect not only information but also the {\em computations} that are performed on data that needs to be secure. Starting with the work on secure multiparty computation \cite{mpc}, and continuing with ZK proofs \cite{zk}, and more recently Fully Homomorphic Encryption \cite{gentry}, verifiable outsourcing computation \cite{muggles,ggp10}, SNARKs \cite{qap,snark-linear} and obfuscation \cite{garg2016candidate} we now have cryptographic tools that protect the secrecy and integrity not only of data, but also of the programs which run on that data. 

Another crucial development in Modern Cryptography has been the adoption of a more ``fine-grained" notion of computational hardness and security. The traditional cryptographic approach modeled computational tasks as ``easy" (for the honest parties to perform) and ``hard" (infeasible for the adversary). Yet we have also seen a notion of {\em moderately hard} problems being used to attain certain security properties. The best example of this approach might be the use of moderately hard inversion problems used in blockchain protocols such as Bitcoin. Although present in many works since the inception of Modern Cryptography, this approach was first 
formalized in a work of Dwork and Naor \cite{dn-spam}. 

In this paper we consider the following model (which can be traced back to the seminal paper by Merkle \cite{merkle} on public key cryptography). Honest parties will run a protocol which will cost\footnote{
We intentionally refer to it as ``cost" to keep the notion generic. For concreteness one can think of $C$ as the running time required to run the protocol.} 
them $C$ while an adversary who wants to compromise the security of the protocol will incur a $C'=\omega(C)$ cost. Note that while $C'$ is asymptotically larger than $C$, it might still be a feasible cost to incur -- the only guarantee is that it is 
substantially larger than the work of the honest parties. For example in Merkle's original proposal for public-key cryptography the honest parties can exchange a key in time $T$ but the adversary can only learn the key in time $T^2$. Other examples include primitives introduced by Cachin and
Maurer \cite{maurer} and Hastad \cite{H87} where 
the cost is the space and parallel time complexity of the parties, respectively. 

Recently there has been renewed interest in this model. Degwekar et al. \cite{fgcrypto} show how to construct certain cryptographic 
primitives in $\NC^1$ [resp. $\AC^0$] which are secure against all adversaries in $\NC^1$ [resp. $\AC^0$]. In conceptually related work Ball et al. \cite{fghardness} present computational problems which are ``moderately hard'' on average, if they are moderately hard in the worst case, a useful property for such problems to be used as cryptographic primitives. 

The goal of this paper is to initiate a study of {\em Fine Grained Secure Computation}. By doing so we connect these two major developments in Modern Cryptography. The question we ask is if it is possible to construct {\em secure computation primitives} that are secure against ``moderately complex" adversaries. We answer this question in the affirmative, by presenting definitions and constructions for the task of Fully Homomorphic Encryption and Verifiable Computation in the fine-grained model. We also present two application scenarios for our model: i) hardware chips that prove their own correctness and ii) protocols against rational adversaries including potential solutions to the {\em Verifier's Dilemma} in smart-contracts transactions such as Ethereum. 

\subsection{Our Results}

Our starting point is the work in \cite{fgcrypto} and specifically their public-key encryption scheme secure against $\NC^1$ circuits. Recall that $\ACzt$ is the class of Boolean circuits with constant depth, unbounded fan-in, 
augmented with parity gates. If the number of $\function{AND}$ gates of non constant fan-in is constant we say that the circuit belongs to the class $\ACztq \subset \ACzt$.

Our results can be summarized as follows
\begin{itemize}
\item We first show that the techniques in \cite{fgcrypto} can be used to build a somewhat homomorphic encryption (SHE) scheme. 
We note that because honest parties are limited to 
$\NC^1$ computations, the best we can hope is to have a scheme that is homomorphic for computations in $\NC^1$. However our scheme can only support computations that can be expressed in 
$\ACztq$. 

\item We then use our SHE scheme, in conjunction with protocols described in \cite{ggp10,ckv10,aik10}, to construct verifiable 
computation protocols for functions in $\ACztq$, secure and input/output private against any adversary in $\NC^1$.

\end{itemize}
Our somewhat homomorphic encryption also allows us to obtain the following protocols secure against $\NC^1$ adversaries: \textit{(i)} constant-round 2PC, secure in the presence of semi-honest static adversaries for functions in $\ACztq$; \textit{(ii)} \textit{Private Function Evaluation} in a two party setting for circuits of constant \textit{multiplicative} depth without relying on universal circuits. These results stem from well-known folklore transformations and we do not prove them formally.

The class $\ACztq$ includes many natural and interesting problems such as: fixed precision arithmetic, evaluation of formulas in 3CNF (or $k$CNF for any constant $k$), a representative subset of SQL queries, and S-Boxes~\cite{sboxes} for symmetric key encryption. 

Our results (like \cite{fgcrypto}) hold under the assumption that $\fgAssump$, a widely believed worst-case assumption on separation of complexity classes. Notice that this assumption does not imply the existence of one-way functions (or even $\P \not = \NP$). Thus, our work shows that it is possible to obtain ``advanced'' cryptographic schemes, such as somewhat homomorphic encryption and verifiable computation, even if we do not live in Minicrypt\footnote{This is a reference to Impagliazzo's ``five possible worlds''~\cite{impagliazzo1995personal}.}\footnote{Naturally the security guarantees of these schemes are more limited compared to their standard definitions.}.

\medskip
\noindent
{\sc Comparison with other approaches.}
One important question is: on what features are our schemes better than ``generic" cryptographic schemes that after all are secure against {\em any} polynomial time  adversary. 

One such feature is the type of assumption one must make to prove security. As we said above, our schemes rely on a very mild worst-case complexity assumption, while cryptographic SHE and VC schemes rely on very specific assumptions, which are much stronger than the above. 

For the case of Verifiable Computation, we also have information-theoretic
protocols which are secure against {\em any} (possibly computationally unbounded) adversary. For example the ``Muggles'' protocol in \cite{muggles} which 
can compute any (log-space uniform) $\NC$ function, 
and is also reasonably efficient
in practice \cite{CMT}.
Or, the more recent work \cite{grlocally}, which obtains efficient VC for functions in a subset of $\NC \cap \class{SC}$.
Compared to these results, one aspect in which our protocol fares better  is that
our Prover/Verifier can be implemented with a constant-depth circuit (in particular in 
$\ACzt$, see Section \ref{sec:VC}) which is not possible for the Prover/Verifier in \cite{muggles,grlocally}, which needs to be in $\TC^0$\footnote{The techniques in  \cite{muggles,grlocally} are based on properties of finite fields. Arithmetic in such fields can be carried out by threshold circuits of constant depth, but not in $\ACzt$.}. Moreover our protocol is non-interactive (while \cite{muggles,grlocally} requires
$\Omega(1)$ rounds of interaction) and because our protocols work in the ``pre-processing model" we do not require any uniformity or regularity condition on the circuit being outsourced (which are required by \cite{muggles} and \cite{CMT}). Finally, out verification scheme achieves input and output privacy.

Finally, we compare our results with the information-theoretic approaches (mostly based on randomized encodings) in  \cite{gghkr07,re,cryptoNC0,SYY}. From the techniques in these works one could obtain somewhat homomorphic encryption and verifiable computation in low-depth circuits (even in $\NC^0$). Here, however, we stress that we are interested in \textit{compact} homomorphic encryption schemes (where the ciphertexts do not grow in size with each homomorphic operation) and in verifiable computation schemes where the total work of the verifier approximately linear in the I/O size (i.e. the size of the verification circuit should be $O(\poly(\lambda)(n+m))$ where $n$ and $m$ are the size of the input and output respectively). The techniques in these works cannot directly achieve these goals. In fact, for homomorphic encryption, they lead to ciphertexts of size exponential in $d$, where $d$ is the depth of the (fan-in two) evaluation circuit. For verifiable computation, they  lead to verification with quadratic running time\footnote{On why this running time: in a straightforward application of these approaches we would have the verifier computing the (randomized) encoding of a function $f \in \NC^1$. The work necessary for this is quadratic in the size of the branching program computing $f$ \cite{re} (this is  cubic if we use the approach in \cite{gghkr07}, described in Guy Rothblum's thesis \cite{rothblum2009delegating}).}. 

% Interactive proofs\footnote{We stress again: with \textit{information-theoretic} soundness.} with verification in constant depth are discussed in \cite{gghkr07} (where the verifier is in 
% $\mathsf{NC}^0$). We point out that, besides achieving non-interactive, constant-depth verification, our schemes also have a verifier running in linear {\em sequential} time in the input/output size (i.e. in size $O(\lambda^c(n+m))$ where $\lambda$ is the security parameter, $n$ the input and $m$ the output sizes of the function being outsourced). 

\subsection{Overview of our Techniques}

In \cite{fgcrypto} the authors already point out that their scheme is 
linearly homomorphic. We make use of the {\em re-linearization} technique 
from \cite{fhe-lwe} to construct a leveled homomorphic encryption. 

Our scheme (as the one in \cite{fgcrypto}) is secure against adversaries in the class of (\textit{non-uniform}) $\NC^1$. This implies that we can only evaluate functions in $\NC^1$ otherwise the evaluator would be able to break the semantic security of the scheme.
However we have to ensure that the \textit{whole} homomorphic evaluation stays in $\NC^1$. The problem is that homomorphically evaluating a function $f$ might 
increase the depth of the computation. 

In terms of circuit depth, the main overhead will be (as usual)
%in the case of homomorphic encryption) 
the computation of multiplication gates. As we 
 show in Section \ref{sec:HE} a single homomorphic multiplication can be performed by a depth two $\ACzt$ circuit, but this requires depth $O(\log(n))$ with a circuit of fan-in two. Therefore, a circuit for $f$ with $\omega(1)$ multiplicative depth would require an evaluation of $\omega(\log(n))$ depth, which would be out of $\NC^1$. Therefore our first scheme can only evaluate
 functions with constant multiplicative depth, as in that case the evaluation stays in $\ACzt$. 
 
We then present a second scheme that extends the class of computable functions 
to $\ACztq$ by allowing for a negligible error in the correctness of the scheme. We use techniques from a work by Razborov \cite{razborov1987lower} on approximating $\ACzt$ circuits with low-degree polynomials -- the correctness of the approximation (appropriately amplified) will be the correctness of our scheme. 



\subsection{Application Scenarios}

The applications described in this section refer to the problem of Verifying Computation, where a Client outsources an algorithm $f$ and an input $x$ to a Server, who returns a value $y$ and a proof that $y=f(x)$. The security property is that it should be infeasible to convince the verifier to accept $y' \neq f(x)$, and the crucial efficiency property is that verifying the proof should cost less than computing $f$ (since avoiding that cost was the reason the Client hired the Server to compute $f$). 

\medskip
\noindent
{\sc Hardware Chips That Prove Their Own Correctness}
Verifiable Computation (VC) can be used to verify the execution of hardware chips designed by untrusted manufacturers. One could envision chips that provide (efficient) \emph{proofs of their correctness} for every input-output computation they perform. These proofs must be  \emph{efficiently verified} in less time and energy than it takes to re-execute the computation itself. 

When working in hardware, however, one may not need the full power of cryptographic protection against {\em any} malicious attacks since one could bound the computational power of the malicious chip. The bound could be obtained by making (reasonable and evidence-based) assumptions on how much computational power can fit in a given chip area. For example one could safely assume that a malicious chip can perform at most a constant factor more work than the original function because of the basic physics of the size and power constraints. In other words, if $C$ is the cost of the honest Server in a VC protocol, then in this model the adversary is limited to $O(C)$-cost computations, and therefore a protocol that guarantees that succesful cheating strategies require $\omega(C)$ cost, will suffice. This is exactly the model in our paper. Our results will apply to the case in which we define the cost as the depth (i.e. the parallel time complexity) of the computation implemented in the chip. 

\medskip
\noindent
{\sc Rational Proofs.}
The problem above is related to the notion of composable Rational Proofs defined in \cite{cg15}. In a Rational Proof (introduced by Azar and Micali~\cite{am,am1}), given a function $f$ and an input $x$, the Server returns the value $y=f(x)$, and (possibly) some auxiliary information, to the Client. The Client in turn 
pays the Server for its work with a reward based on the transcript exchanged with the server and some randomness chosen by the client.  The crucial 
property is that this reward is maximized in expectation when the server 
returns the correct value $y$. Clearly a {\em rational} prover who is only interested 
in maximizing his reward, will always answer correctly. 

The authors of \cite{cg15} 
show however that the definition of Rational Proofs in \cite{am,am1} does not satisfy a basic compositional property needed for the case in which many computations are outsourced to many servers who compete with each other for rewards (e.g. the case of volunteer computations \cite{seti}). 
A ``rational proof" for the single-proof setting may no longer be rational when a large number of ``computation problems" are outsourced. %  Assume that solving each problem takes time $T$. Then in a time interval of length $T$, the honest prover can only solve and receive the reward for a single problem. On the other hand a dishonest prover, can answer up to $T$ problems, for example by answering at random, a strategy that takes $O(1)$ time. 
If one can produce $T$ ``random guesses" to problems in the time it takes to solve 1 problem correctly,  it may be preferable to guess! That's because even if each individual reward for an incorrect answer is lower than the reward for a correct answer, the total reward of $T$ incorrect answers might be higher (and this is indeed the case for some of the protocols presented in \cite{am,am1}). 

The question (only partially answered in \cite{cg15,cg17} for a limited class of computations) is to design protocols where the reward is strictly connected, not just to the correctness of the result, but to the amount of work done by the prover. Consider for example a protocol where the prover collects the reward only if he produces a proof of correctness of the result. Assume that the cost to  produce a valid proof for an incorrect result, is higher than just computing the correct result and the correct proof. Then obviously a rational prover will always answer correctly, because the above strategy of fast incorrect answers will not work anymore. 

While the application is different, the goal is the same as in the previous verifiable hardware scenario. 

\medskip
\noindent
{\sc The Verifier's Dilemma.}
In blockchain systems such as Ethereum, transactions can be expressed by arbitrary programs. To add a transaction to a block miners have to verify its validity, which could be too costly if the program is too complex. This creates the so-called {\em Verifier's Dilemma}~\cite{luu2015demystifying}: given a costly valid transaction $Tr$ a miner who spends time verifying it is at a disadvantage over a miner who does not verify it and accept it ``uncritically" since the latter will produce a valid block faster and claim the reward. On the other hand if the transaction is invalid, accepting it without verifying it first will lead to the rejection of the entire block by the blockchain and a waste of work by the uncritical miner.
The solution is to require efficiently verifiable proofs of validity for transactions, an approach already pursued by various startups in the Ethereum ecosystem (e.g. TrueBit\footnote{TrueBit: \textit{https://truebit.io/}}). We note that it suffices for these proofs to satisfy the condition above: i.e. we do not need the full power of information-theoretic or cryptographic security but it is enough to guarantee that to produce a proof of correctness for a false transaction is more costly than producing a valid transaction and its correct proof, which is exactly the model we are proposing. 


\subsection{Future Directions}

Our work opens up many interesting future directions. 

First of all, it would be nice to extend our results to the case where cost is the actual running time, rather than ``parallel running time"/``circuit depth" as in our model. The techniques in \cite{fghardness} (which presents problems conjectured to have $\Omega(n^2)$ complexity on the average), if not even the original work of Merkle \cite{merkle}, might be useful in building a verifiable computation scheme where if computing the function takes time $T$, then producing a false proof of correctness would have to take $\Omega(T^2)$. 

For the specifics of our constructions it would be nice to ``close the gap" between what we can achieve and the complexity assumption: our schemes can only compute $\ACztq$ against adversaries in $\NC^1$, and ideally we would like to be able to compute all of $\NC^1$ (or at the very least all of $\ACzt$). 

Finally, to apply these schemes in practice it is important to have tight concrete security reductions and a proof-of-concept implementations. 


%\subsection{Other Related Work}

%% Prior Work and Comparison to our results




\section{Notation and Common Preliminaries}

For a distribution $D$, we denote by $x \gets D$ the fact that $x$ is being sampled according to $D$.
We remind the reader that an ensemble $\mathcal{X} = \ens{X}$ is a family of  probability distributions over a family of domains $\mathcal{D}=\ens{D}$. We say two ensembles $\mathcal{D} = \ens{D}$ and $\mathcal{D}' = \ens{D'}$ are statistically indistinguishable if $\frac{1}{2}\Sum_x |D(x)-D'(x)| < \negl(\lambda)$. 

% TODO: Add parts on circuits?

\section{Thesis Roadmap} 
This thesis combines research done in several works.
\textbf{Chapter \ref{chap:RP-expr}} contains results on rational proofs for subclasses of polynomial-time computations in the model of Azar and Micali~\cite{am}. 
\textbf{Chapter \ref{chap:RP-seq}} contains a critique of the original model of raitonal proofs, definitions for sequential composability and related results. The results of these two chapters are joint work with Rosario Gennaro, originally presented in \cite{cg15} and \cite{cg17} (notice that each of the two chapters do not correspond to each of the two works).
\textbf{Chapter \ref{chap:FG}} contains results on homomorphic encryptions and delegating computation in a fine-grained model where adversaries are $\NC^1$ circuits.
 The results in this chapter are joint work with Rosario Gennaro, originally presented in \cite{cg18}.


