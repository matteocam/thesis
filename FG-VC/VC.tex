\section{Fine-Grained Verifiable Computation}
\label{sec:vc}
\label{sec:VC}


In this section we describe our private verifiable computation scheme.
Our constructions are heavily based on the techniques in \cite{ckv10} to obtain (reusable) verifiable computation from fully homomorphic encryption.
In order to guarantee that these techniques also work within $\NC^1$  we prove that: \textit{(i)} the constructions can be computed in low-depth; \textit{(ii)} the reductions in the security proofs can be carried out in low-depth.

\medskip
\noindent{\sc The Scheme from \cite{ckv10}.}
To derive Verifiable Computation from Homomorphic Encryption, \cite{ckv10} follows this approach. The Client, in the expensive preprocessing phase, selects a random input $r$, encrypts it $c_r=E(r)$ and homomorphically compute $c_{f(r)}$ an encryption of $f(r)$. During the online phase, the Client, on input $x$, computes $c_x=E(x)$ and submits
the ciphertexts $c_x,c_r$ in random order to the Server, who homomorphically compute 
$c_{f(r)}=E(f(r))$ and $c_{f(x)}=E(f(x))$ and returns them to the Client. The Client given the message $c_0,c_1$ from the Server, checks that $c_b=c_{f(r)}$ (for the appropriate bit $b$) and if so accepts $y=D(c_{f(x)})$ as $y=f(x)$. The semantic security of $E$ guarantees that this protocol has soundness error $1/2$ (which can be reduced by parallel repetition). This scheme is however one-time, as a malicious server can figure out 
which one is the test ciphertext $c_{f(r)}$ if it is used again. 

To make this scheme ``many time secure", \cite{ckv10} uses the paradigm introduced in 
\cite{ggp10} of running the 1-time scheme ``under the covers" of a different
homomorphic encryption key each time. 

\subsection{A One-time Verification Scheme}

Before we present our variant of the one-time construction in \cite{ckv10}, we present two auxiliary lemmas that guarantee that our protocols are computable in $\ACzt$. We refer the reader to \cite{hag91,matias91} for the proof Lemma \ref{lemma:perm-sampling}. % and we skip the proof of the second lemma since it is trivial.

\begin{lemma}{\cite{hag91,matias91}}
\label{lemma:perm-sampling}
There are uniform $\AC^0$ circuits $C: \bit^{\poly(l)}\to [l]^l$ of size $\poly(l)$ and depth $O(1)$ whose output distribution have statistical distance $\leq 2^{-l}$ from the uniform distribution over permutations of $[l]$.
\end{lemma}

\begin{lemma}
\label{lemma:perm-evaluation}
There are uniform $\AC^0[2]$ circuits $C: [l]^l \times \bit^l \to \bit^l$ of size $O(l^2)$ where $C(\pi, (x_1,\dots, x_l)) = (\pi(1),\dots,\pi(l))$ and $\pi$ is a permutation.
\end{lemma}
\begin{proof}
Let $\vec{x} = (x_1,\dots,x_l)$ the bits to permute and let $\pi$ be a permutation
If $\pi$ is represented as a permutation matrix with rows $\r_1,\dots,\r_l$, we can permute $\vec{x}$ by simply performing $l$ parallel inner products $\inprod{\vec{x}}{\r_i}$-s, which is in $\ACzt$.
We now describe how to generate the permutation matrix from a binary representations $x_1,\dots,x_{\function{lg}(l)}$ of the integers in $[l]$.
Let $f_i : \bit^{\function{lg}(l)} \to \bit^l$ be the function that computes the $i$-th row of the permutation matrix. We can define $f_i$ as follows:
\[
    f_i(x_1,\dots,x_{\function{lg}(l)}) \eqdef \function{eq}([i-1]_2, (x_1,\dots,x_{\function{lg}(l)}) \, ,
\]
where $[i-1]_2$ is the binary representation of $i-1$ and $\function{eq}$ returns 1 if its two inputs (each of lenght $\function{lg}(l)$) are equal.
The function $f_i$ is clearly in $\ACzt$.
% First, for each $i$, we generate the vector of hamming weight $1$ describing $\pi(i)$ by mapping 
\end{proof}

In Figure \ref{fig:one-time} we describe an adaptation of the one-time secure delegation scheme from \cite{ckv10}. We make non-black box use of our homomorphic encryption scheme $\HEp$ (Section \ref{sec:beyond-cm}) with soundness parameter $s= \lambda$. Notice that. during the preprocessing phase, we fix the ``auxiliary randomness'' for $\EvalApprox$ (and thus for $\HEpEval$) once and for all. We will use that same randomness for all the input instances. This choice does not affect the security of the construction.
We remind the reader that we will simplify notation by considering the evaluation key of our somewhat homomorphic encryption scheme as part of its public key.

If $x$ is a vector of bits $x_1, \dots, x_n$, below we will denote with $\HEpEnc(x)$ the concatenation of the bit by bit ciphertexts $\HEpEnc(x_1), \dots, \HEpEnc(x_n)$. We denote by $\HEpEnc(\zero)$ the concatenation of $n$ encryptions of $0$, $\HEpEnc(0)$.

% \item Compute $\hatraux \gets \SampleAuxRandomness_s(\pk, F)$;
% \item Compute $\vec{c} \gets \EvalApprox_s(\pk, F, \hatraux)$;
% \item Output $\vec{c} = (c^{\text{out}}_1, \dots, c^{\text{out}}_{s})$.

\begin{figure}
\begin{framed}
Let $f: \bit^n \to \bit^m$ be a function and $\GenApproxFun, \SampleAuxRandomness$ and $\EvalApprox$ as in Definition \ref{def:aux-he-fns}.
\begin{itemize}
\item $\VCKG(1^\lambda, f) \rightarrow (\pkW, \skD)$: We assume function $f$ represented as 
\begin{enumerate}
\item Generate a pair of keys $(\pk,\sk) \gets \HEpKeygen(1^\lambda)$.
\item Generate the approximating function $f' \gets \GenApproxFun(f)$ and auxiliary string $\auxf \gets \func{GenDecodeAux}(f)$;
\item Generate the ciphertext of the auxiliary random input for homomorphic evaluation $\hatraux \gets \SampleAuxRandomness_{\lambda}(\pk, f')$
\item Compute $t$ independent encryptions $\hatr_i = \HEpEnc_{\pk}(\zero)$ and the homomorphic evaluations $\hatw_i = \hatF(\hatr_i) =  \EvalApprox_s(\pk, f', \hatr_i, \hatraux)$ for $i \in [t]$;
\item $\pkW \gets (\pk, f', \hatraux), \skD \gets (\{ (\hatr_i, \hatw_i)_{i \in [t]}\}, \auxf)$.
\end{enumerate}
\item $\VCPG_{\skD}(x) \rightarrow (\qx, \sx)$: 
\begin{enumerate}
\item Compute $t$ independent encryptions $\hatr_{i+t} = \HEpEnc_{\pk}(x)$ for $i \in [t]$.
\item Sample a random permutation $\pi \sample S_{2t}$.
\item $\qx \gets (\hatzpi{1},\dots,\hatzpi{2t}) = (\hatr_1,\dots,\hatr_{2t})$; $\sx \gets \pi$
\end{enumerate}
\item $\VCCompute_{\pkW}(\qx) \rightarrow \ax$:
\begin{enumerate}
\item Compute $\hat{y}_i = \hatF(\hat{z}_i) = \EvalApprox_s(\pk, f', \hat{z}_i, \hatraux)$ for $i \in [2t]$.
\item $\ax = (\hat{y}_1,\dots,\hat{y}_{2t})$.
\end{enumerate}
\item $\VCVerif_{\skD}(\sx, \ax)$:
\begin{enumerate}
\item Check if $\hatw_i = \hat{y}_i$ for all $i \in [t]$.
\item Check if $\HEpDec_{\sk}(\hat{y}_{\pi(t+1)}, \auxf) = \dots = \HEpDec_{\sk}(\hat{y}_{\pi(2t)}, \auxf)$.
\item If either of the two tests above fails, return $\fail$; otherwise return $\HEpDec_{\sk}(\hat{y}_{\pi(t+1)}, \auxf)$.
\end{enumerate}
\end{itemize}
\end{framed}
\caption{One-Time Delegation Scheme}
\label{fig:one-time}
\end{figure}

\begin{remark}[On deterministic homomorphic evaluation]
As pointed out in \cite{ckv10}, one requirement for the approach in Figure \ref{fig:one-time} to work is for the homomorphic evaluation to be deterministic. We point out that once $\hatraux$ are fixed once and for all the homomorphic evaluation in $\VCCompute$ is deterministic.
\end{remark}

% \matteo{See if this should be fixed}
% \begin{remark}[On including $f'$ in $\pkW$]
% In the construction above we included $f'$ in the public key lengthening the size of the key. We point out this is not necessary and that $f'$ can be computed by the worker on her own during the execution of $\VCCompute$. However this would not allow us to simply homomorphically evaluate $\VCCompute$ in the definition of $\tVC$ in Section \ref{sec:vc-many}. This because the complexity of $\VCCompute$ would go from $\ACztcm$ to $\NC^1$, which our homomorphic schemes cannot handle.
% We point out that it would still be possible to modify the construction of $\tVC$  not including $f'$ in $\pkW$ to obtain the same completeness and soundness properties. However this would come at a cost of a more complex transformation in Figure \ref{fig:reusable-transform}. Including $f'$ in $\pkW$ allowed us to kept the transformation as simple and close to the original description in \cite{ckv10} as possible. 
% \end{remark}


\begin{lemma}[Completeness of $\VC$]
The verifiable computation scheme $\VC$ in Figure \ref{fig:one-time} has overwhelming completeness (Definition \ref{def:vc-completeness}) for the class $\ACztq$.
\end{lemma}
\begin{proof}
The proof is straightforward and stems directly from the homomorphic properties of $\HEp$ (Theorem \ref{thm:hep-homomorphic}).
In fact, by construction and by definition of $\HEp$ (Section \ref{sec:beyond-cm}), the distribution of the $\hatw_i$-s is identical to $\HEpEval_{\pk}(f, \hatr_i)$. Analogously, the distribution of $\hat{y}_i$-s is identical to $\HEpEval_{\pk}(f, \hat{z}_i)$.
\end{proof}


\begin{remark}[Efficiency of $\VC$]
In the following we consider the verifiable computation of a function $f: \bit^n \to \bit^m$ computable by an $\ACztq$ circuit of size $S$.
\begin{itemize}
\item  $\VCKG$ can be computed by an $\ACzt$ circuit of size $O(\poly(\lambda)S)$; 
\item $\VCPG$ can be computed by an $\ACzt$ circuit of size $O(\poly(\lambda)(m+n))$;
\item $\VCCompute$ can be computed by an $\ACzt$ circuit of size $O(\poly(\lambda)S)$; 
\item $\VCVerif$ can be computed by a $\ACzt$ circuit of size $O(\poly(\lambda)(m+n))$ and whose (constant) depth is independent of the depth of $f$.
\end{itemize}
\end{remark}

\begin{lemma}[One-time Soundness] 
\label{lemma:one-time}
Under the assumption that $\fgAssump$ the  scheme in Figure \ref{fig:one-time} is $(1,1)$-sound (one time secure) against $\NC^1$ adversaries whenever t is chosen to be  $\omega(\log(\lambda))$.
\end{lemma}
\begin{proof}
We follow the same proof structure as in the proof of Lemma $12$ in \cite{ckv10}. We will keep part of the analysis informal, emphasizing why this proof still works for low-depth circuits. We refer the reader to \cite{ckv10} for further details.

The following observation will be crucial in the rest of the proof. Notice that, by construction and by definition of $\HEp$ (Section \ref{sec:beyond-cm}), the distribution of the $\hatw_i$-s is identical to $\HEpEval_{\pk}(f, \hatr_i)$. Analogously, the distribution of $\hat{y}_i$-s is identical to $\HEpEval_{\pk}(f, \hat{z}_i)$.

Consider an $\NC^1$ adversary $\advstar$ that cheats with non-negligible probability in the one-time security experiment $\expVCone{\VC}$ (Definition \ref{def:vc-soundness}).
Let $(\hatr_1,\dots,\hatr_{t})$ be the independent copies of $\HEpEnc_{\pkW}(\zero)$ and  $(\hatr_{t+1},\dots,\hatr_{2t})$ the $t$ independent copies of
$\HEpEnc_{\pkW}(x)$ as above. 
Whenever the verification algorithm accepts, the adversary must have responded correctly on $\hatr_1,...,\hatr_t$ and incorrectly (and consistently) on $\hatr_{t+1},\dots,\hatr_{2t}$. 
Our goal is to bound the probability that the adversary succeeds in doing that. 

First, notice that the view of the adversary is
$(\pkW, \hatr_1,\dots,\hatr_{2t})$, and identical to $(\pkW, \HEpEnc_{\pkW}(\zero)^t, \HEpEnc_{\pkW}(x)^t)$.
By semantic security of the homomorphic encryption scheme, there exists an infinitely large set of parameters $\Lambda$ such that $ (\pkW, \HEpEnc_{\pkW}(\zero)^t, \HEpEnc_{\pkW}(x)^t) \lind (\pkW, \HEpEnc_{\pkW}(\zero)^{2t})$. Consider a modified game where the adversary receives  $(\pkW, \HEpEnc_{\pkW}(\zero)^{2t})$. Denote by $p$ the probability that the adversary succeeds in this game. By computational indistinguishability we have
\[
\Pr[\advstar \text{ is correct on } (\hatr_1,\dots,\hatr_{t}) \text{ and incorrect on } (\hatr_{t+1},\dots,\hatr_{2t}) ] \leq p + \negl(\lambda)
\]
for all $\lambda \in \Lambda$.
This inequality holds because we can test in $\NC^1$ whether $\advstar$ cheats only on $(\hatr_{t+1},\dots,\hatr_{2t})$. Therefore, if the adversary's behavior differed significantly between the two games, one would be able to break the semantic security of the homomorphic scheme. Here we made use of the third fact in Lemma $\ref{lemma:facts-lind}$.

We now proceed to upper bound $p$. Observe that 
\[ 
p = \Pr[\advstar \text{ is correct on } (\hatzpi{1},\dots,\hatzpi{t}) \text{ and incorrect on } (\hatzpi{t+1},\dots,\hatzpi{2t})] 
\]
where the $\hatzpi{i}$-s are defined as in Figure \ref{fig:one-time}.
Because of Lemma \ref{lemma:perm-sampling} that the distribution of $\pi$ is statistically indistinguishable from that of a uniformly random permutation. Also, observe that the answers $\hat{y}_i$ of the adversary are independent of $\pi$.
We can then conclude that $p \leq \frac{1}{\binom{2t} {t}} + \negl(t)$, which concludes the security analysis. 
% TODO: define z as in the other case
% \begin{align*}
%     p &= \Pr[\advstar \text{ is correct on } (\hatzpi{1},\dots,\hatzpi{t}) \text{ and incorrect on } (\hatzpi{t+1},\dots,\hatzpi{2t})] \\
%     &\leq \Pr[\advstar \text{ is correct on } (\hatzpi{1},\dots,\hatzpi{t}) \text{ and incorrect on } (\hatzpi{t+1},\dots,\hatzpi{2t}) \given[\Big] \sum^{2t}_{i=1} ]
% \end{align*}
\end{proof}


\subsection{A Reusable Verification Scheme}
\label{sec:vc-many}
We now describe how to obtain a reusable verification scheme $\tVC$ applying the transformation in \cite{ckv10} from one-time sound verification schemes through fully homomorphic encryption. The core idea behind the transformation in \cite{ckv10} is to encapsulate all the operations of a one-time verifiable computation scheme through homomorphic encryption. We instantiate this transformation with the one-time verifiable construction $\VC$, described in Figure \ref{fig:one-time}, and the simplest of our two somewhat homomorphic encryption schemes, $\HE$ (defined in Section \ref{sec:leveled-he-simple}). 

\begin{figure}
\begin{framed}
Let $\VC$ be the verifiable computation scheme defined in Figure \ref{fig:one-time}. The reusable verifiable computation scheme $\tVC = (\tVCKG,\tVCPG,\tVCCompute,\tVCVerif)$ is defined as follows.
\begin{itemize}
\item $\tVCKG(1^\lambda, f) \rightarrow (\tpkW, \tskD)$: The key generation stage is the same as in $\VC$.
\item $\tVCPG_{\tskD}(x) \rightarrow (\tqx, \tsx)$: 
\begin{enumerate}
\item $(\qx, \sx) \gets \VCPG_{\skD}(x)$;
\item Compute a fresh pair of keys $(\pk_x, \sk_x) \gets \HEKeygen(1^\lambda)$;
\item Compute $\hat{q}_x \gets \HEEnc_{\pk_x}(\qx)$;
\item $\tqx \gets (\pk_x, \hat{q}_x)$; $\tsx \gets (\sx, \sk_x)$
% \item Compute $t$ independent encryptions $\hatr_{i+t} = \HEpEnc_{\pk}(x)$ for $i \in [t]$.
% \item Sample a random permutation $\pi \sample S_{2t}$.
% \item $\qx \gets (\hatzpi{1},\dots,\hatzpi{2t}) = (\hatr_1,\dots,\hatr_{2t})$; $\sx \gets \pi$
\end{enumerate}
\item $\tVCCompute_{\tpkW}(\tqx) \rightarrow \tax$:
\begin{enumerate}
\item $\hat{a}_x \gets \HEEval_{\pk_x}(\VCCompute(\cdot, f), \hat{q}_x)$.
\item $\tax \gets \hat{a}_x$.
\end{enumerate}
\item $\tVCVerif_{\tskD}(\tsx, \tax)$:
\begin{enumerate}
\item $\ax \gets \HEDec_{\sk_x}(\hat{a}_x)$.
\item $\function{return}~\VCVerif_{\tskD}(\sx, \ax) $.
\end{enumerate}
\end{itemize}
\end{framed}
\caption{Transformation from one-time $\cal{VC}$ scheme to a \textit{reusable} $\cal{VC}$ scheme}
\label{fig:reusable-transform}
\end{figure}



\begin{corollary}[Completeness of $\tVC$]
The verifiable computation scheme $\tVC$ in Figure \ref{fig:reusable-transform} has overwhelming completeness (Definition \ref{def:vc-completeness}) for the class $\ACztq$.
\end{corollary}
\begin{proof}
The completeness of the scheme above follows directly from the completeness of $\VC$ and the homomorphic properties of $\HE$.
Notice that we can use $\HEEval$ to homomorphically compute $\VCCompute$ as the latter carries out a computation in $\ACztcm$ (although it is \textit{approximating} a computation in $\ACztq$). 
\end{proof}

\begin{remark}[Efficiency of $\tVC$]
The efficiency of $\tVC$ is analogous to that of $\VC$ with the exception of a circuit size overhead of a factor $O(\lambda)$ on the problem generation and verification algorithms and of $O(\lambda^2)$ for the computation algorithm.
All algorithms in $\tVC$ are computable by constant depth circuit (of unbounded fan-in) and the depth of the verification algorithm is independent of the function $F$.
\end{remark}


\begin{theorem}
Under the assumption that $\fgAssump$ the scheme $\tVC$ in Figure \ref{fig:reusable-transform} is $(O(1),\poly(\lambda))$-sound (many-times secure) against $\NC^1$ adversaries whenever t is chosen to be  $\omega(\log(\lambda))$ in the underlying scheme $\VC$. 
\end{theorem}
\begin{proof}
By Lemma \ref{lemma:one-time} there exists an infinite set $\Lambda \subseteq \naturals$ of security parameters for which $\VC$ ``is secure''.
By the proof of Lemma \ref{lemma:one-time}, this set is also the set of parameters where the somewhat homomorphic encryption scheme $\HE$ ``is secure''.
We will show that for all values in this same set $\Lambda$, the probability of success of any  $\NC^1$ adversary in $\expVCmany{\tVC}$ is negligible. 

Assume by contradiction there exists an $\NC^1$ adversary $\advstar$ that achieves non-negligible advantage in $\expVCmany{\tVC}$ for some $\lambda \in \Lambda$.

\textbf{Claim: If $\tVC$ is not secure for some $\lambda^* \in \Lambda$ then we can break the one-time security of $\VC$.}
Let $l = O(1)$ be the number of rounds in the many-time soundness experiment for $\tVC$. Consider the following $\NC^1$ adversary  $\adv_1$ for the experiment $\expVCone{\VC}$:
\begin{itemize}
\item $\adv_1$ obtains a pair a public key $\pkW$ and sends it to $\advstar$;
\item For all rounds $i \in \{1,\dots,l-1 \}$, $\adv_1$ replies to
$\advstar$ queries by generating a fresh pair of keys $(\pk, \sk)$ and sending back encryptions of $\HEEnc_{\pk}(\zero)$;
\item At round $l$, $\adv_1$ responds to all input queries but the last one as above. This, by experiment definition,  is the input where $\advstar$ will try to cheat; we denote this input by $x^*$. Now $\adv_1$ sends $x^*$ as the only input query in the one-time security experiment and will receive back $q^*$. It will then obtain a fresh pair of keys $(\pk^*, \sk^*)$ and send $\HEEnc_{\pk^*}(q^*)$ to $\advstar$.
\item $\advstar$ will respond with $\hata^*$ and $\adv_1$ will send $\HEDec_{\sk^*}(\hata)$ to the challenger for one-time security experiment.
\end{itemize}

The advantage of $\adv_1$ depends on how likely is  $\advstar$ can successfully cheat in that interaction. Let $p$ be the advantage of $\adv_1$ in the one-time security experiment. Clearly, if $p$ is close to the advantage of $\advstar$ in the many-times security experiment $\adv_1$ breaks the security of the one-time scheme.

\textbf{Claim: the advantage of $\adv_1$ is negligibly close to that of $\advstar$ in the many-time security game for security parameter $\lambda^*$.}
We can  prove this by relying on the semantic security of the homomorphic encryption and on a hybrid argument.
% Assume by contradiction that $p$ is noticeably far from the advantage of $\advstar$ in the many-time security game for all but a finite number of values of the security parameter.

Let $L = lm$, the total number of input queries in the many-times security experiment. We now define the hybrids $H^{(j)}$ with $j \in \{0, \dots, L\}$. We define $H^{(0)}$ to be the exactly the many-time security experiment. 
For $j \in [L]$ we define $H^{(j)}$ to be an experiment where we respond to input queries with $\HEEnc_{\pk_f}(\zero)$  where $\pk_f$ is a fresh public key up to input query $j$ and behaves the many-time security experiment from input query $j+1$ on. Notice that $H^{(L)}$ corresponds to the interaction with $\adv_1$ above.

Denote by $A^{(j)}$ the output distribution of $\advstar$ when interacting with $H^{(j)}$.
Intuitively, if the advantage of the $\adv_1$ in the one-time experiment is significantly different from the advantage of $\advstar$ in the many-times security games, then $A^{(0)}$ and $A^{(L)}$ are not $\Lambda$-computationally indistinguishable.

Therefore (by Lemma \ref{lemma:hyb-arg}), there exists $j \in [L]$ such that $A^{(j-1)} \not \sim_{\Lambda} A^{(j)}$.

\def\advcpa{\adv_{\text{CPA}}}
\textbf{Claim: If there exists $j \in [L]$ such that $A^{(j-1)} \not \sim_{\Lambda} A^{(j)}$ then we can break the semantic security of $\HE$.
}
Consider the following $\NC^1$ adversary $\advcpa$ which receives in input a ``challenge'' public key $\pk^*$. $\advcpa$ will interact with $\adv^*$ simulating $H^{(j)}$ until receiving input query $x_{j}$.
At this point it will compute $q_j$ from $\VCPG(x_j)$ and send to the CPA challenger (see Remark \ref{rem:cpa-multiple}) $q_j$ and $\zero$, receiving back an encryption $c^*$ of either message under the public key $\pk^*$. $\advcpa$ will now send $(\pk^*, c^*)$ to $\advstar$ and continue simulating $H^{(j)}$ till the end of the experiment.
The adversary $\advcpa$ will check whether $\advstar$ cheated successfully at the end of the experiment and output (in the multiple-message CPA experiment) $1$ if that is the case and $0$ otherwise. This would allow $\advcpa$ to have a noticeable advantage in the experiment thus breaking the semantic security of $\HE$. 
\end{proof}

% \begin{remark}
% On  the complexity of preparing a circuit version of $\HEpEval(\cdot, F)$ in the offline stage.
% \color{red}{TODO}
% \end{remark}
