The scheme is described in Figure \ref{fig:pke-dvv16}. Its security relies on the
 following result, implicit in \cite{ik00}\footnote{Stated as Lemma 4.3 in \cite{fgcrypto}.}. We will also use this lemma when proving the security of our construction in Section \ref{sec:he}.
 

\begin{lemma}[\cite{ik00}]
\label{lemma:ik00}
If $\fgAssump$ then there exist distribution $\mathcal{D}^{\text{kg}}_{\lambda}$ over  $\bit^{\lambda \times \lambda}$, distribution $\mathcal{D}^{f}_{\lambda}$ over matrices in $\bit^{\lambda \times \lambda}$ \textit{of full rank}, and infinite set $\Lambda \subseteq \naturals$ such that
\[
\Mkg \lind \MD
\]
where $\MD \gets \mathcal{D}^{f}_{\lambda}$ and $\Mkg \gets  \mathcal{D}^{\text{kg}}_{\lambda}$.
\end{lemma}

The following result is central to the correctness of the scheme $\PKE$ 
in Figure \ref{fig:pke-dvv16} and is implicit in \cite{fgcrypto}.

\begin{lemma}[\cite{fgcrypto}]
\label{lemma:dvv16-sampling}
There exists sampling algorithm $\KSample$ such that $(\M, \k) \gets \KSample(\unlambda)$, $\M$ is a matrix distributed according to $\mathcal{D}^{\text{kg}}_{\lambda}$ (as in Lemma \ref{lemma:ik00}), $\k$ is a vector in the kernel of $\M$ and has the form \\$\k = (r_1, \, r_2, \, \dots, \, r_{\lambda-1}, \, 1) \in \bit^{\lambda}$ where $r_i$-s are uniformly distributed bits.
\end{lemma}
%  % Matrices for sampling distributions
%  \def\visMz{
%  \begin{pmatrix}
%   0 & \   & \cdots & 0 & 0 \\
%   1 & 0 & \cdots & \   & 0 \\
%   0 & 1 & \ddots & \   & \vdots  \\
%   \vdots  & \ddots  & \ddots & \  & \\
%   0 & \cdots & 0 & 1 & 0 
%  \end{pmatrix}}


% % Sampling distributions
% \begin{figure}
% \label{fig:sampling-dists}
% \begin{framed}
% \begin{itemize}

% \item  \matteo{TODO: Find ways to set up matrices correctly}

% \end{itemize}
% \end{framed}
% \caption{Sampling Distributions from \cite{ik00}}
% \end{figure}
 
 
 % Definitions 
% \def\Rone{\mathbf{R}_{1}}
% \def\Rtwo{\mathbf{R}_{2}}


%  \newcommand{\LSamp}{\function{LSamp}}
%  \newcommand{\RSamp}{\function{RSamp}}
 
 
 
 % Actual scheme
\begin{figure}
\begin{framed}
% Let $\Mz$ be the $\lambda \times \lambda$ matrix  described in Figure \ref{fig:sampling-dists}.
\begin{itemize}
\item $\PKEKeygen_{\sk}(\unlambda):$
\begin{enumerate}
% \item Sample $\Rone \gets \LSamp (\unlambda)$ and 
% $\Rtwo \gets \RSamp(\unlambda)$;
% \item Let $\k = \transp{(\r \  1)}$ be the last column of $\Rtwo$;
\item Sample $(\M, \k) \gets \KSample(\unlambda)$;
\item Output $(\pk = \M, \sk = \k)$.
\end{enumerate}
\item $\PKEEnc_{\pk = \M}(\mu)):$
\begin{enumerate}
\item Sample $\r \sample \bit^{\lambda}$;
\item Let $\transp{t} = (0 \ \dots 0 \ 1) \in \bit^{\lambda}$;
\item Output $\transp{\c} = \transp{\r}  \M + \mu\transp{\t}$.
\end{enumerate}
\item $\PKEDec_{\sk = \k }(\c):$
\begin{enumerate}
\item Output $\inprod{\k}{\c}$
\end{enumerate}

\end{itemize}
\end{framed}
\caption{PKE construction \cite{fgcrypto}}
\label{fig:pke-dvv16}
\end{figure}


% % Sampling algorithms
% \begin{figure}
% \label{fig:sampling-algs}
% \begin{framed}
% \begin{itemize}

% \item  \matteo{TODO}

% \end{itemize}
% \end{framed}
% \caption{Sampling Algorithms (\cite{fgcrypto})}
% \end{figure}

\begin{theorem}[\cite{fgcrypto}]
Assume $\fgAssump$. Then, the scheme $\PKE = (\PKEKeygen, \PKEEnc, \PKEDec)$ defined in Figure \ref{fig:pke-dvv16} is a Public Key Encryption scheme secure against $\NC^1$ adversaries. All algorithms in the scheme are computable in $\ACzt$.
\end{theorem}

% To guarantee figures are flushed before the next subsection
%\clearpage