Consider the problem of {\sf Outsourced Computation} where a computationally  ``weak'' client hires a more  ``powerful'' server to store data and perform computations on its behalf. This paper is concerned with the problem of designing outsourced computation schemes that incentivize the server to perform correctly the tasks assigned by the client. 

The rise of the {\em cloud computing} paradigm where business do not maintain their own IT infrastructure, but rather hire  ``providers'' to run it, has brought this problem to the forefront of the research community. The goal is to find solutions that are efficient and feasible in practice for problems such as: How do we check the integrity of data that is stored remotely? How do we check computations performed on this remotely stored data? How can a client do this in the most efficient way possible? Or even more generally, how do we incentivize parties to perform correctly in such scenarios?


\subsection{Complexity Theory and Cryptography}

The problem of efficiently checking the correctness of a computation performed by an untrusted party has been central in Complexity Theory for the last 30 years since the introduction of Interactive Proofs by Babai and Goldwasser, Micali and Rackoff \cite{babai,gmr}. 

{\sf Verifiable Outsourced Computation} is now a very active research area in Cryptography and Network Security (see \cite{wb15} for a survey) with the aim to design protocols where it is impossible (under suitable cryptographic assumptions) for a provider to ``cheat" in the above scenarios. While much progress has been done in this area, we are still far from solutions that can be deployed in practice. 

Part of the reason is that Cryptographers consider a very strong adversarial model that prevents {\sf any} adversary from cheating. A different approach is to restrict ourselves to {\em rational adversaries}, whose motivation is not just to disrupt the protocol or computation, but simply to maximize a well defined utility function (e.g. profit).

\subsection{Rational Proofs}

In our work we use the concept of {\sf Rational Proofs} introduced by Azar and Micali in \cite{am} and refined in a subsequent paper \cite{am1}. 

In a Rational Proof, given a function $f$ and an input $x$, the server returns the value $y=f(x)$, and (possibly) some auxiliary information, to the client. The client will in turn 
pay the server for its work with a reward which is a function of the messages 
sent by the server and some randomness chosen by the client.  The crucial 
property is that this reward is maximized in expectation when the server 
returns the correct value $y$. Clearly a rational prover who is only interested 
in maximizing his reward, will always answer correctly. 

The most striking feature of Rational Proofs is their simplicity. For example in \cite{am}, Azar and Micali show {\sf single-message} Rational Proofs for any problem in $\#P$, where an (exponential-time) prover convinces a (poly-time) verifier of the number of satisfying assignment of a Boolean formula. 

For the case of  ``real-life" computations, Azar and Micali in \cite{am1} consider the case of efficient provers (i.e. poly-time) and ``super-efficient" (log-time) verifiers and present $d$-round Rational Proofs for functions computed by (uniform) Boolean circuits of depth $d$, for $d=O(\log n)$. 
% In this case the Verifier runs in logarithmic time.

Recent work \cite{ratsumchecks} shows how to obtain Rational Proofs with sublinear verifiers for languages in $\NC$. Recalling that $\L \subseteq \NL \subseteq \NC_2$, one can use the protocol  in \cite{ratsumchecks} to verify a logspace polytime computation (deterministic or nondeterministic) in $O(\log^2 n )$ rounds and $O(\log^2 n )$ verification.

The work by Chen et al. \cite{chen2016rational} focuses on rational proofs with multiple provers and the related class $\MRIP$ of languages decidable by a polynomial verifier interacting with an arbitrary number of provers. Under standard complexity assumptions, $\MRIP$ includes languages not decidable by a verifier interacting only with one prover. The class $\MRIP$ is equivalent to $\EXP^{||\NP}$.


\subsection{Repeated Executions with a Budget}
%\medskip
%\noindent
%{\sc Compositions of Rational Proofs.}
In \cite{cg15} 
%the authors 
we present a critique of the rational proof model in the case of ``repeated executions with a budget". This model arises in the context of ``volunteer computations" (\cite{seti,folding}) where many computational tasks are outsourced and provers compete in solving as many as possible to obtain rewards. In this scenario assume that a prover has a certain budget $B$ of ``computational effort": how can one  guarantee that the rational strategy is to provide the correct answer in {\em all} the proof he provides? The notion of rational proof guarantees that if the prover engages in a single rational proof then it is in his best interest to provide the correct output. But in \cite{cg15} 
%the authors
we show that in the presence of many computations, it might be more profitable for the prover to use his budget $B$ to provide many incorrect answers than to provide a single correct answer. That's because incorrect (e.g. random) answers are ``cheaper" to compute than the correct one and with the same budget $B$ the prover can provide many of them while the entire budget might be necessary to solve a single problem correctly. If the difference in reward between correct and incorrect answers is not high enough then many incorrect answers may be more profitable and a rational prover will choose that strategy, and indeed this is the case for many of the protocols in \cite{am,am1,ratargs,ratsumchecks}. 

In \cite{cg15} we put forward a stronger notion of {\em sequentially composable rational proofs} which avoids the above problem and guarantees that the rational strategy is always the one to provide correct answers. We also presented sequentially composable rational proofs, but only for some ad-hoc cases, and were not able to generalize them to well-defined complexity classes. 

%for a subset of bounded-depth circuit computations. 

\subsection{Our Contribution}

This paper presents new protocols for the verification of {\em space-bounded polytime computations} against a rational adversary. More specifically consider a language $L \in \DTISP(T(n), S(n))$, i.e. recognized by a deterministic Turing Machine $M_L$ which runs in time $T(n)$ and space $S(n)$. 
We construct a protocol where a rational prover can
convince the verifier that $x \in L$ or $x \notin L$ with the following properties: 
\begin{itemize}
	\item The verifier runs in time $O(S(n) \log n)$
	\item The protocol has $O(\log n)$ rounds and communication complexity $O(S(n) \log n)$
	\item The prover simply runs $M_L(x)$ 
	%and stores all the intermediate configurations (i.e. requires space $O(S(n) T(n))$
\end{itemize}
Our protocol can be proven to correctly incentivize a prover in {\bf both} the stand-alone model of \cite{am} and the sequentially composable definition of \cite{cg15}. This is the first protocol which is sequentially composable for a well-defined complexity class. 

For the case of ``real-life" computations (i.e. poly-time computations verified by a ``super-efficient" verifier) we 
note that for computations in sublinear space our general results yields a protocol in which the verifier is sublinear-time. More specifically, we introduce the first rational proof for $\SC$ (also known as $\DTISP(\poly(n), \polylog(n))$) with polylogarithmic verification and logarithmic rounds. 
%Moreover, our results provide the first efficient rational proof for the 
%non-deterministic class $\NSC = \NTISP(\poly(n), \polylog(n) )$ . 

To compare this with the results in \cite{ratsumchecks}, we note that it is believed that $\NC \not = \SC$ and that the two classes are actually incomparable (see \cite{SCcompleteness} for a discussion). For these computations our results compare
favorably to the one in \cite{ratsumchecks} in at least one aspect: our protocol requires $O(\log n )$ rounds and has the same verification complexity.

We present several extensions of our main result:
\begin{itemize}

	\item Our main protocol can be extended to the case of space-bounded randomized computations using Nisan's 
	pseudo-random generator \cite{nisan1992pseudorandom} to derandomize the computation. 
	\item We also present a different protocol that works for BPNC (bounded error randomized NC) where the Verifier runs in polylog time (note that this class is not covered by our result since we do not know how to express NC with a polylog-space computation). This protocol uses in a crucial way a new {\em composition theorem} for rational proofs which we present in this paper and can be of independent interest. 
	\item We discuss the notion of Rational Arguments (where the adversary is assumed to be both rational and computationally bounded, introduced in \cite{ratargs}) and show that it is achievable for all interesting complexity classes.
	\item Finally we present lower bounds (i.e. conditional impossibility results) for Rational Proofs for various complexity classes.
\end{itemize}

\subsection{The Landscape of Rational Proof Systems}

Rational Proof systems can be divided in roughly two categories, both of them presented in the original work \cite{am}. 

\medskip
\noindent
{\sc Scoring Rules.}
The more ``novel" approach in \cite{am} uses {\em scoring rules} to compute the reward paid by the verifier to the prover. A scoring rule is used to asses the ``quality" of a prediction of a randomized process. Assume that the prover declares that a certain random variable $X$ follows a particular probability distribution $D$. The verifier runs an ``experiment" (i.e. samples the random variable in question) and computes a ``reward" based on the distribution $D$ announced by the prover and the result of the experiment. A scoring rule is maximized if the prover announced the real distribution followed by $X$. The novel aspect of many of the protocols in \cite{am} was how to cast the computation of $y=f(x)$ as the announcement of a certain distribution $D$ that could be tested efficiently by the verifier and rewarded by a scoring rule. 

A simple example is the protocol for $\#P$ in \cite{am} (or its ``scaled-down" version for Hamming weight described more in detail in Section~\ref{sec:example}). Given a Boolean formula $\Phi(x_1,\ldots,x_n)$ the prover announces the number $m$ of satisfying assignments. This can be interpreted as the prover announcing that if one chooses an assignment at random it will be a satisfying one with probability $m \cdot 2^{-n}$. The verifier then chooses a random assignment and checks if it satisfies $\Phi$ or not and uses $m$ and the result of the test to compute the reward via a scoring rule. Since the scoring rule is maximized by the announcement of the correct $m$, a rational prover will announce the correct value. 

As pointed out in \cite{cg15} the problem with the scoring rule approach is that the reward declines slowly as the distribution announced by the Prover becomes more and more distant from the real one. The consequence is that incorrect results still get a substantial reward, even if not a maximal one. Since those incorrect results can be computed faster than the correct one, a Prover with ``budget" $B$ might be incentivized to produce many incorrect answers instead of a single correct one. All of the scoring rule based protocols in \cite{am,am1,ratargs,ratsumchecks} suffer from this problem. 

\medskip
\noindent
{\sc Weak Interactive Proofs.}
The definition of rational proofs requires that the expected reward is maximized for the honest prover. This definition can be made stronger (as done explicitly in \cite{ratargs}) and require a that every systematically dishonest prover would incur a polynomial loss (this property is usually described in terms of a \textit{noticeable reward gap}). As discussed above, the elegant device of scoring rules is the basis for most rational proof protocols in literature, some of which achieve noticeable reward gap. Another simple way in which we can obtain this stronger type of rational proof is the following. Imagine having a test where the prover can be caught cheating with ``low", but non-negligible probability, e.g. $n^{-k}$ for some $k \in \naturals$.
We will informally call this test a \textit{weak interactive proof}\footnote{This is basically the {\em covert adversary} model for multiparty computation introduced in \cite{AL10}.}. Indeed for such proofs we can always pay a fixed reward $R$ to the prover unless we catch him cheating in which case we pay $0$. These are rational proofs since obviously the expected reward of the prover is maximized by the honest behavior. Some of the proofs in \cite{am} and the proofs in \cite{cg15} are weak interactive proofs. Those proofs also turn out to be secure in the sequential model of \cite{cg15} (under appropriate assumptions). 

The protocols in this work are weak interactive proof, which is why we can prove them to be sequentially composable\footnote{
	One exception is the protocol for BPNC, which depends on the underlying protocol for (deterministic) NC. If we use the one in \cite{ratsumchecks}, then the resulting protocol uses scoring rules and is not sequentially composable. However an alterative protocol for NC can be used, based on our work in \cite{cg15}, which is a weak interactive proof and can be proven sequentially composable.}.


\medskip
\noindent
{\sc Scoring rules vs. weak interactive proofs.}
Comparing approaches based on scoring rules and weak interactive proofs the following two questions come up: 
\begin{itemize}
	\item Does one approach systematically lead to more efficient rational proofs (in terms of rounds, communication and verifying complexity) than the other? 
	\item Is one approach more suitable for sequential composability than the other?
\end{itemize}
We do not have a precise answer to the above questions, which we believe are interesting open problems to consider. However we can make the following statements. 

Regarding the first question: in the context of ``stand-alone" (non sequential) rational proofs it is not clear which approach is more powerful. We know that for every language class known to admit a scoring rule based protocol we also have a weak interactive proof with similar performance metrics (i.e. number of rounds, verifier efficiency, etc.). The result in this paper is the first example of a language class for which we have rational proofs based on weak interactive proofs but no example of a scoring rule based protocol exist\footnote{
	We stress that in this comparison we are interested in protocols with similar efficiency parameters. For example, the work in \cite{am} presents several large complexity classes for which we have rational proofs. However, these protocols require a polynomial verifier and do not obtain a noticeable reward gap.}.
This suggests that the weak interactive proof approach might be the more powerful technique. It would be interesting to prove that all rational proofs are indeed weak interactive proofs: i.e. that given a rational proof with certain efficiency parameters, one can construct a weak interactive proof with ``approximately" the same parameters. This question is left as future work.

On the issue of sequential composability, we have already proven in \cite{cg15} that some rational proofs based on scoring rules (such as Brier's scoring rule) are not  sequentially composable. 
This problem might be inherent at least for scoring rules that pay a substantial reward to incorrect computations. What we can say is that all known sequentially composable proofs are based on weak interactive proofs (\cite{cg15}, \cite{am1}\footnote{The construction in Theorem $5.1$ in \cite{am1} is shown to be sequentially composable in \cite{cg15}.} and this work). Again it would be interesting to prove that this is required, i.e. that all sequentially composable rational proofs are weak interactive proofs. 
%An important caveat to keep in mind, however, is that the notion of sequential 
%composability is to some extent flexible and depends
%on the cost model one assumes. % XXX: Maybe say something about the cost model in this work and in the previous one here?

% XXX: Rational arguments here.

\subsection{Other Related  Work}
\label{sec:prior}

{\sc Interactive Proofs.}
Obviously a ``traditional" interactive proof (where security holds against any adversary, even a computationally unbounded one) would work in our model. In this case the most relevant result is 
the recent independent work in \cite{rrr16} that presents breakthrough protocols for the deterministic (and randomized) restriction of the class of language we consider. If $L$ is a language which is recognized by a deterministic (or randomized) Turing Machine $M_L$ which runs in time $T(n)$ and space $S(n)$, then their protocol has the following properties: 
\begin{itemize}
	\item The verifier runs in 
	$O(\poly(S(n)) + n \cdot\polylog(n))$ time;
	\item The prover runs in polynomial time;
	\item The protocol runs in {\em constant} rounds, with communication complexity $O({\sf poly}(S(n)n^{\delta})$ for a constant $\delta$.
\end{itemize}
Apart from round complexity (which is the impressive breakthrough of the result in \cite{rrr16}) our protocols fares better in all other categories. Note in particular that a sublinear space computation does not necessarily yield a sublinear-time verifier in 
\cite{rrr16}. On the other hand, we stress that our protocol only considers weaker rational adversaries. 

\medskip
\noindent{\sc Computational Arguments.}
There is a large class of protocols for {\em arguments} of correctness (e.g. \cite{ggp10,ggpr13,krr14}) even in the rational model \cite{ratargs,ratsumchecks}. Recall that in an argument, security is achieved only against computationally bounded prover. In this case even single round solutions can be achieved. We do not consider this model in this paper, except in Section~\ref{sec:scproof} as one possible option to obtain sequential composability. 

\medskip
\noindent
{\sc Computational Decision Theory.}
Other works in theoretical computer science have studied the connections between cost of computation and utility in decision problems.
The work in \cite{halpern2011don} proposes a framework for \emph{computational decision problems}, where the Decision Maker's (DM) utility depends on the algorithm chosen for computing its strategy.
The Decision Maker runs the algorithm, assumed to be a Turing Machine, on the input to the computational decision problem.
The output of the algorithm determines the DM's strategy. 
Thus the choice of the DM reduces to the choice of a Turing Machine from a certain space. The DM will have beliefs on the running time (cost) of each Turing machine. The actual cost of running the chosen TM will affect the DM's reward.
Rational proofs with costly computation could be formalized in the language of \emph{computational decision problems} in \cite{halpern2011don}. There are similarities between the approach in this
work and that in \cite{halpern2011don}, as both take into account the cost of computation in a decision problem.





