\section{Fine-Grained SHE}
\label{sec:he}
\label{sec:HE}

%\subsection{A Public-Key Encryption Scheme Secure Against $\NC^1$ adversaries %\cite{fgcrypto}}
We start by recalling the public key encryption from \cite{fgcrypto} which is 
secure against adversaries in $\mathsf{NC}^1$. 

The scheme is described in Figure \ref{fig:pke-dvv16}. Its security relies on the
 following result, implicit in \cite{ik00}\footnote{Stated as Lemma 4.3 in \cite{fgcrypto}.}. We will also use this lemma when proving the security of our construction in Section \ref{sec:he}.
 

\begin{lemma}[\cite{ik00}]
\label{lemma:ik00}
If $\fgAssump$ then there exist distribution $\mathcal{D}^{\text{kg}}_{\lambda}$ over  $\bit^{\lambda \times \lambda}$, distribution $\mathcal{D}^{f}_{\lambda}$ over matrices in $\bit^{\lambda \times \lambda}$ \textit{of full rank}, and infinite set $\Lambda \subseteq \naturals$ such that
\[
\Mkg \lind \MD
\]
where $\MD \gets \mathcal{D}^{f}_{\lambda}$ and $\Mkg \gets  \mathcal{D}^{\text{kg}}_{\lambda}$.
\end{lemma}

The following result is central to the correctness of the scheme $\PKE$ 
in Figure \ref{fig:pke-dvv16} and is implicit in \cite{fgcrypto}.

\begin{lemma}[\cite{fgcrypto}]
\label{lemma:dvv16-sampling}
There exists sampling algorithm $\KSample$ such that $(\M, \k) \gets \KSample(\unlambda)$, $\M$ is a matrix distributed according to $\mathcal{D}^{\text{kg}}_{\lambda}$ (as in Lemma \ref{lemma:ik00}), $\k$ is a vector in the kernel of $\M$ and has the form \\$\k = (r_1, \, r_2, \, \dots, \, r_{\lambda-1}, \, 1) \in \bit^{\lambda}$ where $r_i$-s are uniformly distributed bits.
\end{lemma}
%  % Matrices for sampling distributions
%  \def\visMz{
%  \begin{pmatrix}
%   0 & \   & \cdots & 0 & 0 \\
%   1 & 0 & \cdots & \   & 0 \\
%   0 & 1 & \ddots & \   & \vdots  \\
%   \vdots  & \ddots  & \ddots & \  & \\
%   0 & \cdots & 0 & 1 & 0 
%  \end{pmatrix}}


% % Sampling distributions
% \begin{figure}
% \label{fig:sampling-dists}
% \begin{framed}
% \begin{itemize}

% \item  \matteo{TODO: Find ways to set up matrices correctly}

% \end{itemize}
% \end{framed}
% \caption{Sampling Distributions from \cite{ik00}}
% \end{figure}
 
 
 % Definitions 
% \def\Rone{\mathbf{R}_{1}}
% \def\Rtwo{\mathbf{R}_{2}}


%  \newcommand{\LSamp}{\function{LSamp}}
%  \newcommand{\RSamp}{\function{RSamp}}
 
 
 
 % Actual scheme
\begin{figure}
\begin{framed}
% Let $\Mz$ be the $\lambda \times \lambda$ matrix  described in Figure \ref{fig:sampling-dists}.
\begin{itemize}
\item $\PKEKeygen_{\sk}(\unlambda):$
\begin{enumerate}
% \item Sample $\Rone \gets \LSamp (\unlambda)$ and 
% $\Rtwo \gets \RSamp(\unlambda)$;
% \item Let $\k = \transp{(\r \  1)}$ be the last column of $\Rtwo$;
\item Sample $(\M, \k) \gets \KSample(\unlambda)$;
\item Output $(\pk = \M, \sk = \k)$.
\end{enumerate}
\item $\PKEEnc_{\pk = \M}(\mu)):$
\begin{enumerate}
\item Sample $\r \sample \bit^{\lambda}$;
\item Let $\transp{t} = (0 \ \dots 0 \ 1) \in \bit^{\lambda}$;
\item Output $\transp{\c} = \transp{\r}  \M + \mu\transp{\t}$.
\end{enumerate}
\item $\PKEDec_{\sk = \k }(\c):$
\begin{enumerate}
\item Output $\inprod{\k}{\c}$
\end{enumerate}

\end{itemize}
\end{framed}
\caption{PKE construction \cite{fgcrypto}}
\label{fig:pke-dvv16}
\end{figure}


% % Sampling algorithms
% \begin{figure}
% \label{fig:sampling-algs}
% \begin{framed}
% \begin{itemize}

% \item  \matteo{TODO}

% \end{itemize}
% \end{framed}
% \caption{Sampling Algorithms (\cite{fgcrypto})}
% \end{figure}

\begin{theorem}[\cite{fgcrypto}]
Assume $\fgAssump$. Then, the scheme $\PKE = (\PKEKeygen, \PKEEnc, \PKEDec)$ defined in Figure \ref{fig:pke-dvv16} is a Public Key Encryption scheme secure against $\NC^1$ adversaries. All algorithms in the scheme are computable in $\ACzt$.
\end{theorem}

% To guarantee figures are flushed before the next subsection
%\clearpage
 
\subsection{Leveled Homomorphic Encryption for $\ACztcm$ Functions Secure against $\NC^1$}
\label{sec:leveled-he-simple}
%XXX: Can I  use such title even if it's only a subset of $\AC^0[2]$
 % We show how to adapt the relinearization technique from BV13
 
 % -- description of the scheme --

%To highlight what elements are elements in $\GF(2)$, in this section we will 
We denote by $\vec{x}[i]$ the $i$-th bit of a vector of bits $\vec{x}$ . Below, the scheme $\PKE = (\PKEKeygen,\PKEEnc, \PKEDec)$ is the one defined in Figure \ref{fig:pke-dvv16}.
% Here is the way we redefine keygen and how we define eval.
%\begin{figure}
%\label{fig:he-scheme}
%\begin{framed}

Our SHE scheme is defined by the following four algorithms: 
\begin{itemize}
\item $\HEKeygen_{\sk}(\unlambda, L):$ For key generation, sample $L+1$ key pairs
 $(\M_0, \k_0),\dots,\M_0, \k_L) \gets \PKEKeygen(\unlambda)$, and compute, for all $\ell \in \{0, \dots, L-1\}$, $i,j \in [\lambda]$, the value
\[ % XXX: See if something needs to become a vector
    \vec{a}_{\ell,i,j} \gets \PKEEnc_{\M_{\ell+1}}(\k_{\ell}[i]\cdot \k_{\ell}[j]) \in \lambdabits
\]
We define $\A \eqdef \{ a_{\ell,i,j} \}_{\ell,i,j}$ to be the set of all these values.
t then outputs the secret key $\sk = \k_L$, and the public key 
$\pk= (\M_0, \A)$. In the following we call $\evk = \A$ the evaluation key.

We point out a property that will be useful later: by the definition above, for all $\ell \in \{0, \dots, L-1 \}$ we have
\begin{align}
\label{he-keys-invariant}
\inprod{\klp}{\a_{\ell+1,i,j}} = \kl[i] \cdot \kl[j]\,.
\end{align}

\item $\HEEnc_{\pk}(\mu)):$
Recall that $\pk = \M_0$. To encrypt a message $\mu$ we compute 
$\v \gets \PKEEnc_{\M_0}(\mu)$. The output ciphertext contains $\v$ in addition to 
a ``level tag'', an index in $\{0, \dots, L \}$ denoting the ``multiplicative depth'' of the generated ciphertext. The encryption algorithm outputs $c \eqdef (\v, 0)$.

\item $\HEDec_{\k_L}(c):$ To decrypt a ciphertext\footnote{We are only requiring to decrypt ciphertexts that are output by $\HEEval(\cdots)$} $c = (\v,L)$ compute $\PKEDec_{\k_L}(\v)$, i.e.
\[
\inprod{\k_L}{\v}
\]

\item $\HEEval_{\evk}(f, c_1,\dots,c_t):$ where $F : \bit^t \to \bit$: We require that $f$ is represented as an arithmetic circuit in $\GF(2)$ with addition gates  of unbounded fan-in and multiplication gates of fan-in 2. We also require the circuit to be \textit{layered}, i.e. the set of gates can be partitioned in subsets (layers) such that wires are always between adjacent layers. Each layer should be composed homogeneously either of addition or multiplication gates. Finally, we require that the number of multiplications layers (i.e. the multiplicative depth) of $f$ is $L$.

We homomorphically evaluate $f$ gate by gate. We will show how to perform multiplication (resp. addition) of two (resp. many) ciphertexts. Carrying out this procedure recursively, we can homomorphically compute any circuit $f$ of multiplicative depth $L$.
\subsubsection{Ciphertext structure during evaluation.} During the homomorphic evaluation a ciphertext will be of the form $c = (\v, \ell)$ where $\ell$ is the ``level tag'' mentioned above. At any point of the evaluation we will have that $\ell$ is between $0$ (for fresh ciphertexts at the input layer) and $L$ (at the output layer). We define homomorphic evaluation only among ciphertexts at the same level. Since our circuit is layered we will not have to worry about homomorphic evaluation occurring among ciphertexts at different levels. 
Consistently with the fact a level tag represents the multiplicative depth of a ciphertext, addition gates will keep the level of ciphertexts unchanged, whereas multiplication gates will increase it by one.
% say about the invariant
Finally, we will keep the invariant that the output of each gate evaluation $c = (\v, \ell)$ is such that
\begin{align}
\label{he-invariant}
\inprod{\k_{\ell}}{\v} = \mu
\end{align}
where $\mu$ is the correct plaintext output of the gate.


\subsubsection{Homomorphic Evaluation of gates:}
\begin{itemize}
\item \textit{Addition gates.}
Homomorphic evaluation of an addition gates on inputs $c_1,\dots,c_t$ where $c_i = (\v_i, \ell)$ is performed by outputting
\[
 c_{\text{add}} = (\v_{\text{add}}, \ell) \eqdef \Big( \Sum_{i} \v_i, \ell \Big)
\]

Informally, one can see that
\[
\inprod{\k_{\ell}}{\v_{\text{add}}} =  \inprod{\k_{\ell}}{\Sum_i\v_i} = \Sum_i\inprod{\k_{\ell}}{\v_i} = \Sum_i \mu_i
\]
where $\mu_i$ is the plaintext corresponding to $\v_i$. This satisfies the invariant in Eq. \ref{he-invariant}.

\item \textit{Multiplication gates.}
We show how to multiply ciphertexts $c, c'$ where $c = (\v, \ell)$ and $c' = (\v', \ell)$ to obtain an output ciphertext $c_{\text{mult}} = (\v_{\text{mult}}, \ell+1)$.

The homomorphic multiplication algorithm will set
\[
\v_{\text{mult}} \eqdef \Sum_{i,j \in [\lambda]} h_{i,j}\cdot \a_{\ell+1,i,j}
\]
where $h_{i,j} = \v[i] \cdot \v'[j]$ for $i,j \in [\lambda]$.


The final output ciphertext will be
\[
c_{\text{mult}} \eqdef (\v_{\text{mult}}, \ell+1).
\]

This satisfies the invariant in Eq. \ref{he-invariant} as
\begin{align*}
 \inprod{\klp}{\v_{\text{mult}}} &=  \inprod{\klp}{\Sum_{i,j \in [\lambda]} h_{i,j}\cdot \a_{\ell+1,i,j}} \\
                                &= \Sum_{i,j \in [\lambda]} ( h_{i,j} \cdot \inprod{\klp}{\a_{\ell+1,i,j}} )\\
                                &=\Sum_{i,j \in [\lambda]} ( h_{i,j} \cdot \kl[i]\cdot\kl[j] )\\
                                &= \Sum_{i,j \in [\lambda]} ( \v[i] \cdot \v'[j] \cdot \kl[i]\cdot\kl[j] )\\
                                &= \Big(\Sum_{i \in [\lambda]} \v[i] \cdot \kl[i]\Big)\cdot\Big(\Sum_{j \in [\lambda]} \v'[j] \cdot \kl[j]\Big)\\
                                &= \inprod{\kl}{\v}\cdot\inprod{\kl}{\v'}\\
                                &= \mu \cdot \mu'
\end{align*}

where in the third and fourth equality we used respectively Eq. \ref{he-keys-invariant} and the definition of $h_{i,j}$, and $\mu, \mu'$ are the plaintexts corresponding to $\v$ $\v'$ respectively.

\end{itemize}
\end{itemize}

%\end{framed}
%\caption{Leveled Homomorphic Construction}
%\end{figure}

\subsection{Security Analysis}

\begin{theorem}[Security]
The scheme $\HE$ is $\text{CPA}$ secure against $\NC^1$ adversaries (Definition \ref{def:cpa-he}) under the assumption $\fgAssump$.
\end{theorem} 
\begin{proof}
We are going to prove that there exists infinite $\Lambda \subseteq \naturals$ such that $(\pk,\evk, \HEEnc_{\pk}(0)) \lind (\pk,\evk, \HEEnc_{\pk}(1))$.

When using the notations $\Mkg$ and $\MD$ we will always denote matrices to respectively distributed according to $\Dl$ and  $\mathcal{D}^{\text{kg}}$, where $\Dl$ and $\mathcal{D}^{\text{kg}}$ are the distributions defined in Lemma \ref{lemma:ik00}.



We will define the (randomized) encoding procedure $\hEnc : \bit^{\lambda \times \lambda} \to \bit{\lambda}$ defined as 
\[
    \hEnc(\M, b) = \transp{\r} \M + \transp{(0 \ \dots 0 \ b)} \, ,
\]
where $r$ is uniformly distributed in $\bit^{\lambda}$. The functions we will pass to $\hEnc$ will be distributed either according to $\Mkg$ or $\MD$. Notice that: \textit{(i)}  $\hEnc(\Mkg, b)$ is distributed identically to $\HEEnc_{\pk}(b)$; \textit{(ii)} $\hEnc(\MD, b)$ corresponds to the uniform distribution over $\bit^{\lambda}$ because (by Lemma \ref{lemma:ik00}) $\MD$ has full rank and hence $\transp{\r}\MD$ must be uniformly random.

We will denote with $\Mkg_1, \dots, \Mkg_{L}$  the matrices $\M_1, \dots, \M_{\ell}$ used to construct the evaluation key in $\HEKeygen$ (see definition). Recall these matrices are distributed according to $\mathcal{D}^{\text{kg}}$ as in Lemma \ref{lemma:ik00}.

We will also define the following vectors:
\[
\valphakg_{\ell} \eqdef \{ \hEnc(\Mkg_{\ell+1}, \kl[i] \cdot \kl[j]) \ | \ i,j \in [\lambda] \} \qquad \valphaD_{\ell} \eqdef \{ \hEnc(\MD_{\ell+1}, \kl[i] \cdot \kl[j]) \ | \ i,j \in [\lambda] \} \, ,
\]
where $\kl$ is defined as in $\HEKeygen$ and the matrices in input to $\hEnc$ will be clear from the context. Notice that all the elements of $\valphakg_{\ell}$ are encryptions, whereas all the elements of $\valphaD_{\ell}$ are uniformly distributed.

We will use a standard hybrid argument. Each of our hybrids is parametrized by a bit $b$. This bit informally marks whether the hybrid contains an element indistinguishable from an encryption of $b$.
\begin{itemize}

\item $\hE^b \eqdef (\Mkg_0, \hEnc(\Mkg_0, b), \valphakg_1, \dots, \valphakg_L) $ where 
$\Mkg_0$ corresponds to the public key of our scheme.
%$\Mkg_{\ell}$-s with $\ell \in [L]$ correspond to the elements of the evaluation keys as in the definition of $\HEKeygen$.
Notice that $\valphakg_{\ell} \equiv \{ \vec{a}_{\ell, i, j} \ | \ i,j \in [\lambda] \}$ where $ \vec{a}_{\ell, i, j}$ is as defined in $\HEKeygen$. This hybrid corresponds to the distribution $(\pk,\evk, \HEEnc_{\pk}(b))$.
\item $\hH^b_0 \eqdef (\MD_0, \hEnc(\MD, b), \valphakg_1, \dots, \valphakg_L)$. The only difference from $\hE$ is in the first two components where we replaced the actual public key and ciphertext with a full rank matrix distributed according to $\Dl$ and a random vector of bits.
\item For $\ell \in [L]$ we define
\[
\hH^b_{\ell} \eqdef (\MD_0, \hEnc(\MD, b), \valphaD_1, \dots, \valphaD_{\ell}, \valphakg_{\ell+1}, \dots, \valphakg_L) \, .
\]
%where $\valphaD_{l} \eqdef \{ \vec{a}_{\ell, i, j} \ | \ i,j \in [\lambda], \r \sample \bit^{\lambda} \}$
\end{itemize}

We will proceed proving that
\[
\hEz \lind \hHz_0 \lind \hHz_1 \lind \dots \lind \hHz_L \lind \hH^1_L \lind \dots \lind \hH^1_1 \lind \hH^1_0 \lind \hE^1
\]
through a series of smaller claims. In the remainder of the proof $\Lambda$ refers to the set in Lemma \ref{lemma:ik00}.

\begin{itemize}
\item $\hEz \lind \hHz_0$: if this were not the case we would be able to distinguish $\Mkg_0$ from $\MD_0$ for some of the values in the set $\Lambda$ thus contradicting Lemma \ref{lemma:ik00}.
\item $\hHz_{\ell-1} \lind \hHz_{\ell}$ for $\ell \in [L]$: assume by contradiction this statement is false for some $\ell \in [L]$. That is 
\[
(\MD_0, \hEnc(\MD_0, b), \valphaD_1, \dots, \valphaD_{\ell-1}, \valphakg_{\ell}, \dots, \valphakg_L) \not \lind 
(\MD_0, \hEnc(\MD_0, b), \valphaD_1, \dots, \valphaD_{\ell}, \valphakg_{\ell+1}, \dots, \valphakg_L) \, .
\]
Recall that, by definition, the elements of $\valphakg_{\ell}$ are all encryptions whereas the elements of $\valphaD_{\ell}$ are all randomly distributed values. This contradicts the the semantic security of the scheme $\PKE$ (by a standard hybrid argument on the number of ciphertexts). 
%Informally, it is then possible to  define a sub-hybrid that breaks .


\item $\hHz_{L} \lind \hH^1_{L}$: the distributions associated to these two hybrids are identical. In fact, notice the only difference between these two hybrids is in the second component: $\hEnc(\MD, 0)$ in $\hHz_{L}$ and $\hEnc(\MD, 1)$ in $\hH^1_{L}$. As observed above $\hEnc(\MD, b)$ is uniformly distributed, which proves the claim.
\end{itemize}

All the claims above can be proven analogously for  $\hE^1, \hH^1_{0}$  and $\hH^1_{\ell}$-s.
\qed
\end{proof}

\subsection{Efficiency and Homomorphic Properties of Our Scheme}

Our scheme is secure against adversaries in the class $\NC^1$. This implies that we can run $\HEEval$ only on functions $f$ that are in $\NC^1$, otherwise the evaluator would be able to break the semantic security of the scheme.
However we have to ensure that the \textit{whole} homomorphic evaluation stays in $\NC^1$. The problem is that homomorphically evaluating $f$ has an overhead with respect to the "plain" evaluation of $f$. Therefore, we need to determine 
for which functions $f$, we can guarantee that 
%the fact that $F$ is in $\NC^1$ may not be a guarantee for 
$\HEEval(F, \dots)$ will stay in $\NC^1$. 
%We now proceed to analyze for which classes of functions this is the case.

In terms of circuit depth, the main overhead when evaluating $f$ homomorphically is given by the multiplication gates (addition, on the other hand, is ``for free'' --- see definition of $\HEEval$ above). A single homomorphic multiplication can be performed by a depth two $\ACzt$ circuit, but this requires depth $\Omega(\log(n))$ with a circuit of fan-in two. Therefore, a circuit for $f$ with $\omega(1)$ multiplicative depth would require an evaluation of $\omega(\log(n))$ depth, which would be out of $\NC^1$. On the other hand, observe that for any function $f$ in $\ACzt$ with constant multiplicative depth, the evaluation stays in $\ACzt$. This because there is a constant number (depth) of homomorphic multiplications each requiring an $\ACzt$ computation. 

We can now state the following result, derived from the observations above and the fact that the invariant in Eq. \ref{he-invariant} is preserved throughout homomorphic evaluation.

\begin{theorem}
\label{thm:he-homomorphic}
Let $\ACztcm$ the family of circuits in $\ACzt$ with constant multiplicative depth (see Definition \ref{def:ACztcm}).
The scheme $\HE$ is leveled $\ACztcm$-homomorphic. Key generation, encryption, decryption and evaluation are all computable in $\ACztcm$. 
\end{theorem}

\subsection{Beyond Constant Multiplicative Depth}
\label{sec:beyond-cm}

In the previous section we saw how our scheme is homomorphic for a class of constant-depth, unbounded fan-in arithmetic circuits in $\GF(2)$ with \textit{constant multiplicative depth}, i.e. polynomials in $\GF(2)$ of constant degree. We now show how to overcome this limitation by slightly changing our scheme and using techniques from  \cite{razborov1987lower} to approximate $\ACzt$ circuits with low-degree polynomials.



% \begin{definition}[Quasi-Constant Multiplicative Depth]
% Let $C \in \ACzt$ be a circuit. Let $S$ be the number of $\function{AND}$ gates of non constant fan-in. If $S = O(1)$ we say that $C$ has quasi-constant multiplicative depth. We denote with $\ACztq$ the class of circuits with such property.
% \end{definition}

\begin{lemma}[\cite{razborov1987lower}]
\label{lemma:razborov}
Let $C$ be an $\ACztq$ circuit of depth $d$. Then there exists a randomized circuit $C' \in \ACztcm$ such that, for all x,
\[
\Pr[C'(x) \not = C(x)] \leq \epsilon \, ,
\]
where $\epsilon = O(1)$. The circuit $C'$ uses $O(n)$ random bits and its representation can be computed in $\NC^0$ from a representation of $C$.
%\matteo{TODO: Check if representation can be computed in AC0.}
\end{lemma}
\begin{proof}
Consider a circuit $C \in \ACztq$ and let $K = O(1)$ be the total number of $\ANDgt$ and $\ORgt$ gates with non-constant fan-in. We can replace every $\ORgt$ gate of fan-in $m = \omega(1)$ with a randomized ``gadget'' that takes in input $m$ additional random bits and computes the function
$$\hat{g}_{\ORgt}(x_1,\dots,x_m; r_1, \dots, r_m) \eqdef \sum_{i \in [m]} x_i r_i \, .$$
This function can  be implemented in constant multiplicative depth with one $\XORgt$ gate and $m$ $\ANDgt$ gates of fan-in two.
Let $\vec{x} = (x_1,\dots,x_m)$ and $\vec{r} = (r_1,\dots, r_m)$. The probabilistic gadget $\hat{g}_{\ORgt}$ has one-sided error. if  $x_i = 0$ (i.e. if $\ORgt(\vec{x}) = 0$) then $\Pr[\hat{g}_{\ORgt}(\vec{x}; \vec{r}) = 0] = 1$; otherwise $\Pr[\hat{g}_{\ORgt}(\vec{x}; \vec{r}) = 1] = \frac{1}{2}$.

In a similar fashion, we can replace every unbounded fan-in $\ANDgt$ gate with a randomized gadget in computing
$$\hat{g}_{\ANDgt}(x_1,\dots,x_m; r_1, \dots, r_m) \eqdef 1 - \sum_{i \in [m]} (1 - x_i) r_i \, .$$
This gadget can also be implemented in constant-multiplicative depth and has one-sided error $1/2$.
Finally, let us observe that $\Pr[C'(x) \not = C(x)] \leq \epsilon$ with $\epsilon$ being a constant, because
we have only a constant number of gates to be replaced with gadgets for $\hat{g}_{\ORgt}$ or $\hat{g}_{\ANDgt}$.

We only provide the intuition for why the transformations above can be carried out in $\NC^0$. Assume the encoding of a circuit as a list of gates in the form $(g, t_g, in_1, \dots, in_m)$ where $g$ and $t$ are respectively the index of the output wire of the gate and its type (possibly of the form ``input'' or ``random input'') and the $in_i$-s are the indices of the input wire of $g$. The transformation from $C$ to $C'$ needs to simply copy all the items in the list except for the gates of unbounded fan-in. We will assume the encoding conventions of $C$ always puts these gates at the end of the list\footnote{This allows our $\NC^0$ circuit to to ``know'' which gates to copy and which ones to transform based on their position only.}. For each of such gates the transformation circuit needs to: add appropriate $r_1,\dots, r_m$ to the list, add $m$ $\ANDgt$ gates and one $\XORgt$, possibly (if we are transforming an $\ANDgt$ gate) add negation gates. All this can be carried out based on wire connections and the type of the gate (a constant-size string) and thus in $\NC^0$.
% Say this later
% If we repeat the execution of the gadget $s$ times using every time fresh random bit vectors $\vec{r}^{(1)},\dots, \vec{r}^{(s)}$, then we can correctly compute $\ORgt(\vec{x})$ with overwhelming probability. Define $h_{\ORgt}(\vec{x}; \vec{r}^{(1)},\dots, \vec{r}^{(s)}) \eqdef \ORgt(\hat{g}_{\ORgt}(\vec{x}; \vec{r}^{(1)}), \dots, \hat{g}_{\ORgt}(\vec{x}; \vec{r}^{(s)}))$. Clearly $\Pr[h_{\ORgt}(\vec{x}; \vec{r}^{(1)},\dots, \vec{r}^{(s)}) = \ORgt(\vec{x})] \geq 1 - 2^{-s}$.
% Say that you replace every OR with $\hat{g}_{\func{or}}(x_1, \dots, x_m; r_1, \dots r_m) = \sum_i x_i\cdot r_i$ and that we can do this with the appropriate gates. The probability of success is 1/2. 
\end{proof}

In the construction above, we built $C'$ by replacing every gate $g \in \func{S}_{\omega(1)}(C)$ (as in Definition \ref{def:ACztq}) with a (randomized) gadget $G_g$. The output of each these gadgets will be useful in order to keep the low complexity of the decryption algorithm in our next homomorphic encryption scheme. We shall use an ``expanded'' version of $C'$, the multi-output circuit $C'_{exp}$. 

\def\GenApproxFun{\function{GenApproxFun}}
\def\SampleAuxRandomness{\function{SampleAuxRandomness}}
\def\EvalApprox{\function{EvalApprox}}

\begin{definition}[Expanded Approximating Function]
\label{def:exp-approx-fun}
Let $C$ be a circuit in $\ACztq$ and let $C'$ be a circuit as in the proof of Lemma \ref{lemma:razborov}.
We denote by $G_g(\vec{x}; \vec{r})$  the output of the gadget $G_g$ when $C'$ is evaluated on inputs $(\vec{x}; \vec{r})$. On input $(\vec{x}; \vec{r})$, the multi-output circuit $C'_{exp}$  output $C'(\vec{x};\vec{r})$ together with the outputs of the $O(1)$ gadgets $G_g$ for each $g \in \func{S}_{\omega(1)}(C)$. Finally, we denote with $\GenApproxFun$ the algorithm computeing a representation of $C'_{exp}$ from a representation  of $C$.
\end{definition}

\def\auxf{\mathbf{aux}_f}


\begin{lemma}
\label{lemma:decode-approx}

%\label{lemma:}
There exists a deterministic algorithm $\func{DecodeApprox}$  computable in $\ACzt$ with the following properties. 
For every circuit $C$ in $\ACztq$ computing the function $f$, there exists $\auxf \in \bit^{O(1)}$ such that for all $\vec{x} \in \bit^n$
$$\Pr[\func{DecodeApprox}(C'_{exp}(\vec{x}; \vec{r}^{(1)}), \dots, C'_{exp}(\vec{x}; \vec{r}^{(s)}) ) = C(\vec{x})] \geq 1 - \negl(s) \, ,$$
where $C'$ is an approximating circuit as in Lemma \ref{lemma:razborov}, the probability is taken over  the uniformly distributed bit vectors $\vec{r}^{(i)}$-s for $i \in [s]$, $C'_{exp}$ is as in Definition \ref{def:exp-approx-fun}.
Finally, there exists a function $\func{GenDecodeAux}$ that computes $\auxf$ from a representation of $C$ in $\NC^0$.
% \vec{y}^{(1)},\dots, \vec{y}^{(s)}, z^{(1)}, \dots, z^{(s)}
% $\vec{y}^{(i)} \eqdef \{ G_g(\vec{x}, \vec{r}^{(i)}) : g \in \func{S}_{\omega(1)}(C) \}$ 
\end{lemma}
\begin{proof}
Before we provide a construction for $\func{DecodeApprox}$, let us observe how we can amplify the error of $C'$.  Consider for example a gadget $\hat{g}_{\ORgt}$ constructed as in the proof of Lemma \ref{lemma:razborov}, approximating  an $\ORgt$ gate in $C$.
If we repeat the execution of the gadget $s$ times, every time using fresh random bit vectors $\vec{r}'^{(1)},\dots, \vec{r}'^{(s)}$, then we can correctly compute $\ORgt(\vec{x}')$ with overwhelming probability. Define $h_{\ORgt}(\vec{x}'; \vec{r}'^{(1)},\dots, \vec{r}'^{(s)}) \eqdef \ORgt(\hat{g}_{\ORgt}(\vec{x}; \vec{r}'^{(1)}), \dots, \hat{g}_{\ORgt}(\vec{x}'; \vec{r}'^{(s)}))$. Clearly $\Pr[h_{\ORgt}(\vec{x}'; \vec{r}'^{(1)},\dots, \vec{r}'^{(s)}) = \ORgt(\vec{x}')] \geq 1 - 2^{-s}$.
In a similar fashion we can define  $h_{\ANDgt}(\vec{x}'; \vec{r}'^{(1)},\dots, \vec{r}'^{(s)}) \eqdef \ANDgt(\hat{g}_{\ANDgt}(\vec{x}; \vec{r}'^{(1)}), \dots, \hat{g}_{\ANDgt}(\vec{x}'; \vec{r}'^{(s)}))$. It holds that $\Pr[h_{\ANDgt}(\vec{x}'; \vec{r}'^{(1)},\dots, \vec{r}'^{(s)}) = \ANDgt(\vec{x}')] \geq 1 - 2^{-s}$.

If $C'$ were composed by a single gadget $\hat{g}_{\ORgt}$ (resp. $\hat{g}_{\ANDgt}$) we could just let $\func{DecodeApprox}$ be the same as $h_{\ORgt}$ (resp. $h_{\ANDgt}$) and we would be done. To deal with multiple gadgets, however, we need a more general approach. For sake of presentation, assume there are only gadgets approximating $\ORgt$ gates and let us temporarily ignore $\auxf$. We can write each of the $C'_{exp}(\vec{x}; \vec{r}^{(j)})$ input to $\func{DecodeApprox}$ as $(z^{(j)}, y_1^{(j)}, \dots, y_K^{(j)})$ where $K \eqdef |\func{S}_{\omega(1)}|$, $z^{(j)}$ is the output of $C'(\vec{x}, \vec{r}^{(j)})$ and $y_i^{(j)}$ is the output of the $i-th$ gadget when provided random bits from $\vec{r}^{(j)}$. Define $y_i^*$ as $y_i^* \eqdef \ORgt(y_i^{(1)}, \dots,y_i^{(s)})$. 
We then let the output of $\func{DecodeApprox}$ be $z^{j^*}$ where $j^*$ is such that for all $i \in [K]$ it is the case that $y_i^{j^*} = y_i^*$. By the union bound the probability of $z^{j^*} \not = C(\vec{x})$ is upper bounded by $K\cdot 2^{-s}$, which is negligible since $K = O(1)$. 
To generalize this same approach to the scenario including both $\ORgt$ and $\ANDgt$ gadgets we let the string $\auxf$ include information on the type of gates in $\func{S}_{\omega(1)}$. This way  $\func{DecodeApprox}$ can use $\hat{g}_{\ORgt}$ or $\hat{g}_{\ANDgt}$ accordingly.
Clearly the  representation of $\auxf$ can be computed by a representation of $C$ in $\NC^0$.
\end{proof}


%The result above allows 

\subsubsection{Homomorphic Evaluations of $\ACztq$ Circuits}

Below is a variation of our homomorphic scheme that can evaluate all circuits in $\ACztq$ in $\ACzt$. This time, in order to evaluate circuit $C$, we perform several homomorphic evaluations of the randomized circuit $C'$ (as in Lemma \ref{lemma:razborov}). To obtain the plaintext output of $C$ we can decrypt all the ciphertext outputs and take the majority result. Notice that this scheme is still compact.
As we use a randomized approach to evaluate $f$, the scheme $\HEp$ will be implicitly parametrized by a soundness parameter $s$. Intuitively, the probability of a function $f$ being evaluated incorrectly will be upper bounded by $2^{-s}$.

%To simplify notation, in the following paragraphs and in Section \ref{sec:vc} we 
%will slightly abuse the syntax for homomorphic encryption schemes and consider 
%both the public key and evaluation key as part of $\pk$.


For our new scheme we will use the following auxiliary functions:
\begin{definition}[Auxiliary Functions for $\HEp$]
\label{def:aux-he-fns}
\item Let $f: \bit^t \to \bit$ be represented as an arithmetic circuit as in $\HE$ and $\pk$ a public key for the scheme $\HE$ that includes the evaluation key. Let $s$ be a soundness parameter.
We denote by  $f'$ the expanded randomized function approximating $f$ as in Definition \ref{def:exp-approx-fun}; let $t' = O(t)$ be the number of additional random bits $f'$ will take in input.
\begin{itemize}
\item $\GenApproxFun(f):$
\begin{itemize}
\item Computes and returns the representation of the expanded approximating function $f'$ as in Definition \ref{def:exp-approx-fun}.
\end{itemize}
\item $\func{GenDecodeAux}(f):$
\begin{itemize}
\item Computes and returns the auxiliary string $\auxf$ from a representation of $f$ as in Lemma \ref{lemma:decode-approx}.
\end{itemize}
\item $\SampleAuxRandomness_s(\pk, f'):$
\begin{enumerate}
\item We assume  $f'$ is the expanded randomized function approximating $f$ as in Definition \ref{def:exp-approx-fun}; let $t' = O(t)$ be the number of additional random bits $f'$ will take in input.
\item Sample $s\cdot t'$ random bits $r^{(1)}_1,\dots,r^{(1)}_{t'}, \dots, r^{(s)}_1,\dots,r^{(s)}_{t'}$;
\item Compute $\hatraux \eqdef \{  \hat{r}^{(i)}_j \ | \ \hat{r}^{(i)}_j \gets \HEEnc_{\pk}(r^{(i)}_j), i \in [s], j \in [t'] \} $;
\item Output $\hatraux$.
\end{enumerate}
\medskip
\item $\EvalApprox_s(\pk, f', c_1,\dots,c_t, \hatraux):$
\begin{enumerate}
\item Let $\hatraux = \{  \hat{r}^{(i)}_j \ | \  i \in [s], j \in [t'] \} $.
\item For $i \in [s]$, compute $\vec{c}^{\text{out}}_i \gets \HEEval_{\evk}(f', c_1, \dots c_t, \hat{r}^{(i)}_1, \dots, \hat{r}^{(i)}_{t'})$;
\item Output  $\vec{c} = (\vec{c}^{\text{out}}_1, \dots, \vec{c}^{\text{out}}_{s})$\footnote{Recall that the output of the expanded approximating function $f'$ is a bit string and thus each $\vec{c}^{\text{out}}_i$ encrypts a bit string.}.
\end{enumerate}
\end{itemize}
\end{definition}

\bigskip

The new scheme $\HEp$ with soundness parameter $s$ follows. Notice that the evaluation function outputs an auxiliary string $\auxf$ together with the proper ciphertext $\vec{c}$. This is necessary to have a correct decoding in decryption phase.
\begin{framed}
\begin{itemize}
\item Key generation and encryption are the same as in  $\HE$.
\item $\HEpEval_{\pk}(f, c_1,\dots,c_t)$:
\begin{enumerate}
\item Compute $f' \gets \GenApproxFun(f)$;
\item Compute $\hatraux \gets \SampleAuxRandomness_s(\pk, f')$;
\item $\auxf \gets \func{GenDecodeAux}(f)$;
\item $\vec{c} \gets \EvalApprox_s(\pk, f', c_1,\dots, c_t, \hatraux)$;
\item Output $(\vec{c}, \auxf)$.
%\item Output $\vec{c}$. % = (\vec{c}^{\text{out}}_1, \dots, \vec{c}^{\text{out}}_{s})$.
\end{enumerate}
\item $\HEpDec_{\sk}(\vec{c} = (\vec{c}^{\text{out}}_1, \dots, \vec{c}^{\text{out}}_{s}), \auxf)$:
\begin{enumerate}
%\item $\func{DecodeApprox}_f \gets \func{GenDecodeApproxFun}(\auxf)$
\item Let $\vec{y}^{\text{out}}_i \gets \HEDec_{\sk}(\vec{c}^{\text{out}}_i)$ for $i \in [s]$;
\item Output $\func{DecodeApprox}_f(\vec{y}^{\text{out}}_1, \dots, \vec{y}^{\text{out}}_s)$.
\end{enumerate}
\end{itemize}
\end{framed}
% \newcommand{\MACz}{\class{MAC}_0}
% In the theorem below, the complexity class $\MACz$ refers to the 

\begin{remark}
Given in input a function $f$ not necessarily of constant multiplicative depth,  $\GenApproxFun$ returns a function $f'$ of constant multiplicative depth that approximates it.
As stated in Lemma \ref{lemma:razborov}, $\GenApproxFun$ is computable in $\NC^0$ and so is $\func{GenDecodeAux}$. The function $\SampleAuxRandomness$ in $\ACztcm$ and $\EvalApprox$ makes parallel invocations to $\HEEval$ which is computable in $\ACztcm$ when provided in input a function in $\ACztcm$ (Theorem \ref{thm:he-homomorphic}). This fact will be useful when showing the completeness of our verifiable computation schemes in Section \ref{sec:vc}.
\end{remark}

\begin{theorem}
\label{thm:hep-homomorphic}
Let $\ACztq$ the family of circuits in $\ACzt$ with quasi-constant multiplicative depth as in Definition \ref{def:quasi-constant}.
The scheme $\HEp$ above with soundness parameter $s = \Omega(\lambda)$ is leveled $\ACztq$-homomorphic. Key generation, encryption and evaluation can be computed in $\ACztcm$. Decryption is computable in $\ACzt$.
\end{theorem}