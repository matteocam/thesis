%\subsection{Public-Key Encryption}
A public-key encryption scheme \\
$\PKE = (\PKEKeygen, \PKEEnc, \PKEDec)$ is a triple of algorithms which operate as follow:
\begin{itemize}
\item \textbf{Key Generation.} The algorithm $(\pk,\sk)\leftarrow$ \PKEKeygen$(1^{\lambda})$ takes a unary representation of the security parameter and outputs a public key encryption key $\pk$ and a secret decryption key $\sk$.
\item \textbf{Encryption.} The algorithm $c\leftarrow$ \PKEEnc$_{\pk}(\mu)$ takes the public key $\pk$ and a single bit message $\mu\in\bit$ and outputs a ciphertext $c$. The notation \PKEEnc$_{\pk}(\mu;r)$ will be used to represent the encryption of a bit $\mu$ using randomness $r$. 
\item \textbf{Decryption.} The algorithm $\mu^*\leftarrow$ \PKEDec$_{\sk}(c)$ takes the secret key $\sk$ and a ciphertext $c$ and outputs a message $\mu^*\in\bit$. 
\end{itemize}
Obviously we require that $\mu=$\PKEDec$_{\sk}($\PKEEnc$_{\pk}(\mu))$

\begin{definition}[CPA Security for PKE]
\label{def:cpa-pke}
\label{def:cpa-he}
A scheme $\PKE$ is IND-CPA secure if for an infinite $\Lambda \subseteq \naturals$ we have
$$ (\pk,\PKEEnc_{\pk}(0)) \lind (\pk,\PKEEnc_{\pk}(1)) $$
where $(\pk,\sk)\leftarrow$ \PKEKeygen($1^{\lambda}$).
\end{definition}

\begin{remark}[Security for Multiple Messages]
\label{rem:cpa-multiple}
Notice that by a standard hybrid argument and Lemma \ref{lemma:hyb-arg} we can prove that any scheme secure according to Definition \ref{def:cpa-pke} is also secure for multiple messages (i.e. the two sequences of encryptions bit by bit of two bit strings are computationally indistinguishable).
We will use this fact in the proofs in Section $\ref{sec:vc}$, but we do not provide the formal definition for this type of security. We refer the reader to 5.4.2 in~\cite{goldreich2009foundations2}.
\end{remark}

\subsubsection{Somewhat Homomorphic Encryption}
A public-key encryption scheme is said to be homomorphic if there is an 
additional algorithm $\mathsf{Eval}$ which takes a input the public key $\pk$, 
the representation of a function $f:\bit^l\rightarrow\bit$ and a set of $l$ ciphertexts $c_1,\ldots,c_l$, and outputs a ciphertext $c_f$. 
%$\HE = (\HEKeygen, \HEEnc, \HEDec, \HEEval)$ is a public-key encryption scheme 
%with a few additions. It is a quadruple of algorithms which operate as follow:
%\begin{itemize}
%\item \textbf{Key Generation.} The algorithm $(\pk,\evk,\sk)\leftarrow$ %\HEKeygen$(1^{\lambda})$ takes a unary representation of the security parameter 
%and outputs a public key encryption key $\pk$, a public evaluation key $\evk$ and 
%a secret decryption key $\sk$.
%\item \textbf{Encryption.} The algorithm $c\leftarrow$ \HEEnc$_{\pk}(\mu)$ 
%operates exactly as $\PKEEnc$.
%\item \textbf{Decryption.} The algorithm $\mu^*\leftarrow$ \HEDec$_{\sk}(c)$ 
%operates exactly as $\PKEDec$.
%\item \textbf{Homomorphic Evaluation} The algorithm $c_{f}\leftarrow$ 
%$\HEEval}_{\pk}(f,c_1,\ldots,c_l)$ takes the evaluation key $\evk$, a function 
%$f:\bit^l\rightarrow\bit$ and a set of $l$ ciphertexts $c_1,\ldots,c_l$, and 
%outputs a ciphertext $c_f$. 

% It must be the case that:
% \begin{equation}
% \mathrm{\HEDec}_{\sk}(c_f) = f(\mathrm{\HEDec}_{\sk}(c_1),\ldots,\mathrm{\HEDec}_{\sk}(c_l))
% \end{equation}
% with probability $\geq 1- \gamma(\lambda)$. 
%\end{itemize}


%A homomorphic encryption scheme is said to be secure if it meets the following 
%notion of semantic security:

%\begin{definition}[CPA Security for HE]\label{def:cpa-he}
%A scheme $\HE$ is IND-CPA secure if for an infinite $\Lambda \subseteq \naturals$ 
%we have
%$$ (\pk,\evk,\mathrm{\HEEnc}_{\pk}(0)) \lind (\pk,\evk,\mathrm{\HEEnc}_{\pk}(1)) $$
%where $(\pk,\evk,\sk)\leftarrow$ \HEKeygen($1^{\lambda}$).
%\end{definition}

\newcommand{\cclass}{\mathcal{C}}
We proceed to define the homomorphism property. The next notion of $\cclass$-homomorphism is sometimes also referred to as ``somewhat homomorphism''.

\begin{definition}[$\cclass$-homomorphism]
Let $\cclass$ be a class of functions (together with their respective representations).
An encryption scheme $\PKE$ is $\cclass$-homomorphic (or, homomorphic for the class $\cclass$) if for every function $f_\lambda$ where  $f_\lambda \in \mathcal{F} \funfam{f} \in \cclass$ and respective inputs $\mu_1,\dots,\mu_l \in \bit$ (where $l = l(\lambda)$), it holds that if 
$(\pk, \sk) \gets \PKEKeygen(1^{\lambda})$ and $c_i \gets \PKEEnc_{\pk}(\mu_i)$
then 
\[
    \Pr[\PKEDec_{\sk}(\Eval_{\pk}(F, c_1,\dots, c_l)) \not = F(\mu_1,\dots,\mu_l)] = \negl(\lambda),
    % Note: this definition is not infinitely-often, but for large enough values. It's correct this way.
\]
\end{definition}


%As pointed out in~\cite{fhe-lwe}, there are two important properties that the 
%above definition does not require. First, it does not require that the ciphertexts
%$c_i$ are decryptable themselves, only that they become
%decryptable after homomorphic evaluation. Finally, it does not require that the 
%output of $\HEEval$
%can undergo additional homomorphic evaluation.


% \begin{definition}[CPA Security for HE]\label{def:cpa-he}
% A scheme \textnormal{\HE} is IND-CPA secure if, for any polynomial time adversary $\mathcal{A}$, there exists a negligible function $\mu(\cdot)$ such that 
% \begin{equation}
% \mathrm{Adv}_{\mathrm{CPA}}[\mathcal{A}] \EqDef  
% |\Pr[\mathcal{A}(\pk,\evk,\mathrm{\HEEnc}_{\pk}(0)) = 1] - 
% \Pr[\mathcal{A}(\pk,\evk,\mathrm{\HEEnc}_{\pk}(1)) = 1]|=\mu(\lambda)
% \end{equation}
% where $(\pk,\evk,\sk)\leftarrow$ \textnormal{\HEKeygen}($1^{\lambda}$). We further say that a scheme \textnormal{\HE} has CPA security $\delta$ for some negligible function $\delta(\cdot)$ if the above indistinguishability gap $\mu(\lambda)$ is smaller than $\delta(\lambda)^{\Omega(1)}$. 
% \end{definition}

As usual we require the scheme to be non-trivial by requiring that the output 
of $\mathsf{Eval}$ is compact:
\begin{definition}[\bfseries Compactness]\label{def:compact}
A homomorphic encryption scheme $\PKE$ is compact if there exists a polynomial $s$ in $\lambda$ such that the output length of $\mathsf{Eval}$ is at most $s(\lambda)$ bits long (regardless of the function $f$ being computed or the number of inputs). 
\end{definition}


\begin{definition}
\label{def:leveled}
Let $\cclass = \funfam{\cclass}$ of arithmetic circuits in $\GF(2)$. A scheme 
$\PKE$ is leveled  $\cclass$-homomorphic if it takes $1^L$ as additional input in 
key generation, and can only evaluate depth-$L$ arithmetic circuits from 
$\cclass$. The bound $s(\lambda)$ on the ciphertext must remain independent of 
$L$.
\end{definition}
