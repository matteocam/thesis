We now show how in repeated executions of a Rational Proof a nd where the Prover has a "budget" of computation cost that he is willing to invest, then there is no guarantee anymore that the profit is maximized by the honest prover. The reason is that it might be more profitable for the prover to use his budget to provide many incorrect answers than to provide a single correct answer. That's because incorrect (e.g. random) answers are ``cheaper" to compute than the correct one and with the same budget $B$ the prover can provide many of them while the entire budget might be necessary to solve a single problem correctly. If incorrect answers still receive a substantial reward then many incorrect answers may be more profitable and a rational prover will choose that strategy. 


This motivated us to consider a stronger definition which requires the reward to be 
somehow connected to the "effort" paid by the prover. The definition (stated below) basically says that if a (possibly dishonest) prover invests less computation than the honest prover then he must collect a smaller reward. 

Consider the protocol in the previous section for the computation of the function 
$G_{n,k}(\cdot)$. Assume that the honest execution of the protocol (including the computation of $G_{n,k}(\cdot)$) has cost $C=n$. 

Assume now that we are given a sequence of $n$ inputs 
$x^{(1)},\ldots,x^{(i)},\ldots$ where each $x^{(i)}$ is an $n$-bit string. 
In the following let $m_i$ be the Hamming weight of $x^{(i)}$ and $p_i=m_i/n$.

Therefore the honest prover investing $C=n$ cost, will be able to execute the protocol only only once, say on input $x^{(i})$.  By setting $p=\tilde{p}=p_i$ in Eq.~\ref{eq:bsr}, we see that $P$ obtains reward 
\[ R(x^{(i)}) = 2(p_i^2-p_i+1) \leq 2 \]
Consider instead a prover $\disP$ which in the execution of the protocol outputs a random value $\tilde{m} \in [0..n]$. The expected reward of $\disP$ on {\sf any} input $x^{(i)}$ is (by setting $p=p_i$ and $\tilde{p}=m/n$ in Eq.~\ref{eq:bsr} and taking 
expectations):
\begin{align*}
\tilde{R}(x^{(i)}) &= \expectation_{m,b}[BSR(\frac{m}{n}, b)] \\
                &= \frac{1}{n+1}\sum_{m=0}^{n}\expectation_b[BSR(\frac{m}{n},b] \\
                &= 
                \frac{1}{n+1}\sum_{m=0}^{n}(2(2p_i \cdot \frac{m}{n}-\frac{m^2}{n^2}-p_i+1))
                 \\
                 &=2-\frac{2n+1}{3n} > 1 \; \mbox{ for $n>1$.}
\end{align*}
Therefore by "solving" just two computations $\disP$ earns more than $P$. Moreover t
the strategy of $\disP$ has cost $1$ and 
therefore it earns more than $P$ by investing a lot less cost\footnote{
If we think of cost as time, then in the same time interval in which $P$ solves one problem, $\disP$ can solve up to $n$ problems, earning a lot 
more money, by answering fast and incorrectly.}.

Note that "scaling" the reward by a multiplier $M$ does not help in this case, since both the honest and dishonest prover's rewards would be multiplied by 
the same multipliers, without any effect on the above scenario. 

We have therefore shown a rational strategy, where cheating many times and collecting many rewards is more profitable than collecting a single reward for an honest computation. 
