
In the following we will adopt a "concrete-security" version of the "asymptotic" definitions and theorems in \cite{am1,rosen}. We assume the reader is familiar with the notion of interactive proofs \cite{gmr}. 

\noindent
\begin{definition}[Rational Proof]
\label{def:RP-delta}
\label{def:RP}
A function $f:$ $\binstring^n$ $\to$ $\binstring^n$ admits a rational proof if there exists an interactive proof $(P,V)$ and a randomized reward function
$\rew : \binstrings \to \posreals$ such that

\begin{enumerate}
\item \emph{(Rational completeness)} For any input $x \in 
\binstring^n$, $\Pr[\out((P,V)(x)) = f(x)] = 1.$

\item For every prover $\disP$, and for any input $x \in 
\binstring^n$ there exists a $\delta_{\disP}(x) \geq 0$ such that 
$ \expRewProtDis + \delta_{\disP}(x) \leq \expRewProtHon. $
\end{enumerate}
The expectations and the probabilities are taken over the random coins of the prover and verifier.
\end{definition} 

\noindent
Let $\epsilon_{\disP} = \Pr[\out((P,V)(x)) \neq f(x)]$. 
Following \cite{rosen} we define the {\sf reward gap} as 
\[ \Delta(x) = min_{P^* : \epsilon_{P^*}=1}[\delta_{P^*}(x)]  \]
i.e. the minimum reward gap over the provers that always report the incorrect value. 
It is easy to see that for arbitrary prover $\disP$ we have $\delta_{\disP}(x) \geq 
\epsilon_{\disP} \cdot \Delta(x)$. Therefore it suffices to prove that a protocol has 
a strictly positive reward gap $\Delta(x)$ for all $x$. 

